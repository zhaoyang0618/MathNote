\part{单复变函数}
《单复变函数》(Functions of One Complex Variable)的作者是J.B.康威(John B. Conway). 参考: \cite{FunctionsofOneComplexVariable1978}. 

\chapter{复数系}\label{section00201}

\section{实数}\label{subsection0020101}
我们用$\mathbb{R}$表示所有实数组成的集. 假定读者熟悉实数系及其性质, 特别地, 假定读者具备下面的知识: $\mathbb{R}$的序, 上确界和下确界的定义和性质, 以及$\mathbb{R}$的完备性($\mathbb{R}$中的每一个有上界的集必有上确界). 我们也假定读者熟知$\mathbb{R}$中的序列的收敛性与无穷级数. 最后, 一个人只有在单变量实函数方面有了坚实的基础之后, 才可以着手学习复变函数. 虽然在学习解析函数理论之前, 传统上是先学习多变数实函数. 但是对于本书来说, 本质上这不是必要的条件, 因为本书中任何地方都不需要这个领域里深入的结果. 

\section{复数域}\label{subsection0020102}
我们把复数集$\mathbb{C}$定义为所有有序数对$(a, b)$的集, 其中$a,b$是实数. 加法和乘法由下式定义: 
\begin{gather*}
(a, b) + (c, d) = (a+c, b+d), \\
(a, b)(c, d) = (ac-bd, bc + ad).
\end{gather*}
容易验证, 这样定义后, $\mathbb{C}$满足域(field)的所有公理. 这就是说, $\mathbb{C}$满足加法和乘法的结合律、交换律、分配了;$(0, 0)$和$(1,0)$分别是加法和乘法的单位元素, 并且$\mathbb{C}$内的每一个非零元素有加法和乘法的逆元素. 

对于复数$(a, 0)$, 我们将写为$a$, 这个映照$a \mapsto (a, 0)$定义了一个$\mathbb{R}$到$\mathbb{C}$的域同构\footnote{这恐怕不能说是同构, 因为明显不是一一映射, 应该是$\mathbb{R}$和$\mathbb{C}$的一个子集同构}, 所以我们可以把$\mathbb{R}$考虑为$\mathbb{C}$的一个子集. 如果令$i=(0, 1)$, 那么$(a, b) = a + ib$, 从现在起, 我们对复数就不再使用有序数对的记号了. 

注意到$i^2=-1$, 所以方程$z^2+1=0$在$\mathbb{C}$内有根. 事实上, 对于$\mathbb{C}$内的每个$z$, $z^2+1=(z+i)(z-i)$. 更一般地, 如果$z$和$w$是复数, 我们得到
\[
z^2+w^2 = (z+iw)(z-iw),
\]
令$z$和$w$是实数$a$和$b$($a$和$b$都不为0\footnote{这里只需要$a$和$b$不全为0即可. }), 我们得到
\[
\frac{1}{a+bi} = \frac{a-bi}{a^2+b^2} = \frac{a}{a^2+b^2} - i(\frac{b}{a^2+b^2}),
\]
这样我们就有了一个复数的倒数的公式. 

当我们写$z = a + bi$($a, b \in \mathbb{R}$)时, 我们称$a$, $b$为$z$的实部和虚部, 并且用$a = \Re{z}$, $b=\Im{z}$表示. 

作为本节的结尾, 我们在$\mathbb{C}$内引进两个运算. 这两个运算不是域的运算. 如果$z = x+iy$($x, y \in \mathbb{R}$), 那么我们定义$|z| = (x^2+y^2)^{\frac{1}{2}}$为$z$的绝对值, $\bar{z}=x-iy$为$z$的共轭数. 注意: 
\begin{gather}\label{equ00202001}
|z|^2=z\bar{z},
\end{gather}
特别地, 如果$z \neq 0$, 那么
\[
\frac{1}{z} = \frac{\bar{z}}{|z|^2}.
\]

下面是绝对值和共轭数的基本性质, 其证明留给读者. 
\begin{gather}
\Re{z} = \frac{1}{2}(z + \bar{z}), \quad \Im{z} = \frac{1}{2i}(z - \bar{z}). \label{equ00202002} \\
(\bar{z+w}) = \bar{z} + \bar{w}, \quad \bar{zw} = \bar{z}\bar{w}. \label{equ00202003}\\
|zw| = |z||w|. \label{equ00202004} \\
|z/w| = |z|/|w|.\label{equ00202005}\\
|\bar{z}| = |z|.\label{equ00202006}
\end{gather}
读者证明后面三个式子的时候, 应当尽量避免将$z$和$w$展开为它们的实部和虚部, 而最好利用(\ref{equ00202001}), (\ref{equ00202002})和(\ref{equ00202003}). 

\begin{exercise}
求下列各复数的实部和虚部: 
\begin{gather*}
\begin{aligned}
&\frac{1}{z}; \frac{z-a}{z+a}(a \in \mathbb{R}); z^2; \frac{3+5i}{7i+1}; (\frac{-1+i\sqrt{3}}{2})^3;\\
&(\frac{-1-i\sqrt{3}}{2})^6; i^n; (\frac{1+i}{\sqrt{2}})^n, 2 \le n \le 8.
\end{aligned}
\end{gather*}
\end{exercise}

\begin{exercise}
求下列各复数的绝对值和共轭数: 
\[
\begin{aligned}
&-2+i; -3; (2+i)(4+3i);\frac{3-i}{\sqrt{2}+3i};\frac{i}{i+3}; \\
&(1+i)^6; i^{17}.
\end{aligned}
\]
\end{exercise}

\begin{exercise}
证明: 当且仅当$z=\bar{z}$时, $z$才是实数. 
\end{exercise}

\begin{exercise}\label{exer002010204}
若$z$和$w$是复数, 证明下列等式: 
\[
\begin{aligned}
&|z+w|^2 = |z|^2 + 2\Re{z\bar{w}} + |w|^2 \\
&|z-w|^2 = |z|^2 - e\Re{z\bar{w}} + |w|^2 \\
&|z+w|^2 + |z-w|^2 = 2(|z|^2+|w|^2)
\end{aligned}
\]
\end{exercise}

\begin{exercise}
设$z=z_1+\cdots+z_n$, $w= w_1+\cdots+w_n$, 利用归纳法证明: 
\[
|w| = |w_1|\cdots|w_n|; \bar{z}=\bar{z_1}+\cdots+\bar{z_n}; \bar{w}=\bar{w_1}\cdots\bar{w_n}.
\]
\end{exercise}

\begin{exercise}
设$R(z)$是$z$的有理函数, 如果$R(z)$的所有系数是实数, 则$\overline{R(z)} = R(\bar{z})$. 
\end{exercise}

\section{复平面}\label{subsection0020103}
从复数的定义易见, $\mathbb{C}$中每一点$z$都可以和平面$\mathbb{R}^2$上唯一确定的点$(\Re{z}, \Im{z})$相等同. 复数的加法恰好就是向量空间$\mathbb{R}^2$的加法, 如果$z$和$w$是$\mathbb{C}$中的点, 那么从$z$和$w$到$0(=(0,0))$画两条直线, 这两条直线形成了以$0$、$z$、$w$为三个顶点的平行四边形的两条边, 平行四边形的第四个顶点就是$z+w$. 

注意, $|z-w|$恰好是$z$和$w$之间的距离, 理会到这一点, 上节习题\ref{exer002010204}中的最后一个等式说的就是平行四边形法则: 平行四边形各边长的平方和等于其对角线的平方和. 

距离函数的基本性质是它满足三角不等式(见下一章). 在这种情况下, 对复数$z_1$, $z_2$, $z_3$, 这个不等式变为
\[
|z_1-z_2| \le |z_1-z_3| + |z_3-z_2|.
\]

利用$z_1-z_2 = (z_1-z_3) + (z_3-z_2)$, 容易看出, 我们只需证明
\begin{gather}\label{equ002010301}
|z+w| \le |z| + |w| \quad (z, w \in \mathbb{C})
\end{gather}
为了证明这个不等式, 首先看出, 对于$\mathbb{C}$中任意$z$, 
\begin{gather}\label{equ002010302}
\begin{aligned}
&-|z| \le \Re{z} \le |z|, \\
&-|z| \le \Im{z} \le |z|.
\end{aligned}
\end{gather}
因此, $\Re{z\bar{w}} \le |z\bar{w}| = |z||w|$. 于是
\[
\begin{aligned}
|z+w|^2 &= |z|^2 + 2\Re{z\bar{w}} + |w|^2\\
&\le |z|^2 + 2|z||w| + |w|^2 = (|z|+|w|)^2,
\end{aligned}
\]
由此推出\ref{equ002010301}. (这个式子称为三角不等式, 因为如果我们把$z$和$w$表示在平面上, (\ref{equ002010301})式表明, 三角形$[0,z,z+w]$的一边的长度小于另外两边长度的和. 或者说两点间的最短距离是直线). 在遇到一个不等式时, 人们总应当问一问等号成立的必要充分条件是什么, 考察一个三角形并考虑到(\ref{equ002010301})的几何意义, 我们就引出条件$z=tw$, 对某一$t \in \mathbb{R}$, $t \ge 0$. (或者如果$w=0$, 则$w=tz$). 显然, 当这两点和原点共线时, 等号成立. 事实上, 如果我们看一下(\ref{equ002010301})式的证明, 便知道$|z+w|=|z|+|w|$成立的必要充分条件是$|z\bar{w}|=\Re(z\bar{w})$. 这等价于$z\bar{w}\ge 0$(即$z\bar{w}$使非负实数). 如果$w \neq 0$, 两边乘以$w/w$, 我们得到$|w|^2(z/w) \ge 0$, 令
\[
t = z/w = (\frac{1}{|w|^2})|w|^2(z/w),
\]
那么$z=tw$, $t \ge 0$. 

由归纳法, 我们也有
\begin{gather}\label{equ002010303}
|z_1+z_2+\cdots+z_n| \le |z_1|+|z_2+\cdots+|z_n|,|
\end{gather}
不等式
\begin{gather}\label{equ002010304}
||z| - |w|| \le |z-w|
\end{gather}
也是有用的. 

既然我们给出了绝对值的几何解释, 让我们再来看一看, 平面上一点的共轭复数是什么, 这是容易的, 事实上, $\bar{z}$是$z$关于$x$轴(即实轴)的对称点. 

\begin{exercise}
证明(\ref{equ002010304})并给出等号成立的必要充分条件. 
\end{exercise}

\begin{exercise}
证明: (\ref{equ002010302})中的等号成立, 当且仅当, 对任意整数$k$和$l$, $1 \le k,l \le n$, 只要$z_l \neq 0$, 就有$z_k/z_l \le 0$. 
\end{exercise}

\begin{exercise}
设$a \in \mathbb{R}$, $c > 0$是固定的. 对于每个可能选取的$a$和$c$, 试描画出满足条件
\[
|z-a| - |z+a|=2c
\]
的点集. 现在设$a$施任意复数, 利用平面的旋转画出满足上述方程的点的轨迹. 
\end{exercise}

\section{复数的极坐标表示与复数的方根}\label{subsection0020104}
考虑复平面$\mathbb{C}$的点$z=x+iy$. 这个点有极坐标$(r, \theta)$: $x=r\cos{\theta}$, $y=r\sin{\theta}$. 显然$r=|z|$, $\theta$是正实轴与从0到$z$的直线段的夹角. 注意, 在上述等式中的$\theta$可以代之以$\theta$加上$2\pi$的任意整数倍, 角$\theta$称为$z$的幅(还是这个“辐)”角, 记为$\theta=\arg{z}$. 由于$\theta$的不确定性, “$\arg$”不是一个函数. 我们引进记号
\begin{gather}\label{equ002010401}
\cis{\theta}=\cos{\theta}+i\sin{\theta}.
\end{gather}

设$z_1=r_1\cis{\theta_1}$, $z_2=r_2\cis{\theta_2}$, 那么
\[
\begin{aligned}
z_1z_2=r_1r_2\cis{\theta_1}\cis{\theta_2}&=r_1r_2[(\cos{\theta_1}\cos{\theta_2}-\sin{\theta_1}\sin{}\theta_2)\\
+i(\sin{\theta_1}\cos{\theta_2}+\sin{\theta_2}\cos{\theta_1})]
\end{aligned}
\]
由正弦和余弦的和角公式, 我们得到
\begin{gather}\label{equ002010402}
z_1z_2 = r_1r_2\cis(\theta_1+\theta_2).
\end{gather}
换句话说$\arg{z_1z_2} = \arg{z_1} + \arg{z_2}$(什么实函数把成绩变为和?\footnote{对数函数$\ln{(ab)}=\ln{a}+\ln{b}$}). 由归纳法, 对于$z_k=r_k\cis{\theta_k}$, $1 \le k \le n$, 我们有
\begin{gather}\label{equ002010403}
z_1z_2\cdots{}z_n=r_1r_2\cdots{}r_n\cis{(\theta_1+\theta_2+\cdots+\theta_n)}.
\end{gather}
特别地, 对于每个整数$n \ge 0$, 有
\begin{gather}\label{equ002010404}
z^n = r^n\cis(n\theta).
\end{gather}
进而若$z \neq 0$, 则$z[r^{-1}\cis(-\theta)]=1$;所以如果$z \neq 0$, 那么对于一切整数$n$, 正的, 负的. 或0, (\ref{equ002010404})也成立. 作为(\ref{equ002010404})的一个特别情形, 我们得到棣莫佛(de Moivre)公式: 
\[
(\cos{\theta}+i\sin{\theta})^n=\cos{n\theta} + i\sin{n\theta}.
\]

现在我们可以来考虑下面的问题了. 对于给定的一个复数$a \neq 0$, 和一个整数$n \ge 2$, 你能否找到满足$z^n=a$的数$z$?这样的$z$你能找到多少个?由于(\ref{equ002010404})式, 解答这个问题是容易的. 设$a=|a|\cis{\alpha}$;由(\ref{equ002010404}), $z=|a|^{\frac{1}{n}}\cis{(\alpha/n)}$就满足要求. 但是这个解不是唯一解, 因为$z'=|a|^{\frac{1}{n}}\cis{\frac{1}{n}(\alpha+2\pi)}$也满足$(z')^n=a$. 事实上, 每一个数
\begin{gather}\label{equ002010405}
|a|^{\frac{1}{n}}\cis{\frac{1}{n}(\alpha+2\pi{}k)}, \quad 0 \le k \le n-1.
\end{gather}
都是$a$的$n$次方根. 借助(\ref{equ002010404})我们得到下述结果: 对于$\mathbb{C}$中的每一个不等于零的数$a$, 都有$a$的$n$个不同的$n$次方根, 它们由公式(\ref{equ002010405})给出. 

\textbf{例子} \quad 计算$n$次单位根. 由于$1=\cis{0}$, (\ref{equ002010405})式给出如下这些根: 
\[
1,\cis{\frac{2\pi}{n}},\cis{\frac{4\pi}{n}},\cdots\cis{\frac{2\pi}{n}(n-1)}.
\]
特别地, 立方单位根是
\[
1, \frac{1}{2}(-1 + i\sqrt{3}),\frac{1}{2}(-1-i\sqrt{3}).
\]

\begin{exercise}
求出6次单位根. 
\end{exercise}

\begin{exercise}
计算: 
\begin{enumerate}
\item[(a)]$i$的平方根;
\item[(b)]$i$的立方根;
\item[(c)]$\sqrt{3}+3i$的平方根. 
\end{enumerate}
\end{exercise}

\begin{exercise}
$n$次单位原根是一复数$a$, 使得$1,a,a^2,\cdots,a^{n-1}$是$n$个不同的$n$次单位根. 证明: 如果$a,b$分别是$n$次和$m$次单位原根, 则$ab$是$k$次单位根, $k$是某一整数. $k$的最小值是什么?如果$a,b$是非单位原根, 你能说些什么?
\end{exercise}

\begin{exercise}
试利用二项式
\[
(a+b)^n=\sum_{k=0}^{n}{{n \choose k}a^{n-k}b^{k}},
\]
其中${n \choose k}=\frac{n!}{k!(n-k)!}$, 并比较棣莫佛公式两边的实部和虚部, 得到公式
\[
\begin{aligned}
&\cos{n\theta} = \cos^{n}{\theta} - {n \choose 2}\cos^{n-2}{\theta}\sin^{2}{\theta} + {n \choose 4}\cos^{n-4}{\theta}\sin^{4}{\theta}-\cdots\\
&\sin{n\theta} = {n \choose 1}\cos^{n-1}{\theta}\sin{theta}-{n \choose 3}\cos^{n-3}{\theta}\sin^{3}{\theta}+\cdots.
\end{aligned}
\]
\end{exercise}

\begin{exercise}
设$z=\cis{\frac{2\pi}{n}}$, 整数$n \ge 2$. 证明: $1+z+\cdots+z^{n-1}=0$.
\end{exercise}

\begin{exercise}
证明: $\phi(t)=\cis{t}$是加法群$\mathbb{R}$到乘法群$T=\{z:|z|=1\}$上的群同态. 
\end{exercise}

\begin{exercise}
如果$z \in \mathbb{C}$, 并且对于每个正整数$n$, $\Re{z^n} \ge 0$, 证明: $z$是正实数. 
\end{exercise}

\section{复平面上的直线和半平面}\label{subsection0020105}
设$L$表示$\mathbb{C}$中的直线. 从初等解析几何知道, $L$是由$L$上的一个点和一个方向向量决定的. 于是, 如果$a$是$L$上任一点, $b$是它的方向向量, 那么
\[
L = \{z = a+tb:-\infty < t < \infty\}.
\]
由于$b \neq 0$, 这就给出, 对于$L$上的$z$, 有
\[
\Im{(\frac{z-a}{b})} = 0.
\]
事实上, 如果$z$满足等式
\[
0 = \Im{(\frac{z-a}{b})},
\]
那么
\[
t = \frac{z-a}{b},
\]
蕴含$z = a + tb$, $-\infty < t < \infty$. 这就是说
\begin{gather}\label{equ002010501}
L = \big{\{} z: \Im{(\frac{z-a}{b})} = 0 \big{\}}.
\end{gather}
集合
\[
\begin{aligned}
&\big{\{} z: \Im{(\frac{z-a}{b})} > 0 \big{\}},\\
&\big{\{} z: \Im{(\frac{z-a}{b})} < 0 \big{\}}.
\end{aligned}
\]
的轨迹是什么呢?作为回答这个问题的第一步, 注意到$b$是一个方向, 我们可以假定$|b|=1$. 我们暂时考虑$a=0$的情形. 并且令$H_0=\{z:\Im{(z/b)}>0\}$, $b = \cis{\beta}$. 如果$z = r\cis{\theta}$, 那么$z/b = r\cis{(\theta-\beta)}$. 于是$z$在$H_0$中, 当且仅当$\sin(\theta-\beta)>0$, 即$\beta < \theta < \pi + \beta$. 所以, 如果我们“按照$b$的方向沿着$L$前进”, $H_0$是位于$L$左边的半平面. 如果我们令
\[
H_a = \big{\{} z: \Im{(\frac{z-a}{b})} > 0 \big{\}},
\]
那么容易看出, $H_a = a + H_0 \equiv \{ a + w: w \in H_0\}$;即$H_a$是由$H_0$平移$a$而得到的, 因此, $H_a$是位于$L$的左边的半平面. 类似地, 
\[
K_a = \big{\{} z: \Im{(\frac{z-a}{b})} < 0 \big{\}}
\]
是位于$L$的右边的半平面. 

\begin{exercise}
设$C$是圆周$\{z:|z-c|=r\}$, $r > 0$, $a = c + r\cis{\alpha}$;并且令
\[
L_{\beta} = \big{\{} z: \Im{(\frac{z-a}{b})} = 0 \big{\}},
\]
其中$b=\cis{\beta}$. 找出$L_{\beta}$在$a$处切于圆周$C$的关于$\beta$的充分必要条件. 
\end{exercise}

\section{扩充平面及其球面表示}\label{subsection0020106}
在复分析中, 我们常常涉及到这样一些函数, 当自变量趋于给定点时, 它们趋于无穷. 为了讨论这种情形, 我们引进扩充平面$\mathbb{C}_{\infty}\equiv \mathbb{C} \bigcup \{\infty\}$. 同时为了讨论到取到无穷作为它的值的函数的连续性. 我们也希望在$\mathbb{C}_{\infty}$内引进距离函数. 为了这个目的以及为了给出$\mathbb{C}_{\infty}$的具体图像, 我们把$\mathbb{C}_{\infty}$表示为$\mathbb{R}^3$中的单位球面
\[
S = \{(x_1,x_2,x_3) \in \mathbb{R}^3:x_1^2 + x_2^2 + x_3^2=1\}.
\]

设$N=(0, 0,1)$;即$N$是$S$上的北极. 同时, 把$\mathbb{C}$等同于$\{(x_1,x_2,0):x_1,x_2 \in \mathbb{R}\}$, 于是$\mathbb{C}$沿赤道切割$\mathbb{C}$. 现在对于$\mathbb{C}$中每个点$z$, 考虑$\mathbb{R}^3$中通过$z$和$N$的直线. 这条直线与球面恰好交于一点$Z \neq N$. 如果$|z| > 1$, 那么$Z$在北半球面上;如果$|z|<1$, 那么$Z$在南半球面上;如果$|z|=1$, 那么$Z = z$. 当$|z| \to \infty$时, $Z$怎样呢?很清楚, $Z$趋于$N$. 因此, 我们就把$N$与$\mathbb{C}_{\infty}$中的$\infty$等同起来. 这样一来, $\mathbb{C}_{\infty}$就被表示为球面$S$了. 

让我们来考察这种表示法. 令$z=x+iy$,设$Z = (x_1, x_2, x_3)$是$S$上相应的点, 我们要找出用$x$, $y$表示$x_1, x_2, x_3$的方程. 在$\mathbb{R}^3$中通过$z$和$N$的直线由$\{tN + (1-t)z:-\infty<t<\infty\}$或
\begin{gather}\label{equ002010601}
\{((1-t)x, (1-t)y, t): -\infty < t < \infty\}
\end{gather}
给出. 因此, 如果能够找到直线和$S$的交点的$t$值, 我们就能够找到$Z$的坐标. 如果$t$是这个值, 那么
\[
1 = (1-t)^2x^2 + (1-t)^2y^2 + t^2 = (1-t)^2|z|^2 + t^2.
\]
由此注意到
\[
1-t^2 = (1-t)^2|z|^2.
\]
因为$t \neq 1$($z \neq \infty$), 所以
\[
t = \frac{|z|^2-1}{|z|^2+1}.
\]
于是
\begin{gather}\label{equ002010602}
x_1 = \frac{2x}{|z|^2+1}, x_2 = \frac{2y}{|z|^2+1}, x_3 = \frac{|z|^2-1}{|z|^2+1}.
\end{gather}
这就给出
\begin{gather}\label{equ002010603}
x_1 = \frac{z + \bar{z}}{|z|^2+1}, x_2 = \frac{-i(z-\bar{z})}{|z|^2+1}, x_3 = \frac{|z|^2-1}{|z|^2+1}.
\end{gather}

如果$Z$是给定的($Z \neq N$), 我们希望找$z$. 这时, 通过令$t = x_3$并利用(\ref{equ002010601}), 我们得到
\begin{gather}\label{equ002010604}
z = \frac{x_1 + ix_2}{1 - x_3}
\end{gather}
现在让我们用下面的方式定义扩充平面上点之间的距离函数: 对于$\mathbb{C}_{\infty}$中的$z$, $z'$, 定义$z$到$z'$的距离$d(z,z')$为它们在$\mathbb{R}^3$中相应两点$Z$和$Z'$的距离. 如果$Z=(x_1,x_2,x_3)$, $Z'=(x_1',x_2',x_3')$, 那么
\begin{gather}\label{equ002010605}
d(z,z') = [(x_1-x_1')^2 + (x_2-x_2')^2 + (x_3-x_3')^2]^{\frac{1}{2}}.
\end{gather}
利用$Z$和$Z'$在$S$上这一事实, (\ref{equ002010605})给出
\begin{gather}\label{equ002010606}
[d(z,z')]^2 = 2 - 2(x_1x_1' + x_2x_2'+x_3x_3').
\end{gather}
由(\ref{equ002010603}), 我们得到
\begin{gather}\label{equ002010607}
d(z,z') = \frac{2|z-z'|}{[(1+|z|^2)(1+|z'|^2)]^{\frac{1}{2}}}, \quad (z,z' \in \mathbb{C}).
\end{gather}
用类似的方法, 对于$\mathbb{C}$中的$z$, 我们得到
\begin{gather}\label{equ002010608}
d(z, \infty) = \frac{2}{(1 + |z|^2)^{\frac{1}{2}}},
\end{gather}
球面$S$和$\mathbb{C}_{\infty}$的点之间这种对应关系称为球极平面投影. 

\begin{exercise}
给出(\ref{equ002010607})和(\ref{equ002010608})的详细推导. 
\end{exercise}

\begin{exercise}
对于下列$\mathbb{C}$中的点给出$S$上对应的点: $0, 1+i,3+2i$. 
\end{exercise}

\begin{exercise}
$S$上哪些子集对应$\mathbb{C}$中的实轴和虚轴. 
\end{exercise}

\begin{exercise}
设$\Lambda$是$S$上的一个圆周, 那么在$\mathbb{R}^3$中有唯一的平面$P$, 使得$P \bigcap S = \Lambda$. 由解析几何知道
\[
P = \{(x_1,x_2,x_3):x_1\beta_1 + x_2\beta_2 + x_3\beta_3 = l\},
\]
其中$(\beta_1,\beta_2,\beta_3)$是与$P$正交的一个向量, $l$是某一实数. 可以假设$\beta_1^2+\beta_2^2+\beta_3^2=1$. 利用这一事实, 证明: 如果$\Lambda$包含点$N$, 则它在$\mathbb{C}$上的投影是一直线. 否则, $\Lambda$投影到$\mathbb{C}$中的一个圆周上. 
\end{exercise}

\begin{exercise}
设$Z$和$Z'$是$S$上分别与$z$和$z'$相应的两点. $W$是$S$上与$z+z'$对应的点. 试用$Z$和$Z'$的坐标表示出$W$的坐标. 
\end{exercise}

\chapter{度量空间与$\mathbb{C}$的拓扑}\label{section00202}

\section{度量空间的定义和例子}\label{subsection0020201}
一个度量空间是一个序偶$(X, d)$, 这里$X$是一个集, $d$是一个从$X \times X$到$\mathbb{R}$的函数, 称之为距离函数或度量, 它满足下列条件: 
\[
d(x, y) \ge 0;
\]
当且仅当$x=y$时, $d(x, y)=0$;
\begin{gather*}
d(x, y) = d(y,x) \quad (\text{对称性});\\
d(x, z) \le d(x, y) + d(y, z)\quad (\text{三角不等式}).
\end{gather*}
如果$x$和$r > 0$是固定的, 那么定义
\begin{gather*}
B(x; r) = \{y \in X: d(x, y) < r\},\\
\bar{B}(x; r) = \{y \in X: d(x, y) \le r\}.
\end{gather*}
$B(x; r)$和$\bar{B}(x; r)$分别称为以$x$为中心, $r$为半径的开球和闭球. 

\textbf{例子}

\begin{example}\label{exam002020101}
设$X = \mathbb{R}$或$\mathbb{C}$, 定义$d(z, w)=|z-w|$, 这就使$(\mathbb{R}, d)$和$(\mathbb{C}, d)$都成为度量空间. 事实上, $(\mathbb{C}, d)$将是我们最感兴趣的例子. 如果读者在此以前从来未接触过度量空间的概念, 那么在学习这一章的过程中应当时常想到$(\mathbb{C}, d)$. 
\end{example}

\begin{example}\label{exam002020102}
设$(X, d)$是一个度量空间, $Y \subset X$;那么$(Y, d)$也是一个度量空间. 
\end{example}

\begin{example}\label{exam002020103}
设$X = \mathbb{C}$, 定义$d(x+iy, a+ib)=|x-a|+|y-b|$. 那么$(\mathbb{C}, d)$是一个度量空间. 
\end{example}

\begin{example}\label{exam002020104}
设$X = \mathbb{C}$, 定义$d(x+iy, a+ib)=\max{|x-a|, |y-b|}$. 
\end{example}

\begin{example}\label{exam002020105}
设$X$是任意一个集, 定义$d(x, y) = 0$, 如果$x=y$;$d(x, y)=1$, 如果$x \neq y$. 为了证明函数$d$满足三角不等式, 只要考虑在$x,y,z$当中出现相等的各种可能情形. 注意, 如果$r \le 1$, 则$B(x; r)$只由一个点$x$所组成;如果$r > 1$, 则$B(x;r) = X$. 这个度量空间在解析函数论的研究中并不出现. 
\end{example}

\begin{example}\label{exam002020106}
设$X = \mathbb{R}^n$, 对于$\mathbb{R}^n$中的$x=(x_1,\cdots, x_n)$和$y=(y_1,\cdots,y_n)$定义
\[
d(x, y) = [\sum_{j=1}^{n}{(x_j-y_j)^2}]^{\frac{1}{2}}.
\]
\end{example}

\begin{example}\label{exam002020107}
设$S$是任意一个集, $B(S)$表示满足条件
\[
\left\|f\right\|_{\infty} = \sup{\{\left|f(s)\right|: s \in S\}} < \infty
\]
的函数$f: S \to \mathbb{C}$的集. 这就是说, $B(S)$由所有其值域位于某一有穷半径的圆内的复值函数所构成. 对于$B(S)$中的$f$和$g$定义$d(f, g) = \left\|f-g\right\|_{\infty}$. 我们来证明$d$满足三角不等式. 事实上, 如果$f$, $g$和$h$在$B(S)$中, $s$是$S$中的任意一点, 那么$|f(s)-g(s)| = |f(s)-h(s)+h(s)-g(s)| \le |f(s)-h(s)| + |h(s)-g(s)| \le \left\|f-h\right\|_{\infty} + \left\|h - g\right\|_{\infty}$. 于是若对于$S$中所有的$s$取上确界, 则有$\left\|f-g\right\|_{infty} \le \left\|f-h\right\|_{\infty} + \left\|h - g\right\|_{\infty}$, 这就是对于$d$的三角不等式. 
\end{example}

\begin{definition}{开集}{def002020101}
对于度量空间$(X, d)$, 一个集$G \subset X$是开集, 如果$G$内的每一个$x$, 都存在一个$\epsilon > 0$, 使得$B(x;\epsilon) \subset G$. 
\end{definition}

于是, 一个集在$\mathbb{C}$内是开的, 如果它没有“边”. 例如, 
\[
G = \{z \in G: a < \Re{(z)} < b\}
\]
是开的;但是$\{z: \Re{(z)} < 0\} \bigcap \{0\}$不是开的, 因为不管我们把$\epsilon$取得多么小, $B(0;\epsilon)$都不能包含在这个集内. 

我们用$\emptyset$表示空集, 就是一个元素也没有的集. 

\begin{proposition}{}{prop002020101}
设$(X, d)$是一个度量空间, 那么:
\begin{enumerate}
\item[(a)]集$X$和$\emptyset$是开集. 
\item[(b)]如果$G_1,\cdots, G_n$是$X$中的开集, 则$\bigcap_{k=1}^{n}{G_k}$也是$X$中的开集. 
\item[(c)]如果$\{G_j:j \in J\}$是$X$中的开集族, $J$是任一指标集, 则$G = \bigcup\{G_j:j \in J\}$也是开集. 
\end{enumerate}
\end{proposition}

\begin{proof}
(a)的证明是平凡的. 为了证明(b), 设$x \in G = \bigcap_{k=1}^{n}{G_k}$;那么$x \in G_k$, $k=1,2,\cdots,n$. 于是由定义, 对于每个$k$有$\epsilon_k > 0$, 使得$B(x;\epsilon_k) \subset G_k$. 如果取$\epsilon=\min(\epsilon_1,\epsilon_2,\cdots, \epsilon_n)$, 那么, 对于$1 \le k \le n$, $B(x;\epsilon) \subset B(x;\epsilon_k) \subset G_k$, 于是$B(x; \epsilon) \subset G$, $G$是开集. 

(c)的证明留给读者作为习题. 
\end{proof}

在度量空间里还有另一类著名的子集. 这类子集包含它们的全部“边”, 换一个说法, 它们的余集没有“边”. 

\begin{definition}{闭集}{def002020102}
一个集$F \subset X$是闭的, 如果它的余集$X-F$是开的. 
\end{definition}

下面的命题是命题\ref{pro:prop002020101}的补命题. 对于前一命题应用Morgan法则便可完成其证明, 我们把它留给读者. 

\begin{proposition}{}{prop002020102}
设$(X, d)$是一个度量空间, 那么: 
\begin{enumerate}
\item[(a)]集$X$和$\emptyset$是闭的. 
\item[(b)]如果$F_1,\cdots, F_n$是$X$中的闭集, 则$\bigcup_{k=1}^{n}{F_k}$也是$X$中的闭集. 
\item[(c)]如果$\{F_j:j \in J\}$是$X$中的闭集族, $J$是任一指标集, 则$F = \bigcap\{F_j:j \in J\}$也是闭集. 
\end{enumerate}
\end{proposition}

在学习开集和闭集时, 最普遍的错误是把闭集的定义解释为一个集不是开集便是闭集. 这种理解当然是错误的. 只要看集$\{z \in \mathbb{C} : \Re{(z)} > 0\} \cup \{0\}$就清楚了, 这个集既不是开的, 也不是闭的. 

\begin{definition}{}{def002020103}
设$A$是$X$的子集, 那么, $A$的内部$\intset{A}$就是集合$\bigcup\{G: G\text{是开集, 且}G \subset A\}$. $A$的闭包$A^-$就是集$\bigcap\{F: F\text{是闭集, 且}F \supset A\}$. 注意, $\intset{A}$可以是空集, $A^-$可以是$X$. 如果$A = \{a + ib : a\text{和}b\text{是有理数}\}$, 那么同时有$A^-=\mathbb{C}$和$\intset{A} = \emptyset$. 根据命题\ref{pro:prop002020101}和\ref{pro:prop002020102}, $A^-$是闭集, $\intset{A}$是开集. $A$的边界记为$\partial{A}$, 定义为$\partial{A} = A^- \cap (X-A)^-$. 
\end{definition}

\begin{proposition}{}{prop002020103}
设$A$和$B$是度量空间$(X, d)$的子集, 那么: 
\begin{enumerate}
\item[(a)]当且仅当$A = \intset{A}$时$A$是开集. 
\item[(b)]当且仅当$A = A^-$时$A$是闭集. 
\item[(c)]$\intset{A} = X - (X-A)^-$;$A^- = X- \intset(X-A)$;$\partial{A} = A^- - \intset{A}$. 
\item[(d)]$(A \bigcup B)^- = A^- \bigcup B^-$. 
\item[(e)]当且仅当存在$\epsilon > 0$, 使得$B(x_0; \epsilon) \subset A$时, $x_0 \in \intset{A}$. 
\item[(f)]当且仅当, 对每一$\epsilon > 0$, $B(x_0; \epsilon) \cap A \neq \emptyset$时, $x_0 \in A^-$. 
\end{enumerate}
\end{proposition}
\begin{proof}
(a)至(e)的证明留给读者. 为了证明(f), 假设$x_0 \in A^- = X - \intset(X-A)$;于是, $x_0 \not\in \intset(X-A)$. 由(e), 对于每一$\epsilon > 0$, $B(x_0; \epsilon)$不包含在$X-A$内, 这就是说, 存在一个点$y \in B(x_0; \epsilon)$, $y$不在$X-A$内. 所以$y \in B(x_0;\epsilon) \cap A$. 现在设$x_0 \not\in A^- = X - \intset{(X-A)}$, 那么$x_0 \in \intset(X-A)$, 由(e), 存在$\epsilon > 0$, 使得$B(x_0; \epsilon) \subset X-A$. 即$B(x_0; \epsilon) \cap A = \emptyset$, 所以$x_0$不满足条件. 
\end{proof}

最后, 再定义一类著名的集合. 

\begin{definition}{稠密}{def002020104}
度量空间$X$的一个子集$A$是稠密的, 如果$A^- = X$. 
\end{definition}

有理数集$\mathbb{Q}$在$\mathbb{R}$中是稠密的, $\{x + iy : x, y, \in \mathbb{Q}\}$在$\mathbb{C}$中是稠密的. 

\begin{exercise}
证明: (\ref{exam002020102})至(\ref{exam002020106})中给出的那些例子都确实是度量空间, 只有例子(\ref{exam002020106})的证明可能会有些困难, 对于这些例子给出$B(x;r)$. 
\end{exercise}

\begin{exercise}
$\mathbb{C}$的下列子集, 哪些是开集, 哪些是闭集?
\begin{enumerate}
\item[(a)]$\{z:|z|<1\}$;
\item[(b)]实轴;
\item[(c)]$\{z: z^n=1,\text{对某一整数}n \ge 1\}$;
\item[(d)]$\{z \in \mathbb{C}: z\text{是实数, 且}0 \le z <1\}$;
\item[(e)]$\{z \in \mathbb{C}: z\text{是实数, 且}0 \le z \le 1\}$. 
\end{enumerate}
\end{exercise}

\begin{exercise}
如果$(X, d)$是任一度量空间, 证明: 每一个开球是开集, 每一个闭球是闭集. 
\end{exercise}

\begin{exercise}
给出(\ref{pro:prop002020101}c)的详细证明. 
\end{exercise}

\begin{exercise}
证明命题\ref{pro:prop002020102}. 
\end{exercise}

\begin{exercise}
证明: 一个集$G \subset X$是开的, 当且仅当$X-G$是闭的. 
\end{exercise}

\begin{exercise}
证明: $(\mathbb{C}_{\infty}, d)$是一度量空间, 其中$d$是由第一章的(\ref{equ002010607}), (\ref{equ002010608})给出的. 
\end{exercise}

\begin{exercise}\label{exer002020108}
设$(X, d)$是一度量空间, $Y \subset X$, 又设$G \subset X$是开的, 证明$G \cap Y$是$(Y, d)$中的开集. 反之, 如果$G_1 \subset Y$是$(Y, d)$中的开集, 则存在开集$G \subset X$, 使得$G_1 = G \cap Y$. 
\end{exercise}

\begin{exercise}\label{exer002020109}
在上题中用“闭的”代替“开的”. 
\end{exercise}

\begin{exercise}
证明命题\ref{pro:prop002020103}
\end{exercise}

\begin{exercise}
证明: $\{\cis{k}:k \ge 0\}$在$T=\{z \in \mathbb{C}:|z|=1\}$中是稠密的. 对于哪些$\theta$的值, $\{\cis(k\theta):k>0\}$在$T$中是稠密的?
\end{exercise}

\section{连通性}\label{subsection0020202}
作为这一节的开始, 让我们先给出一个例子. 设$X = \{ x \in \mathbb{C} : |z| \le 1\} \cup \{z: |z-3| < 1\}$, 并且把$\mathbb{C}$的度量赋于$X$(今后, 当我们把$\mathbb{R}$或$\mathbb{C}$的子集$X$看作一个度量空间时, 如果不作相反的声明, 总假定$X$继承度量$d(z, w) = |z-w|$), 那么集合$A = \{z:|z| \le 1\}$既是开的, 又是闭的. 它是闭的, 因为它在$X$中的余集$B = X-A = \{z:|z-3| < 1\}$是开的;$A$是开的, 因为如果$a \in A$, 那么$B(a;1) \subset A$(注意: $\{z \in \mathbb{C}: |z-a| < 1\}$并不总包含在$A$中, 当$a=1$时就是一例. 但当按定义, $B(a;1)$是$z \in X: |z-a|<1$, 它是包含在$A$中的). 类似的, $B$在$X$中也是既开又闭的. 

这是一个非连通空间的例子. 

\begin{definition}{连通}{def002020201}
一个度量空间$(X, d)$是连通的, 如果只有$\emptyset$和$X$既是开的又是闭的. 设$A \subset X$, 如果度量空间$(A, d)$是连通的, 那么$A$是$X$的连通子集. 
\end{definition}

连通性的一个等价说法是: $X$是不连通的, 如果存在$X$中的互不相交的非空开集$A$和$B$, 使得$X = A \cup B$. 事实上, 如果这个条件成立, 那么$A = X-B$也是闭的. 

\begin{proposition}{}{prop002020201}
一个集$X \subset \mathbb{R}$是连通的, 当且仅当$X$是一个区间. 
\end{proposition}

\begin{proof}
设$X = [a, b]$, $a$, $b$是$\mathbb{R}$的元素. 设$A \subset X$是$X$的开子集, 满足$a \in A$, $A \neq X$. 我们将证明$A$不可能也是闭的, 因此$X$必是连通的. 因为$A$是开的, $a \in A$, 所以存在$\epsilon > 0$, 使得$[a, a+\epsilon) \subset A$, 设
\[
r = \sup\{\epsilon: [a, a+\epsilon) \subset A\}.
\]
则有断言: $[a, a+r) \subset A$. 事实上, 如果$a \le x < a+r$, 令$h = a + r -x > 0$, 由上确界的定义, 存在$\epsilon$, $r - h < \epsilon < r$且$[a, a+\epsilon) \subset A$. 但是$a \le x = a + (r-h) < a + \epsilon$蕴含$x \in A$. 断言得证. 

但是$a + r \not\in A$\footnote{这里有两种可能: (1)$a+r=b$;(2)$a+r < b$.当$a+r=b$时, $a + r \in A$导致$A=X$, 与原来的假定$A \neq X$矛盾;作者忽略了这种情况. }, 因为在相反的情形, $a + r \in A$, 那么由于$A$是开的, 存在$\delta > 0$, 使得$[a + r, a+r+\delta) \subset A$. 但这就给出$[a, a+r+\delta) \subset A$. 这与$r$的定义相矛盾. 现在假定$A$也是闭的, 那么$a + r \in B = X- A$, $B$是开的, 因此我们可以找到$\delta > 0$, 使得$(a+r-\delta, a+r] \subset B$. 这和上述断言矛盾. 

其他类型的区间的连通性的证明是类似的, 留给读者作为习题. 

$\mathbb{R}$中的连通集必是一区间, 其证明留做习题. 
\end{proof}

如果$w$和$z$是$\mathbb{C}$中的两点, 那么我们用
\[
[z, w] = \{tw + (1-t)z:0 \le t \le 1\}
\]
表示从$z$到$w$的直线段, 从$a$到$b$的折线是集$P = \bigcup_{k=1}^{n}{[z_k, w_k]}$. 其中$z_1 = a$, $w_n = b$, 并且对于$1 \le k \le n-1$, $w_k = w_{k+1}$;或者写成$P=[a,z_1,\cdots, z_n, b]$. 

\begin{theorem}{}{thm002020201}
一个开集$G \subset \mathbb{C}$是连通的, 当且仅当, 对于$G$的任意两点$a$, $b$, 存在一条从$a$到$b$的折线, 这一折线整个地位于$G$内. 
\end{theorem}

\begin{proof}
设$G$满足定理的条件, 假定$G$不是连通的, 我们将得到一个矛盾. 由定义, $G = A \cup B$, 其中$A$, $B$既是开集又是闭集, 且$A \cap B = \emptyset$, $A$, $B$都是非空的, 设$a \in A$, $b \in B$;按照假定, 存在从$a$到$b$的一条折线$P$, $P \subset G$. 稍加考虑, 便可看出, 在组成$P$的某一线段上,  有一点在$A$内, 而另一点在$B$内, 所以我们可以假定$P=[a, b]$. 我们定义
\begin{gather*}
S = \{s \in [0, 1] : sb + (1-s)a \in A\},\\
T = \{t \in [0, 1] : tb + (1-t)a \in B\}.
\end{gather*}
那么, $S \cap T = \emptyset$, $S \cup T = [0,1]$, $0 \in S$, $1 \in T$. 但是能够证明$S$和$T$都是开集(习题\ref{exer002020202}), 这就和$[0,1]$的连通性矛盾. 于是$G$一定是连通的. 

现在设$G$是连通的, 并且在$G$内固定一点$a$, 要指出如何构造从$a$到$b$的折线(在$G$内!)是困难的, 但是我们并不需要实现这种构造, 而只要证明这一折线是存在的. 对于$G$内固定的一点$a$, 定义
\[
A = \{ b \in G: \text{存在}a\text{到}b\text{的折线}P \subset G\}.
\]
我们要证明$A$在$G$内既是开的又是闭的. 由于$a \in A$和$G$是连通的, 所以$A = G$, 定理便得证. 

为了证明$A$是开的, 设$b \in A$, $P = [a, z_1, \cdots, z_n, b]$是从$a$到$b$的折线, $P \subset G$. 由于$G$是开的(这对于定理的前半部分并不需要), 存在$\epsilon > 0$, 使得$B(b; \epsilon) \subset G$, 但是如果$z \in B(b; \epsilon)$, 那么$[b, z] \subset B(b;\epsilon) \subset G$, 因此, $Q = P \cup [b,z]$是$G$内从$a$到$z$的折线, 这就表明$B(b; \epsilon) \subset A$, 所以$A$是开的. 

为了证明$A$是闭的, 假设在$G-A$内有一点$z$, 及$\epsilon > 0$使得$B(z; \epsilon) \subset G$. 如果$A \cap B(z; \epsilon)$内存在一点$b$, 那么如上所述, 我们能够构造一条从$a$到$z$的折线. 于是我们必有$B(z;\epsilon) \cap A = \emptyset$, 或者$B(z;\epsilon) \subset G-A$. 即$G-A$是开的, 所以$A$是闭的. 
\end{proof}

\begin{corollary}{}{coro002020201}
如果$G\subset{}\mathbb{C}$是开的, 连通的, $a$, $b$是$G$内的点, 那么在$G$内存在一条从$a$到$b$的折线, 这一折线由平行于实轴和平行于虚轴的线段所组成. 
\end{corollary}

\begin{proof}
证明这个推论的方法有两个. 一个方法是先在$G$内求得一条从$a$到$b$的折线. 然后修改其每一线段, 使得新的折线具有所要的性质. 利用紧性比较容易实现这个证明(见本章\ref{section0020205}节习题\ref{exer002020507}). 另一个证明可以由修改定理\ref{thm:thm002020201}的证明而得到. 和定理\ref{thm:thm002020201}的证明一样, 定义集$A$, 但附加一个限制, 就是折线的线段都平行于一个坐标轴. 往下的证明仍然有效, 只有一点例外, 就是如果$z \in B(b; \epsilon)$, 那么$[b,z]$可能不平行于坐标轴, 但是容易看出, 如果$z=x+iy$, $b=p+iq$, 那么折线$[b,p+iy]\cup [p+iy, z] \subset B(b;\epsilon)$, 且它的线段平行于坐标轴. 
\end{proof}

现在我们将证明, 度量空间的任意一个集可以用典型的方法表示为连通块的和. 

\begin{definition}{}{def002020205}
度量空间$X$的子集$D$是$X$的一个分支, 如果它是$X$的最大连通子集. 即$D$是连通的, 并且不存在$X$的连通子集, $D$是它的真子集. 
\end{definition}

如果读者考察这一节一开始给出的例子, 就会发现$A$, $B$都是分支. 并且$X$只有这两个分支. 作为另一个例子, 设$X = \{0, 1, \frac{1}{2},\frac{1}{3}, \cdots\}$, 这时显然$X$的每一个分支都是一个点, 并且它的每一个点都是一个分支. 注意, 分支$\{\frac{1}{n}\}$都是$X$中的开集, 分支$\{0\}$不是$X$中的开集. 

\begin{lemma}{}{lemma002020206}
设$x_0 \in X$, $\{D_j : j \in J\}$是$X$的连通子集族, 对于$J$中的每一个$j$, $x_0 \in D_j$. 则$D = \bigcup{\{D_j : j \in J\}}$是连通的. 
\end{lemma}

\begin{proof}
设$A$是度量空间$(D, d)$的子集, 它既是开的又是闭的, 且设$A \neq \emptyset$. 那么对于每个$j$, $A \cap D_j$是$(D_j, d)$中的开集, 也是$(D_j, d)$中的闭集(见\ref{section0020201}节中的习题\ref{exer002020108}和习题\ref{exer002020109}). 由于$D_j$是连通的, 所以, 或者$A \cap D_j = \emptyset$, 或者$A \cap D_j = D_j$. 因为$A \neq \emptyset$, 所以至少存在一个$k$, 使得$A \cap D_k \neq \emptyset$;因此$A \cap D_k = D_k$, 特别地, $x_0 \in A$. 所以, 对于每个$j$, $x_0 \in A \cap D_j$, 于是对于每个$j$, $A \cap D_j = D_j$, 或者说$D_j \subset A$. 这就得到$D = A$, 所以$D$是连通的. 
\end{proof}

\begin{theorem}{}{thm002020207}
设$(X, d)$是一个度量空间, 则
\begin{enumerate}
\item[(a)]$X$中的每一个$x_0$包含在$X$的一分支中;
\item[(b)]$X$的不同分支是互不相交的. 
\end{enumerate}
\end{theorem}

注意, (a)表示$X$是它的分支的和. 

\begin{proof}
(a)设$\mathscr{D}$是包含$x_0$的$X$的连通子集族. 注意到$\{x_0\} \in \mathscr{D}$, 所以$\mathscr{D} \neq \emptyset$. 也注意到上述引理的假设适用于族$\mathscr{D}$, 因此$C = \bigcup\{D; D \in \mathscr{D}\}$是连通的, 且$x_0 \in C$. $C$必定是一个分支. 事实上, 如果$D$是连通的, $C \subset D$, 那么$x_0 \in D$, 所以$D \in \mathscr{D}$. 但是这样一来,$D \subset C$, 所以$C=D$. 于是$C$是最大的. (a)得证. 

(b)设$C_1$, $C_2$是两个分支, $C_1 \neq C_2$, 假定在$C_1 \cap C_2$内存在一点$x_0$, 再由引理, $C_1 \cup C_2$是连通的, 由于$C_1$, $C_2$都是分支, 这就给出$C_1 = C_1 \cup C_2 = C_2$, 矛盾. 
\end{proof}

\begin{proposition}{}{prop002020208}
(a)如果$A \subset X$是连通的, $A \subset B \subset A^-$, 那么$B$是连通的;(b)如果$C$是$X$的一分支, 那么$C$是闭的. 
\end{proposition}

证明留给读者作为习题. 

\begin{theorem}{}{thm002020209}
设$G$是$\mathbb{C}$中的开集, 那么$G$的分支是开集, 并且$G$只有可数个分支.  
\end{theorem}

\begin{proof}
设$C$是$G$的一分支, $x_0 \in C$. 由于$G$是开集, 所以存在$\epsilon > 0$, 使得$B(x_0;\epsilon) \subset G$. 根据引理, $B(x_0;\epsilon) \cup C$是连通的, 所以它必是$C$. 即$B(x_0;\epsilon) \subset C$, 所以$C$是开的. 

为了看出分支的个数是可数的, 设$S = \{a + ib: a, b\text{是有理数, 且}a+ib \in G\}$, 那么$S$是可数的. $G$的每个分支包含$S$的一点, 所以分支的个数是可数的. 
\end{proof}

\begin{exercise}\label{exer002020201}
本习题的目的在于证明$\mathbb{R}$的连通子集是一个区间. 
\begin{enumerate}
\item[(a)]证明: 当且仅当对于$A$中的任意两点$a$, $b$, $a < b$, 有$[a, b] \subset A$时, 集$A \subset \mathbb{R}$是一个区间. 
\item[(b)]利用(a)证明: 如果$A \subset \mathbb{R}$是连通的, 那么$A$是一个区间. 
\end{enumerate}
\end{exercise}

\begin{exercise}\label{exer002020202}
证明定理\ref{thm:thm002020201}的证明中的集$S$和$T$是开集. 
\end{exercise}

\begin{exercise}\label{exer002020203}
$\mathbb{C}$中的下列子集$X$, 哪些是连通的?如果$X$不是连通的, 它的分支是什么?
\begin{enumerate}
\item[(a)]$X = \{z: |z| \le 1\} \cup \{z:|z-2| < 1\}$;
\item[(b)]$X = [0, 1] \cup \{1 + \frac{1}{n}: n > 1\}$;
\item[(c)]$X = \mathbb{C} - (A \cup B)$, 其中$A = [0, \infty)$, $B = \{z = r\cis{\theta} : r = \theta, 0 \le \theta \le \infty \}$. 
\end{enumerate}
\end{exercise}

\begin{exercise}\label{exer002020204}
证明引理\ref{lem:lemma002020206}的下述推广: 如果$\{D_j:j \in J\}$是$X$的连通子集族, 且对于$J$中的每个$j$和$k$, 有$D_j \cap D_k \neq \emptyset$, 那么$D = \bigcup\{D_j: j \in J\}$是连通的. 
\end{exercise}

\begin{exercise}\label{exer002020205}
证明: 如果$F \subset X$是闭的、连通的, 那么对于$F$中的每对点$a$, $b$和每个$\epsilon > 0$, 在$F$中存在点$z_0, z_1,\cdots z_n$, $z_0=a$, $z_n=b$, 且对于$1 \le k \le n$, $d(z_{k-1}, z_k) < \epsilon$, $F$是闭的这个假定是必要的吗?如果$F$是一个满足这个性质的集, 即使$F$是闭的, 也不一定是连通的. 试举例说明之. 
\end{exercise}


\section{序列与完备性}\label{section0020203}
在度量空间中, 最有用的概念之一是收敛序列的概念, 这一概念在度量空间和复分析中与在微积分中一样起着中心的作用. 

\begin{definition}{}{def002020301}
设$\{x_1,x_2,\cdots\}$是度量空间$(X,d)$中的一个序列, 说$\{x_n\}$收敛到$x$, 如果对于每个$\epsilon > 0$, 存在正整数$N$, 使得$n > N$时, $d(x, x_n) < \epsilon$, 记为$x = \lim{x_n}$或$x_n \to x$. 
\end{definition}

换言之, $x = \lim{x_n}$, 如果$0 = \lim{d(x, x_n)}$. 

如果$X = \mathbb{C}$, 那么$z = \lim{z_n}$意味着, 对于每个$\epsilon > 0$, 存在正整数$N$, 使得当$n > N$时, $|z - z_n| < \epsilon$. 

在度量空间的理论中, 许多概念可以借助于序列来叙述. 下面是一个例子. 

\begin{proposition}{}{prop002020302}
一个集$F \subset X$是闭的, 当且仅当对于$F$中的每个序列$\{x_n\}$, 若$x = \lim{x_n}$, 则$x \in F$. 
\end{proposition}

\begin{proof}
设$F$是闭的, $x = \lim{x_n}$, 其中每个$x_n$在$F$中. 所以对于每个$\epsilon > 0$, 在$B(x;\epsilon)$中有一点$x_n$;即$B(x;\epsilon) \cap F \neq \emptyset$. 所以由命题\ref{pro:prop002020103}(f), $x \in F^- = F$. 

现在设$F$不是闭的, 所以在$F^-$中有$x_0$, $x_0$不在$F$中. 由命题\ref{pro:prop002020103}(f), 对于每个$\epsilon > 0$, 有$B(x_0;\epsilon) \cap F \neq \emptyset$. 特别地, 对于每个正整数$n$, 在$B(x_0; \frac{1}{n}) \cap F$中有点$x_n$. 于是$d(x_0, x_n) < \frac{1}{n}$, 这就蕴含$x_n \to x_0$. 由于$x_0 \not\in F$, 这就是说定理的条件不满足. 
\end{proof}

\begin{definition}{}{def002020303}
设$A \subset X$. 那么$X$中的点$x$是$A$的极限点, 如果在$A$中存在由不同点构成的序列$\{x_n\}$, 使得$x = \lim{x_n}$. 
\end{definition}

在这个定义中“不同”二字的理由可由下面的例子得到解释. 设$X = \mathbb{C}$, $A = [0, 1] \cup \{i\}$;$[0, 1]$中的每一点是$A$的极限点, 但$i$不是$A$的极限点, 我们不能指望把$i$这样的点叫做极限点. 但是如果把“不同”二字从定义中删去, 我们就可以对每个$n$取$x_n = i$, 有$i = \lim{x_n}$. 

\begin{proposition}{}{prop002020304}
(a)一个集合是闭的, 当且仅当, 它包含它的所有极限点;(b)如果$A \subset X$, 那么$A^-=A \cup \{x: x\text{是}A\text{的极限点}\}$. 
\end{proposition}

证明留做习题. 

从实分析中我们知道, $\mathbb{R}$的基本性质是: 任意一个序列, 当$n$增大时它的项变得越来越接近, 则它一定是收敛的. 这种序列称为Cauchy序列. 这种序列的属性之一是它的极限一定存在, 尽管你不能求出它. 

\begin{definition}{Cauchy序列}{def002020305}
序列$\{x_n\}$称为Cauchy序列, 如果对于每个$\epsilon > 0$, 都存在一个正整数$N$, 使得对所有的$n$, $m \ge N$, 有$d(x_n, x_m) < \epsilon$. 
\end{definition}

如果$(X, d)$有性质: 每个Cauchy序列在$X$中有极限, 那么$(X, d)$是完备的. 

\begin{proposition}{}{prop002020306}
$\mathbb{C}$是完备的. 
\end{proposition}

\begin{proof}
如果$\{x_n + iy_n\}$是$\mathbb{C}$中的Cauchy序列, 那么$\{x_n\}$和$\{y_n\}$是$\mathbb{R}$中的Cauchy序列, 由于$\mathbb{R}$是完备的, 所以$x_n \to x$, $y_n \to y$, $x, y$在$\mathbb{R}$中. 由此推出, $x+iy=\lim{(x_n + iy_n)}$, 所以$\mathbb{C}$是完备的. 
\end{proof}

考虑具有度量$d$(见第\ref{section00201}章的\ref{equ002010608}和\ref{equ002010607})的$C_{\infty}$. 设$z_n$, $z$是$\mathbb{C}$中的点, 可以证明$d(z_n, z) \to 0$, 当且仅当$|z_n-z| \to 0$. 尽管如此, 注意序列$\{z_n\}$, $\lim{|z_n|} = \infty$是$\mathbb{C}_{\infty}$中的Cauchy序列, 但是它不是$\mathbb{C}$中的Cauchy序列. 

如果$A \subset X$, 我们把$\diam{A} = \sup{\{d(x, y):x\text{和}y\text{在}A\text{中}\}}$定义为$A$的直径. 

\begin{theorem}{Cantor定理}{thm002020307}
度量空间$(X, d)$是完备的, 当且仅当, 任意满足条件$F_1 \supset F_2 \supset \cdots$和$\diam{F_n} \to 0$, 非空闭集序列$\{F_n\}$, 其交集$\bigcap_{n=1}^{\infty}{F_n}$是由一个点所组成. 
\end{theorem}

\begin{proof}
设$(X, d)$是完备的, $\{F_n\}$是一个闭集序列, 具有性质: (i)$F_1 \supset F_2 \supset \cdots$;(ii)$\lim{\diam{F_n}} = 0$. 对于每个$n$, 设$x_n$是$F_n$种的任意一点, 如果$n, m \ge N$, 那么$x_n$, $x_m$在$F_N$中, 由定义, $d(x_n, x_m) \le \diam{F_N}$. 由假定, $N$可玄德充分大, 使得$\diam{F_N} < \epsilon$;这就表明$\{x_n\}$是Cauchy序列. 由于$X$是完备的, 所以$x_0 = \lim{x_n}$存在. 又对于所有的$n \ge N$, 因为$F_n \subset F_N$, 所以$x_n$在$F_N$中;因此, 对于每个$N$, $x_0$在$F_N$中, 这就给出$x_0 \in \bigcap_{n=1}^{\infty}{F_n} =F$. 所以$F$至少包含一个点. 如果$y$也在$F$中, 那么对于每个$n$, $x_0$和$y$都在$F_n$中, 这就给出$d(x_0, y) \le \diam{F_n} \to 0$, 所以$d(x_0, y) = 0$, 或者$x_0 = y$. 

现在, 如果$X$满足所述的条件, 我们来证明$X$是完备的. 设$\{x_n\}$是$X$中的Cauchy序列, 又设$F_n = \{x_n, x_{n+1},\cdots\}^-$;那么$F_1 \supset F_2 \supset \cdots$. 如果$\epsilon > 0$, 选取$N$, 使得对于每个$n, m \ge N$, 都有
\[
d(x_n, x_m) < \epsilon;
\]
这就表示对于$n \ge N$, $\diam\{x_n, x_{n+1}, \cdots\} \le \epsilon$, 所以对于$n \ge N$, $\diam{F_n} \le \epsilon$(习题\ref{exer002020303}). 于是$\diam{F_n} \to 0$, 并且, 按照假设, 在$X$中存在点$x_0$, $\{x_0\} = F_1 \cap F_2 \cap \cdots$. 特别地, $x_0$在$F_n$中, $d(x_0, x_n) \le \diam{F_n} \to 0$, 所以$x_0 = \lim{x_n}$. 
\end{proof}

有一个典型习题与这个定理有联系, 就是在$\mathbb{R}$中找一个集序列$\{F_n\}$, 它满足下面的条件中的两个条件: 
\begin{enumerate}
\item[(a)]每个$F_n$是闭的;
\item[(b)]$F_1 \supset F_2 \supset \cdots$;
\item[(c)]$\diam{F_n} \to 0$. 
\end{enumerate}
但是$F = F_1 \cap F_2 \cap \cdot$或者是空的, 或者多于一点, 对于两个条件的各种可能选择, 读者都应举出相应的例子. 

\begin{proposition}{}{prop002020308}
设$(X, d)$是一个完备的度量空间, $Y \subset X$, 当且仅当$Y$在$X$中是闭时$(Y, d)$是一个完备度量空间. 
\end{proposition}

\begin{proof}
当$Y$是闭子集时, $(Y, d)$是完备的. 其证明留给读者作为习题. 现在设$(Y, d)$是完备的, $x_0$是$Y$的极限点, 那么在$Y$中有序列$\{y_n\}$, 使得$x_0 = \lim{y_n}$. 因此$\{y_n\}$是Cauchy序列(习题\ref{exer002020305}), 并且因为$(Y, d)$是完备的, 所以$\{y_n\}$一定收敛到$Y$中的$y_0$. 由此推得$y_0 = x_0$, 所以$Y$包含它的所有极限点. 由命题\ref{pro:prop002020304}, $Y$是闭的. 
\end{proof}

\begin{exercise}
证明命题\ref{pro:prop002020304}. 
\end{exercise}

\begin{exercise}
完成命题\ref{pro:prop002020308}的详细证明. 
\end{exercise}

\begin{exercise}\label{exer002020303}
证明: $\diam{A} = \diam{A^-}$. 
\end{exercise}

\begin{exercise}\label{exer002020304}
设$z_n$, $z$是$\mathbb{C}$中的点, $d$是$\mathbb{C}_{\infty}$中的度量, 证明$|z_n - z| \to 0$, 当且仅当, $d(z_n, z) \to 0$. 证明: 如果$|z_n| \to \infty$, 那么$\{z_n\}$是$\mathbb{C}_{\infty}$中的Cauchy序列. ($\{z_n\}$在$\mathbb{C}_{\infty}$中一定收敛吗?)
\end{exercise}

\begin{exercise}\label{exer002020305}
证明: $(X, d)$中的每个收敛序列一定是Cauchy序列. 
\end{exercise}

\begin{exercise}\label{exer002020306}
给出三个不完备度量空间的例子. 
\end{exercise}

\begin{exercise}\label{exer002020307}
在$\mathbb{R}$上作一度量$d$, 满足条件: $|x_n - x| \to 0$, 当且仅当, $d(x_n, x) \to 0$, 而当$|x_n| \to \infty$时, $\{x_n\}$是$(\mathbb{R}, d)$中的Cauchy序列. (提示: 从$\mathbb{C}_{\infty}$得到启示. )
\end{exercise}

\begin{exercise}\label{exer002020308}
设$\{x_n\}$是Cauchy序列, 且$\{x_{n_k}\}$是收敛子序列, 证明: $\{x_n\}$一定是收敛的. 
\end{exercise}



\section{紧性}\label{section0020204}
紧性的概念是把有限集中一些好的性质推广到无穷集去, 紧集的大部分性质类似于有限集的性质, 这些性质在有限集是很平凡的. 例如, 有限集的每个序列有收敛子序列. 这是平凡的, 因为至少有一点重复无穷多次. 当我们把“有限”代之以“紧”时, 这个结果仍然成立. 

\begin{definition}{紧集}{def002020401}
度量空间$X$的子集$K$是紧的, 如果对于$X$中的每个具有性质
\begin{gather}\label{equ002020402}
K \subset \bigcup\{G: G \in \mathscr{G}\},
\end{gather}
的开集族$\mathscr{G}$都可在$\mathscr{G}$中找到有限个集$G_1, G_2, \cdots, G_n$, 使得$K \subset G_1 \cup G_2 \cup \cdots \cup G_n$. 满足(\ref{equ002020402})的集族$\mathscr{G}$称为$K$的覆盖;如果$\mathscr{G}$的每个集是开的, 则称它是$K$的开覆盖. 
\end{definition}

显然, 空集和所有的有限集是紧的. $D = \{z \in \mathbb{C}:|z|<1\}$是一个非紧集的例子. 如果$G_n = \big\{z : |z| < 1 - \frac{1}{n}\big\}$, $n=2,3,\cdots$, 那么$\{G_2, G_3, \cdots\}$是$D$的一个开覆盖, 但它没有有限子覆盖. 

\begin{proposition}{}{prop002020403}
设$K$是$X$的一个紧子集, 那么
\begin{enumerate}
\item[(a)]$K$是闭的;
\item[(b)]如果$F$是闭的, 且$F \subset K$, 则$F$是紧的. 
\end{enumerate}
\end{proposition}

\begin{proof}
为了证明(a), 我们要证明$F = F^-$. 设$x_0 \in K^-$, 由命题\ref{pro:prop002020103}(f), 对于每个$\epsilon > 0$, $B(x_0;\epsilon) \cap K \neq \emptyset$. 设
\[
G_n = X - \bar{B}(x_0; \frac{1}{n}),
\]
并假定$x_0 \not\in K$, 那么每个$G_n$是开集, 且$K \subset \bigcup_{n=1}^{\infty}{G_n}$(因为$\bigcap_{n=1}^{\infty}{\bar{B}(x_0; \frac{1}{n})} = \{x_0\}$). 因为$K$是紧的, 所以存在正整数$m$, 使得$K \subset \bigcup_{n=1}^{m}{G_n}$. 但是$G_1 \subset G_2 \subset \cdots$, 所以$K \subset G_m = X - \bar{B}(x_0;\frac{1}{m})$. 但这就给出$B(x_0; \frac{1}{m}) \cap K = \emptyset$, 从而得到一个矛盾. 于是$K = K^-$. 

为了证明(b), 设$\mathscr{G}$是$F$的一个开覆盖. 那么由于$F$是闭的, $\mathscr{G} \cup \{X-F\}$是$K$的开覆盖. 设$G_1, G_2, \cdots, G_n$是$\mathscr{G}$中的集, 使得$K \subset G_1 \cup \cdots \cup G_n \cup (X-F)$. 显然, $F \subset G_1 \cup \cdots \cup G_n$, 所以$F$是紧的. 
\end{proof}

设$\mathscr{F}$是$X$的子集族, 我们说$\mathscr{F}$有有限交性质$(f, i, p)$如果, 只要$\{F_1, F_2, \cdots, F_n\} \subset \mathscr{F}$, 总有$F_1 \cap F_2 \cap \cdots \cap F_n \neq \emptyset$. 这种子集族的一个例子是$\{D - G_2, D-G_3, \cdots\}$, 其中集$G_n$是命题\ref{pro:prop002020403}之前所述例子中的集. 

\begin{proposition}{}{prop002020404}
一个集合$K \subset X$是紧的, 当且仅当, $K$中每个具有$(f,i,p)$的闭子集\footnote{这里的“闭子集”应是“相对于集合$K$的闭子集”. 否则命题的充分性不真, 必要性的证明应作相应的修改. }族$\mathscr{F}$, 都有$\bigcap\{F: F \in \mathscr{F}\} \neq \emptyset$. 
\end{proposition}

\begin{proof}
设$K$是紧的, $\mathscr{F}$是具有$(f,i,p)$的$K$中的闭子集族. 假设$\bigcap\{F: F \in \mathscr{F}\} = \emptyset$. 令$\mathscr{P} = \{X - F: F \in \mathscr{F}\}$, 那么由假设$\bigcup\{X - F: F \in \mathscr{F}\} = X - \bigcap\{F: F \in \mathscr{F}\} = X$. 特别地, $\mathscr{P}$是$K$的开覆盖, 于是存在$F_1, \cdots, F_n \in \mathscr{F}$, 使得$K \subset \bigcup_{k=1}^{n}{(X - F_k)} = X - \bigcap_{k=1}^{n}{F_k}$. 但这就给出$\bigcap_{k=1}^{n}{F_k} = X - K$, 由于每个$F_k$是$K$的子集, 所以必有$\bigcap_{k=1}^{n}{F_k} = \emptyset$. 这与$\mathscr{F}$具有$(f,i,p)$相矛盾. 

条件的充分性的证明留给读者作为习题. 
\end{proof}

\begin{corollary}{}{coro002020405}
每个紧的度量空间是完备的. 
\end{corollary}

\begin{proof}
这容易由上面的命题和定理\ref{thm:thm002020307}推出. 
\end{proof}

\begin{corollary}{}{coro002020406}
如果$X$是紧的, 那么每个无穷集在$X$中至少有一个极限点. 
\end{corollary}

\begin{proof}
设$S$是$X$的一无穷子集, 假设$S$没有极限点. 设$\{a_1, a_2, \cdots\}$是$S$中不同点的序列, 那么$F_n = \{a_n, a_{n+1}, \cdots\}$也没有极限点. 但是如果一个集合没有极限点, 那么也可以说它包含了它的所有极限点, 因而它是闭集!于是每个$F_n$是闭的, 且$\{F_n : n \ge 1\}$具有$(f,i,p)$. 但是由于点$a_1, a_2,\cdots$是不同的, 所以$\bigcap_{n=1}^{\infty}{F_n} = \emptyset$. 这与上面的命题\ref{pro:prop002020404}相矛盾. 
\end{proof}

\begin{definition}{列紧性}{def002020407}
称一个度量空间$(X, d)$是列紧的, 如果$X$中的每个序列都有收敛子序列. 
\end{definition}

我们将证明度量空间的紧性和列紧性是一回事, 为此需要下面的引理. 

\begin{lemma}{Lebesque覆盖引理}{lemma002020408}
如果$(X, d)$是列紧的, $\mathscr{G}$是$X$的开覆盖, 那么存在$\epsilon > 0$, 使得$X$中的每个$x$, 都存在$\mathscr{G}$中的一个集$G$, 满足$B(x; \epsilon) \subset G$. 
\end{lemma}

\begin{proof}
用反证法, 设$\mathscr{G}$是$X$的开覆盖, 而这样的$\epsilon$不存在. 特别地, 对于每个正整数$n$, 在$X$中有点$x_n$, 使得$B(x_n;\frac{1}{n})$不包含在$\mathscr{G}$中任一个集$G$内. 因为$X$是列紧的, 所以在$X$中存在点$x_0$和序列$\{x_{n_k}\}$, 使得$x_0 = \lim{x_{n_k}}$. 设$G_0 \in \mathscr{G}$, $x_0 \in G_0$;选取$\epsilon >0$使得$B(x_0;\epsilon) \subset G_0$. 现在设$N$是这样的正整数, 对于所有的$n_k \ge N$, 都有$d(x_0;x_{n_k}) \le \epsilon / 2$. 设$n_k$是比$N$和$2/\epsilon$都大的任意正整数, $y \in B(x_{n_k};\frac{1}{n_k})$, 那么$d(x_0, y) \le d(x_0, x_{n_k}) + d(x_{n_k}, y) < \epsilon / 2 + 1/n_k < \epsilon$. 即$B(x_{n_k}; \frac{1}{n_k}) \subset B(x_0;\epsilon) \subset G_0$, 这和$x_{n_k}$的取法相矛盾. 
\end{proof}

对于Lebesque覆盖引理通常有两种误解. 一是言之未及, 一是言过其实. 由于$\mathscr{G}$是$X$的开覆盖, 所以$X$的每个$x$包含在$\mathscr{G}$的某一个$G$内;因为$G$是开集, 于是存在$\epsilon > 0$使得$B(x;\epsilon) \subset G$. 但是引理所给出的$\epsilon > 0$是使得对于任意的$x$, $B(x; \epsilon)$都包含在$\mathscr{G}$的某一集内. 另一种误解是, 以为对于引理中所得到的$\epsilon > 0$, $B(x; \epsilon)$包含在$\mathscr{G}$中含有$x$的每个$G$内. 

\begin{theorem}{}{thm002020409}
设$(X, d)$是一个度量空间, 那么下列条件是等价的: 
\begin{enumerate}
\item[(a)]$X$是紧的;
\item[(b)]$X$中的每个无穷集至少有一个极限点;
\item[(c)]$X$是列紧的;
\item[(d)]$X$是完备的, 并且对于每个$\epsilon > 0$, $X$内存在有穷多个点$x_1,x_2,\cdots, x_n$, 使得
\[
X = \bigcup_{k=1}^{n}{B(x_k;\epsilon)}.
\]
((d)中所述的性质称为完全有界性. )
\end{enumerate}
\end{theorem}

\begin{proof}
由推论\ref{cor:coro002020406}, (a)蕴含(b). 

(b)蕴含(c). 设$\{x_n\}$是$X$中的一个序列, 不失一般性, 假定点$x_1,x_2,\cdots$是不同的. 由(b), 集合$\{x_1, x_2,\cdots\}$有一个极限点$x_0$. 于是有点$x_{n_1} \in B(x_0; 1)$, 类似的, 有正整数$n_2 > n_1$, $x_{n_2} \in B(x_0; \frac{1}{2})$, 如此继续下去, 我们得到正整数$n_1 < n_2 < \cdots$, $x_{n_k} \in B(x_0;\frac{1}{k})$. 于是$x_0 = \lim{x_{n_k}}$. 所以$X$是列紧的. 

(c)蕴含(d). 设$\{x_n\}$是Cauchy序列, 应用列紧性的定义和借助于\ref{section0020203}节的习题\ref{exer002020308}, 便可看出$X$是完备的. 

现在设$\epsilon > 0$, 固定$x_1 \in X$. 如果$X = B(x_1;\epsilon)$, 那么结论得证. 否则选取$x_2 \in X - B(x_1; \epsilon)$. 如果$X = B(x_1;\epsilon) \cup B(x_2;\epsilon)$, 结论也得证. 否则设$x_3 \in X - [B(x_1;\epsilon) \cup B(x_2;\epsilon)]$. 如果这个过程不会终止, 我们就得到一序列$\{x_n\}$, 使得
\[
x_{n+1} \in X - \bigcup_{k=1}^{n}{B(x_k; \epsilon)}.
\]
但是这蕴含对于$n \neq m$, $d(x_n, x_m) \ge \epsilon > 0$. 于是$\{x_n\}$没有收敛子列, 这与(c)相矛盾. 

(d)蕴含(c). 这部分证明用到“鸽巢原理”, 这个原理可表述为: 如果物件数多于容器数, 那么至少有一个容器里装的物件多于一个. 进而, 如果无穷多个点包含在有穷多个球里, 那么有一个球包含无穷多个点, 所以(d)是说, 对于每个$\epsilon > 0$和$X$中的无穷集, 存在点$y \in X$, 使得$B(y; \epsilon)$包含这个集的无穷多个点. 设$\{x_n\}$是一个由不同点组成的序列. 在$X$中存在点$y_1$和$\{x_n\}$的子序列$\{x_{n}^{(1)}\}$, 使得$\{x_n^{(1)}\} \subset B(y_1;1)$. 又存在$X$中的$y_2$和$\{x_n^{(1)}\}$的子序列$\{x_n^{(2)}\}$, 使得$\{x_n^{(2)}\} \subset B(y_2; 1/2)$. 如此继续下去, 对于每个正整数$k \ge 2$, 存在$X$中的$y_k$和$\{x_n^{(k-1)}\}$的子序列$\{x_n^{(k)}\}$, 使得$\{x_n^{(k)}\} \subset B(y_k; 1/k)$. 设$F_k = \{x_n^{(k)}\}^-$, 那么$\diam{F_k} \le 2/k$, 且$F_1 \supset F_2 \supset \cdots$. 根据定理\ref{thm:thm002020407}, $\bigcap_{k=1}^{\infty}{F_k} = \{x_0\}$. 我们断言$x_k^{(k)} \to x_0$($x_k^{(k)}$是$\{x_n\}$的子序列). 事实上, $x_0 \in F_k$, 所以$d(x_0;x_k^{(k)}) \le \diam{F_k} \le 2/k$, $x_0 = \lim{x_k^{(k)}}$. 

(c)蕴含(a). 设$\mathscr{G}$是$X$的一个开覆盖. 上面的引理给出一个$\epsilon > 0$, 使得对于每个$x \in X$, $\mathscr{G}$中存在一个集$G$, $B(x; \epsilon) \subset G$. 现在已知(c)蕴含(d), 因此, 在$X$中存在点$x_1, \cdots, x_n$, 使得$X = \bigcup_{k=1}^{n}{B(x_k;\epsilon)}$. 现在对于$1 \le k \le n$, 存在集$G_k \in \mathscr{G}$, $B(x_k;\epsilon) \subset G_k$. 因此$X = \bigcup_{k=1}^{n}{G_k}$, 即$\{G_1,\cdots, G_n\}$是$\mathscr{G}$的有限子覆盖. 
\end{proof}

\begin{theorem}{Heine-Borel定理}{thm002020410}
当且仅当$K$是有界闭集时, $R^{n}$($n \ge 1$)是一个集$K$是紧的. 
\end{theorem}

\begin{proof}
如果$K$是紧的, 那么由前一定理的(d), $K$是完全有界的. 由命题\ref{pro:prop002020403}推出$K$一定是闭的. 容易证明完全有界的集也是有界的. 

现在假设$K$是有界闭集. 因此存在实数$a_1,a_2,\cdots,a_n$和$b_1,b_2,\cdots,b_n$, 使得$K \subset F = [a_1, b_1] \times \cdots [a_n, b_n]$. 如果能够证明$F$是紧的, 那么因为$K$是闭的, 就可推知$K$是紧的(命题\ref{pro:prop002020403}(b)). 由于$\mathbb{R}^n$是完备的和$F$是闭的, 推知$F$是完备的. 因此, 再次应用前一定理中的(d), 我们只需证明$F$是完全有界的. 这是容易的, 虽然写起来有点繁. 设$\epsilon > 0$;现在我们将$F$写成$n$维矩形的和, 其中每个矩形的直径小于$\epsilon$. 这样, 我们有$F \subset \bigcup_{k=1}^{\infty}{B(x_k; \epsilon)}$, 其中每个$x_k$属于前面提到的矩形中的某一个. 这个做法的细节留给读者作为习题去完成(习题\ref{exer002020403}). 
\end{proof}

\begin{exercise}
完成命题\ref{exer002020404}的证明. 
\end{exercise}

\begin{exercise}\label{exer002020402}
设$p = (p_1,p_2,\cdots, p_n)$, $q = (q_1, q_2, \cdots, q_n)$是$\mathbb{R}$中的点, 并且对于每个$k$, $p_k < q_k$. 设$R = [p_1, q_1] \times \cdots \times [p_n,q_n]$. 证明
\[
\diam{R} = d(p, q) = [\sum_{k=1}^{n}{(q_k - p_k)^2}]^{\frac{1}{2}}.
\]
\end{exercise}

\begin{exercise}\label{exer002020403}
设$F = [a_1, b_1] \times \cdots \times [a_n, b_n] \subset \mathbb{R}^n$, $\epsilon > 0$, 利用习题\ref{exer002020402}证明: 存在矩形$R_1, R_2, \cdots, R_m$, 使得$F = \bigcup_{k=1}^{m}{R_k}$, 并且对于每个$k$, $\diam{R_k} < \epsilon$. 如果$x_k \in R_k$, 那么由此推出$R_k \subset B(x_k; \epsilon)$. 
\end{exercise}

\begin{exercise}\label{exer002020404}
证明: 有穷多个紧集的和是紧的. 
\end{exercise}

\begin{exercise}
设$X$是所有有界复数序列的集. 也就是$\{x_k\} \in X$, 当且仅当, $\sup\{|x_n|:n \ge 1\} < \infty$. 如果$x = \{x_n\}$和$y = \{y_n\}$, 定义$d(x, y) = \sup\{|x_n-y_n|: n \ge 1\}$. 证明: 对于$X$中的每个$x$和$\epsilon > 0$, $\bar{B}(x; \epsilon)$不是完全有界的, 尽管它是完备的. (提示: 如果首先证明可以假定$x=(0, 0, \cdots, 0)$, 事情就容易了). 
\end{exercise}

\begin{exercise}
证明: 完全有界的集的闭包是完全有界的. 
\end{exercise}



\section{连续性}\label{section0020205}
函数最基本的性质之一是连续性. 有了连续性就保证了一定程度的正则性和光滑性. 否则, 要得到度量空间上的任何函数理论是困难的. 由于本书的主题是具有导数的(所以也是连续的)一个复变数的函数论, 所以连续性的研究是基本的. 

\begin{definition}{连续}{def002020501}
设$(X, d)$和$(\Omega, \rho)$是度量空间, $f:X \to \Omega$是一个函数. 设$a \in X$, $\omega \in \Omega$, 如果对于每个$\epsilon > 0$, 都存在$\delta > 0$, 使得只要$0 < d(x, a) < \delta$就有$\rho(f(x), \omega) < \epsilon$, 那么就说$\lim\limits_{x \to a}{f(x)} = \omega$. 如果$\lim\limits_{x \to a}{f(x)} = f(a)$就说函数$f$在点$a$是连续的, 如果$f$在$X$的每一点都是连续的, 那么就称$f$是从$X$到$\Omega$的连续函数. 
\end{definition}

\begin{proposition}{}{prop002020502}
设$f: (X, d) \to (\Omega, \rho)$是一个函数, $a \in X$, $\alpha = f(a)$. 下列事实是等价的: 
\begin{enumerate}
\item[(a)]$f$在$a$点是连续的;
\item[(b)]对于每个$\epsilon > 0$, $f^{-1}(B(\alpha; \epsilon))$包含一个以$a$为中心的球;
\item[(c)]$\alpha = \lim{f(x_n)}$, 只要$a = \lim{x_n}$. 
\end{enumerate}
\end{proposition}

证明留给读者作为习题. 

这是关于函数在一点的连续性的最后一个命题, 从现在起, 我们将只涉及在$X$的所有点上连续的函数. 

\begin{proposition}{}{prop002020503}
设$f: (X, d) \to (\Omega, \rho)$是一个函数, 下列事实是等价的: 
\begin{enumerate}
\item[(a)]$f$是连续的;
\item[(b)]如果$\Delta$是$\Omega$中的开集, 那么$f^{-1}(\Delta)$是$X$中的开集;
\item[(c)]如果$\Gamma$是$\Omega$中的闭集, 那么$f^{-1}(\Gamma)$是$X$中的闭集. 
\end{enumerate}
\end{proposition}

\begin{proof}
(a)蕴含(b). 设$\Delta$是$\Omega$中的开集, $x \in f^{-1}(\Delta)$. 如果$\omega = f(x)$, 那么$\omega$在$\Delta$内;由定义, 存在$\epsilon > 0$, 使得$B(\omega, \epsilon) \subset \Delta$. 由于$f$是连续的, 所以由上一命题的(b)给出一$\delta > 0$, 使得$B(x, \delta) \subset f^{-1}(B(\omega; \epsilon)) \subset f^{-1}(\Delta)$. 因此$f^{-1}(\Delta)$是开的. 

(b)蕴含(c). 如果$\Gamma \subset \Omega$是闭的, 那么$\Delta = \Omega - \Gamma$是开的. 由(b), $f^{-1}(\Delta) = X - f^{-1}(\Gamma)$是开的, 所以$f^{-1}(\Gamma)$是闭的. 

(c)蕴含(a). 假设在$X$中存在一点$x$, $f$在这点不连续, 那么存在$\epsilon > 0$, 和一个序列$\{x_n\}$, 使得$x = \lim{x_n}$, 但是对于每个$n$, 都有$\rho(f(x_n), f(x)) \ge \epsilon$. 令$\Gamma = \Omega - B(f(x); \epsilon)$, 那么$\Gamma$是闭的, 并且$x_n$在$f^{-1}(\Gamma)$中. 由于$f^{-1}(\Gamma)$是闭的(根据(c)), 我们有$x \in f^{-1}(\Gamma)$. 但这就蕴含$\rho(f(x);f(x)) \ge \epsilon > 0$, 故矛盾. 
\end{proof}

下述类型的结果大概易为读者所理解, 所以证明留给读者作为习题. 
\begin{proposition}{}{prop002020504}
设$f$和$g$是$X$到$\mathbb{C}$内的连续函数, $\alpha, \beta \in \mathbb{C}$. 那么$\alpha{}f + \beta{}g$和$fg$也是连续的. 如果对于$X$中的每个$x$, $g(x) \neq 0$, 那么$f/g$也是连续的. 
\end{proposition}

\begin{proposition}{}{prop002020505}
设$f: X \to Y$及$g:Y \to Z$是连续函数, 那么$g \circ f$(这里$g \circ f(x) = g(f(x))$)是$X$到$Z$内的一个连续函数. 
\end{proposition}

\begin{proof}
如果$U$是$Z$中的开集, 那么$g^{-1}(U)$是$Y$中的开集. 因此$f^{-1}(g^{-1}(U))=(g \circ f)^{-1}(U)$是$X$中的开集. 
\end{proof}

\begin{definition}{一致连续}{def002020506}
函数$f:(X, d) \to (\Omega, \rho)$是一致连续的, 如果对于每个$\epsilon > 0$, 存在$\delta > 0$(只依赖于$\epsilon$), 使得当$d(x, y) < \delta$时, 就有$\rho(f(x), f(y)) < \epsilon$. 我们称$f$是一个Lipschitz函数. 如果存在常数$M > 0$, 使得对于$X$中的所有$x$和$y$, 都有$\rho(f(x), f(y)) \le Md(x, y)$. 
\end{definition}

容易看出, 每个Lipschitz函数是一致连续的. 事实上, 如果给定$\epsilon > 0$, 取$\delta = \epsilon / M$即可. 也容易看出每个一致连续的函数是连续的. 上述诸类函数有些什么例子呢?如果$X = \Omega = \mathbb{R}$, 那么$f(x)=x^2$是连续的, 但不是一致连续的. 如果$X = \Omega = [0,1]$, 那么$f(x) = x^{\frac{1}{2}}$是一致连续的, 但不是Lipschitz函数. 下述命题为Lipschitz函数提供了一个丰富的来源. 

设$A \subset X$, $x \in X$. 我们定义$x$到集$A$的距离$d(x, A)$为
\[
d(x, A) = \inf\{d(x, a) : a \in A\}.
\]

\begin{proposition}{}{prop002020507}
设$A \subset X$, 那么: 
\begin{enumerate}
\item[(a)]$d(x, A) = d(x, A^-)$. 
\item[(b)]$d(x, A) = 0$, 当且仅当, $x \in A^-$. 
\item[(c)]对于$X$中的所有$x$, 有$|d(x, A) - d(y, A)| \le d(x, y)$. 
\end{enumerate}
\end{proposition}

\begin{proof}
(a) 如果$A \subset B$, 那么由定义显然有$d(x, B) \le d(x, A)$. 因此$d(x, A^-) \le d(x, A)$. 另一方面, 如果$\epsilon > 0$, 则存在$A^-$中的一点$y$, 使得$d(x, A^-) \ge d(x, y) - \epsilon / 2$. 再在$A$中找一点$a$, 满足$d(y, a) < \epsilon/2$. 但是由三角不等式$|d(x, y) - d(x, a)| \le d(y, a) < \epsilon/2$. 特别地, $d(x, y) > d(x, a) - \epsilon/2$. 这就给出$d(x, A^-) \ge d(x, a) - \epsilon \ge d(x, A) - \epsilon$. 因为$\epsilon$是任意的, 所以$d(x, A^-) \ge d(x, A)$. (a)得证. 

(b) 如果$x \in A^-$, 那么$0 = d(x, A^-) = d(x, A)$. 现在对于$X$中的任意$x$, 在$A$中存在一最小序列$\{a_n\}$, 使得$d(x, A) = \lim{d(x, a_n)}$. 所以如果$d(x, A) = 0$, 那么$\lim{d(x, a_n)} = 0$, 即$x = \lim{a_n}$, $x \in A^-$. 

(c) 对于$A$中的$a$, $d(x, a) \le d(x, y) + d(y, a)$, 因此$d(x, A) = \inf\{d(x, a):a \in A\} \inf\{d(x, y) + d(y, a):a \in A\} = d(x, y) + d(y, A)$. 这就给出了$d(x, A) - d(y, A) \le d(x, y)$. 类似的, $d(y, A) - d(x, A) \le d(x, y)$. 所以不等式得证. 
\end{proof}

注意, 命题的(c)是说: 由$f(x)=d(x, A)$所定义的函数$f: X \to \mathbb{R}$是一个Lipschitz函数. 如果我们变动集$A$, 便得到许多这种函数. 

两个一致连续的(Lipschitz)函数的乘积仍是一致连续的(Lipschitz)函数, 这一命题是不真的. 例如, $f(x)=x$是Lipschitz函数, 但是$f \cdot f$甚至都不是一致连续的. 不过如果$f$和$g$都是有界的, 那么命题就成立了(见习题\ref{exer002020503}). 

下面的定理包含连续函数的两个最重要的性质. 

\begin{theorem}{}{thm002020508}
设$f:(X, d) \to (\Omega, \rho)$是一个连续函数. (a) 如果$X$是紧的, 那么$f(X)$是$\Omega$种的紧子集;(b) 如果$X$是连通的, 那么$f(X)$是$\Omega$中的连通子集. 
\end{theorem}

\begin{proof}
为了证明(a)和(b), 不失一般性, 可以假设$f(X) = \Omega$. (a) 设$\{\omega_n\}$是$\Omega$中的一个序列, 那么对于每个$n \ge 1$, 在$X$中存在一点$x_n$, 使得$\omega_n = f(x_n)$. 由于$X$是紧的, 所以在$X$中存在一点$x$和一个子序列$\{x_{n_k}\}$, 使得$x = \lim{x_{n_k}}$. 设$\omega = f(x)$, 那么由$f$的连续性, $\omega = \lim{\omega_{n_k}}$, 因此根据定理\ref{thm:thm002020409}, $\Omega$是紧的. (b)设$\Sigma \subset \Omega$在$\Omega$中既是开的, 又是闭的, 且$\Sigma \neq \emptyset$, 那么因为$f(X) = \Omega$, 所以$\emptyset \neq f^{-1}(\Sigma)$. 因为$f$是连续的, 所以$f^{-1}(\Sigma)$也既是开的, 又是闭的. 根据$X$的连通性, $f^{-1}(\Sigma) = X$, 这就给出$\Omega = \Sigma$, 于是$\Omega$是连通的. 
\end{proof}

\begin{corollary}{}{coro002020509}
如果$f:X \to \Omega$是连续的, $K$是$X$中的紧集或连通集, 那么相应地$f(K)$是$\Omega$中的紧集或连通集. 
\end{corollary}

\begin{corollary}{}{coro002020510}
如果$f:X \to \mathbb{R}$是连续的, $X$是连通的, 那么$f(X)$是一个区间. 
\end{corollary}

这可由$\mathbb{R}$的连通子集的特征是区间这一事实推出. 

\begin{theorem}{中值定理}{thm002020511}
如果$f:[a, b] \to \mathbb{R}$是连续的, 且$f(a) \le \xi \le f(b)$, 那么存在一点$x$, $a \le x \le b$, 使得$f(x) = \xi$. 
\end{theorem}

\begin{corollary}{}{coro002020512}
如果$f:X \to \mathbb{R}$是连续的, $K \subset X$是紧的, 那么在$K$中存在点$x_0$和$y_0$, 使得$f(x_0) = \sup\{f(x):x \in K\}$, $f(y_0) = \inf\{f(x): x \in K\}$. 
\end{corollary}

\begin{proof}
如果$\alpha = \sup\{f(x):x\in K\}$, 那么因为$f(K)$在$\mathbb{R}$中是有界闭集, 所以$\alpha$在$f(K)$内. 类似地, $\beta = \inf\{f(x): x \in K\}$在$f(K)$内. 
\end{proof}

\begin{corollary}{}{coro002020513}
如果$K \subset X$是紧的, $f: X \to \mathbb{C}$是连续的, 那么在$K$内存在点$x_0$和$y_0$, 使得
\[
\begin{aligned}
|f(x_0)| &= \sup\{|f(x)|:x \in K\},\\ 
|f(y_0)| &= \inf\{|f(x)|: x \in K\}.
\end{aligned}
\]
\end{corollary}

\begin{proof}
这个系由上一系推出, 因为$g(x)=|f(x)|$定义一个从$X$到$\mathbb{R}$的连续函数. 
\end{proof}

\begin{corollary}{}{coro002020514}
如果$K$是$X$的紧子集, $x$在$X$内, 那么在$K$内存在一点$y$, 使得$d(x, y) = d(x, K)$. 
\end{corollary}

\begin{proof}
定义$f: X \to \mathbb{R}$为$f(y) = d(x, y)$, 那么$f$是连续的, 并且由系\ref{cor:coro002020512}, 在$K$上取到最小值. 这就是说, 在$K$内存在一点$y$, 使得对于每个$z \in K$, 都有$f(y) \le f(z)$. 这就给出了$d(x, y) = d(x, K)$. 
\end{proof}

下面的两个定理极为重要, 在全书中将反复用到它, 用时不再注明. 

\begin{theorem}{}{thm002020515}
设$f:X \to \Omega$是连续的, $X$是紧的, 那么$f$是一致连续的. 
\end{theorem}

\begin{proof}
设$\epsilon > 0$, 我们要找一个$\delta > 0$, 使得$d(x, y) < \delta$蕴含$\rho(f(x), f(y)) < \epsilon$. 假如不存在这样的$\delta$;特别地, 每个$\delta = \frac{1}{n}$都不满足上述要求. 那么对于每个$n \ge 1$, 在$X$内有点$x_n, y_n$, 使得$d(x_n, y_n) < \frac{1}{n}$, 但是$\rho(f(x_n), f(y_n)) \ge \epsilon$. 因为$X$是紧的, 所以存在子序列$\{x_{n_k}\}$和点$x \in X$, 使得$x = \lim{x_{n_k}}$. 

我们断言$x = \lim{y_{n_k}}$. 事实上, $d(x, y_{n_k}) \le d(x, x_{n_k}) + \frac{1}{n_k}$;当$k$趋于$\infty$时, 它是趋于零的. 

设$\omega = f(x)$, 那么$\omega = \lim{f(x_{n_k})} = \lim{f(y_{n_k})}$, 所以不等式
\[
\epsilon \le \rho(f(x_{n_k}), f(y_{n_k})) \le \rho(f(x_{n_k}), \omega) + \rho(\omega, f(y_{n_k})),
\]
的右边趋于零. 这是一个矛盾. 因而定理得证. 
\end{proof}

\begin{definition}{}{def002020516}
如果$A$, $B$是$X$的子集, 那么定义$A$到$B$的距离$d(A, B)$为
\[
d(A, B) = \inf\{d(a, b): a \in A, b \in B\}.
\]
\end{definition}

注意, 如果$B$是一个点所组成的集$\{x\}$, 那么$d(A, \{x\}) = d(x, A)$. 如果$A = \{y\}$, $B = \{x\}$. 那么$d(\{x\}, \{y\}) = d(x, y)$. 又如果$A \cap B \neq \emptyset$, 那么$d(A, B) = 0$. 但是$A, B$不相交, 我们也可能有$d(A, B) = 0$. 最典型的例子是取$A = \{(x, 0): x \in \mathbb{R}\}$, $B = \{(x, e^x): x \in \mathbb{R}\}$. 注意$A, B$都是闭的且不相交, 但仍有$d(A, B) = 0$. 

\begin{theorem}{}{thm002020517}
如果$A$, $B$是$X$中不相交的集合, $B$是闭的, $A$是紧的, 那么$d(A, B) > 0$. 
\end{theorem}

\begin{proof}
定义$f: X \to \mathbb{R}$为$f(x) = d(x, B)$. 因为$A \cap B = \emptyset$及$B$是闭集, 所以对于$A$内的每一个$a$, $f(a) > 0$. 但是因为$A$是紧的, 所以在$A$内存在一点$a$, 使得$0 < f(a) = \inf\{f(x): x \in A\} = d(A, B)$. 
\end{proof}

\begin{exercise}
证明命题\ref{pro:prop002020502}. 
\end{exercise}

\begin{exercise}
如果$f$和$g$是从$X$到$\mathbb{C}$的一致连续(Lipschitz)函数, 那么$f+g$也是一致连续(Lipschitz)函数. 
\end{exercise}

\begin{exercise}\label{exer002020503}
我们说$f:X \to \mathbb{C}$是有界的, 如果存在一常数$M > 0$, 使得对于$X$中的所有$x$, 有$|f(x)| \le M$. 证明: 如果$f$和$g$是从$X$到$\mathbb{C}$的有界的, 一致连续的(Lipschitz)函数, 则$fg$也是这样的函数.
\end{exercise}

\begin{exercise}
两个一致连续的(Lipschitz)函数的复合函数仍然是一致连续(Lipschitz)函数吗?
\end{exercise}

\begin{exercise}
设$f:X \to \Omega$是一致连续的. 证明: 如果$\{x_n\}$是$X$中的Cauchy序列,则$\{f(x_n)\}$是$\Omega$中的Cauchy序列。如果我们只假定$f$是连续的,结论仍对吗?(证明或举出反例)
\end{exercise}

\begin{exercise}\label{exer002020506}
回忆稠密集的定义(\ref{def:def002020104}). 设$\Omega$是完备的度量空间,$f: (D, d) \to (\Omega,\rho)$是一致连续的,其中$D$在$(X, d)$中稠密。利用上题证明存在一致连续的函数$g:X \to \Omega$,使得对$D$内的每一点,$g(x)=f(x)$.
\end{exercise}

\begin{exercise}\label{exer002020507}
设$G$是$\mathbb{C}$中的一个开子集,$P$是$G$内一条从$a$到$b$的折线。用定理\ref{thm:thm002020515}和\ref{thm:thm002020517}证明:在$G$内存在一条从$a$到$b$的折线,它由平行于实轴或虚轴的线段所组成。
\end{exercise}

\begin{exercise}
利用Lebesque覆盖引理\ref{lem:lemma002020408}给出定理\ref{thm:thm002020515}的另一个证明。
\end{exercise}

\begin{exercise}
证明\ref{section0020202}节的习题\ref{exer002020205}的下述逆命题。设$(X, d)$是紧的度量空间,具有性质:对于每个$\epsilon > 0$,及$X$中的任意点$a$和$b$,在$X$中存在点$z_0,z_1,\cdots, z_n$, $z_0 = a$, $z_n = b$, 使得对于$1 \le k \le n$, 有$d(z_{k-1}, z_k) < \epsilon$,则$(X, d)$是连通的。(提示:利用定理\ref{thm:thm002020517})
\end{exercise}

\begin{exercise}
设$f$和$g$是从$(X, d)$到$(\Omega, \rho)$的连续函数,$D$是$X$中的稠密子集。证明:如果对于$D$内的$x$,$f(x)=g(x)$,则$f=g$。利用这一事实证明第\ref{exer002020506}题中所得到的函数是唯一的。
\end{exercise}


\section{一致收敛性}\label{section0020206}
设$X$是一个集,$(\Omega, \rho)$是一个度量空间,$f, f_1, f_2, \cdots$是$X$到$\Omega$内的函数。我们说序列$\{f_n\}$一致收敛到$f$,记成$f=\ulim{f_n}$,如果对于每个$\epsilon > 0$,存在正整数$N$(只依赖于$\epsilon$),使得当$n \ge N$时,对$X$中的所有$x$,都有$\rho(f(x), f_n(x)) < \epsilon$. 因此,只要$n \ge N$,就有
\[
\sup\{\rho(f(x), f_n(x)) : x \in X\} \le \epsilon.
\]

第一个问题是:如果$X$不仅是一个集,而且还是一个度量空间,并且每一个$f_n$是连续的,那么$f$是连续的吗?回答是肯定的。
\begin{theorem}{}{thm002020601}
设$f_n:(X, d) \to (\Omega, \rho)$对于每个$n$是连续的,$f=\ulim{f_n}$,则$f$是连续的。
\end{theorem}

\begin{proof}
在$X$中固定$x_0$,且取定$\epsilon > 0$,我们希望找出一个$\delta > 0$,使得当$d(x_0, x)< \delta$时, $\rho(f(x_0), f(x)) < \epsilon$. 因为$f = \ulim{f_n}$,所以存在一个函数$f_n$,使得对于$X$中的所有$x$,$\rho(f(x), f_n(x)) < \epsilon/3$. 因为$f_n$是连续的,所以存在$\delta>0$,使得当$d(x_0, x)< \delta$时,$\rho(f_n(x_0), f_n(x)) < \epsilon/3$. 所以如果$d(x_0, x)<\delta$, 则有$\rho(f(x_0), f(x)) \le \rho(f(x_0), f_n(x_0)) + \rho(f_n(x_0), f_n(x)) + \rho(f_n(x), f(x)) < \epsilon$,
\end{proof}

让我们来考虑特别情形$\Omega = \mathbb{C}$。如果$u_n: X \to \mathbb{C}$,令$f_n(x) = u_1(x)+\cdots+u_n(x)$,我们说$f(x) = \sum\limits_{n=1}^{\infty}{u_n(x)}$,当且仅当,$f(x)=\lim{f_n(x)}$对于$X$中的所有$x$成立。称级数$\sum\limits_{1}^{\infty}{u_n(x)}$一致收敛到$f$,当且仅当,$f = \ulim{f_n}$.

\begin{theorem}{Weierstrass M-判别法}{thm002020602}
设函数$u_n:X \to \mathbb{C}$对于$X$中的每个$x$有$|u_n(x)| \le M_n$,而这些常数满足$\sum\limits_{n=1}^{\infty}{M_n} < \infty$, 那么$\sum\limits_{n=1}^{\infty}{u_n}$是一致收敛的。
\end{theorem}

\begin{proof}
设$f_n(x)= u_1(x)+\cdots+u_n(x)$,那么当$n \ge m$时,对于每个$x$,有
\[
|f_n(x)-f_m(x0| = |u_{m+1}(x) + \cdots + u_n(x)| \le \sum_{k=m+1}^{n}{M_k}.
\]
因为$\sum\limits_{1}^{\infty}{M_k}$收敛,所以$\{f_n(x)\}$是$\mathbb{C}$中的Cauchy序列,于是存在一点$\xi \in \mathbb{C}$,使得$\xi = \lim{f_n(x)}$。定义$f(x) = \xi$。这就给出了一个函数$f: X \to \mathbb{C}$。现在有
\[
\begin{aligned}
|f(x)-f_n(x)| &= |\sum_{k=n+1}^{\infty}{u_k(x)}| \\
& \le \sum_{k=n+1}^{\infty}{|u_k(x)|} \le \sum_{k=n+1}^{\infty}{M_k};
\end{aligned}
\]
因为$\sum\limits_{1}^{\infty}{M_k}$是收敛的,所以对于任意的$\epsilon > 0$,存在正整数$N$,使得当$n \ge N$时,$\sum\limits_{k=n+1}^{\infty}{M_k} < \epsilon$。这就给出了当$n \ge N$时,对于$X$中的所有$x$,$|f(x)-f_n(x)| < \epsilon$.
\end{proof}

\begin{exercise}
设$\{f_n\}$是从$(X, d)$到$(\Omega, \rho)$内的一致连续的函数所组成的序列,并且$f = \ulim{f_n}$,证明:$f$是一致连续的。如果每一个$f_n$是带有常数$M_n$的Lipschitz函数,且$\sup{M_n} < \infty$,则$f$是Lipschitz函数。如果$\sup{M_n} = \infty$,说明$f$可能不是Lipschitz函数。
\end{exercise}

\chapter{解析函数的初等性质和例子}\label{chapter00203}

\section{幂级数}\label{section0020301}
在这一节里,我们将给出幂级数的定义和基本性质。然后利用幂级数来给出解析函数的例子。为此有必要先给出有关$\mathbb{C}$内无穷级数的某些初等事实。对$\mathbb{R}$内的无穷级数,这些事实读者应该是熟知的。设对于每个$n \ge 0$,$a_n$在$\mathbb{C}$内,称级数$\sum\limits_{n=0}^{\infty}{a_n}$收敛到$z$,当且仅当,对于每一个$\epsilon > 0$,存在一正整数$N$,使得当$m \ge N$时,$|\sum\limits_{n=0}^{m}{a_n} - z| < \epsilon$。如果$\sum\limits_{n=0}^{\infty}{|a_n|}$收敛,则称级数$\sum\limits_{n=0}^{\infty}{a_n}$绝对收敛。

\begin{proposition}{}{prop002030101}
如果$\sum{a_n}$绝对收敛,那么$\sum{a_n}$收敛。
\end{proposition}

\begin{proof}
设$\epsilon > 0$,令$z_n = a_0 + a_1 + \cdots + a_n$,因为$\sum{|a_n|}$收敛,所以存在正整数$N$,使得$\sum\limits_{n=N}^{\infty}{|a_n|} < \epsilon$. 于是,如果$m \ge k \ge N$,
\[
|z_m - z_k| = |\sum_{n=k+1}^{m}{a_n}| \le \sum_{n=N}^{\infty}{|a_n|} < \epsilon.
\]
即$\{z_n\}$是Cauchy序列。所以在$\mathbb{C}$内有一点$z$,$z = \lim{z_n}$。因此$\sum{z_n} = z$。
\end{proof}

回顾$\mathbb{R}$内的序列上极限和下极限的定义。如果$\{a_n\}$是$\mathbb{R}$内的序列,那么定义
\[
\begin{aligned}
\lim\inf{a_n} &= \lim_{n \to \infty}{[\inf\{a_n, a_{n+1}, \cdots\}]}, \\
\lim\sup{a_n} &= \lim_{n \to \infty}{[\sup\{a_n, a_{n+1}, \cdots\}]},
\end{aligned}
\]
$\lim\inf{a_n}$和$\lim\sup{a_n}$的另一个记号是$\varliminf{a_n}$和$\varlimsup{a_n}$。如果$b_n = \inf\{a_n, a_{n+1}, \cdots\}$,那么$\{b_n\}$是实的递增序列,或是$\{-\infty\}$。因此,$\lim\inf{a_n}$总是存在的,虽然它可能是$\pm{\infty}$。类似地,$\lim\sup{a_n}$总存在,虽然它可能是$\pm{\infty}$。

$\lim\inf$和$\lim\sup$的若干性质包含在这一节的习题中。在$a$点附近的幂级数是形如$\sum\limits_{n=0}^{\infty}{a_n(z-a)^n}$的无穷级数。幂级数的一个最容易的例子(也是最有用的)是几何级数$\sum\limits_{n=0}^{\infty}{z^n}$。对于$z$的哪些值这个级数是收敛的?什么时候这个级数是发散的?容易看出,$1 - z^{n+1} = (1-z)(1 + z + z^2 + \cdots + z^n)$,所以
\begin{gather}\label{equ002030102}
1 + z + \cdots + z^n = \frac{1-z^{n+1}}{1-z}.
\end{gather}

如果$|z| < 1$,那么$0 = \lim{z^n}$。所以几何级数是收敛的,并且有
\[
\sum_{n=0}^{\infty}{z^n} = \frac{1}{1-z}.
\]
如果$|z|>1$,那么$\lim{z^n} = \infty$,级数发散。这个结果不仅是一般幂级数的收敛情况的模型,而且是探讨幂级数的收敛性质的工具。

\begin{theorem}{}{thm002030103}
对于给定的幂级数$\sum\limits_{n=0}^{\infty}{a_n(z-a)^n}$, 由
\[
\frac{1}{R} = \lim\sup{|a_n|^{frac{1}{n}}},
\]
定义数$R$,$0 \le R \le \infty$。那么:
\begin{enumerate}
\item[(a)]如果$|z-a| < R$,则级数绝对收敛;
\item[(b)]如果$|z-a|>R$,则级数的项无界,所以级数发散;
\item[(c)]如果$0 < r < R$,则级数在$\{z:|z-a| \le r\}$上一致收敛。并且具有性质(a)和(b)的数$R$是唯一的。
\end{enumerate}
\end{theorem}

\begin{proof}
我们可以假定$a = 0$。如果$|z|<R$,那么有一$r$,满足$|z| < r < R$。于是存在正整数$N$,使得$|a_n|^{\frac{1}{n}} < \frac{1}{r}$,对于所有的$n > N$成立(因为$\frac{1}{r} > \frac{1}{R}$)。但是这时,$|a_n| < \frac{1}{r^n}$,所以对于所有的$n \ge N$, $|a_nz^n| < (\frac{|z|}{r})^n$。这就是说,余项$\sum\limits_{n=N}^{\infty}{a_nz^n}$囿于级数$\sum{(\frac{|z|}{r})^n}$,并且因为$\frac{|z|}{r} < 1$, 所以对于每个$z$,$|z|<R$,这个幂级数绝对收敛。

现在设$r < R$,选取$\rho$,使得$r < \rho < R$,与上面一样,设$N$是一正整数,使得对于所有的$n \ge N$,$|a_n| < \frac{1}{\rho^n}$,那么,如果$|z|<r$,便有$|a_nz^n| < (\frac{r}{\rho})^n$, $(\frac{r}{\rho})<1$,因此由Weierstrass M-判别法,幂级数在$\{z: |z| \le r\}$上一致收敛。这就证明了(a)和(c)。

为了证明(b), 设$|z|>R$。选取$r$,使得$|z| > r > R$。因此$\frac{1}{r} < \frac{1}{R}$,由$\lim\sup$的定义,有无穷多个$n$使得$\frac{1}{r} < |a_n|^{\frac{1}{r}}$。由此推出$|a_nz^n|>(\frac{|z|}{r})^n$。因为$(\frac{|z|}{r}) > 1$,所以这些项是无界的。
\end{proof}

数$R$称为幂级数的收敛半径。

\begin{proposition}{}{prop002030104}
如果$\sum{a_n(z-a)^n}$是一个给定的幂级数,收敛半径为$R$,则
\[
R = \lim{|a_n/a_{n+1}|},
\]
如果右边的极限存在。
\end{proposition}

\begin{proof}
仍然假定$a = 0$. 设$\alpha = \lim{|a_n /a_{n+1}|}$,我们假定这个极限存在。设$|z|<r < \alpha$,并且取正整数$N$,使得对于所有的$n \ge N$,有$r < |a_n / a_{n+1}|$, 令$B = |a_N|r^N$, 那么$|a_{N+1}|r^{N+1} = |a_{N+1}|rr^N < |a_N|r^N = B$; $|a_{N+2}|r^{N+2} = |a_{N+2}|r \cdot r^{N+1} < |a_{N+1}|r^{N+1} < B$; 如此继续下去,我们得到$|a_nr^n| \le B$对于所有的$n \ge N$成立。但是这时对于所有的$n \ge N$,$|a_nz^n| = |a_nr^n|\frac{|z|^n}{r^n} \le B\frac{|z|^n}{r^n}$. 因为$|z|<r$, 所以我们得到$\sum\limits_{n=1}^{\infty}{|a_nz^n|}$囿于一收敛级数,因此是收敛的。由于$r < \alpha$是任意的,所以$\alpha \le R$.

另一方面,如果$|z| > r > \alpha$,那么对于大于某一正整数$N$的所有的$n$,$|a_n|< r|a_{n+1}|$. 如前所述,我们得到, 对于$n \ge N$,有$|a_nr^n| \ge B = |a_Nr^N|$,这就给出$a_nz^n > B\frac{|z|^n}{r^n}$,它当$n$趋于$\infty$时趋于$\infty$。因此级数$\sum{a_nz^n}$发散。所以$R \le \alpha$,于是$R = \alpha$.
\end{proof}

考虑级数$\sum\limits_{n=0}^{\infty}{\frac{z^n}{n!}}$,由命题\ref{pro:prop002030104},这个级数的收敛半径是$\infty$,因此对于每个复数,它是收敛的,并且在$\mathbb{C}$内的每个紧子集上一致收敛。为了和微积分学一致,我们把这个级数称为指数级数或指数函数
\[
e^z = \exp{z} = \sum_{n=0}^{\infty}{\frac{z^n}{n!}}.
\]

回顾无穷级数理论中的下述命题(不予证明)。

\begin{proposition}{}{prop002030105}
设$\sum{a_n}$和$\sum{b_n}$是两个绝对收敛的级数,令
\[
c_n = \sum_{k=0}^{n}{a_kb_{n-k}},
\]
那么$\sum{c_n}$是绝对收敛的,其和为$(\sum{a_n})(\sum{b_n})$.
\end{proposition}

\begin{proposition}{}{prop002030106}
设$\sum{a_n(z-a)^n}$, $\sum{b_n(z-a)^n}$是收敛半径$\ge r > 0$的两个幂级数,令
\[
c_n = \sum_{k=0}^{n}{a_kb_{n-k}},
\]
那么幂级数$\sum(a_n + b_n)(z-a)^n$和$\sum{c_n(z-a)^n}$的收敛半径都大于或等于$r$,并且对于$|z-a|<r$,有
\[
\begin{aligned}
\sum{(a_n+b_n)(z-a)^n} &= [\sum{a_n(z-a)^n} + \sum{b_n(z-a)^n}],\\
\sum{c_n(c-a)^n} &= [\sum{a_n(z-a)^n}][\sum{b_n(z-a)^n}].
\end{aligned}
\]
\end{proposition}

\begin{proof}
我们只给出证明的梗概。如果$0 < s < r$,那么对于$|z| < s$,我们得到$\sum{|a_n+b_n||z|^n} \le \sum{|a_n|s^n} + \sum{|b_n|s^n} < \infty$;$\sum{|c_n||z|^n} \le (\sum{|a_n|s^n})(\sum{|b_n|s^n}) < \infty$。由此容易完成命题的证明。
\end{proof}

\begin{exercise}
证明命题\ref{pro:prop002030105}
\end{exercise}

\begin{exercise}
给出命题\ref{pro:prop002030106}的详细证明。
\end{exercise}

\begin{exercise}
设$\{a_n\}$,$\{b_n\}$是实数序列,证明:$\lim\sup{(a_n + b_n)} \le \lim\sup{a_n} + \lim\sup{b_n}$, $\lim\inf{(a_n+b_n)} \ge \lim\inf{a_n} + \lim\inf{b_n}$.
\end{exercise}

\begin{exercise}
证明:对于$\mathbb{R}$中的任意序列,有$\lim\sup{a_n} \ge \lim\inf{a_n}$.
\end{exercise}

\begin{exercise}
如果$\{a-n\}$是$\mathbb{R}$中的收敛序列,$a = \lim{a_n}$,证明$a = \lim\inf{a_n} = \lim\sup{a_n}$.
\end{exercise}

\begin{exercise}
求下列幂级数的收敛半径:
\begin{enumerate}
\item[(a)]$\sum\limits_{n=0}^{\infty}{a^nz^n}$, $a \in \mathbb{C}$;
\item[(b)]$\sum\limits_{n=0}^{\infty}{a^{n^2}z^n}$, $a \in \mathbb{C}$;
\item[(c)]$\sum\limits_{n=0}^{\infty}{k^nz^n}$, 整数$k \neq 0$;
\item[(d)]$\sum\limits_{n=0}^{\infty}{z^{n!}}$.
\end{enumerate}
\end{exercise}

\begin{exercise}
证明幂级数
\[
\sum_{n=1}^{\infty}{\frac{(-1)^n}{n}z^{n(n+1)}}
\]
的收敛半径等于1.讨论$z-1,-1, i$时幂级数的收敛性。(提示:这个幂级数的第$n$个系数不是$\frac{(-1)^n}{n}$.)
\end{exercise}

\section{解析函数}\label{section0020302}
在这节里,我们将定义解析函数并给出某些例子。还要证明解析函数的实部和虚部满足Cauchy-Riemann方程。

\begin{definition}{}{def002030201}
设$G$是$\mathbb{C}$中的开集,$f: G \to \mathbb{C}$。说$f$在$G$内的一点$a$是可微的,如果
\[
\lim_{h \to 0}{\frac{f(a+h) - f(a)}{h}}
\]
存在。这个极限值用$f'(a)$来表示,称为$f$在$a$点的导数。如果$f$在$G$的每一点是可微的,我们就称$f$在$G$内是可微的。注意,如果$f$在$G$内是可微的,那么$f'(a)$就定义了一个函数$f':G \to \mathbb{C}$。如果$f'$是连续的,那么我们就说$f$是连续可微的。如果$f'$是可微的,那么就说$f$是二次可微的;如此等等,一个可微函数,如果它的各阶导数都是可微的,就称它是无穷次可微的。
\end{definition}

(今后,除非作相反的声明,我们将假定所有的函数都在$\mathbb{C}$中取值。)

读者一定预料到下面的事实:
\begin{proposition}{}{prop002030202}
如果$f:G \to \mathbb{C}$在$G$内的$a$点是可微的,那么$f$在$a$点连续。
\end{proposition}

\begin{proof}
事实上
\[
\begin{aligned}
\lim_{z \to a}{|f(z)-f(a)|} &=[\lim_{z \to a}{\frac{|f(z)-f(a)|}{|z-a|}}] \cdot [\lim_{z \to a}{|z-a|}] \\
&=|f'(a)| \cdot 0 = 0.
\end{aligned}
\]
\end{proof}

\begin{definition}{}{def002030203}
称函数:$f: G \to \mathbb{C}$是解析的,如果$f$在$G$内是连续可微的。
\end{definition}

如同在微积分学中一样,容易推知,在$G$内的解析函数的和,乘积仍是解析函数。还有,如果$f$和$g$在$G$内是解析的,$G_1$是$G$内的点集,$g$在$G_1$内不等于零,那么$f/g$在$G_1$内是解析的。

由于常数函数与函数$f(z)=z$显然是解析的。由此推出,所有的有理函数在分母的零点集的余集内是解析的。

此外,对于和,积,商的导数的通常法则仍然成立。
\begin{proposition}{链式法则}{prop002030204}
设$f$和$g$分别在$G$和$\Omega$内解析,$f(G) \subset \Omega$, 那么$g \circ f$在$G$内是解析的,并且对于$G$内的所有$z$,有
\[
(g \circ f)'(z) = g'(f(z))f'(z).
\]
\end{proposition}

\begin{proof}
在$G$内固定$z_0$并选取正数$r$使得$B(z_0; r) \subset G$。我们必须证明,如果$0 < |h_n| < r$,$\lim{h_n} = 0$,则$\lim\{h_n^{-1}[g(f(z_0+hn)) - g(f(z_0))]\}$存在且等于$g'(f(z_0))f'(z_0)$。(为什么这对于证明是充分的? )

\begin{description}
\item[情形1]设对于所有的$n$, $f(z_0) \neq f(z_0 + h_n)$。在这种情形,
\[
\begin{aligned}
&\frac{g \circ f(z_0 + h_n) - g \circ f(z_0)}{h_n} \\
&\quad = \frac{g \circ f(z_0 + h_n) - g \circ f(z_0)}{f(z_0+h_n) - f(z_0)} \cdot \frac{f(z_0+h_n) - f(z_0)}{h_n}
\end{aligned}
\]
因为根据(\ref{pro:prop002030202})$\lim{[f(z_0 + h_n) - f(z_0)]} = 0$,所以我们有
\[\lim{h_n^{-1}[g \circ f(z_0 + h_n) - g \circ f(z_0)]} = g'(f(z_0))f'(z_0).
\]
\item[情形2]设对于无穷多个$n$的值,$f(z_0) = f(z_0 + h_n)$。将$h_n$表示为两个序列$\{k_n\}$和$\{l_n\}$的和,其中$f(z_0) \neq f(z_0 + k_n)$和$f(z_0) = f(z_0 + l_n)$对所有的$n$成立。由于$f$是可微的,所以$f'(z_0) = \lim{l_n^{-1}[f(z_0 + l_n) - f(z_0)]} = 0$。也有$\lim{l_n^{-1}[g \circ f(z_0 + l_n) - g \circ f(z_0)]} = 0$。由情形1,$\lim{k_n^{-1}[g \circ f(z_0+k_n) - g \circ f(z_0)]} = g'(f(z_0))f'(z_0) = 0$,所以
\[
\lim{h_n^{-1}[g \circ f(z_0+h_n) - g \circ f(z_0)]} = 0 = g'(f(z_0))f'(z_0) = 0.
\]
\end{description}
一般情形容易由上面两种情形推出。
\end{proof}

为了定义导数,我们假定函数是定义在开集内的。如果我们说$f$是在集$A$上解析的,而$A$不是开集,我们的意思是指$f$在包含$A$的一个开集内是解析的。

解析函数的这个定义也许对许多读者来说有点反常。但是,在看过解析函数论的书籍,并且上了一年解析函数的课程和讨论班之后,他们会发现这个定义在微积分学中已经出现过,从而会解除一定的疑虑。但这个理论是微积分学的简单推广吗?回答是否定的。为了表明这两者之间有多么巨大的差别,让我们提一下,我们以后将证明\textbf{可微函数是解析的}。这的确是一个奇特的结果,在实变数函数的理论中是没有与此相应的结果的(例如考虑$x^2\sin{\frac{1}{x}}$)。另一个同样值得注意的结果是:\textbf{每个解析函数是无穷次可微的,并且在它的域内的每一点有幂级数展开式。}为什么如此弱的假设竟有如此深刻的结论呢?如果考虑一下导数的定义,便可从中找到出现这种现象的某些征兆。

在复变数的情形,变数可以沿无穷多个方向趋于一点$a$,但在实变数的情形,只有两个趋近的途径。例如,定义在$\mathbb{R}$上的函数的连续性,可以通过它的左连续性和右连续性来讨论。这与复变数函数的情形是大不相同的。所以复变数函数有导数这句话比说实变数函数有导数更强。甚至,如果我们令$g(x, y) = f(x + iy)$,把定义在$G \subset \mathbb{C}$内的函数$f$看作为两个实变数的函数,那么,即使要求$f$是Frechet可微的\footnote{应该是多元函数中的微分定义,等涉及到之后补上这里的参考资料。},也不能保证$f$在我们的意义下有导数。

在习题中,我们要求读者证明$f(z)=|z|^2$仅在$z=0$有导数;但是$g(x,y) = f(x+iy) = x^2+y^2$是Frechet可微的。

可微性蕴含解析性在第\ref{chapter00204}章中证明。现在我们来证明幂级数表示的函数是解析的。

\begin{proposition}{}{prop002030205}
设$f(z) = \sum\limits_{n=0}^{\infty}{a_n(z - a)^n}$的收敛半径$R > 0$,那么:
\begin{enumerate}
\item[(a)]对于每个$k \ge 1$,级数
\begin{gather}\label{equ002030206}
\sum_{n=k}^{\infty}{n(n-1)\cdots(n-k+1)a_n(z-a)^{n-k}}
\end{gather}
的收敛半径为$R$;
\item[(b)]函数$f$在$B(a;R)$内是无穷次可微的,并且对于所有的$k \ge 1$及$|z - a| < R$,$f^{(k)}(z)$由级数(\ref{equ002030206})给出;
\item[(c)]对于$n \ge 0$,
\begin{gather}\label{equ002030207}
a_n = \frac{1}{n!}f^{(n)}(a).
\end{gather}
\end{enumerate}
\end{proposition}

\begin{proof}
仍假定$a = 0$.

(a)我们首先注意到,如果对于$k=1$,(a)被证明了,那么(a)在$k=2,\cdots$时也将随之成立。事实上,(a)在$k=1$的情形应用到级数$\sum{na_n(z-a)^{n-1}}$上,便得到$k=2$的情形。我们已知$R^{-1} = \lim\sup{|a_n|^{\frac{1}{n}}}$,而希望证明$R^{-1} = \lim\sup{|na_n|^{\frac{1}{n-1}}}$。由l'H\"ospital法则\footnote{$l'H\hat{o}spital$,洛必达。},$\lim\limits_{n \to \infty}{\frac{\log{n}}{n-1}} = 0$,所以$\lim\limits_{n \to \infty}{n^{\frac{1}{n-1}}} = 1$。如果能够证明
\[
\lim\sup{|a_n|^{\frac{1}{n-1}}} = R^{-1},
\]
我们的结果便从习题\ref{exer002030202}推出。

设$(R')^{-1} = \lim\sup{|a_n|^{\frac{1}{n-1}}}$, 那么$R'$是$\sum\limits_{1}^{\infty}{a_nz^{n-1}} = \sum\limits_{0}^{\infty}{a_{n+1}z^n}$的收敛半径,注意到$z\sum{a_{n+1}z^n} + a_0 = \sum{a_nz^n}$,所以如果$|z| < R'$,那么$\sum{|a_nz^n|} \le |z_0| + |z|\sum{|a_{n+1}z^n|} < \infty$. 这就给出$R' \le R$。如果$|z| < R$,且$z \neq 0$,那么$\sum{|a_nz^n|} < \infty$,$\sum{|a_{n+1}z^n|} \le \frac{1}{|z|} \cdot \sum{|a_nz^n| + \frac{1}{|z|}|a_0|} < \infty$,所以$R \le R'$。这就得到$R = R'$。(a)证毕。

(b)对于$|z| < R$,设$g(z) = \sum\limits_{n=1}^{\infty}{na_nz^{n-1}}$,$s_n(z) = \sum\limits_{k=0}^{n}{a_kz^k}$,$R_n(z) = \sum\limits_{k=n+1}^{\infty}{a_kz^k}$。固定$B(0;R)$内的一点$w$,固定$r$使得$|w|<r < R$。我们要证明$f'(w)$存在且等于$g(w)$。为此,设$\delta > 0$是任意的,只要使得$\bar{B}(w; \delta) \subset B(0; r)$(在后面的证明中我们将进一步限定$\delta$),设$z \in B(w;\delta)$,那么
\begin{gather}\label{equ002030208}
\begin{aligned}
&\quad\frac{f(z)-f(w)}{z - w} - g(w) \\
&=[\frac{s_n(z)-s_n(w)}{z-w} = s_n'(w)] + [s_n'(w) - g(w)] \\
& \quad \quad + [\frac{R_n(z) - R_n(w)}{z-w}],
\end{aligned}
\end{gather}
现在
\[
\frac{R_n(z)-R_n(w)}{z-w} = \frac{1}{z-w}\sum_{k=n+1}^{\infty}{a_k(z^k - w^k)} = \sum_{k=n+1}^{\infty}{a_k(\frac{z^k - w^k}{z-w})}.
\]
而
\[
|\frac{z^k - w^k}{z-w}| = |z^{k-1} + z^{k-2}w + \cdots + zw^{k-2} + w^{k-1}| \le kr^{k-1}.
\]
因此
\[
|\frac{R_n(z) - R_n(w)}{z-w}| \le \sum_{k=n+1}^{\infty}{|a_k|kr^{k-1}}.
\]
因为$r < R$,所以$|\sum\limits_{k=1}^{\infty}{|a_k|kr^{k-1}}|$收敛,于是对于任意的$\epsilon > 0$,存在正整数$N_1$,使得当$n \ge N_1$时,
\[
|\frac{R_n(z) - R_n(w)}{z-w}| < \epsilon/3, \quad (z \in B(w; \delta))
\]
又$\lim{s_n'(w)} = g(w)$,所以存在正整数$N_2$,使得当$n \ge N_2$时,$|s_n'(w) - g(w)| < \epsilon/3$。设$n = \max{(N_1, N_2)}$,这时我们可选取$\delta > 0$,使得当$0 < |z - w| < \delta$时,
\[
|\frac{s_n(z) - s_n(w)}{z-w} - s_n'(w)| < \epsilon / 3.
\]
把这些不等式代入(\ref{equ002030208}),便得到
\[
|\frac{f(z)-f(w)}{z-w} - g(w)| < \epsilon.
\]
只要$0 < |z-w| < \delta$。这就是$f'(w) = g(w)$。

(c)直接计算,我们得到$f(0) = f^{(0)}(0)=a_0$。利用(\ref{equ002030206})(当$a=0$时),我们得到$f^{(k)}(0) = k!a_k$,这就给出了公式(\ref{equ002030207})。
\end{proof}

\begin{corollary}{}{coro002030209}
如果级数$\sum\limits_{n=0}^{\infty}{a_n(z-a)^n}$的收敛半径$R>0$,那么$f(z)=\sum\limits_{n=0}^{\infty}{a_n(z-a)^n}$ 在 $B(a; R)$ 内是解析的.
\end{corollary}

因此$\exp{z} = \sum\limits_{n=0}^{\infty}{z^n/n!}$在$\mathbb{C}$内是解析的。在进一步考察指数函数与定义$\cos{z}$, $\sin{z}$以前,必须证明下面的结果。

\begin{proposition}{}{prop002030210}
如果$G$是连通开集,$f: G \to \mathbb{C}$是可微的,并且对于$G$内所有的$z$,$f'(z)=0$,那么$f$是常数。
\end{proposition}

\begin{proof}
在$G$内固定$z_0$,设$w_0 = f(z_0)$,$A = \{z \in G: f(z)=w_0\}$.我们将通过证明$A$在$G$内既是开的,又是闭的,来证明$A = G$。设$z \in G$,$\{z_n\} \subset A$, $z = \lim{z_n}$,因为$f(z_n) = w_0$对于每个$n \ge 1$成立,以及$f$是连续的,所以我们得到,$f(z)=w_0$,或者说$z \in A$。于是$A$在$G$内是闭的。现在在$A$内固定$a$,设$\epsilon > 0$,使得$B(a;\epsilon) \subset G$。如果$z \in B(a;\epsilon)$, 令$g(t) = f(tz + (1-t)a)$,$0 \le t \le 1$,那么
\begin{gather}\label{equ002030211}
\frac{g(t)-g(s)}{t - s} = \frac{g(t)-g(s)}{(t-s)z + (s-t)a} \cdot \frac{(t-s)z + (s-t)a}{t - s}.
\end{gather}
如果我们令$t \to s$,便得到
\[
\lim_{t \to s}{\frac{g(t)-g(s)}{t - s}} = f'(sz + (1-s)a) \cdot (z-a) = 0.
\]
即对于$0 \le s \le t$,$g'(s) = 0$,因此$g$是一个常数。所以$f(z)=g(1) = g(0) = f(a)=w_0$. 即$B(a; \epsilon) \subset A$, $A$也是开集。
\end{proof}

现在对$f(z)=e^z$,求导数,由命题\ref{pro:prop002030205},
\[
\begin{aligned}
f'(z) &= \sum_{n=1}^{\infty}{\frac{n}{n!}z^{n-1}} = \sum_{n=1}^{\infty}{\frac{1}{(n-1)!}z^{n-1}}\\
&= \sum_{n=0}^{\infty}{\frac{z^n}{n!}} = f(z).
\end{aligned}
\]
于是复的指数函数和实的情形有相同的性质,即
\begin{gather}\label{equ002030212}
\frac{d}{dz}{e^z} = e^z.
\end{gather}
对于$\mathbb{C}$内某一固定的$a$,设$g(z) = e^ze^{a-z}$,那么$g'(z) = e^ze^{a-z} + e^z(-e^{a-z}) = 0$, 因此对于$\mathbb{C}$中的所有$z$和某一常数$\omega$, $g(z) = \omega$. 特别地,利用$e^0 = 1$,我们得到$\omega = g(0) = e^a$. 所以对于所有的$z$,$e^ze^{a-z}=e^a$。于是对于$\mathbb{C}$中所有的$a$, $b$,$e^{a+b} = e^a \cdot e^b$. 这也给出$1 = e^ze^{-z}$,它蕴含对于任意的$z$,$e^z \neq 0$以及$e^{-z} = 1/e^z$。我们再回到$e^z$的幂级数。因为这个级数的所有系数是实的,所以我们得到$\exp{\bar{z}} = \overline{\exp{z}}$. 特别地,对于实数$\theta$, 我们得到$|e^{i\theta}|^2 = e^{i\theta}e^{-i\theta} = e^0 = 1$. 更一般地,$|e^z|^2 = e^ze^{\bar{z}} = e^{z = \bar{z}} = \exp{(2\Re{z})}$, 于是 
\begin{gather}\label{equ002030213}
|\exp{z}| = \exp{(\Re{z})}.
\end{gather}
所以我们看出$e^z$和实函数$e^x$有同样的性质。仍和实的幂级数类似,我们也用幂级数定义$\cos{z}$和$\sin{z}$. 
\[
\begin{aligned}
\cos{z} &= 1 - \frac{z^2}{2!} + \frac{z^4}{4!} + \cdots + (-1)^n\frac{z^{2n}}{(2n)!} + \cdots,\\
\sin{z} &= z - \frac{z^3}{3!} + \frac{z^5}{5!} + \cdots + (-1)^n\frac{z^{2n+1}}{(2n+1)!} + \cdots.
\end{aligned}
\]
这两个幂级数的收敛半径都是$\infty$,所以$\cos{z}$和$\sin{z}$在$\mathbb{C}$内是解析的。利用命题\ref{pro:prop002030205},我们得到$(\cos{z})' = -\sin{z}$, $(\sin{z})' = \cos{z}$。通过幂级数的运算(因为幂级数是绝对收敛的, 这是允许的),得到
\begin{gather}\label{equ002030214}
\cos{z} = \frac{1}{2}(e^{iz} + e^{-iz}), \quad \sin{z} = \frac{1}{2i}(e^{iz} - e^{-iz}).
\end{gather}
这就得到,对于$\mathbb{C}$中的$z$,有$\cos^2{z} + \sin^2{z} = 1$以及
\begin{gather}\label{equ002030215}
e^{iz} = \cos{z} + i\sin{z}.
\end{gather}
特别地,如果在(\ref{equ002030215})中设$z$是实数$\theta$,我们得到$e^{i\theta} = \cis{\theta}$。因此,对于$\mathbb{C}$中的$z$
\begin{gather}\label{equ002030216}
z = |z|e^{i\theta},
\end{gather}
其中$\theta = \arg{z}$。因为$e^{x+iy} = e^x+e^{iy}$,所以我们有$|e^z| = \exp{(\Re{z})}$及$\arg{e^z} = \Im{z}$.

说$f$是以$c$为周期的周期函数,如果$f(z+c) = f(z)$对于$\mathbb{C}$中的所有$z$成立。如果$c$是$e^z$的周期,那么$e^z = e^{z+c} = e^ze^c$蕴含$e^c=1$。因为$1 = |e^c| = \exp{(\Re{c})}$,$\Re{(c)} = 0$。于是对于$\mathbb{R}$中的某一$\theta$,$c = i\theta$. 但是由于$1 = e^c = e^{i\theta} = \cos{\theta} + i\sin{theta}$, 给出$e^z$的周期是$2\pi{i}$的倍数。于是如果我们用直线$\Im{z} = \pi(2k-1)$,$k$是任意整数,把平面分成无穷多个水平带形,指数函数在每个带形内有相同的性质。这个周期性是实指数所不具备的一个性质。注意,通过考察复函数,我们证明了指数函数和三角函数之间的关系式(\ref{equ002030215})。这个关系式单凭我们关于实函数的知识是料想不到的。

现在我们来定义$\log{z}$,我们可以采取和前面一样的办法,设$\log{z}$是实对数函数在某点附近的幂级数展开式。但是这仅给出在某一圆内的$\log{z}$。定义对数为$t^{-1}$从$1$到$x$($x > 0$)的积分,这个方法是可行,但在复的情形,这样定义有点冒险,并且是不能令人满意的。又因为与实的情形不同,$e^z$并不是一一的映照,所以$\log{z}$不能定义为$e^z$的反函数。但是我们可以有与此类似的做法。

我们要这样定义$\log{w}$, 使得当$z = \log{w}$时,$w = e^z$。现在,因为对于任意$z$, $e^z \neq 0$, 所以我们不能定义$\log{0}$,设$e^z = w$,$w \neq 0$; 如果$z = x + iy$,那么$|w| = e^x$,且对于某个$k$,$y = \arg{w} + 2\pi{}k$. 因此
\begin{gather}\label{equ002030217}
\{\log{|w|} + i(\arg{w} + 2\pi{}k) : k\text{是任意整数}\}
\end{gather}
是$e^z = w$的解的集。(注意:$\log{|w|}$是通常实的对数。)

\begin{definition}{}{def002030218}
如果$G$是$\mathbb{C}$中的连通开集,$f: G \to \mathbb{C}$是一个连续函数,使得对于$G$内的所有$z$,$z = \exp{f(z)}$,那么$f$是对数的一个分支。
\end{definition}

注意$0 \not\in G$。

设$f$是连通集$G$内的一个给定的对数分支,$k$是整数。$g(z)=f(z) + 2\pi{}ki$,那么$\exp{g(z)} = \exp{f(z)} = z$,所以$g$也是一个对数分支。反之,如果$f$, $g$都是$\log{z}$的分支,那么对于$G$内的每个$z$, $f(z) = g(z) + 2\pi{}ki$,$k$是依赖于$z$的某一整数。对于$G$内的每个$z$, $k$是相同的吗?回答是肯定的。事实上,如果$h(z) = \frac{1}{2\pi{}i}[f(z)-g(z)]$, 那么$h(z)$在$G$内是连续的,并且$h(G) \subset \mathbb{Z}$(整数集)。因为$G$是连通的,所以$h(G)$也是连通的(第\ref{chapter00202}章定理\ref{thm:thm002020508})。因此在$\mathbb{Z}$中有一个整数$k$,使得$f(z) + 2\pi{}ik = g(z)$对于$G$内的所有$z$都成立。这就得到下面的命题。

\begin{proposition}{}{prop002030219}
如果$G \subset \mathbb{C}$是连通开集, $f$是$G$内$\log{z}$的一个分支,那么$\log{z}$的所有分支可表示为$f(z) + 2\pi{}ki$,$k \in \mathbb{Z}$。
\end{proposition}

现在让我们在某一个连通开集内至少作出$\log{z}$的一个分支。设
\[
G = \mathbb{C} - \{z: z \le 0\};
\]
即,沿负实轴“切开”平面。显然$G$是连通的,并且对于$G$内的每个$z$,能够唯一地表示为$z = |z|e^{i\theta}$,其中$-\pi < \theta < \pi$,对于$\theta$在这个范围内,定义$f(re^{i\theta}) = \log{r} + i\theta$。我们把连续性的证明留给读者(习题\ref{exer002030209}),由此推出$f$是$G$内对数的一个分支。

$f$是解析的吗?为了回答这个问题,我们先证明一个一般性的事实。

\begin{proposition}{}{prop002030220}
设$G$和$\Omega$是$\mathbb{C}$的开子集。假定$f: G \to \mathbb{C}$,$g: \Omega \to \mathbb{C}$是连续函数,使得$f(G) \subset \Omega$,且$g(f(z)) = z$对于$G$内的所有$z$成立。如果$g$是可微的,且$g'(z) \neq 0$,那么$f$是可微的,且
\[
f'(z) = \frac{1}{g'(f(z))}.
\]
如果$g$是解析的,那么$f$也是解析的。
\end{proposition}

\begin{proof}
在$G$内固定一点$a$. 设$h \in \mathbb{C}$,$h \neq 0$, 使得$a + h \in G$. 因此$a = g(f(a))$, $a + h = g(f(a + h))$蕴含$f(a) \neq f(a + h)$. 又
\[
\begin{aligned}
1 &= \frac{g(f(a + h)) - g(f(a))}{h}\\
&=\frac{g(f(a + h)) - g(f(a))}{f(a + h) - f(a)} \cdot \frac{f(a + h) - f(a)}{h}.
\end{aligned}
\]

现在当$h \to 0$时,左边的极限当然是1,所以右边的极限也存在。因为$\lim\limits_{h \to 0}{[f(a + h) - f(a)]} = 0$, 所以
\[
\lim_{h \to 0}{\frac{g(f(a + h)) - g(f(a))}{f(a+h) - f(a)}} = g'(f(a)).
\]
因此,由$g'(f(a)) \neq 0$可知
\[
\lim_{h \to 0}{\frac{f(a + h) - f(a)}{h}}
\]
存在,并且$1 = g'(f(a))f'(a)$。

于是$f'(z) = [g'(f(a))]^{-1}$。如果$g$是解析的,那么$g'$是连续的,由此得到$f$是解析的。
\end{proof}

\begin{corollary}{}{coro002030221}
对数函数的分支是解析的,它的导数是$z^{-1}$.
\end{corollary}

我们把上面定义在$\mathbb{C} - \{z:z \le 0\}$的对数的特殊分支,称为对数的主分支(principal branch)。如果不作别的声明,我们总是把$\log{z}$当作对数的主分支。

如果$f$是对数函数在连通开集$G$内的分支,$b$是$\mathbb{C}$中的固定点,那么定义$g: G \to \mathbb{C}$为$g(z) = \exp{(bf(z))}$. 如果$b$是一个整数,那么$g(z) = z^b$。对于具有$\log{z}$的分支的连通开集,我们用这种方式定义$z^b$的分支,其中$b$在$\mathbb{C}$中。如果我们把$g(z)=z^b$作为一个函数,我们总是把这个函数理解为$z^b = \exp{b\log{z}}$,其中$\log{z}$是对数的主分支,因为$\log{z}$是解析的,所以$z^b$也是解析的。

从刚才的考虑可以明显看出,连通性在解析函数论中起着重要作用。例如,如果$G$不是连通的,那么命题\ref{pro:prop002030210}是不对的。连通性在这里所起的作用类似于区间在微积分中所起的作用。因为这个原因,引进术语“域\footnote{这个域不是代数中的域,这里的域应该对应英文domain,而代数中的域对应field。}”是方便的。一个域是平面上的一个连通开子集。

这一节以讨论Cauchy-Riemann方程作为结束。设$f: G \to \mathbb{C}$是解析的。对于$G$内的$x + iy$,令$u(x, y) = \Re{f(x + iy)}$, $v(x, y) = \Im{f(x + iy)}$。我们用两种不同的方法计算极限
\[
f'(z) = \lim_{h \to 0}{\frac{f(z+h) - f(z)}{h}}.
\]
先让$h$取实值趋于零。对于$h \neq 0$,$h$是实的,我们得到
\[
\begin{aligned}
\frac{f(z+h) - f(z)}{h} &= \frac{f(x+ h + iy) - f(x + iy)}{h}\\
&= \frac{u(x+h, y) - u(x, y)}{h} \\
&\quad + i\frac{v(x+h, y) - v(x, y)}{h}
\end{aligned}
\]
令$h \to 0$,得到
\begin{gather}\label{equ002030222}
f'(z) = \frac{\partial{u}}{\partial{x}}(x, y) + i\frac{\partial{v}}{\partial{x}}(x, y).
\end{gather}
现在让$h$取纯虚值趋于零,即对于$h \neq 0$,$h$是实的,
\[
\begin{aligned}
\frac{f(z+ih) - f(z)}{ih} &= -i\frac{u(x, y+h) - u(x, y)}{h} \\
&\quad + \frac{v(x, y+h) - v(x, y)}{h}
\end{aligned}
\]
于是
\begin{gather}\label{equ002030223}
f'(z) = -i\frac{\partial{u}}{\partial{y}}(x, y) + \frac{\partial{v}}{\partial{y}}(x, y).
\end{gather}
令(\ref{equ002030222})和(\ref{equ002030223})的实部虚部相等,我们便得到Cauchy-Riemann方程
\begin{gather}\label{equ002030224}
\frac{\partial{u}}{\partial{x}} = \frac{\partial{v}}{\partial{y}}, \quad \frac{\partial{u}}{\partial{y}} = -\frac{\partial{v}}{\partial{x}}
\end{gather}
假定$u$和$v$有二阶连续偏导数(我们最终将证明它们是无穷次可微的)。对Cauchy-Riemann方程求微商,我们得到
\[
\frac{\partial^2{u}}{\partial{x^2}} = \frac{\partial^2{v}}{\partial{x}\partial{y}}, \quad
\frac{\partial^2{u}}{\partial{y^2}} = -\frac{\partial^2{v}}{\partial{y}\partial{x}}.
\]
因此
\begin{gather}\label{equ002030225}
\frac{\partial^{u}}{\partial{x^2}} + \frac{\partial^2{u}}{\partial{y^2}} = 0,
\end{gather}
满足(\ref{equ002030225})的任意一个具有连续的二阶偏导数的函数称为调和函数。类似地,$v$也是调和的。我们将在第\ref{chapter00210}章中研究调和函数。

设$G$是平面上的一个域,$u$和$v$是定义在$G$内具有连续偏导数的函数。进而设$u$, $v$满足Cauchy-Riemann方程。为此,设$z = x + iy \in G$, $B(z;r) \subset G$. 如果$h = s + it \in B(0; r)$, 那么
\[
\begin{aligned}
u(x+s, y+t) - u(x, y) &= [u(x+s, y+t) - u(x, y + t)]\\
&\quad + [u(x, y + t) - u(x, y)].
\end{aligned}
\]
把一元函数导数的中值定理应用到这两个括号内的式子上,那么对于$B(0; r)$内的每个$s + it$,都有数$s_1$和$t_1$,$|s_1| < |s|$, $|t_1| < |t|$,使得
\begin{gather}\label{equ002030226}
\left\{
\begin{aligned}
&u(x+s, y+t) - u(x, y+t) = u_x(x+s_1, y+ t)s,\\
&u(x, y+t) - u(x, y) = u_y(x, y + t_1)t.
\end{aligned}
\right.
\end{gather}
令$\varphi(s, t) = [u(x+s, y+t) - u(x, y)] - [u_x(x, y)s +u_y(x, y)t]$, 由(\ref{equ002030226})得到
\[
\begin{aligned}
\frac{\varphi(s, t)}{s + it} &= \frac{s}{s + it}[u_x(x+s_1, y + t) - u_x(x, y)]\\
&\quad + \frac{t}{s+it}[u_y(x, y+t_1) - u_y(x, y)].
\end{aligned}
\]
但是由$|s| < |s + it|$, $|t| < |s + it|$, $|s_1| < |s|$, $|t_1| < |t|$,以及$u_x$, $u_y$是连续的,得到
\begin{gather}\label{equ002030227}
\lim_{s + it \to 0}{\frac{\varphi(s, t)}{s + it}} = 0.
\end{gather}
因此
\[
\begin{aligned}
u(x+s, y+t) - u(x, y) &= u_x(x, y)s + u_y(x, y)t \\
&\quad + \varphi(s, t),
\end{aligned}
\]
其中$\varphi$满足(\ref{equ002030227})。类似地,
\[
\begin{aligned}
v(x+s, y+t) - v(x, y) = v_x(x, y)s + v_y(x, y)t + \psi(s, t).
\end{aligned}
\]
其中$\psi$满足
\begin{gather}\label{002030228}
\lim_{s +it \to 0}{\frac{\psi(s, t)}{s+it}} = 0.
\end{gather}
利用$u$, $v$满足Cauchy-Riemann方程这个事实,容易看出
\[
\frac{f(z + s + it) - f(z)}{s+it} = u_x(z) + iv_x(z) + \frac{\varphi(s, t) + i\psi(s, t)}{s+it}.
\]
根据(\ref{equ002032027})和(\ref{equ002030228}),$f$是可微的,并且$f'(z) = u_x(z) + iv_x(z)$. 因为$u_x$, $v_x$是连续的,所以$f'$是连续的,$f$是解析的。我们把这些结果总结如下。
\begin{theorem}{}{thm002030229}
设$u$, $v$是定义在区域$G$内的实值函数,假定$u$, $v$有连续的偏导数。$f: G \to \mathbb{C}$定义为$f(z) = u(z) + iv(z)$。当且仅当$u$, $v$满足Cauchy-Riemann方程时,$f$是解析的。
\end{theorem}

\textbf{例子}\quad $u(x, y) = \log{(x^2+y^2)^{\frac{1}{2}}}$在$G = \mathbb{C} - \{0\}$内是调和的吗?回答是肯定的。对$u$求微商可看出它满足(\ref{002030225})。但是也可通过观察下述事实来证明:在$G$内的每一点的邻域内,$u$是定义在该邻域的一个解析函数(哪个函数?)的实部。

下面是关于调和函数的另一个问题。这个问题将在第\ref{chapter00208}章\ref{section0020803}中作详细的研究。设$G$是平面上的一个域,$u: G \to \mathbb{R}$是调和的. 是否存在一个调和函数$v: G \to \mathbb{R}$,使得$f = u+iv$在$G$内是解析的?如果这样的函数$v$存在, 就称它是$u$的共轭调和函数。如果$v_1$和$v_2$是$u$的两个共轭调和函数,那么$i(v_1-v_2) = (u+iv_1)-(u+iv_2)$在$G$内是解析的,并且仅取纯虚值。由此推出,一个调和函数的两个共轭调和函数相差为一常数(习题\ref{exer002030214})。

回到共轭调和函数的存在性问题上来。在域$G = \mathbb{C} - \{0\}$内调和函数的上述例子$u(z) = \log{|z|}$,没有共轭调和函数。事实上,如果它有共轭调和函数,那么便可在$G$内定义对数函数的一个解析分支,而这是办不到的(习题\ref{exer002030221})。但是存在一些区域,在这些区域内每个调和函数有共轭调和函数。特别地,现在证明,当$G$是任意圆或是整个平面时,就是这种情形。
\begin{theorem}{}{thm002030230}
设$G$或者是整个平面,或者是某一个开圆。如果$u: G \to \mathbb{R}$是一个调和函数,那么$u$有共轭调和函数。
\end{theorem}

\begin{proof}
为了完成定理的证明,需要用到积分号下求微商的Leibniz法则(这个法则将在第\ref{chapter00204}章的\ref{pro:prop002040201}的命题中叙述和证明),设$G = B(0; R)$,$0 < R \le \infty$。又设$u : G \to \mathbb{R}$是调和函数。我们通过寻求调和函数$v$,使得$u$, $v$满足Cauchy-Riemann方程,来完成定理的证明。为此定义
\[
v(x,y) = \int_{0}^{y}{u_x(x, t)dt} + \varphi(x),
\]
并确定$\varphi$,使得$v_x = -u_y$。上述等式两边对$x$求微商,得到
\[
\begin{aligned}
v_x(x, y) &= \int_{0}^{y}{u_{xx}(x, t)dt} + \varphi'(x) = -\int_{0}^{y}{u_{yy}(x, t)dt} + \varphi'(x)\\
&= -u_y(x, y) + u_y(x, 0) + \varphi'(x).
\end{aligned}
\]
所以,必定有$\varphi'(x) = -u_y(x, 0)$。容易验证$u$和
\[
v(x, y) = \int_{0}^{y}{u_x(x, t)dt} - \int_{0}^{x}{u_y(s, 0)ds}
\]
确实满足Cauchy-Riemann方程。
\end{proof}

$G$是圆或是$\mathbb{C}$这个条件 用在何处?这个证明方法为什么不能作足够的修改使之适用于一般的域?当$G = \mathbb{C} - \{0\}$, $u = \log{|z|}$时,这个证明在何处失效?


\begin{problemset}
\item 证明:$f(z) = |z|^2 = x^ + y^2$仅在原点有导数。

\item\label{exer002030202}证明:如果$b_n$, $a_n$是正实数,$0 < b = \lim{b_n}$, $a = \lim\sup{a_n}$, 则$ab = \lim\sup{(a_nb_n)}$. 如果正德这一要求去掉,结论仍成立吗?

\item 证明:$\lim{n^{\frac{1}{n}}} = 1$.

\item 证明:$(\cos{z})' = -\sin{z}$, $(\sin{z})' = \cos{z}$.

\item 导出公式(\ref{equ002030214})。

\item 描画出下列各集:$\{z:e^z = i\}$, $\{z:e^z=-1\}$, $\{z:e^z=-i\}$, $\{z:\cos{z}=0\}$, $\{z:\sin{z}=0\}$.

\item 证明对于$\cos{(z+w)}$, $\sin{(z+w)}$的公式。

\item 定义$\tan{z} = \frac{\sin{z}}{\cos{z}}$; 这个函数在何处有定义?在何处是解析的?

\item\label{exer002030209}设$z_n$, $z \in G= \mathbb{C} - \{z: z \le 0\}$, 且$z_n = r_ne^{i\theta_n}$, $z = re^{i\theta}$,其中$-\pi < \theta, \theta_n < \pi$. 证明:如果$z_n \to z$,那么$\theta_n \to \theta$, $r_n \to r$.

\item 证明命题\ref{pro:prop002030220}的下述推广:设$G$和$\Omega$是$\mathbb{C}$中的开集又设$f$和$h$是定义在$G$内的函数,$g: \Omega \to \mathbb{C}$,$f(G) \subset \Omega$, $g$和$h$是解析的, 对于任意$w$, $g'(w) \neq 0$, $f$是连续的,$h$是一一的,并且对于$G$内的$z$,它们满足$h(z) = g(f(z))$. 证明$f$是解析的,给出$f'(z)$的公式。

\item 设$f:G \to \mathbb{C}$是对数函数的一个分支,$n$是一个整数,证明对于$G$内的所有的$z$, $z^n = \exp{(nf(z))}$.

\item 证明:函数$z^{\frac{1}{2}}$的实部总是正的。

\item 设$G = \mathbb{C} - \{z:z \le 0\}$, $n$是正整数。试求出所有的解析函数$f:G \to \mathbb{C}$, 使得对于所有的$z \in G$, $z = (f(z))^n$.

\item\label{exer002030214}设$f: G \to \mathbb{C}$是解析的,$G$是连通的,证明:如果$f(z)$对于$G$内的所有$z$是实的,那么$f$是常数。

\item 对于$r > 0$, 设$A = \{w: w = \exp{(\frac{1}{z})}, 0 < |z| < r\}$, 试确定这个集$A$. 

\item 找出一个连通开集$G \subset \mathbb{C}$和$G$内的两个连续函数$f$和$g$,使得$f(z)^2 = g(z)^2 = 1 - z^2$对于$G$内的所有$z$都成立。你能使$G$最大吗?$f$和$g$是解析的吗? 

\item 给出$\sqrt{1-z}$的主分支。

\item 设$f$和$g$分别是$z^a$和$z^b$的分支。证明:$fg$是$z^{a+b}$的分支,$f/g$是$z^{a-b}$的分支。设$f(G) \subset G$, $g(G) \subset G$, 证明:$f \circ g$和$g \circ f$都是$z^{ab}$的分支。

\item 设$G$是一个域,定义$G^* - \{z: \bar{z} \in G\}$. 如果$f: G \to \mathbb{C}$是解析的,证明:$f^*: G^* \to \mathbb{C}$, $f^*(z) = \overline{f(\bar{z})}$也是解析的。

\item 设$z_1$, $z_2$, $\cdots$, $z_n$是复数,并且对于$1 \le k \le n$, $\Re{z_k} > 0$, $\Re{(z_1\cdots{}z_n)} > 0$. 证明$\log{(z_1\cdots{}z_n)} = \log{z_1} + \cdots + \log{z_n}$, 其中$\log{z}$是对数函数的主分支。如果关于$z_k$的限制取消,这个公式仍成立吗?

\item\label{exer002030221}证明:不存在定义于$G = \mathbb{C} - \{0\}$的对数分支。(提示:假如这样的分支存在,将它与主分支比较。)
\end{problemset}


\section{作为映照得解析函数. M\"{o}bius变换}\label{section0020303}
考虑函数$f(z)=z^2$. 如果$z = x+iy$, $\mu + i\nu = f(z)$,那么$\mu = x^2 - y^2$,$\nu = 2xy$. 因此双曲线$x^2 - y^2= c$和$2xy = d$由$f$映为直线$\mu = c$和$\nu = d$. 一个有趣的事实是:当$c$和$d$不为0时,这些双曲线正如它们的像一样是正交的。这个事实并不是孤立的现象。在这一节稍后,我们将对这个性质作一般性的探讨。

现在考察直线$x=c$和$y=d$变成什么。先考察$x=c$($y$任意);$f$把这条直线映为$\mu = c^2 - y^2$, $\nu = 2cy$。消去$y$,我们得到$x=c$被映为抛物线$\nu^2 = -4c^2(\mu - c^2)$. 类似地,$f$把$y=d$映为抛物线$\nu^2 = 4d(\mu + d^2)$. 这些抛物线在$(c^2 - d^2, \pm{}2|cd|)$相交。应当指出,当$c \to 0$时,抛物线$\nu^2 = -4c^2(\mu - c^2)$逐渐合拢,而接近于负实轴。这与函数$z^{\frac{1}{2}}$把$G = \mathbb{C} - \{z:z \le 0\}$映为$\{z:\Re{z}>0\}$这一事实相吻合。还要注意,$x=c$和$x=-c$($y=d$和$y=-d$)被映为同一抛物线。

中心在原点的圆周变成什么呢?若$z = re^{i\theta}$,那么$f(z) = r^2e^{2i\theta}$。于是以原点为中心,半径为$r$的圆周被映为半径为$r^2$的圆周,两个点映为同一点\footnote{$(r, \theta)$和$(r, \theta+\pi)$映为同一个点}。

最后,扇形$S(\alpha, \beta)=\{z:\alpha < \arg{z} < \beta\}$, $\alpha < \beta$变为什么?容易看出$S(\alpha, \beta)$的像是扇形$S(2\alpha, 2\beta)$, 当$\beta-\alpha < \pi$时,$f$限制在$S(\alpha, \beta)$上是一一的。

上面的讨论阐明了$f(z)=z^2$的性质,而且对于研究其他解析函数的映照性质也是有用的。在解析函数的理论中,下述问题占有重要的地位:给定两个连通集$G$和$\Omega$,是否存在定义在$G$内的解析函数,使得$f(G) = \Omega$? 除了本身的兴趣外,这个问题的解(或者宁可说是关于解的存在性问题)是很有用的。

\begin{definition}{}{def002030301}
域$G \subset \mathbb{C}$内的一条路径是一个连续函数$\gamma: [a,b] \to G$, $[a, b]$是$\mathbb{R}$中的某一区间。如果对于$[a, b]$内的每一个$t$,$\gamma'(t)$存在,并且$\gamma':[a, b] \to \mathbb{C} $是连续的,那么$\gamma$是光滑路径。$\gamma$是分段光滑的,如果存在$[a, b]$的一个分割,$a = t_0 < t_1 < \cdots < t_n=b$, 使得$\gamma$在每个子区间$[t_{j-1}, t_j]$上,$1 \le j \le n$,是光滑的。
\end{definition}

称函数$\gamma:[a, b] \to \mathbb{C}$对于$[a, b]$内的每个$t$都有导数$\gamma'(t)$, 意思是指:对于$a < t < b$,
\[
\lim_{h \to 0}{\frac{\gamma{(t+h)} - \gamma{(t)}}{h}} = \gamma'(t)
\]
存在;对于$t=a$和$t=b$分别存在右极限和左极限。当然,这等价于说$\Re{(\gamma)}$和$\Im{(\gamma)}$有导数。

设$\gamma:[a,b] \to G$是光滑路径,并且对于$(a, b)$内的某一$t_0$,$\gamma'(t_0) \neq 0$,那么$\gamma$在点$z_0 = \gamma(t_0)$有切线。这条切线通过点$z_0$,方向是(向量)$\gamma'(t_0)$的方向,或者说这条线的斜率是$\tan{\arg{\gamma'(t_0)}}$. 如果$\gamma_1$和$\gamma_2$是两条光滑路径,$\gamma_1(t_1) = \gamma_2(t_2)=z_0$,$\gamma_1'(t_1) \neq 0$, $\gamma_2'(t_2) \neq 0$, 那么路径$\gamma_1$和$\gamma_2$在$z_0$的夹角定义为
\[
\arg{\gamma_2'(t_2)} - \arg{\gamma_1'(t_1)}.
\]

设$\gamma$是$G$内的光滑路径, $f:G \to \mathbb{C}$是解析的,那么$\sigma = f \circ \gamma$也是一条光滑路径,并且$\sigma'(t) = f'(\gamma(t))\gamma'(t)$. 设$z_0 = \gamma(t_0)$, 假定$\gamma'(t_0) \neq 0$, $f'(z_0) \neq 0$, 那么$\sigma'(t_0) \neq 0$, 并且$\arg{\sigma'(t_0)} = \arg{f'(z_0)} + \arg{\gamma'(t_0)}$, 即
\begin{gather}\label{equ002030302}
\arg{\sigma'(t_0)} - \arg{\gamma'(t_0)} = \arg{f'(z_0)}.
\end{gather}

现在设$\gamma_1$, $\gamma_2$是光滑路径,$\gamma_1(t_1) = \gamma_2(t_2) = z_0$, 并且$\gamma_1'(t_1) \neq 0 \neq \gamma_2'(t_2)$. 设$\sigma_1 = f \circ \gamma_1$, $\sigma_2 = f \circ \gamma_2$. 还假定路径$\gamma_1$和$\gamma_2$在$z_0$彼此不相切,即假定$\arg{\gamma_1(t_1)} \neq \arg{\gamma_2(t_2)}$. 由等式(\ref{equ002030302})得到
\begin{gather}\label{equ002030303}
\arg{\gamma_2'(t_2)} - \arg{\gamma_1'(t_1)} = \arg{\sigma_2'(t_2)} - \arg{\sigma_1'(t_1)}.
\end{gather}
这就是说,任意给定过$z_0$的两条路径,$f$把这两条路径映过$w_0 = f(z_0)$的两条路径,当$f'(z) \neq 0$时,曲线的夹角的大小和方向都是保持不变的。综上所述有下面的定理。
\begin{theorem}{}{thm002030304}
如果$f:G \to \mathbb{C}$是解析的,那么在$f'(z_0) \neq 0$的每一点$z_0$,$f$保持角度不变。
\end{theorem}

函数$f: G \to \mathbb{C}$保持角度不变,并且
\[
\lim_{z \to a}{\frac{|f(z)-f(a)|}{|z-a|}}
\]
也存在,这种函数称为共形映照。如果$f$是解析的,且对于任意$z$,$f'(z) \neq 0$,则$f$是共形的,反之亦然。

如果$f(z) = e^z$,那么$f$在整个$\mathbb{C}$中是共形的。让我们进一步察看一下指数函数。若$z = c + iy$, $c$是固定的。那么$f(z) = re^{iy}$, $r=e^c$, 即$f$把直线$x=c$映为中心在原点,半径为$e^c$的圆周,又,$f$把直线$y=d$映为无穷射线$\{re^{id}: 0 < r < \infty\}$.

我们已经看到,在任何宽度$<2\pi$的水平带形上,$e^z$是一一的,设$G = \{z:-\pi < \Im{z} < \pi\}$, 那么$f(G) = \Omega = \mathbb{C} - \{z:z \le 0\}$. $f$也把垂直线段$\{z = c+iy, -\pi < y < \pi\}$映为部分圆周$\{e^ce^{i\theta}, -\pi < \theta < \pi\}$. 把水平直线$y = d$, $-\pi < d < \pi$,映为和正实轴的夹角等于$d$的射线。

注意,对数的主分支$\log{z}$则反过来,它把$\Omega$映为带形$G$,把圆周映为$G$内的垂直线段,把射线映为$G$内的水平直线。

$\cos{z}$, $\sin{z}$以及其他解析函数的映照性质的研究将在习题中进行。现在着手研究一类奇特的映照---M\"obius映照。
\begin{definition}{}{def002030305}
形如$S(z) = \frac{az+b}{cz+d}$的映照称为分式线性变换。如果$a$,$b$,$c$,$d$满足$ad-bc \neq 0$,那么$S(z)$称为M\"obius(麦比乌斯)变换。
\end{definition}

如果$S$是M\"obius变换,那么$S^{-1} = \frac{dz - b}{-cz+a}$满足$S(S^{-1}(z)) = S^{-1}(S(z)) = z$,即$S^{-1}$是$S$的逆映照,如果$S$和$T$都是分式线性变换,那么$S \circ T$也是分式线性变换,因此,M\"obius变换的集在复合变换下构成一个群。如果不作别的声明,我们考虑的分式线性变换总是M\"obius变换。

设$S(x) = \frac{az+b}{cz+d}$, 如果$\lambda$是任意非零复数,那么
\[
S(z) = \frac{(\lambda{}a)z + (\lambda{}b)}{(\lambda{}c)z + (\lambda{}d)}.
\]
即系数$a,b,c,d$不是唯一的(见习题\ref{exer002030320})。

我们也可以把$S$看作是定义在$\mathbb{C}_{\infty}$上的,它满足$S(\infty) = a/c$, $S(-d/c) = \infty$. (注意,我们不可能有$a = 0 =c$,或$d = 0 = c$,因为在这两种情形下都和$ad - bc \neq 0$相矛盾)因为$S$有逆变换,所以$S$把$\mathbb{C}_{\infty}$映为$\cinfty$。

如果$S(z) = z+a$,那么$S$称为平移;如果$S(z)=az$,$a \neq 0$, 那么$S$是一伸缩;如果$S(z)=e^{i\theta}z$\footnote{奇怪,从这个定义来看,前面的$az$中的$a$应该是实数才是合理的。},那么它是一个旋转。最后,如果$S(z) = 1/z$,那么它是一个反演。

\begin{proposition}{}{prop002030306}
如果$S$是一个M\"obius变换,那么$S$是平移,伸缩\footnote{从这个命题看,旋转是包含在伸缩里面的。},反演的复合。(当然,其中有的可能不出现。)
\end{proposition}

\begin{proof}
首先,设$c = 0$,因此$S(z) = (a/d)z + (b/d)$, 所以如果$S_1(z)=(a/d)z$, $S_2(z) = z + (b/d)$, 那么$S_2 \circ S_1 = S$, 命题得证。

现在设$c \neq 0$, 那么令$S_1(z)=z + d/c$, $S_2(z) = 1/z$, $S_3(z) = \frac{bc - ad}{c^2}z$, $S_4(z) = z + a/c$, 那么$S = S_4 \circ S_3 \circ S_2 \circ S_1$.
\end{proof}

$S$的不动点是什么呢?即哪些点满足$S(z)=z$. 如果$z$满足这个条件,那么
\[
cz^2 + (d-a)z - b = 0.
\]
因此,一个M\"obius变换至多有两个不动点。除非$S(z)=z$对所有的$z$成立。


现在设$S$是一个M\"obius变换,$a,b,c$是$\cinfty$中的不同点。$\alpha=S(a)$, $\beta = S(b)$, $\gamma = S(c)$. 假定$T$是另一个具有这种性质的变换。那么$T^{-1}\circ S$以$a, b, c$作为不动点,所以$T^{-1}\circ S = I =$恒同变换,即$S=T$. 因此一个M\"obius变换由$\cinfty$中的任意给定的三个点唯一确定。

设$z_2, z_3, z_4$是$\cinfty$中的点。定义$S: \cinfty \to \cinfty$为:
\[
\begin{aligned}
S(z) &= (\frac{z - z_3}{z-z_4})/(\frac{z_2-z_3}{z_2-z_4}), \quad z_2,z_3, z_4 \in \mathbb{C};\\
S(z) &= \frac{z- z_3}{z-z_4},\quad z_2 = \infty;\\
S(z) &= \frac{z_2- z_4}{z-z_4},\quad z_3 = \infty;\\
S(z) &= \frac{z- z_3}{z_2-z_3},\quad z_4 = \infty.
\end{aligned}
\]

在任意情况下,$S(z_2) = 1$, $S(z_3) = 0$, $S(z_4) = \infty$.并且$S$是具有这种性质的唯一变换。
\begin{definition}{交比}{def002030307}
如果$z_1 \in \cinfty$,那么$(z_1,z_2,z_3,z_4)$($z_1,z_2,z_3,z_4$的交比)是使$z_2$变为1,$z_3$变为0,$z_4$变为$\infty$这一唯一的M\"obius变换下$z_1$的像。
\end{definition}

例如, $(z_2,z_2, z_3, z_4)=1$,$(z, 1, 0, \infty)=z$. 如果$M$是任意M\"obius映照,$w_2,w_3,w_4$是使得$Mw_2=1$, $Mw_3=0$, $Mw_4=\infty$的点,那么$Mz=(z,w_2,w_3,w_4)$.

\begin{proposition}{}{prop002030308}
如果$z_2,z_3,z_4$是不同的点,$T$是任意M\"obius变换,那么对于任意点$z_1$,
\[
(z_1,z_2,z_3,z_4) = (Tz_1, Tz_2, Tz_3, Tz_4).
\]
\end{proposition}

\begin{proof}
设$S(z) = (z, z_2, z_3, z_4)$, 那么$S$是一个M\"obius映照。如果$M = ST^{-1}$, 那么$M(Tz_2) = 1$, $M(Tz_3) = 0$, $M(Tz_4) = \infty$,因此$ST^{-1} = (z, Tz_2, Tz_3, Tz_4)$对于$\cinfty$中的所有$z$成立。特别地,令$z = Tz_1$便得到所要求的结果。
\end{proof}

\begin{proposition}{}{prop002030309}
如果$z_2, z_3, z_4$是$\cinfty$中的不同点, $\omega_2, \omega_3, \omega_4$也是$\cinfty$中的不同点,那么有且只有一个M\"obius变换,使得$Sz_2=\omega_2$, $Sz_3 = \omega_3$, $Sz_4 = \omega_4$.
\end{proposition}

\begin{proof}
设$Tz = (z, z_2, z_3, z_4)$, $Mz = (z, \omega_2, \omega_3, \omega_4)$, $S = M^{-1}T$. 显然$S$具有所要求的性质。如果$R$是另一个M\"obius变换,使得$Rz_j=\omega_j$, $j=2,3,4$, 那么$R^{-1}\circ S$有三个不动点($z_2, z_3$和$z_4$),因此$R^{-1} \circ S = I$,或者说$S = R$.
\end{proof}

从中学几何中已经熟知,平面上的三点决定一个圆周。(注意$\cinfty$中过$\infty$的圆周相应于$\mathbb{C}$中的直线,因此,在上面的叙述中不需要预先声明三点“不共线”,平面上的直线将称为圆周。)下述结果说明在什么时候四点位于一个圆周上。

\begin{proposition}{}{prop002030310}
设$z_1,z_2,z_3,z_4$是$\cinfty$中的四个不同点,那么$(z_1,z_2,z_3,z_4)$是实数,当且仅当,这四个点位于一个圆周上。
\end{proposition}

\begin{proof}
设$S:\cinfty \to \cinfty$定义为$Sz = (z, z_2, z_3, z_4)$, 那么$S^{-1}(\mathbb{R})=$使得$(z,z_2,z_3,z_4)$是实数的$z$的集。因此,如果我们能够在M\:obius变换下,$\mathbb{R}_{\infty}$的像是一个圆周,命题就得证。

设$Sz = \frac{az+b}{cz+d}$, 如果$z=x \in \mathbb{R}$, $\omega = S^{-1}(x)$, 那么$x = S\omega$九蕴含$S(\omega) = \overline{S(\omega)}$, 即
\[
\frac{a\omega + b}{c\omega+d} = \frac{\overline{a\omega} + \bar{b}}{\overline{c\omega} + \bar{d}}
\]
交叉相乘得到
\begin{gather}\label{equ002030311}
\begin{aligned}
&(a\bar{c} - \bar{a}c)|\omega|^2 + (a\bar{d}-\bar{d}c)\omega\\
&+(b\bar{c} - d\bar{a})\bar{\omega} + (b\bar{d} - \bar{b}d) = 0.
\end{aligned}
\end{gather}
如果$a\bar{c}$是实的,那么$a\bar{c} - \bar{a}c = 0$, 令$\alpha = 2(a\bar{d} - \bar{b}c)$, $\beta = i(b\bar{d}-\bar{b}d)$. 用$i$乘(\ref{equ002030311}),因为$\beta$是实的,所以得到
\begin{gather}\label{equ002030312}
0 = \Im{(\alpha\omega)} - \beta = \Im{(\alpha\omega - \beta)}.
\end{gather}
即$\omega$位于由(\ref{equ002030312})决定的直线上,$\alpha$, $\beta$是固定的。如果$a\bar{c}$不是实的,那么(\ref{equ002030311})变成
\[
|\omega|^2 + \bar{\gamma}\omega + \gamma\bar{\omega} - \delta = 0,
\]
其中$\gamma$是某一个复常数,$\delta$是某一个实常数。因此
\begin{gather}\label{equ002030313}
|\omega + \gamma| = \lambda,
\end{gather}
其中
\[
\lambda = (|\gamma|^2 + \delta)^{\frac{1}{2}} = |\frac{ad-bc}{\bar{a}c - a\bar{c}}| > 0.
\]
因为$\gamma$和$\lambda$不依赖于$x$, (\ref{equ002030313})是圆周的方程,所以命题得证。
\end{proof}

\begin{theorem}{}{thm002030314}
M\"obius变换把圆周变为圆周。
\end{theorem}

\begin{proof}
设$\Gamma$是$\cinfty$中的任意的一个圆周,$S$是任意一个M\"obius变换。$z_2$, $z_3$, $z_4$是$\Gamma$上的三个不同的点,记$\omega_j = Sz_j$, $j=2,3,4$, 那么$\omega_2, \omega_3, \omega_4$决定一圆周$\Gamma'$。我们断言$S(\Gamma) = \Gamma'$.事实上,对于$\cinfty$中的任意一点$z$,由命题\ref{pro:prop002030308}得到
\begin{gather}\label{equ002030315}
(z, z_2, z_3, z_4) = (Sz, \omega_2, \omega_3, \omega_4).
\end{gather}
根据上面的命题,如果$z$在$\Gamma$上,那么(\ref{equ002030315})的两边是实的,但这就是说$Sz \in \Gamma'$.
\end{proof}

现在设$\Gamma$和$\Gamma'$是$\cinfty$中的两个圆周。$z_2,z_3,z_4 \in \Gamma$, $\omega_2, \omega_3, \omega_4 \in \Gamma'$. 令$Rz = (z, z_2, z_3, z_4)$, $Sz=(z, \omega_2, \omega_3, \omega_4)$, 那么$T = S^{-1}\circ R$把$\Gamma$映为$\Gamma'$。事实上,$Tz_j = \omega_j$, $j=2,3,4$,和上面的证明一样,推出$T(\Gamma) = \Gamma'$.

\begin{proposition}{}{prop002030316}
对于$\cinfty$中任意给定的圆周$\Gamma$和$\Gamma'$,存在一个M\"obius变换$T$,使得$T(\Gamma) = \Gamma'$. 并且我们能够指定$\Gamma$上的任意三点变为$\Gamma'$上的任意三点。如果我们指定$Tz_j$, $j=2,3,4$($\Gamma$上的$z_j$互不相同),那么$T$是唯一的。
\end{proposition}

\begin{proof}
除了唯一性外,其余的已在上段中给出了。唯一性的证明对于读者来说是一个简单的习题。
\end{proof}

现在我们知道M\"obius变换把圆周变为圆周。下一个问题是这些圆周的内部和外部变为什么?为了回答这个问题,我们引进一些新的概念。

\begin{definition}{}{def002030317}
设$\Gamma$是通过$z_2, z_3, z_4$的圆周,$\cinfty$中的点$z$和$z^*$称为关于$\Gamma$是对称的,如果
\begin{gather}\label{equ002030318}
(z^*, z_2, z_3, z_4) = \overline{(z, z_2, z_3, z_4)}.
\end{gather}
\end{definition}
这个式子表示,这个定义不仅依赖于圆周,而且也依赖于圆周上的点$z_2, z_3, z_4$,作为一个习题(习题\ref{exer002030311}),留给读者去证明对称性和这些点的选取无关。

由命题\ref{pro:prop002030310}, $z$关于圆周$\Gamma$与自己对称,当且仅当,$z \in \Gamma$.

让我们来研究$z$和$z^*$是对称的意味着什么。设$\Gamma$是一条直线,那么顾名思义,我们相信$z$和$z^*$关于$\Gamma$是对称的,如果通过$z$和$z^*$的直线垂直于$\Gamma$,并且$z$和$z^*$在$\Gamma$的两侧,到$\Gamma$的有相同的距离。事实正是如此。

设$\Gamma$是一直线。取$z_4 = \infty$。等式(\ref{equ002030318})变成
\[
\frac{z^* - z_3}{z_2-z_3} = \frac{\bar{z}-\bar{z_3}}{\bar{z_2} - \bar{z_3}}.
\]
这就得到$|z^* - z_3| = |z - z_3|$;因为$z_3$是不确定的,所以$z$, $z^*$到$\Gamma$上每一点的距离相同。又
\[
\Im{\frac{z^*-z_3}{z_2-z_3}} = \Im{\frac{\bar{z}-\bar{z_3}}{\bar{z_2} - \bar{z_3}}} = -\Im{\frac{z-z_3}{z_2-z_3}}.
\]
因此,我们得到(除非$z \in \Gamma$),$z$, $z^*$位于$\Gamma$确定的不同的半平面内。由此推出,$[z, z^*]$垂直于$\Gamma$.

现在设$\Gamma = \{z:|z - a|=R\}$($0 < R < \infty$),$z_2,z_3, z_4$在$\Gamma$上。根据(\ref{equ002030318}),并对若干M\"obius变换多次应用命题\ref{pro:prop002030308},得到
\[
\begin{aligned}
(z^*, z_2, z_3, z_4) &= \overline{(z, z_2, z_3, z_4)}\\
&= \overline{(z-a, z_2-a, z_3-a, z_4-a)} \\
&= (\bar{z}- \bar{a}, \frac{R^2}{z_2 - a}, \frac{R^2}{z_3 - a}, \frac{R^2}{z_4 - a})\\
&= (\frac{R^2}{\bar{z} - \bar{a}}, z_2 - a, z_3 - a, z_4 - a)\\
&= (\frac{R^2}{\bar{z}-\bar{a}} + a, z_2, z_3, z_4). 
\end{aligned}\footnote{第一个等式是对称的定义,第二个等式使用变换$Tz = z-a$,第三个等式是把共轭作用于每一项,这个是成立的,第四个等式是使用变换$Tz = R^2/z$,第五个等式是使用变换$Tz = z+a$。}
\]
因此。$z^* = a + R^2(\bar{z} - \bar{a})^{-1}$, 或者$(z^* - a)(\bar{z} - \bar{a}) = R^2$. 由此得到
\[
\frac{z^*-a}{z-a} = \frac{R^2}{|z-a|^2} > 0;
\]
所以$z^*$位于从$a$出发,通过$z$的射线$\{a + t(z-a): 0 < t < \infty\}$上。利用$|z-a||z-a| = R^2$这一事实,如下图所示,我们能够由$z$得到$z^*$(若$z$位于$\Gamma$内部)。设$L$是从$a$出发,通过$z$的射线,过$z$点作一条直线$P$垂直于$L$, 在$P$和$\Gamma$的交点处作$\Gamma$的切线,这条切线和$L$的交点就是$z^*$. 点$a$和$\infty$关于$\Gamma$是对称的。

\begin{theorem}{对称原理}{thm002030319}
如果M\"obius变换$T$把圆周$\Gamma_1$变为圆周$\Gamma_2$,那么关于$\Gamma_1$的一对对称点被$T$映为关于$\Gamma_2$的一对对称点。
\end{theorem}

\begin{proof}
设$z_2, z_3, z_4 \in \Gamma_1$, 如果$z, z^*$关于$\Gamma_1$是对称的,那么由命题\ref{pro:prop002030308}得到
\[
\begin{aligned}
(Tz^*, Tz_2, Tz_3, Tz_4) &= (z^*, z_2, z_3, z_4)\\
&= \overline{(z, z_2, z_3, z_4)}\\
&=\overline{(Tz, Tz_2, Tz_3, Tz_4)}.
\end{aligned}
\]
因此$Tz^*$和$Tz$关于$\Gamma_2$是对称的。
\end{proof}

现在我们讨论在$\cinfty$中圆周的定向。这将能够使我们去区分$\cinfty$中圆周的“内部”和“外部”。注意,在$\cinfty$(球面)上,对于圆周的内部和外部没有明显的选择。

\begin{definition}{}{def002030320}
如果$\Gamma$是一个圆周,那么$\Gamma$的定向是$\Gamma$上的有序的三个点$(z_1, z_2, z_3)$。
\end{definition}

直观上,这三个点给了$\Gamma$的一个方向。这就是我们从$z_1$“走”到$z_2$再到$z_3$,如果只给定两个点,当然意思就含糊了。

设$\Gamma = \mathbb{R}$, $z_1, z_2, z_3 \in \mathbb{R}$,又设$Tz = (z, z_1, z_2, z_3) = \frac{az+b}{cz+d}$. 由于$T(\mathbb{R}_{\infty}) = \mathbb{R}_{\infty}$,由此推出,$a,b,c,d$可以取为实数(见习题\ref{exer002030308}),因此
\[
\begin{aligned}
Tz &= \frac{az+b}{cz+d} = \frac{az+b}{|cz+d|^2}(c\bar{z} = d)\\
&=\frac{1}{|cz+d|^2}[ac|z|^2 + bd + bc\bar{z} + adz].
\end{aligned}
\]
所以
\[
\Im{(z, z_1, z_2, z_3)} = \frac{(ad-bc)}{|cz+d|^2}\Im{z}.
\]
于是,$\{z:\Im{(z, z_1, z_2, z_3)<0}\}$或者是上半平面,或者是下半平面,取决于$(ad-bc)<0$还是$(ad-bc)>0$.(注意,$ad-bc$是$T$的“行列式”。)

现在设$\Gamma$是任意的,$z_1, z_2, z_3$在$\Gamma$上。对于任意的M\"obius变换$S$,我们有(根据命题\ref{pro:prop002030308})
\[
\begin{aligned}
\{z:\Im{(z, z_1, z_2, z_3)} > 0\} &= \{z:\Im{(Sz, Sz_1, Sz_2, Sz_3)} > 0\}\\
&= S^{-1}\{z:\Im{(z, Sz_1, Sz_2, Sz_3)} > 0\}.
\end{aligned}
\]
特别地,如果选取$S$,使得$S$把$\Gamma$映为$\mathbb{R}_{\infty}$,那么$\{z: \Im{(z, z_1,z_2, z_3)} > 0\}$等于上半平面或下半平面在$S^{-1}$下的像。

如果$(z_1, z_2, z_3)$是$\Gamma$的定向,那么我们定义$\Gamma$(关于$(z_1, z_2, z_3)$)的右边为
\[
\{z:\Im{(z,z_1, z_2, z_3)} > 0\}.
\]
类似地,我们定义$\Gamma$的左边为
\[
\{z:\Im{(z,z_1, z_2, z_3)} < 0\}.
\]

下述定理的证明留给读者作为习题。
\begin{theorem}{定向原理}{thm002030321}
设$\Gamma_1$和$\Gamma_2$是$\cinfty$中的两个圆周,$T$是M\"obius变换,$T(\Gamma_1) = \Gamma_2$, $(z_1, z_2, z_3)$是$\Gamma_1$的定向,那么$T$把$\Gamma_1$的右边和左边变为$\Gamma_2$关于定向$(Tz_1, Tz_2, Tz_3)$的右边和左边。
\end{theorem}

考虑$\mathbb{R}$的定向$(1, 0, \infty)$。由交比的定义,$(z, 1, 0, \infty) = z$。因此,$\mathbb{R}$关于$(1,0,\infty)$的右边是上半平面。这和我们的直观是一致的“当我们沿$\mathbb{R}$从$1$走到$0$,再到$\infty$,上半平面在我们的右边。

作为例子,考虑下述问题:找出一个解析函数$f: G \to \mathbb{C}$,$G = \{z:\Re{z} > 0\}$,使得$f(G) = D = \{z:|z|<1\}$. 我们是通过寻找一个M\"obius变换来解决这个问题的。这个变换把虚轴变为单位圆周。由定向原理,它把$G$变为$D$(即我们必须细心选取这个映照,使得它不是把$D$变为$\{z:|z|>1\}$)。

如果我们给定虚轴的定向为$(-i, 0, i)$,那么$\{z:\Re{z} > 0\}$是虚轴的右边。事实上,
\[
\begin{aligned}
(z, -i, 0, i) &= \frac{2z}{z-i} = \frac{2z}{z-i} \cdot \frac{\bar{z}+i}{\bar{z}+i}\\
&=\frac{2}{|z-i|^2}(|z|^2 + iz).
\end{aligned}
\]
因此, $\{z: \Im{(z, -i, 0, i)} > 0\} = \{z: \Im{iz} > 0\} = \{z:\Re{z} > 0\}$.给$\Gamma$以定向$(-i, -1, i)$,我们得到$D$位于$\Gamma$的右边。又
\[
(z, -i, -1, i) = \frac{2i}{i-1} \cdot \frac{z+1}{z-i}.
\]
如果
\[
Sz = \frac{2z}{z-i}, Rz = (\frac{2i}{i-1})(\frac{z+1}{z-i}),
\]
那么$T = R^{-1}S$把$G$映为$D$(并把虚轴映为$\Gamma$)。由代数运算,我们有
\[
Tz = \frac{z-1}{z+1}.
\]

把这个结果和前面的结果结合起来,我们得到, $g(z) = \frac{e^z-1}{e^z+1}$把无穷带形$\{z:|\Im{z}|<\pi/2\}$映为开单位圆$D$。(值得一提的是$\frac{e^z-1}{e^z+1} = \tanh(\frac{z}{2})$)。

设$G_1, G_2$是连通开集,为了找出一个解析函数$f$,使得$f(G_1)=G_2$,我们试图把$G_1, G_2$都映为开单位圆。如果这一点能够办到,$f$就能由一个函数和另一个函数的反函数的符合而得到。

作为一个例子,设$G$是两个相交于$a, b$($a \neq b$)的圆周的内部开集,$L$是过$a, b$的直线,其定向是$(\infty, a, b)$。那么$Tz = (z, \infty, a, b) = \frac{z-a}{z-b}$把$L$映为实轴$(T\infty = 1,Ta=0, Tb=\infty)$.由于$T$把圆周映为圆周,所以$T$把$\Gamma_1$和$\Gamma_2$映为过$0$和$\infty$的圆周,即$T(\Gamma_1)$, $T(\Gamma_2)$是直线。利用定向,我们有$T(G) = \{\theta_0 - \alpha < \arg{\omega} < \theta_0\}$\footnote{原书误为$T(G) =\{\omega-\alpha < \arg{\omega} < \alpha\}$, $\alpha>0$.---译注}, $\alpha>0$,或者是某一这种闭扇形的余集。利用幂函数,或许还有一个旋转,我们可以把这个楔形映为右半平面。现在与$\frac{z-1}{z+1}$复合起来,就得到$G$到$D = \{z:|z|<1\}$的映照。

\begin{problemset}
\item 求$\{z:\Re{z} < 0, |\Im{z}| < \pi\}$在指数函数下的像。

\item 对于集$\{z:|\Im{z}|< \pi/2\}$做上一习题

\item 讨论$\cos{z}$和$\sin{z}$的映照性质。

\item 讨论$z^n$和$z^{1/n}$, $n>2$的映照性质。(提示:利用极坐标。)

\item 求出$\cinfty$中伸缩、平移、反演的不动点。

\item 计算下列交比:(a) $(7+i,1,0,\infty)$; (b)$(2, 1-i, 1, 1+i)$; (c)$(0, 1, i, -1)$; (d)$(i-1, \infty, 1+i, 0)$.

\item 如果$Tz = \frac{az+b}{cz+d}$,求出$z_2,z_3, z_4$(用$a,b,c,d$表示),使得$Tz = (z, z_2, z_3, z_4)$.

\item\label{exer002030308}如果$Tz = \frac{az+b}{cz+d}$, 证明:当且仅当我们可以选取$a,b,c,d$为实数时$T(\mathbb{R}) = \mathbb{R}$.

\item 如果$Tz = \frac{az+b}{cz+d}$,求出$T(\Gamma) = \Gamma$的充分必要条件,其中$\Gamma$是单位圆周$\{z:|z|=1\}$.

\item 设$D = \{z:|z|<1\}$, 求出所有满足$T(D)=D$的M\"obius变换。

\item\label{exer002030311}证明:对称的定义(\ref{def:def002030317})不依赖于$z_2,z_3,z_4$的选取,即证明:如果$\omega_2, \omega_2, \omega_4$也在$\Gamma$上,那么它们满足等式(\ref{equ002030318}),当且仅当$(z^*, \omega_2, \omega_3, \omega_4) = \overline{(z, \omega_2, \omega_3, \omega_4)}$.(提示:利用习题\ref{exer002030308})

\item 证明定理\ref{thm:thm002030304}。

\item 讨论映照$f(z) = \frac{1}{2}(z + \frac{1}{z})$.

\item 设一个圆包含在另一个圆内,并在一点$a$相切,$G$是这两个圆周之间的域,将$G$共形映照为开单位圆。(提示:先用$(z-a)^{-1}$).

\item 设$G = \{z: 0 < |z| < 1\}$. 能否将$G$共形映照为开单位圆?

\item 试用解析函数$f$把$G = \mathbb{C} - \{z:-1  \le z \le 1\}$映为开单位圆,$f$能是一一的吗?

\item 设$G$是一个域,$f: G \to \mathbb{C}$是解析的,$f(G)$是圆周的一子集,证明:$f$是一常数。

\item 设$-\infty < a < b < \infty$. $Mz = \frac{z-ia}{z-ib}$. 定义直线$L_1 = \{z:\Im{z}=b\}$, $L_2 = \{z:\Im{z}=a\}$, $L_3=\{z:\Re{z}=0\}$.确定图中的$A, B, C, D, E, F$中的哪一个能被$M$映为图中的$U,V,W,X,Y,Z$.

\item 设$a,b,M$如上题所述,$\log{z}$是对数的主分支。
\begin{enumerate}
\item[(a)]证明:$\log{(Mz)}$对于所有的$z$,除去$z=ic$,$a<c<b$外有定义。并且如果$h(z)=\Im{[\log{Mz}]}$, 那么对于$\Re{z} > 0$, 有$0 < h(z) < \pi$.
\item[(b)]证明:对于$\Re{z} > 0$及任意实数$c$, $\log{(z-ic)}$有定义,并证明:如果$\Re{z}>0$,有$|\Im{\log{(z-ic)}}|<\pi/2$. 
\item[(c)]设$h$如(a)中所述,证明:$h(z)=\Im{[\log{z-ia} - \log{(z-ib)}]}$.
\item[(d)]证明:
\[
\int_{a}^{b}{\frac{dt}{z-it}} = i[\log{(z-ib)} - \log{(z-ia)}].
\]
(提示:利用微积分学的基本定理。)
\item[(e)]结合(c)和(d),得到
\[
h(x+iy) = \int_{a}^{b}{\frac{dt}{x^2 + (y-t)^2}} = \arctan{(\frac{y-a}{x})} - \arctan{(\frac{y-b}{x})}.
\]
\item[(f)]解释(e)的几何意义,并证明:当$\Re{z}>0$时,$h(z)$是图中所画的角度。
\end{enumerate}

\item\label{exer002030320}设$Sz = \frac{az+b}{cz+d}$, $Tz = \frac{\alpha{}z + \beta}{\gamma{}z + \delta}$, 证明:当且仅当存在非零复数$\lambda$,使得$\alpha = \lambda{}a$, $\beta = \lambda{}b$, $\gamma = \lambda{}c$, $\delta = \lambda{}d$时$S = T$.

\item\label{exer002030321} 设$T$是具有不定点$z_1$和$z_2$的M\"obius变换。如果$S$是一个M\"obius变换,证明:$S^{-1}TS$有不动点$S^{-1}z_1$和$S^{-1}z_2$.

\item\label{exer002030322} (a)证明:一个M\"obius变换,当且仅当它是一个伸缩时,仅以$0$和$\infty$为它的不动点。

(b)证明:一个M\"obius变换,当且仅当它是一个平移时,仅以$\infty$为它的不动点。

\item 证明:一个M\"obius变换$T$,当且仅当, $Tz = a/z$, $a \in \mathbb{C}$时,满足$T(0) = \infty$, $T(\infty) = 0$。

\item 设$T$是一个M\"obius变换,且不失恒同变换。证明:一个M\"obius变换$S$, 当且仅当$S$和$T$有相同的不动点时,满足$ST=TS$。(提示:利用习题\ref{exer002030321}和\ref{exer002030322})

\item 求出M\"obius变换群的所有Abel子群。

\item (a)设$GL_2(\mathbb{C})=$所有的$2 \times 2$可逆复数矩阵。$\mathscr{M}$是M\"obius变换群。$\varphi: GL_2(\mathbb{C}) \to \mathscr{U}$定义为
\[
\varphi\begin{pmatrix}a & b \\ c & d\end{pmatrix} = \frac{az+b}{cz+d}.
\]
证明:$\varphi$是$GL_2(\mathbb{C})$到$\mathscr{M}$上的群同态。求出$\varphi$的核。

(b) 设$SL_2(\mathbb{C})$是$GL_2(\mathbb{C})$的子群,它是由行列式为1的所有矩阵所组成的。证明:$SL_2(\mathbb{C})$在$\varphi$下的像是$\mathscr{M}$的全体。$\varphi$的核的哪部分在$SL_2(\mathbb{C})$中?

\item 如果$\mathscr{G}$是一个群,$\mathscr{N}$是一个子群,那么$\mathscr{N}$称为$\mathscr{G}$的正规子群,如果当$T \in \mathscr{N}$,$S \in \mathscr{G}$时$S^{-1}TS \in \mathscr{N}$。如果$\mathscr{G}$的仅有的正规子群是$\{I\}$($I$是$\mathscr{G}$的单位)和$\mathscr{G}$本身,则称$\mathscr{G}$是单纯群。证明:M\"obius变换群$\mathscr{M}$是一个单纯群。

\item 讨论$(1-z)^i$的映照性质。

\item 对于复数$\alpha$和$\beta$,$|\alpha|^2 + |\beta|^2 = 1$,
\[
u_{\alpha,\beta} = \frac{\alpha{}z - \bar{\beta}}{\beta{}z - \bar{\alpha}}, U = \{u_{\alpha,\beta}:|\alpha|^2+|\beta|^2=1\}.
\]
(a)证明:在变换的复合运算下$U$构成一个群。

(b)如果$SU_2$是所有的行列式为1的酉矩阵所构成的集,证明:$SU_2$在矩阵的乘法下是一个群,并且对于$SU_2$中的每一个$A$, 有唯一的复数$\alpha,\beta$,它们满足$|\alpha|^2+|\beta|^2=1$,使得$A = \begin{pmatrix}\alpha & \beta \\ -\bar{\beta} & \bar{\alpha}\end{pmatrix}$.

(c)证明:$\begin{pmatrix}\alpha & \beta \\ -\bar{\beta} & \bar{\alpha}\end{pmatrix} \to u_{\alpha, \beta}$是群$SU_2$到$U$上的同构。

(d)如果$l \in \{0, \frac{1}{2}, 1, \frac{3}{2}, \cdots\}$, 设$H_l=$所有其次数$\le 2l$的多项式。对于$U$中的$u_{\alpha,\beta}$定义$T_u^{(l)}:H_l \to H_l$为$(T_u^{(l)}f)(z) = (\beta{}z+\bar{\alpha})^{2l}f(u(z))$.证明:$T_u^{(l)}$是$H_l$上的一个可逆的线性变换。并且$u \mapsto T_u^{(l)}$是$U$到一个可逆线性变换群里($H_l$到$H_l$上)的一一同态。

\item 对于$|z|<1$定义$f(z)$为
\[
f(z) = \exp\{-i\log{[i(\frac{1+z}{1-z})]^{1/2}}\}.
\]

(a)证明: $f$将$D = \{z:|z|<1\}$共形映照为一个圆环$G$.

(b)求出所有把$D$映为$D$的M\"obius变换$S(z)$,使得当$|z|<1$时,$f(s(z)) = f(z)$.
\end{problemset}


\chapter{复积分}\label{chapter00204}
本章导出的结果在解析函数的研究中是基本的,这里给出的定理是整个数学知识的基础之一,并且具有一系列范围广泛的应用。

\section{Riemann-Stieltjes积分}\label{section0020401}
为了定义函数沿$\mathbb{C}$内一条路径的积分,我们首先来定义Riemann-Stieltjes积分。这种积分的讨论是远不完备的,而仅限于讨论为了有力的解释线积分是必需的那些结果。

\begin{definition}{有界变差函数}{def002040101}
设$[a, b] \subset \mathbb{R}$,称函数$\gamma:[a, b] \to \mathbb{C}$是有界变差函数,如果存在一个常数$M>0$,使得对于$[a,b]$的每一个分割$P=\{a=t_0 < t_1<\cdots<t_m=b\}$, 总有
\[
v(\gamma:P) = \sum_{k=1}^{m}{|\gamma(t_k) - \gamma(t_{k-1})|} \le M.
\]
$\gamma$的全变差$V(\gamma)$定义为
\[
V(\gamma) = \sup\{v(\gamma:P):P\text{是}[a,b]\text{的任一个分割}\}.
\]
显然, $V(\gamma) \le M < \infty$.
\end{definition}

容易看出。当且仅当,$\Re{\gamma}$和$\Im{\gamma}$是有界变差函数时,$\gamma$才是有界变差函数。当$\gamma$是实的,非减函数时,那么$\gamma$是有界变差函数,并且$V(\gamma) = \gamma(b) - \gamma(a)$(习题\ref{exer002040101})。我们还要给出别的例子,但首先给出这种函数的一些容易导出的性质。
\begin{proposition}{}{prop002040102}
设$\gamma:[a, b] \to \mathbb{C}$是有界变差函数,那么:
\begin{enumerate}
\item[(a)]如果$P$和$Q$是$[a,b]$的分割,且$P \subset Q$,则
\[
v(\gamma:P) \le v(\gamma:Q);
\]
\item[(b)]如果$\sigma:[a, b] \to \mathbb{C}$也是有界变差函数,且$\alpha, \beta \in \mathbb{C}$, 则$\alpha\gamma+\beta\sigma$是有界变差函数,且
\[
V(\alpha\gamma+\beta\sigma) \le |\alpha|V(\gamma) + |\beta|V(\sigma).
\]
\end{enumerate}
\end{proposition}

证明留给读者。

下面的命题给出一类广泛的有界变差函数,实际上,这是我们主要考虑的一类函数。

\begin{proposition}{}{prop00203040103}
如果$\gamma:[a,b]\to \mathbb{C}$是分段光滑函数,则$\gamma$是有界变差函数,且
\[
V(\gamma) = \int_{a}^{b}{|\gamma'(t)|dt}.
\]
\end{proposition}

\begin{proof}
设$\gamma$是光滑的(对于分段光滑情形的证明容易由此推出),我们记得,当我们说$\gamma$是光滑时,就意味着$\gamma'$是连续的。设$P=\{a=t_0<t_1<\cdots<t_m=b\}$,那么,由定义,
\[
\begin{aligned}
v(\gamma;P) &= \sum_{k=1}^{m}{|\gamma(t_k) - \gamma(t_{k-1})|} = \sum_{k=1}^{m}{|\int_{t_{k-1}}^{t_k}{\gamma'(t)dt}|} \\
&\le\sum_{k=1}^{m}{\int_{t_{k-1}}^{t_k}{|\gamma'(t)|dt}} = \int_{a}^{b}{|\gamma'(t)|dt}.
\end{aligned}
\]
因此$V(\gamma) \le \int_{a}^{b}{|\gamma'(t)|dt}$, $\gamma$是有界变差函数。

由于$\gamma'$连续,因而一致连续;所以,如果给定$\epsilon>0$, 我们可以选取$\delta_1 >0$,使得当$|s-t|<\delta_1$时,有$|\gamma'(s)-\gamma'(t)|<\epsilon$, 又可选取$\delta_2 > 0$,使得如果$P=\{a=t_0<t_1<\cdots<t_m = b\}$, $\left\|P\right\| = \max\{(t_k - t_{k-1}):1 \le k \le m\} \le \delta_2$时,有
\[
|\int_{a}^{b}{|\gamma'(t)|dt} + \sum_{k=1}^{m}{|\gamma'(\tau_k)|(t_k - t_{k-1})}| < \epsilon,
\]
其中$\tau_k$是$[t_{k-1}, t_k]$中的任意一点。因此,
\[
\begin{aligned}
\int_{a}^{b}{|\gamma'(t)|dt} &\le \epsilon + \sum_{k=1}^{m}{|\gamma'(\tau_k)|(t_k-t_{k-1})}\\
&=\epsilon + \sum_{k=1}^{m}{|\int_{t_{k-1}}^{t_k}{\gamma'(\tau_k)dt}|}\\
&\le \epsilon + \sum_{k=1}^{m}{|\int_{t_{k-1}}^{t_k}{[\gamma'(\tau_k) - \gamma'(t)]dt}|} \\
&\quad + \sum_{k=1}^{m}{|\int_{t_{k-1}}^{t_k}{\gamma'(t)dt}|}.
\end{aligned}
\]
如果$\left\|P\right\| \le \delta = \min(\delta_1, \delta_2)$,那么对于$[t_{k-1}, t_k]$中的$\tau_k$,有$|\gamma'(t_k) - \gamma'(t_{k-1})|<\epsilon$, 且
\[
\begin{aligned}
\int_{a}^{b}{|\gamma'(t)|dt} &\le \epsilon + \epsilon(b-a) + \sum_{k=1}^{m}{|\gamma(t_k) - \gamma(t_{k-1})|}\\
&=\epsilon[1 + (b-a)] + v(\gamma;P)\\
&\le \epsilon[1 + (b-a)] + V(\gamma).
\end{aligned}
\]
令$\epsilon \to 0+$, 给出
\[
\int_{a}^{b}{|\gamma'(t)|dt} \le V(\gamma),
\]
这就得到了要证明的等式。
\end{proof}

\begin{theorem}{}{thm002040104}
设$\gamma:[a,b]\to \mathbb{C}$是有界变差函数,$f:[a,b]\to \mathbb{C}$是连续函数,则存在一个复常数$I$, 使得对于每个$\epsilon>0$,存在$\delta>0$,当$P = \{t_0<t_1<\cdots<t_m\}$是$[a,b]$的一个分割,且$\left\|P\right\|=\max\{(t_k - t_{k-1}):1\le k \le m\} < \delta$时,
\[
|I - \sum_{k=1}^{m}{f(\tau_k)[\gamma(t_k) - \gamma(t_{k-1})]}|<\epsilon
\]
其中$\tau_k$可以在$[t_{k-1}, t_k]$上任意选择。

数$I$就称为$f$在$[a,b]$上关于$\gamma$的积分,记为
\[
I = \int_{a}^{b}{fd\gamma} = \int_{a}^{b}{f(t)d\gamma(t)}.
\]
\end{theorem}

\begin{proof}
由于$f$是连续的,因而是一致连续的;于是我们可以(归纳地)取到正数$\delta_1>\delta_2>\delta_3>\cdots$,使得当$|s-t|<\delta_m$时,有$|f(s)-f(t)|<\frac{1}{m}$.对每一个$m \ge 1$,设$\mathscr{P}_m$是$[a,b]$的满足$\|P\| < \delta_m$的所有分割$P$作成的集,那么$\mathscr{P}_1 \supset \mathscr{P}_2 \supset\cdots$, 再定义$F_m$是集
\begin{gather}\label{equ002040105}
\{\sum_{k=1}^{m}{f(\tau_k)[\gamma(t_k) - \gamma(t_{k-1})]}: P \in \mathscr{P}_m, t_{k-1} \le \tau_k \le t_k\}
\end{gather}
的闭包。

我们先假定下述断言成立。
\begin{gather}\label{equ002040106}
\left\{
\begin{aligned}
&F_1 \supset F_2 \supset F_3 \supset \cdots,\\
&\diam{F_m} \le \frac{2}{m}V(\gamma).
\end{aligned}
\right.
\end{gather}
这时由Cantor定理(第\ref{section00202}章\ref{thm:thm002020307}),恰好存在一个复数$I$,使得对每一个$m \ge 1$都有$I \in F_m$. 让我们来说明,这将完成定理的证明。如果$\epsilon>0$, 令$m > \frac{2}{\epsilon}V(\gamma)$, 则$\epsilon > \frac{2}{m}V(\gamma) \le \diam{F_m}$,由于$I \in F_m$,所以$F_m \subset B(I;\epsilon)$. 于是取$\delta=\delta_m$,定理便得证。

现在来证明断言(\ref{equ002040106})。
\end{proof}

\begin{problemset}
\item\label{exer002040101}设$\gamma:[a, b] \to \mathbb{R}$是非减的,
\end{problemset}


\section{解析函数的幂级数表示}\label{section0020402}



\begin{proposition}{}{prop002040201}
设$\varphi$
\end{proposition}

\chapter{Runge定理}\label{chapter00208}

\section{Runge定理}\label{section0020801}


\section{单连通性}\label{section0020802}


\section{Mittag-Leffler定理}\label{section0020803}


\chapter{调和函数}\label{chapter00210}















