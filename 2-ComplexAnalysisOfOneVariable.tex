\chapter{单复变函数}
《单复变函数》(Functions of One Complex Variable)的作者是J.B.康威(John B. Conway)。参考:\cite{FunctionsofOneComplexVariable1978}。

\section{复数系}\label{section00201}

\subsection{实数}\label{subsection0020101}
我们用$\mathbb{R}$表示所有实数组成的集。假定读者熟悉实数系及其性质,特别地,假定读者具备下面的知识:$\mathbb{R}$的序,上确界和下确界的定义和性质,以及$\mathbb{R}$的完备性($\mathbb{R}$中的每一个有上界的集必有上确界)。我们也假定读者熟知$\mathbb{R}$中的序列的收敛性与无穷级数。最后,一个人只有在单变量实函数方面有了坚实的基础之后,才可以着手学习复变函数。虽然在学习解析函数理论之前,传统上是先学习多变数实函数。但是对于本书来说,本质上这不是必要的条件,因为本书中任何地方都不需要这个领域里深入的结果。

\subsection{复数域}\label{subsection0020102}
我们把复数集$\mathbb{C}$定义为所有有序数对$(a, b)$的集,其中$a,b$是实数。加法和乘法由下式定义:
\begin{gather*}
(a, b) + (c, d) = (a+c, b+d), \\
(a, b)(c, d) = (ac-bd, bc + ad).
\end{gather*}
容易验证,这样定义后,$\mathbb{C}$满足域(field)的所有公理。这就是说,$\mathbb{C}$满足加法和乘法的结合律、交换律、分配了;$(0, 0)$和$(1,0)$分别是加法和乘法的单位元素,并且$\mathbb{C}$内的每一个非零元素有加法和乘法的逆元素。

对于复数$(a, 0)$,我们将写为$a$,这个映照$a \mapsto (a, 0)$定义了一个$\mathbb{R}$到$\mathbb{C}$的域同构\footnote{这恐怕不能说是同构,因为明显不是一一映射,应该是$\mathbb{R}$和$\mathbb{C}$的一个子集同构},所以我们可以把$\mathbb{R}$考虑为$\mathbb{C}$的一个子集。如果令$i=(0, 1)$,那么$(a, b) = a + ib$,从现在起,我们对复数就不再使用有序数对的记号了。

注意到$i^2=-1$,所以方程$z^2+1=0$在$\mathbb{C}$内有根。事实上,对于$\mathbb{C}$内的每个$z$,$z^2+1=(z+i)(z-i)$。更一般地,如果$z$和$w$是复数,我们得到
\[
z^2+w^2 = (z+iw)(z-iw),
\]
令$z$和$w$是实数$a$和$b$($a$和$b$都不为0\footnote{这里只需要$a$和$b$不全为0即可。}),我们得到
\[
\frac{1}{a+bi} = \frac{a-bi}{a^2+b^2} = \frac{a}{a^2+b^2} - i(\frac{b}{a^2+b^2}),
\]
这样我们就有了一个复数的倒数的公式。

当我们写$z = a + bi$($a, b \in \mathbb{R}$)时,我们称$a$,$b$为$z$的实部和虚部,并且用$a = \Re{z}$,$b=\Im{z}$表示。

作为本节的结尾,我们在$\mathbb{C}$内引进两个运算。这两个运算不是域的运算。如果$z = x+iy$($x, y \in \mathbb{R}$),那么我们定义$|z| = (x^2+y^2)^{\frac{1}{2}}$为$z$的绝对值,$\bar{z}=x-iy$为$z$的共轭数。注意:
\begin{gather}\label{equ00202001}
|z|^2=z\bar{z},
\end{gather}
特别地,如果$z \neq 0$,那么
\[
\frac{1}{z} = \frac{\bar{z}}{|z|^2}.
\]

下面是绝对值和共轭数的基本性质,其证明留给读者。
\begin{gather}
\Re{z} = \frac{1}{2}(z + \bar{z}), \quad \Im{z} = \frac{1}{2i}(z - \bar{z}). \label{equ00202002} \\
(\bar{z+w}) = \bar{z} + \bar{w}, \quad \bar{zw} = \bar{z}\bar{w}. \label{equ00202003}\\
|zw| = |z||w|. \label{equ00202004} \\
|z/w| = |z|/|w|.\label{equ00202005}\\
|\bar{z}| = |z|.\label{equ00202006}
\end{gather}
读者证明后面三个式子的时候,应当尽量避免将$z$和$w$展开为它们的实部和虚部,而最好利用(\ref{equ00202001}),(\ref{equ00202002})和(\ref{equ00202003})。

\begin{exercise}
求下列各复数的实部和虚部:
\begin{gather*}
\begin{aligned}
&\frac{1}{z}; \frac{z-a}{z+a}(a \in \mathbb{R}); z^2; \frac{3+5i}{7i+1}; (\frac{-1+i\sqrt{3}}{2})^3;\\
&(\frac{-1-i\sqrt{3}}{2})^6; i^n; (\frac{1+i}{\sqrt{2}})^n, 2 \le n \le 8.
\end{aligned}
\end{gather*}
\end{exercise}

\begin{exercise}
求下列各复数的绝对值和共轭数:
\[
\begin{aligned}
&-2+i; -3; (2+i)(4+3i);\frac{3-i}{\sqrt{2}+3i};\frac{i}{i+3}; \\
&(1+i)^6; i^{17}.
\end{aligned}
\]
\end{exercise}

\begin{exercise}
证明:当且仅当$z=\bar{z}$时,$z$才是实数。
\end{exercise}

\begin{exercise}\label{exer002010204}
若$z$和$w$是复数,证明下列等式:
\[
\begin{aligned}
&|z+w|^2 = |z|^2 + 2\Re{z\bar{w}} + |w|^2 \\
&|z-w|^2 = |z|^2 - e\Re{z\bar{w}} + |w|^2 \\
&|z+w|^2 + |z-w|^2 = 2(|z|^2+|w|^2)
\end{aligned}
\]
\end{exercise}

\begin{exercise}
设$z=z_1+\cdots+z_n$,$w= w_1+\cdots+w_n$,利用归纳法证明:
\[
|w| = |w_1|\cdots|w_n|; \bar{z}=\bar{z_1}+\cdots+\bar{z_n}; \bar{w}=\bar{w_1}\cdots\bar{w_n}.
\]
\end{exercise}

\begin{exercise}
设$R(z)$是$z$的有理函数,如果$R(z)$的所有系数是实数,则$\overline{R(z)} = R(\bar{z})$。
\end{exercise}

\subsection{复平面}\label{subsection0020103}
从复数的定义易见,$\mathbb{C}$中每一点$z$都可以和平面$\mathbb{R}^2$上唯一确定的点$(\Re{z}, \Im{z})$相等同。复数的加法恰好就是向量空间$\mathbb{R}^2$的加法,如果$z$和$w$是$\mathbb{C}$中的点,那么从$z$和$w$到$0(=(0,0))$画两条直线,这两条直线形成了以$0$、$z$、$w$为三个顶点的平行四边形的两条边,平行四边形的第四个顶点就是$z+w$。

注意,$|z-w|$恰好是$z$和$w$之间的距离,理会到这一点,上节习题\ref{exer002010204}中的最后一个等式说的就是平行四边形法则:平行四边形各边长的平方和等于其对角线的平方和。

距离函数的基本性质是它满足三角不等式(见下一章)。在这种情况下,对复数$z_1$,$z_2$,$z_3$,这个不等式变为
\[
|z_1-z_2| \le |z_1-z_3| + |z_3-z_2|.
\]

利用$z_1-z_2 = (z_1-z_3) + (z_3-z_2)$,容易看出,我们只需证明
\begin{gather}\label{equ002010301}
|z+w| \le |z| + |w| \quad (z, w \in \mathbb{C})
\end{gather}
为了证明这个不等式,首先看出,对于$\mathbb{C}$中任意$z$,
\begin{gather}\label{equ002010302}
\begin{aligned}
&-|z| \le \Re{z} \le |z|, \\
&-|z| \le \Im{z} \le |z|.
\end{aligned}
\end{gather}
因此,$\Re{z\bar{w}} \le |z\bar{w}| = |z||w|$。于是
\[
\begin{aligned}
|z+w|^2 &= |z|^2 + 2\Re{z\bar{w}} + |w|^2\\
&\le |z|^2 + 2|z||w| + |w|^2 = (|z|+|w|)^2,
\end{aligned}
\]
由此推出\ref{equ002010301}。(这个式子称为三角不等式,因为如果我们把$z$和$w$表示在平面上,(\ref{equ002010301})式表明,三角形$[0,z,z+w]$的一边的长度小于另外两边长度的和。或者说两点间的最短距离是直线)。在遇到一个不等式时,人们总应当问一问等号成立的必要充分条件是什么,考察一个三角形并考虑到(\ref{equ002010301})的几何意义,我们就引出条件$z=tw$,对某一$t \in \mathbb{R}$,$t \ge 0$。(或者如果$w=0$,则$w=tz$)。显然,当这两点和原点共线时,等号成立。事实上,如果我们看一下(\ref{equ002010301})式的证明,便知道$|z+w|=|z|+|w|$成立的必要充分条件是$|z\bar{w}|=\Re(z\bar{w})$。这等价于$z\bar{w}\ge 0$(即$z\bar{w}$使非负实数)。如果$w \neq 0$,两边乘以$w/w$,我们得到$|w|^2(z/w) \ge 0$,令
\[
t = z/w = (\frac{1}{|w|^2})|w|^2(z/w),
\]
那么$z=tw$,$t \ge 0$。

由归纳法,我们也有
\begin{gather}\label{equ002010303}
|z_1+z_2+\cdots+z_n| \le |z_1|+|z_2+\cdots+|z_n|,|
\end{gather}
不等式
\begin{gather}\label{equ002010304}
||z| - |w|| \le |z-w|
\end{gather}
也是有用的。

既然我们给出了绝对值的几何解释,让我们再来看一看,平面上一点的共轭复数是什么,这是容易的,事实上,$\bar{z}$是$z$关于$x$轴(即实轴)的对称点。

\begin{exercise}
证明(\ref{equ002010304})并给出等号成立的必要充分条件。
\end{exercise}

\begin{exercise}
证明:(\ref{equ002010302})中的等号成立,当且仅当,对任意整数$k$和$l$,$1 \le k,l \le n$,只要$z_l \neq 0$,就有$z_k/z_l \le 0$。
\end{exercise}

\begin{exercise}
设$a \in \mathbb{R}$,$c > 0$是固定的。对于每个可能选取的$a$和$c$,试描画出满足条件
\[
|z-a| - |z+a|=2c
\]
的点集。现在设$a$施任意复数,利用平面的旋转画出满足上述方程的点的轨迹。
\end{exercise}

\subsection{复数的极坐标表示与复数的方根}\label{subsection0020104}
考虑复平面$\mathbb{C}$的点$z=x+iy$。














