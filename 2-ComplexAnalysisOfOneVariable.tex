\part{单复变函数}
《单复变函数》(Functions of One Complex Variable)的作者是J.B.康威(John B. Conway)。参考:\cite{FunctionsofOneComplexVariable1978}。

\chapter{复数系}\label{section00201}

\section{实数}\label{subsection0020101}
我们用$\mathbb{R}$表示所有实数组成的集。假定读者熟悉实数系及其性质,特别地,假定读者具备下面的知识:$\mathbb{R}$的序,上确界和下确界的定义和性质,以及$\mathbb{R}$的完备性($\mathbb{R}$中的每一个有上界的集必有上确界)。我们也假定读者熟知$\mathbb{R}$中的序列的收敛性与无穷级数。最后,一个人只有在单变量实函数方面有了坚实的基础之后,才可以着手学习复变函数。虽然在学习解析函数理论之前,传统上是先学习多变数实函数。但是对于本书来说,本质上这不是必要的条件,因为本书中任何地方都不需要这个领域里深入的结果。

\section{复数域}\label{subsection0020102}
我们把复数集$\mathbb{C}$定义为所有有序数对$(a, b)$的集,其中$a,b$是实数。加法和乘法由下式定义:
\begin{gather*}
(a, b) + (c, d) = (a+c, b+d), \\
(a, b)(c, d) = (ac-bd, bc + ad).
\end{gather*}
容易验证,这样定义后,$\mathbb{C}$满足域(field)的所有公理。这就是说,$\mathbb{C}$满足加法和乘法的结合律、交换律、分配了;$(0, 0)$和$(1,0)$分别是加法和乘法的单位元素,并且$\mathbb{C}$内的每一个非零元素有加法和乘法的逆元素。

对于复数$(a, 0)$,我们将写为$a$,这个映照$a \mapsto (a, 0)$定义了一个$\mathbb{R}$到$\mathbb{C}$的域同构\footnote{这恐怕不能说是同构,因为明显不是一一映射,应该是$\mathbb{R}$和$\mathbb{C}$的一个子集同构},所以我们可以把$\mathbb{R}$考虑为$\mathbb{C}$的一个子集。如果令$i=(0, 1)$,那么$(a, b) = a + ib$,从现在起,我们对复数就不再使用有序数对的记号了。

注意到$i^2=-1$,所以方程$z^2+1=0$在$\mathbb{C}$内有根。事实上,对于$\mathbb{C}$内的每个$z$,$z^2+1=(z+i)(z-i)$。更一般地,如果$z$和$w$是复数,我们得到
\[
z^2+w^2 = (z+iw)(z-iw),
\]
令$z$和$w$是实数$a$和$b$($a$和$b$都不为0\footnote{这里只需要$a$和$b$不全为0即可。}),我们得到
\[
\frac{1}{a+bi} = \frac{a-bi}{a^2+b^2} = \frac{a}{a^2+b^2} - i(\frac{b}{a^2+b^2}),
\]
这样我们就有了一个复数的倒数的公式。

当我们写$z = a + bi$($a, b \in \mathbb{R}$)时,我们称$a$,$b$为$z$的实部和虚部,并且用$a = \Re{z}$,$b=\Im{z}$表示。

作为本节的结尾,我们在$\mathbb{C}$内引进两个运算。这两个运算不是域的运算。如果$z = x+iy$($x, y \in \mathbb{R}$),那么我们定义$|z| = (x^2+y^2)^{\frac{1}{2}}$为$z$的绝对值,$\bar{z}=x-iy$为$z$的共轭数。注意:
\begin{gather}\label{equ00202001}
|z|^2=z\bar{z},
\end{gather}
特别地,如果$z \neq 0$,那么
\[
\frac{1}{z} = \frac{\bar{z}}{|z|^2}.
\]

下面是绝对值和共轭数的基本性质,其证明留给读者。
\begin{gather}
\Re{z} = \frac{1}{2}(z + \bar{z}), \quad \Im{z} = \frac{1}{2i}(z - \bar{z}). \label{equ00202002} \\
(\bar{z+w}) = \bar{z} + \bar{w}, \quad \bar{zw} = \bar{z}\bar{w}. \label{equ00202003}\\
|zw| = |z||w|. \label{equ00202004} \\
|z/w| = |z|/|w|.\label{equ00202005}\\
|\bar{z}| = |z|.\label{equ00202006}
\end{gather}
读者证明后面三个式子的时候,应当尽量避免将$z$和$w$展开为它们的实部和虚部,而最好利用(\ref{equ00202001}),(\ref{equ00202002})和(\ref{equ00202003})。

\begin{exercise}
求下列各复数的实部和虚部:
\begin{gather*}
\begin{aligned}
&\frac{1}{z}; \frac{z-a}{z+a}(a \in \mathbb{R}); z^2; \frac{3+5i}{7i+1}; (\frac{-1+i\sqrt{3}}{2})^3;\\
&(\frac{-1-i\sqrt{3}}{2})^6; i^n; (\frac{1+i}{\sqrt{2}})^n, 2 \le n \le 8.
\end{aligned}
\end{gather*}
\end{exercise}

\begin{exercise}
求下列各复数的绝对值和共轭数:
\[
\begin{aligned}
&-2+i; -3; (2+i)(4+3i);\frac{3-i}{\sqrt{2}+3i};\frac{i}{i+3}; \\
&(1+i)^6; i^{17}.
\end{aligned}
\]
\end{exercise}

\begin{exercise}
证明:当且仅当$z=\bar{z}$时,$z$才是实数。
\end{exercise}

\begin{exercise}\label{exer002010204}
若$z$和$w$是复数,证明下列等式:
\[
\begin{aligned}
&|z+w|^2 = |z|^2 + 2\Re{z\bar{w}} + |w|^2 \\
&|z-w|^2 = |z|^2 - e\Re{z\bar{w}} + |w|^2 \\
&|z+w|^2 + |z-w|^2 = 2(|z|^2+|w|^2)
\end{aligned}
\]
\end{exercise}

\begin{exercise}
设$z=z_1+\cdots+z_n$,$w= w_1+\cdots+w_n$,利用归纳法证明:
\[
|w| = |w_1|\cdots|w_n|; \bar{z}=\bar{z_1}+\cdots+\bar{z_n}; \bar{w}=\bar{w_1}\cdots\bar{w_n}.
\]
\end{exercise}

\begin{exercise}
设$R(z)$是$z$的有理函数,如果$R(z)$的所有系数是实数,则$\overline{R(z)} = R(\bar{z})$。
\end{exercise}

\section{复平面}\label{subsection0020103}
从复数的定义易见,$\mathbb{C}$中每一点$z$都可以和平面$\mathbb{R}^2$上唯一确定的点$(\Re{z}, \Im{z})$相等同。复数的加法恰好就是向量空间$\mathbb{R}^2$的加法,如果$z$和$w$是$\mathbb{C}$中的点,那么从$z$和$w$到$0(=(0,0))$画两条直线,这两条直线形成了以$0$、$z$、$w$为三个顶点的平行四边形的两条边,平行四边形的第四个顶点就是$z+w$。

注意,$|z-w|$恰好是$z$和$w$之间的距离,理会到这一点,上节习题\ref{exer002010204}中的最后一个等式说的就是平行四边形法则:平行四边形各边长的平方和等于其对角线的平方和。

距离函数的基本性质是它满足三角不等式(见下一章)。在这种情况下,对复数$z_1$,$z_2$,$z_3$,这个不等式变为
\[
|z_1-z_2| \le |z_1-z_3| + |z_3-z_2|.
\]

利用$z_1-z_2 = (z_1-z_3) + (z_3-z_2)$,容易看出,我们只需证明
\begin{gather}\label{equ002010301}
|z+w| \le |z| + |w| \quad (z, w \in \mathbb{C})
\end{gather}
为了证明这个不等式,首先看出,对于$\mathbb{C}$中任意$z$,
\begin{gather}\label{equ002010302}
\begin{aligned}
&-|z| \le \Re{z} \le |z|, \\
&-|z| \le \Im{z} \le |z|.
\end{aligned}
\end{gather}
因此,$\Re{z\bar{w}} \le |z\bar{w}| = |z||w|$。于是
\[
\begin{aligned}
|z+w|^2 &= |z|^2 + 2\Re{z\bar{w}} + |w|^2\\
&\le |z|^2 + 2|z||w| + |w|^2 = (|z|+|w|)^2,
\end{aligned}
\]
由此推出\ref{equ002010301}。(这个式子称为三角不等式,因为如果我们把$z$和$w$表示在平面上,(\ref{equ002010301})式表明,三角形$[0,z,z+w]$的一边的长度小于另外两边长度的和。或者说两点间的最短距离是直线)。在遇到一个不等式时,人们总应当问一问等号成立的必要充分条件是什么,考察一个三角形并考虑到(\ref{equ002010301})的几何意义,我们就引出条件$z=tw$,对某一$t \in \mathbb{R}$,$t \ge 0$。(或者如果$w=0$,则$w=tz$)。显然,当这两点和原点共线时,等号成立。事实上,如果我们看一下(\ref{equ002010301})式的证明,便知道$|z+w|=|z|+|w|$成立的必要充分条件是$|z\bar{w}|=\Re(z\bar{w})$。这等价于$z\bar{w}\ge 0$(即$z\bar{w}$使非负实数)。如果$w \neq 0$,两边乘以$w/w$,我们得到$|w|^2(z/w) \ge 0$,令
\[
t = z/w = (\frac{1}{|w|^2})|w|^2(z/w),
\]
那么$z=tw$,$t \ge 0$。

由归纳法,我们也有
\begin{gather}\label{equ002010303}
|z_1+z_2+\cdots+z_n| \le |z_1|+|z_2+\cdots+|z_n|,|
\end{gather}
不等式
\begin{gather}\label{equ002010304}
||z| - |w|| \le |z-w|
\end{gather}
也是有用的。

既然我们给出了绝对值的几何解释,让我们再来看一看,平面上一点的共轭复数是什么,这是容易的,事实上,$\bar{z}$是$z$关于$x$轴(即实轴)的对称点。

\begin{exercise}
证明(\ref{equ002010304})并给出等号成立的必要充分条件。
\end{exercise}

\begin{exercise}
证明:(\ref{equ002010302})中的等号成立,当且仅当,对任意整数$k$和$l$,$1 \le k,l \le n$,只要$z_l \neq 0$,就有$z_k/z_l \le 0$。
\end{exercise}

\begin{exercise}
设$a \in \mathbb{R}$,$c > 0$是固定的。对于每个可能选取的$a$和$c$,试描画出满足条件
\[
|z-a| - |z+a|=2c
\]
的点集。现在设$a$施任意复数,利用平面的旋转画出满足上述方程的点的轨迹。
\end{exercise}

\section{复数的极坐标表示与复数的方根}\label{subsection0020104}
考虑复平面$\mathbb{C}$的点$z=x+iy$。这个点有极坐标$(r, \theta)$:$x=r\cos{\theta}$,$y=r\sin{\theta}$。显然$r=|z|$,$\theta$是正实轴与从0到$z$的直线段的夹角。注意,在上述等式中的$\theta$可以代之以$\theta$加上$2\pi$的任意整数倍,角$\theta$称为$z$的幅(还是这个“辐)”角,记为$\theta=\arg{z}$。由于$\theta$的不确定性,“$\arg$”不是一个函数。我们引进记号
\begin{gather}\label{equ002010401}
\cis{\theta}=\cos{\theta}+i\sin{\theta}.
\end{gather}

设$z_1=r_1\cis{\theta_1}$,$z_2=r_2\cis{\theta_2}$,那么
\[
\begin{aligned}
z_1z_2=r_1r_2\cis{\theta_1}\cis{\theta_2}&=r_1r_2[(\cos{\theta_1}\cos{\theta_2}-\sin{\theta_1}\sin{}\theta_2)\\
+i(\sin{\theta_1}\cos{\theta_2}+\sin{\theta_2}\cos{\theta_1})]
\end{aligned}
\]
由正弦和余弦的和角公式,我们得到
\begin{gather}\label{equ002010402}
z_1z_2 = r_1r_2\cis(\theta_1+\theta_2).
\end{gather}
换句话说$\arg{z_1z_2} = \arg{z_1} + \arg{z_2}$(什么实函数把成绩变为和?\footnote{对数函数$\ln{(ab)}=\ln{a}+\ln{b}$})。由归纳法,对于$z_k=r_k\cis{\theta_k}$,$1 \le k \le n$,我们有
\begin{gather}\label{equ002010403}
z_1z_2\cdots{}z_n=r_1r_2\cdots{}r_n\cis{(\theta_1+\theta_2+\cdots+\theta_n)}.
\end{gather}
特别地,对于每个整数$n \ge 0$,有
\begin{gather}\label{equ002010404}
z^n = r^n\cis(n\theta).
\end{gather}
进而若$z \neq 0$,则$z[r^{-1}\cis(-\theta)]=1$;所以如果$z \neq 0$,那么对于一切整数$n$,正的,负的。或0,(\ref{equ002010404})也成立。作为(\ref{equ002010404})的一个特别情形,我们得到棣莫佛(de Moivre)公式:
\[
(\cos{\theta}+i\sin{\theta})^n=\cos{n\theta} + i\sin{n\theta}.
\]

现在我们可以来考虑下面的问题了。对于给定的一个复数$a \neq 0$,和一个整数$n \ge 2$,你能否找到满足$z^n=a$的数$z$?这样的$z$你能找到多少个?由于(\ref{equ002010404})式,解答这个问题是容易的。设$a=|a|\cis{\alpha}$;由(\ref{equ002010404}),$z=|a|^{\frac{1}{n}}\cis{(\alpha/n)}$就满足要求。但是这个解不是唯一解,因为$z'=|a|^{\frac{1}{n}}\cis{\frac{1}{n}(\alpha+2\pi)}$也满足$(z')^n=a$。事实上,每一个数
\begin{gather}\label{equ002010405}
|a|^{\frac{1}{n}}\cis{\frac{1}{n}(\alpha+2\pi{}k)}, \quad 0 \le k \le n-1.
\end{gather}
都是$a$的$n$次方根。借助(\ref{equ002010404})我们得到下述结果:对于$\mathbb{C}$中的每一个不等于零的数$a$,都有$a$的$n$个不同的$n$次方根,它们由公式(\ref{equ002010405})给出。

\textbf{例子} \quad 计算$n$次单位根。由于$1=\cis{0}$,(\ref{equ002010405})式给出如下这些根:
\[
1,\cis{\frac{2\pi}{n}},\cis{\frac{4\pi}{n}},\cdots\cis{\frac{2\pi}{n}(n-1)}.
\]
特别地,立方单位根是
\[
1, \frac{1}{2}(-1 + i\sqrt{3}),\frac{1}{2}(-1-i\sqrt{3}).
\]

\begin{exercise}
求出6次单位根。
\end{exercise}

\begin{exercise}
计算:
\begin{enumerate}
\item[(a)]$i$的平方根;
\item[(b)]$i$的立方根;
\item[(c)]$\sqrt{3}+3i$的平方根。
\end{enumerate}
\end{exercise}

\begin{exercise}
$n$次单位原根是一复数$a$,使得$1,a,a^2,\cdots,a^{n-1}$是$n$个不同的$n$次单位根。证明:如果$a,b$分别是$n$次和$m$次单位原根,则$ab$是$k$次单位根,$k$是某一整数。$k$的最小值是什么?如果$a,b$是非单位原根,你能说些什么?
\end{exercise}

\begin{exercise}
试利用二项式
\[
(a+b)^n=\sum_{k=0}^{n}{{n \choose k}a^{n-k}b^{k}},
\]
其中${n \choose k}=\frac{n!}{k!(n-k)!}$,并比较棣莫佛公式两边的实部和虚部,得到公式
\[
\begin{aligned}
&\cos{n\theta} = \cos^{n}{\theta} - {n \choose 2}\cos^{n-2}{\theta}\sin^{2}{\theta} + {n \choose 4}\cos^{n-4}{\theta}\sin^{4}{\theta}-\cdots\\
&\sin{n\theta} = {n \choose 1}\cos^{n-1}{\theta}\sin{theta}-{n \choose 3}\cos^{n-3}{\theta}\sin^{3}{\theta}+\cdots.
\end{aligned}
\]
\end{exercise}

\begin{exercise}
设$z=\cis{\frac{2\pi}{n}}$,整数$n \ge 2$。证明:$1+z+\cdots+z^{n-1}=0$.
\end{exercise}

\begin{exercise}
证明:$\phi(t)=\cis{t}$是加法群$\mathbb{R}$到乘法群$T=\{z:|z|=1\}$上的群同态。
\end{exercise}

\begin{exercise}
如果$z \in \mathbb{C}$,并且对于每个正整数$n$,$\Re{z^n} \ge 0$,证明:$z$是正实数。
\end{exercise}

\section{复平面上的直线和半平面}\label{subsection0020105}
设$L$表示$\mathbb{C}$中的直线。从初等解析几何知道,$L$是由$L$上的一个点和一个方向向量决定的。于是,如果$a$是$L$上任一点,$b$是它的方向向量,那么
\[
L = \{z = a+tb:-\infty < t < \infty\}.
\]
由于$b \neq 0$,这就给出,对于$L$上的$z$,有
\[
\Im{(\frac{z-a}{b})} = 0.
\]
事实上,如果$z$满足等式
\[
0 = \Im{(\frac{z-a}{b})},
\]
那么
\[
t = \frac{z-a}{b},
\]
蕴含$z = a + tb$,$-\infty < t < \infty$。这就是说
\begin{gather}\label{equ002010501}
L = \big{\{} z: \Im{(\frac{z-a}{b})} = 0 \big{\}}.
\end{gather}
集合
\[
\begin{aligned}
&\big{\{} z: \Im{(\frac{z-a}{b})} > 0 \big{\}},\\
&\big{\{} z: \Im{(\frac{z-a}{b})} < 0 \big{\}}.
\end{aligned}
\]
的轨迹是什么呢?作为回答这个问题的第一步,注意到$b$是一个方向,我们可以假定$|b|=1$。我们暂时考虑$a=0$的情形。并且令$H_0=\{z:\Im{(z/b)}>0\}$,$b = \cis{\beta}$。如果$z = r\cis{\theta}$,那么$z/b = r\cis{(\theta-\beta)}$。于是$z$在$H_0$中,当且仅当$\sin(\theta-\beta)>0$,即$\beta < \theta < \pi + \beta$。所以,如果我们“按照$b$的方向沿着$L$前进”,$H_0$是位于$L$左边的半平面。如果我们令
\[
H_a = \big{\{} z: \Im{(\frac{z-a}{b})} > 0 \big{\}},
\]
那么容易看出,$H_a = a + H_0 \equiv \{ a + w: w \in H_0\}$;即$H_a$是由$H_0$平移$a$而得到的,因此,$H_a$是位于$L$的左边的半平面。类似地,
\[
K_a = \big{\{} z: \Im{(\frac{z-a}{b})} < 0 \big{\}}
\]
是位于$L$的右边的半平面。

\begin{exercise}
设$C$是圆周$\{z:|z-c|=r\}$,$r > 0$,$a = c + r\cis{\alpha}$;并且令
\[
L_{\beta} = \big{\{} z: \Im{(\frac{z-a}{b})} = 0 \big{\}},
\]
其中$b=\cis{\beta}$。找出$L_{\beta}$在$a$处切于圆周$C$的关于$\beta$的充分必要条件。
\end{exercise}

\section{扩充平面及其球面表示}\label{subsection0020106}
在复分析中,我们常常涉及到这样一些函数,当自变量趋于给定点时,它们趋于无穷。为了讨论这种情形,我们引进扩充平面$\mathbb{C}_{\infty}\equiv \mathbb{C} \bigcup \{\infty\}$。同时为了讨论到取到无穷作为它的值的函数的连续性。我们也希望在$\mathbb{C}_{\infty}$内引进距离函数。为了这个目的以及为了给出$\mathbb{C}_{\infty}$的具体图像,我们把$\mathbb{C}_{\infty}$表示为$\mathbb{R}^3$中的单位球面
\[
S = \{(x_1,x_2,x_3) \in \mathbb{R}^3:x_1^2 + x_2^2 + x_3^2=1\}.
\]

设$N=(0, 0,1)$;即$N$是$S$上的北极。同时,把$\mathbb{C}$等同于$\{(x_1,x_2,0):x_1,x_2 \in \mathbb{R}\}$,于是$\mathbb{C}$沿赤道切割$\mathbb{C}$。现在对于$\mathbb{C}$中每个点$z$,考虑$\mathbb{R}^3$中通过$z$和$N$的直线。这条直线与球面恰好交于一点$Z \neq N$。如果$|z| > 1$,那么$Z$在北半球面上;如果$|z|<1$,那么$Z$在南半球面上;如果$|z|=1$,那么$Z = z$。当$|z| \to \infty$时,$Z$怎样呢?很清楚,$Z$趋于$N$。因此,我们就把$N$与$\mathbb{C}_{\infty}$中的$\infty$等同起来。这样一来,$\mathbb{C}_{\infty}$就被表示为球面$S$了。

让我们来考察这种表示法。令$z=x+iy$,设$Z = (x_1, x_2, x_3)$是$S$上相应的点,我们要找出用$x$,$y$表示$x_1, x_2, x_3$的方程。在$\mathbb{R}^3$中通过$z$和$N$的直线由$\{tN + (1-t)z:-\infty<t<\infty\}$或
\begin{gather}\label{equ002010601}
\{((1-t)x, (1-t)y, t): -\infty < t < \infty\}
\end{gather}
给出。因此,如果能够找到直线和$S$的交点的$t$值,我们就能够找到$Z$的坐标。如果$t$是这个值,那么
\[
1 = (1-t)^2x^2 + (1-t)^2y^2 + t^2 = (1-t)^2|z|^2 + t^2.
\]
由此注意到
\[
1-t^2 = (1-t)^2|z|^2.
\]
因为$t \neq 1$($z \neq \infty$),所以
\[
t = \frac{|z|^2-1}{|z|^2+1}.
\]
于是
\begin{gather}\label{equ002010602}
x_1 = \frac{2x}{|z|^2+1}, x_2 = \frac{2y}{|z|^2+1}, x_3 = \frac{|z|^2-1}{|z|^2+1}.
\end{gather}
这就给出
\begin{gather}\label{equ002010603}
x_1 = \frac{z + \bar{z}}{|z|^2+1}, x_2 = \frac{-i(z-\bar{z})}{|z|^2+1}, x_3 = \frac{|z|^2-1}{|z|^2+1}.
\end{gather}

如果$Z$是给定的($Z \neq N$),我们希望找$z$。这时,通过令$t = x_3$并利用(\ref{equ002010601}),我们得到
\begin{gather}\label{equ002010604}
z = \frac{x_1 + ix_2}{1 - x_3}
\end{gather}
现在让我们用下面的方式定义扩充平面上点之间的距离函数:对于$\mathbb{C}_{\infty}$中的$z$,$z'$,定义$z$到$z'$的距离$d(z,z')$为它们在$\mathbb{R}^3$中相应两点$Z$和$Z'$的距离。如果$Z=(x_1,x_2,x_3)$,$Z'=(x_1',x_2',x_3')$,那么
\begin{gather}\label{equ002010605}
d(z,z') = [(x_1-x_1')^2 + (x_2-x_2')^2 + (x_3-x_3')^2]^{\frac{1}{2}}.
\end{gather}
利用$Z$和$Z'$在$S$上这一事实,(\ref{equ002010605})给出
\begin{gather}\label{equ002010606}
[d(z,z')]^2 = 2 - 2(x_1x_1' + x_2x_2'+x_3x_3').
\end{gather}
由(\ref{equ002010603}),我们得到
\begin{gather}\label{equ002010607}
d(z,z') = \frac{2|z-z'|}{[(1+|z|^2)(1+|z'|^2)]^{\frac{1}{2}}}, \quad (z,z' \in \mathbb{C}).
\end{gather}
用类似的方法,对于$\mathbb{C}$中的$z$,我们得到
\begin{gather}\label{equ002010608}
d(z, \infty) = \frac{2}{(1 + |z|^2)^{\frac{1}{2}}},
\end{gather}
球面$S$和$\mathbb{C}_{\infty}$的点之间这种对应关系称为球极平面投影。

\begin{exercise}
给出(\ref{equ002010607})和(\ref{equ002010608})的详细推导。
\end{exercise}

\begin{exercise}
对于下列$\mathbb{C}$中的点给出$S$上对应的点:$0, 1+i,3+2i$。
\end{exercise}

\begin{exercise}
$S$上哪些子集对应$\mathbb{C}$中的实轴和虚轴。
\end{exercise}

\begin{exercise}
设$\Lambda$是$S$上的一个圆周,那么在$\mathbb{R}^3$中有唯一的平面$P$,使得$P \bigcap S = \Lambda$。由解析几何知道
\[
P = \{(x_1,x_2,x_3):x_1\beta_1 + x_2\beta_2 + x_3\beta_3 = l\},
\]
其中$(\beta_1,\beta_2,\beta_3)$是与$P$正交的一个向量,$l$是某一实数。可以假设$\beta_1^2+\beta_2^2+\beta_3^2=1$。利用这一事实,证明:如果$\Lambda$包含点$N$,则它在$\mathbb{C}$上的投影是一直线。否则,$\Lambda$投影到$\mathbb{C}$中的一个圆周上。
\end{exercise}

\begin{exercise}
设$Z$和$Z'$是$S$上分别与$z$和$z'$相应的两点。$W$是$S$上与$z+z'$对应的点。试用$Z$和$Z'$的坐标表示出$W$的坐标。
\end{exercise}

\chapter{度量空间与$\mathbb{C}$的拓扑}\label{section00202}

\section{度量空间的定义和例子}\label{subsection0020201}
一个度量空间是一个序偶$(X, d)$,这里$X$是一个集,$d$是一个从$X \times X$到$\mathbb{R}$的函数,称之为距离函数或度量,它满足下列条件:
\[
d(x, y) \ge 0;
\]
当且仅当$x=y$时,$d(x, y)=0$;
\begin{gather*}
d(x, y) = d(y,x) \quad (\text{对称性});\\
d(x, z) \le d(x, y) + d(y, z)\quad (\text{三角不等式}).
\end{gather*}
如果$x$和$r > 0$是固定的,那么定义
\begin{gather*}
B(x; r) = \{y \in X: d(x, y) < r\},\\
\bar{B}(x; r) = \{y \in X: d(x, y) \le r\}.
\end{gather*}
$B(x; r)$和$\bar{B}(x; r)$分别称为以$x$为中心,$r$为半径的开球和闭球。

\textbf{例子}

\begin{example}\label{exam002020101}
设$X = \mathbb{R}$或$\mathbb{C}$,定义$d(z, w)=|z-w|$,这就使$(\mathbb{R}, d)$和$(\mathbb{C}, d)$都成为度量空间。事实上,$(\mathbb{C}, d)$将是我们最感兴趣的例子。如果读者在此以前从来未接触过度量空间的概念,那么在学习这一章的过程中应当时常想到$(\mathbb{C}, d)$。
\end{example}

\begin{example}\label{exam002020102}
设$(X, d)$是一个度量空间,$Y \subset X$;那么$(Y, d)$也是一个度量空间。
\end{example}

\begin{example}\label{exam002020103}
设$X = \mathbb{C}$,定义$d(x+iy, a+ib)=|x-a|+|y-b|$。那么$(\mathbb{C}, d)$是一个度量空间。
\end{example}

\begin{example}\label{exam002020104}
设$X = \mathbb{C}$,定义$d(x+iy, a+ib)=\max{|x-a|, |y-b|}$。
\end{example}

\begin{example}\label{exam002020105}
设$X$是任意一个集,定义$d(x, y) = 0$,如果$x=y$;$d(x, y)=1$,如果$x \neq y$。为了证明函数$d$满足三角不等式,只要考虑在$x,y,z$当中出现相等的各种可能情形。注意,如果$r \le 1$,则$B(x; r)$只由一个点$x$所组成;如果$r > 1$,则$B(x;r) = X$。这个度量空间在解析函数论的研究中并不出现。
\end{example}

\begin{example}\label{exam002020106}
设$X = \mathbb{R}^n$,对于$\mathbb{R}^n$中的$x=(x_1,\cdots, x_n)$和$y=(y_1,\cdots,y_n)$定义
\[
d(x, y) = [\sum_{j=1}^{n}{(x_j-y_j)^2}]^{\frac{1}{2}}.
\]
\end{example}

\begin{example}\label{exam002020107}
设$S$是任意一个集,$B(S)$表示满足条件
\[
\left\|f\right\|_{\infty} = \sup{\{\left|f(s)\right|: s \in S\}} < \infty
\]
的函数$f: S \to \mathbb{C}$的集。这就是说,$B(S)$由所有其值域位于某一有穷半径的圆内的复值函数所构成。对于$B(S)$中的$f$和$g$定义$d(f, g) = \left\|f-g\right\|_{\infty}$。我们来证明$d$满足三角不等式。事实上,如果$f$,$g$和$h$在$B(S)$中,$s$是$S$中的任意一点,那么$|f(s)-g(s)| = |f(s)-h(s)+h(s)-g(s)| \le |f(s)-h(s)| + |h(s)-g(s)| \le \left\|f-h\right\|_{\infty} + \left\|h - g\right\|_{\infty}$。于是若对于$S$中所有的$s$取上确界,则有$\left\|f-g\right\|_{infty} \le \left\|f-h\right\|_{\infty} + \left\|h - g\right\|_{\infty}$,这就是对于$d$的三角不等式。
\end{example}

\begin{definition}{开集}{def002020101}
对于度量空间$(X, d)$,一个集$G \subset X$是开集,如果$G$内的每一个$x$,都存在一个$\epsilon > 0$,使得$B(x;\epsilon) \subset G$。
\end{definition}

于是,一个集在$\mathbb{C}$内是开的,如果它没有“边”。例如,
\[
G = \{z \in G: a < \Re{(z)} < b\}
\]
是开的;但是$\{z: \Re{(z)} < 0\} \bigcap \{0\}$不是开的,因为不管我们把$\epsilon$取得多么小,$B(0;\epsilon)$都不能包含在这个集内。

我们用$\emptyset$表示空集,就是一个元素也没有的集。

\begin{proposition}{}{prop002020101}
设$(X, d)$是一个度量空间,那么:
\begin{enumerate}
\item[(a)]集$X$和$\emptyset$是开集。
\item[(b)]如果$G_1,\cdots, G_n$是$X$中的开集,则$\bigcap_{k=1}^{n}{G_k}$也是$X$中的开集。
\item[(c)]如果$\{G_j:j \in J\}$是$X$中的开集族,$J$是任一指标集,则$G = \bigcup\{G_j:j \in J\}$也是开集。
\end{enumerate}
\end{proposition}

\begin{proof}
(a)的证明是平凡的。为了证明(b),设$x \in G = \bigcap_{k=1}^{n}{G_k}$;那么$x \in G_k$,$k=1,2,\cdots,n$。于是由定义,对于每个$k$有$\epsilon_k > 0$,使得$B(x;\epsilon_k) \subset G_k$。如果取$\epsilon=\min(\epsilon_1,\epsilon_2,\cdots, \epsilon_n)$,那么,对于$1 \le k \le n$,$B(x;\epsilon) \subset B(x;\epsilon_k) \subset G_k$,于是$B(x; \epsilon) \subset G$,$G$是开集。

(c)的证明留给读者作为习题。
\end{proof}

在度量空间里还有另一类著名的子集。这类子集包含它们的全部“边”,换一个说法,它们的余集没有“边”。

\begin{definition}{闭集}{def002020102}
一个集$F \subset X$是闭的,如果它的余集$X-F$是开的。
\end{definition}

下面的命题是命题\ref{pro:prop002020101}的补命题。对于前一命题应用Morgan法则便可完成其证明,我们把它留给读者。

\begin{proposition}{}{prop002020102}
设$(X, d)$是一个度量空间,那么:
\begin{enumerate}
\item[(a)]集$X$和$\emptyset$是闭的。
\item[(b)]如果$F_1,\cdots, F_n$是$X$中的闭集,则$\bigcup_{k=1}^{n}{F_k}$也是$X$中的闭集。
\item[(c)]如果$\{F_j:j \in J\}$是$X$中的闭集族,$J$是任一指标集,则$F = \bigcap\{F_j:j \in J\}$也是闭集。
\end{enumerate}
\end{proposition}

在学习开集和闭集时,最普遍的错误是把闭集的定义解释为一个集不是开集便是闭集。这种理解当然是错误的。只要看集$\{z \in \mathbb{C} : \Re{(z)} > 0\} \cup \{0\}$就清楚了,这个集既不是开的,也不是闭的。

\begin{definition}{}{def002020103}
设$A$是$X$的子集,那么,$A$的内部$\intset{A}$就是集合$\bigcup\{G: G\text{是开集,且}G \subset A\}$。$A$的闭包$A^-$就是集$\bigcap\{F: F\text{是闭集,且}F \supset A\}$。注意,$\intset{A}$可以是空集,$A^-$可以是$X$。如果$A = \{a + ib : a\text{和}b\text{是有理数}\}$,那么同时有$A^-=\mathbb{C}$和$\intset{A} = \emptyset$。根据命题\ref{pro:prop002020101}和\ref{pro:prop002020102},$A^-$是闭集,$\intset{A}$是开集。$A$的边界记为$\partial{A}$,定义为$\partial{A} = A^- \cap (X-A)^-$。
\end{definition}

\begin{proposition}{}{prop002020103}
设$A$和$B$是度量空间$(X, d)$的子集,那么:
\begin{enumerate}
\item[(a)]当且仅当$A = \intset{A}$时$A$是开集。
\item[(b)]当且仅当$A = A^-$时$A$是闭集。
\item[(c)]$\intset{A} = X - (X-A)^-$;$A^- = X- \intset(X-A)$;$\partial{A} = A^- - \intset{A}$。
\item[(d)]$(A \bigcup B)^- = A^- \bigcup B^-$。
\item[(e)]当且仅当存在$\epsilon > 0$,使得$B(x_0; \epsilon) \subset A$时,$x_0 \in \intset{A}$。
\item[(f)]当且仅当,对每一$\epsilon > 0$,$B(x_0; \epsilon) \cap A \neq \emptyset$时,$x_0 \in A^-$。
\end{enumerate}
\end{proposition}
\begin{proof}
(a)至(e)的证明留给读者。为了证明(f),假设$x_0 \in A^- = X - \intset(X-A)$;于是,$x_0 \not\in \intset(X-A)$。由(e),对于每一$\epsilon > 0$,$B(x_0; \epsilon)$不包含在$X-A$内,这就是说,存在一个点$y \in B(x_0; \epsilon)$,$y$不在$X-A$内。所以$y \in B(x_0;\epsilon) \cap A$。现在设$x_0 \not\in A^- = X - \intset{(X-A)}$,那么$x_0 \in \intset(X-A)$,由(e),存在$\epsilon > 0$,使得$B(x_0; \epsilon) \subset X-A$。即$B(x_0; \epsilon) \cap A = \emptyset$,所以$x_0$不满足条件。
\end{proof}

最后,再定义一类著名的集合。

\begin{definition}{稠密}{def002020104}
度量空间$X$的一个子集$A$是稠密的,如果$A^- = X$。
\end{definition}

有理数集$\mathbb{Q}$在$\mathbb{R}$中是稠密的,$\{x + iy : x, y, \in \mathbb{Q}\}$在$\mathbb{C}$中是稠密的。

\begin{exercise}
证明:(\ref{exam002020102})至(\ref{exam002020106})中给出的那些例子都确实是度量空间,只有例子(\ref{exam002020106})的证明可能会有些困难,对于这些例子给出$B(x;r)$。
\end{exercise}

\begin{exercise}
$\mathbb{C}$的下列子集,哪些是开集,哪些是闭集?
\begin{enumerate}
\item[(a)]$\{z:|z|<1\}$;
\item[(b)]实轴;
\item[(c)]$\{z: z^n=1,\text{对某一整数}n \ge 1\}$;
\item[(d)]$\{z \in \mathbb{C}: z\text{是实数,且}0 \le z <1\}$;
\item[(e)]$\{z \in \mathbb{C}: z\text{是实数,且}0 \le z \le 1\}$。
\end{enumerate}
\end{exercise}

\begin{exercise}
如果$(X, d)$是任一度量空间,证明:每一个开球是开集,每一个闭球是闭集。
\end{exercise}

\begin{exercise}
给出(\ref{pro:prop002020101}c)的详细证明。
\end{exercise}

\begin{exercise}
证明命题\ref{pro:prop002020102}。
\end{exercise}

\begin{exercise}
证明:一个集$G \subset X$是开的,当且仅当$X-G$是闭的。
\end{exercise}

\begin{exercise}
证明:$(\mathbb{C}_{\infty}, d)$是一度量空间,其中$d$是由第一章的(\ref{equ002010607}),(\ref{equ002010608})给出的。
\end{exercise}

\begin{exercise}\label{exer002020108}
设$(X, d)$是一度量空间,$Y \subset X$,又设$G \subset X$是开的,证明$G \cap Y$是$(Y, d)$中的开集。反之,如果$G_1 \subset Y$是$(Y, d)$中的开集,则存在开集$G \subset X$,使得$G_1 = G \cap Y$。
\end{exercise}

\begin{exercise}\label{exer002020109}
在上题中用“闭的”代替“开的”。
\end{exercise}

\begin{exercise}
证明命题\ref{pro:prop002020103}
\end{exercise}

\begin{exercise}
证明:$\{\cis{k}:k \ge 0\}$在$T=\{z \in \mathbb{C}:|z|=1\}$中是稠密的。对于哪些$\theta$的值,$\{\cis(k\theta):k>0\}$在$T$中是稠密的?
\end{exercise}

\section{连通性}\label{subsection0020202}
作为这一节的开始,让我们先给出一个例子。设$X = \{ x \in \mathbb{C} : |z| \le 1\} \cup \{z: |z-3| < 1\}$,并且把$\mathbb{C}$的度量赋于$X$(今后,当我们把$\mathbb{R}$或$\mathbb{C}$的子集$X$看作一个度量空间时,如果不作相反的声明,总假定$X$继承度量$d(z, w) = |z-w|$),那么集合$A = \{z:|z| \le 1\}$既是开的,又是闭的。它是闭的,因为它在$X$中的余集$B = X-A = \{z:|z-3| < 1\}$是开的;$A$是开的,因为如果$a \in A$,那么$B(a;1) \subset A$(注意:$\{z \in \mathbb{C}: |z-a| < 1\}$并不总包含在$A$中,当$a=1$时就是一例。但当按定义,$B(a;1)$是$z \in X: |z-a|<1$,它是包含在$A$中的)。类似的,$B$在$X$中也是既开又闭的。

这是一个非连通空间的例子。

\begin{definition}{连通}{def002020201}
一个度量空间$(X, d)$是连通的,如果只有$\emptyset$和$X$既是开的又是闭的。设$A \subset X$,如果度量空间$(A, d)$是连通的,那么$A$是$X$的连通子集。
\end{definition}

连通性的一个等价说法是:$X$是不连通的,如果存在$X$中的互不相交的非空开集$A$和$B$,使得$X = A \cup B$。事实上,如果这个条件成立,那么$A = X-B$也是闭的。

\begin{proposition}{}{prop002020201}
一个集$X \subset \mathbb{R}$是连通的,当且仅当$X$是一个区间。
\end{proposition}

\begin{proof}
设$X = [a, b]$,$a$,$b$是$\mathbb{R}$的元素。设$A \subset X$是$X$的开子集,满足$a \in A$,$A \neq X$。我们将证明$A$不可能也是闭的,因此$X$必是连通的。因为$A$是开的,$a \in A$,所以存在$\epsilon > 0$,使得$[a, a+\epsilon) \subset A$,设
\[
r = \sup\{\epsilon: [a, a+\epsilon) \subset A\}.
\]
则有断言:$[a, a+r) \subset A$。事实上,如果$a \le x < a+r$,令$h = a + r -x > 0$,由上确界的定义,存在$\epsilon$,$r - h < \epsilon < r$且$[a, a+\epsilon) \subset A$。但是$a \le x = a + (r-h) < a + \epsilon$蕴含$x \in A$。断言得证。

但是$a + r \not\in A$\footnote{这里有两种可能:(1)$a+r=b$;(2)$a+r < b$.当$a+r=b$时,$a + r \in A$导致$A=X$,与原来的假定$A \neq X$矛盾;作者忽略了这种情况。},因为在相反的情形,$a + r \in A$,那么由于$A$是开的,存在$\delta > 0$,使得$[a + r, a+r+\delta) \subset A$。但这就给出$[a, a+r+\delta) \subset A$。这与$r$的定义相矛盾。现在假定$A$也是闭的,那么$a + r \in B = X- A$,$B$是开的,因此我们可以找到$\delta > 0$,使得$(a+r-\delta, a+r] \subset B$。这和上述断言矛盾。

其他类型的区间的连通性的证明是类似的,留给读者作为习题。

$\mathbb{R}$中的连通集必是一区间,其证明留做习题。
\end{proof}

如果$w$和$z$是$\mathbb{C}$中的两点,那么我们用
\[
[z, w] = \{tw + (1-t)z:0 \le t \le 1\}
\]
表示从$z$到$w$的直线段,从$a$到$b$的折线是集$P = \bigcup_{k=1}^{n}{[z_k, w_k]}$。其中$z_1 = a$,$w_n = b$,并且对于$1 \le k \le n-1$,$w_k = w_{k+1}$;或者写成$P=[a,z_1,\cdots, z_n, b]$。

\begin{theorem}{}{thm002020201}
一个开集$G \subset \mathbb{C}$是连通的,当且仅当,对于$G$的任意两点$a$,$b$,存在一条从$a$到$b$的折线,这一折线整个地位于$G$内。
\end{theorem}

\begin{proof}
设$G$满足定理的条件,假定$G$不是连通的,我们将得到一个矛盾。由定义,$G = A \cup B$,其中$A$,$B$既是开集又是闭集,且$A \cap B = \emptyset$,$A$,$B$都是非空的,设$a \in A$,$b \in B$;按照假定,存在从$a$到$b$的一条折线$P$,$P \subset G$。稍加考虑,便可看出,在组成$P$的某一线段上, 有一点在$A$内,而另一点在$B$内,所以我们可以假定$P=[a, b]$。我们定义
\begin{gather*}
S = \{s \in [0, 1] : sb + (1-s)a \in A\},\\
T = \{t \in [0, 1] : tb + (1-t)a \in B\}.
\end{gather*}
那么,$S \cap T = \emptyset$,$S \cup T = [0,1]$,$0 \in S$,$1 \in T$。但是能够证明$S$和$T$都是开集(习题\ref{exer002020202}),这就和$[0,1]$的连通性矛盾。于是$G$一定是连通的。

现在设$G$是连通的,并且在$G$内固定一点$a$,要指出如何构造从$a$到$b$的折线(在$G$内!)是困难的,但是我们并不需要实现这种构造,而只要证明这一折线是存在的。对于$G$内固定的一点$a$,定义
\[
A = \{ b \in G: \text{存在}a\text{到}b\text{的折线}P \subset G\}.
\]
我们要证明$A$在$G$内既是开的又是闭的。由于$a \in A$和$G$是连通的,所以$A = G$,定理便得证。

为了证明$A$是开的,设$b \in A$,$P = [a, z_1, \cdots, z_n, b]$是从$a$到$b$的折线,$P \subset G$。由于$G$是开的(这对于定理的前半部分并不需要),存在$\epsilon > 0$,使得$B(b; \epsilon) \subset G$,但是如果$z \in B(b; \epsilon)$,那么$[b, z] \subset B(b;\epsilon) \subset G$,因此,$Q = P \cup [b,z]$是$G$内从$a$到$z$的折线,这就表明$B(b; \epsilon) \subset A$,所以$A$是开的。

为了证明$A$是闭的,假设在$G-A$内有一点$z$,及$\epsilon > 0$使得$B(z; \epsilon) \subset G$。如果$A \cap B(z; \epsilon)$内存在一点$b$,那么如上所述,我们能够构造一条从$a$到$z$的折线。于是我们必有$B(z;\epsilon) \cap A = \emptyset$,或者$B(z;\epsilon) \subset G-A$。即$G-A$是开的,所以$A$是闭的。
\end{proof}

\begin{corollary}{}{coro002020201}
如果$G\subset{}\mathbb{C}$是开的,连通的,$a$,$b$是$G$内的点,那么在$G$内存在一条从$a$到$b$的折线,这一折线由平行于实轴和平行于虚轴的线段所组成。
\end{corollary}

\begin{proof}
证明这个推论的方法有两个。一个方法是先在$G$内求得一条从$a$到$b$的折线。然后修改其每一线段,使得新的折线具有所要的性质。利用紧性比较容易实现这个证明(见本章\ref{section0020205}节习题\ref{exer002020507})。另一个证明可以由修改定理\ref{thm:thm002020201}的证明而得到。和定理\ref{thm:thm002020201}的证明一样,定义集$A$,但附加一个限制,就是折线的线段都平行于一个坐标轴。往下的证明仍然有效,只有一点例外,就是如果$z \in B(b; \epsilon)$,那么$[b,z]$可能不平行于坐标轴,但是容易看出,如果$z=x+iy$,$b=p+iq$,那么折线$[b,p+iy]\cup [p+iy, z] \subset B(b;\epsilon)$,且它的线段平行于坐标轴。
\end{proof}

现在我们将证明,度量空间的任意一个集可以用典型的方法表示为连通块的和。

\begin{definition}{}{def002020205}
度量空间$X$的子集$D$是$X$的一个分支,如果它是$X$的最大连通子集。即$D$是连通的,并且不存在$X$的连通子集,$D$是它的真子集。
\end{definition}

如果读者考察这一节一开始给出的例子,就会发现$A$,$B$都是分支。并且$X$只有这两个分支。作为另一个例子,设$X = \{0, 1, \frac{1}{2},\frac{1}{3}, \cdots\}$,这时显然$X$的每一个分支都是一个点,并且它的每一个点都是一个分支。注意,分支$\{\frac{1}{n}\}$都是$X$中的开集,分支$\{0\}$不是$X$中的开集。

\begin{lemma}{}{lemma002020206}
设$x_0 \in X$,$\{D_j : j \in J\}$是$X$的连通子集族,对于$J$中的每一个$j$,$x_0 \in D_j$。则$D = \bigcup{\{D_j : j \in J\}}$是连通的。
\end{lemma}

\begin{proof}
设$A$是度量空间$(D, d)$的子集,它既是开的又是闭的,且设$A \neq \emptyset$。那么对于每个$j$,$A \cap D_j$是$(D_j, d)$中的开集,也是$(D_j, d)$中的闭集(见\ref{section0020201}节中的习题\ref{exer002020108}和习题\ref{exer002020109})。由于$D_j$是连通的,所以,或者$A \cap D_j = \emptyset$,或者$A \cap D_j = D_j$。因为$A \neq \emptyset$,所以至少存在一个$k$,使得$A \cap D_k \neq \emptyset$;因此$A \cap D_k = D_k$,特别地,$x_0 \in A$。所以,对于每个$j$,$x_0 \in A \cap D_j$,于是对于每个$j$,$A \cap D_j = D_j$,或者说$D_j \subset A$。这就得到$D = A$,所以$D$是连通的。
\end{proof}

\begin{theorem}{}{thm002020207}
设$(X, d)$是一个度量空间,则
\begin{enumerate}
\item[(a)]$X$中的每一个$x_0$包含在$X$的一分支中;
\item[(b)]$X$的不同分支是互不相交的。
\end{enumerate}
\end{theorem}

注意,(a)表示$X$是它的分支的和。

\begin{proof}
(a)设$\mathscr{D}$是包含$x_0$的$X$的连通子集族。注意到$\{x_0\} \in \mathscr{D}$,所以$\mathscr{D} \neq \emptyset$。也注意到上述引理的假设适用于族$\mathscr{D}$,因此$C = \bigcup\{D; D \in \mathscr{D}\}$是连通的,且$x_0 \in C$。$C$必定是一个分支。事实上,如果$D$是连通的,$C \subset D$,那么$x_0 \in D$,所以$D \in \mathscr{D}$。但是这样一来,$D \subset C$,所以$C=D$。于是$C$是最大的。(a)得证。

(b)设$C_1$,$C_2$是两个分支,$C_1 \neq C_2$,假定在$C_1 \cap C_2$内存在一点$x_0$,再由引理,$C_1 \cup C_2$是连通的,由于$C_1$,$C_2$都是分支,这就给出$C_1 = C_1 \cup C_2 = C_2$,矛盾。
\end{proof}

\begin{proposition}{}{prop002020208}
(a)如果$A \subset X$是连通的,$A \subset B \subset A^-$,那么$B$是连通的;(b)如果$C$是$X$的一分支,那么$C$是闭的。
\end{proposition}

证明留给读者作为习题。

\begin{theorem}{}{thm002020209}
设$G$是$\mathbb{C}$中的开集,那么$G$的分支是开集,并且$G$只有可数个分支。 
\end{theorem}

\begin{proof}
设$C$是$G$的一分支,$x_0 \in C$。由于$G$是开集,所以存在$\epsilon > 0$,使得$B(x_0;\epsilon) \subset G$。根据引理,$B(x_0;\epsilon) \cup C$是连通的,所以它必是$C$。即$B(x_0;\epsilon) \subset C$,所以$C$是开的。

为了看出分支的个数是可数的,设$S = \{a + ib: a, b\text{是有理数,且}a+ib \in G\}$,那么$S$是可数的。$G$的每个分支包含$S$的一点,所以分支的个数是可数的。
\end{proof}

\begin{exercise}\label{exer002020201}
本习题的目的在于证明$\mathbb{R}$的连通子集是一个区间。
\begin{enumerate}
\item[(a)]证明:当且仅当对于$A$中的任意两点$a$, $b$,$a < b$,有$[a, b] \subset A$时,集$A \subset \mathbb{R}$是一个区间。
\item[(b)]利用(a)证明:如果$A \subset \mathbb{R}$是连通的,那么$A$是一个区间。
\end{enumerate}
\end{exercise}

\begin{exercise}\label{exer002020202}
证明定理\ref{thm:thm002020201}的证明中的集$S$和$T$是开集。
\end{exercise}

\begin{exercise}
$\mathbb{C}$中的下列子集$X$,哪些是连通的?如果$X$不是连通的,它的分支是什么?
\begin{enumerate}
\item[(a)]$X = \{z: |z| \le 1\} \cup \{z:|z-2| < 1\}$;
\item[(b)]$X = [0, 1] \cup \{1 + \frac{1}{n}: n > 1\}$;
\item[(c)]$X = \mathbb{C} - (A \cup B)$,其中$A = [0, \infty)$,$B = \{z = r\cis{\theta} : r = \theta, 0 \le \theta \le \infty \}$。
\end{enumerate}
\end{exercise}

\begin{exercise}
证明引理\ref{lem:lemma002020206}的下述推广:如果$\{D_j:j \in J\}$是$X$的连通子集族,且对于$J$中的每个$j$和$k$,有$D_j \cap D_k \neq \emptyset$,那么$D = \bigcup\{D_j: j \in J\}$是连通的。
\end{exercise}

\begin{exercise}
证明:如果$F \subset X$是闭的、连通的,那么对于$F$中的每对点$a$,$b$和每个$\epsilon > 0$,在$F$中存在点$z_0, z_1,\cdots z_n$,$z_0=a$,$z_n=b$,且对于$1 \le k \le n$,$d(z_{k-1}, z_k) < \epsilon$,$F$是闭的这个假定是必要的吗?如果$F$是一个满足这个性质的集,即使$F$是闭的,也不一定是连通的。试举例说明之。
\end{exercise}


\section{序列与完备性}\label{section0020203}
在度量空间中,最有用的概念之一是收敛序列的概念,这一概念在度量空间和复分析中与在微积分中一样起着中心的作用。

\begin{definition}{}{def002020301}
设$\{x_1,x_2,\cdots\}$是度量空间$(X,d)$中的一个序列,说$\{x_n\}$收敛到$x$,如果对于每个$\epsilon > 0$,存在正整数$N$,使得$n > N$时,$d(x, x_n) < \epsilon$,记为$x = \lim{x_n}$或$x_n \to x$。
\end{definition}

换言之,$x = \lim{x_n}$,如果$0 = \lim{d(x, x_n)}$。

如果$X = \mathbb{C}$,那么$z = \lim{z_n}$意味着,对于每个$\epsilon > 0$,存在正整数$N$,使得当$n > N$时,$|z - z_n| < \epsilon$。

在度量空间的理论中,许多概念可以借助于序列来叙述。下面是一个例子。

\begin{proposition}{}{prop002020302}
一个集$F \subset X$是闭的,当且仅当对于$F$中的每个序列$\{x_n\}$,若$x = \lim{x_n}$,则$x \in F$。
\end{proposition}

\begin{proof}
设$F$是闭的,$x = \lim{x_n}$,其中每个$x_n$在$F$中。所以对于每个$\epsilon > 0$,在$B(x;\epsilon)$中有一点$x_n$;即$B(x;\epsilon) \cap F \neq \emptyset$。所以由命题\ref{pro:prop002020103}(f),$x \in F^- = F$。

现在设$F$不是闭的,所以在$F^-$中有$x_0$,$x_0$不在$F$中。由命题\ref{pro:prop002020103}(f),对于每个$\epsilon > 0$,有$B(x_0;\epsilon) \cap F \neq \emptyset$。特别地,对于每个正整数$n$,在$B(x_0; \frac{1}{n}) \cap F$中有点$x_n$。于是$d(x_0, x_n) < \frac{1}{n}$,这就蕴含$x_n \to x_0$。由于$x_0 \not\in F$,这就是说定理的条件不满足。
\end{proof}

\begin{definition}{}{def002020303}
设$A \subset X$。那么$X$中的点$x$是$A$的极限点,如果在$A$中存在由不同点构成的序列$\{x_n\}$,使得$x = \lim{x_n}$。
\end{definition}

在这个定义中“不同”二字的理由可由下面的例子得到解释。设$X = \mathbb{C}$,$A = [0, 1] \cup \{i\}$;$[0, 1]$中的每一点是$A$的极限点,但$i$不是$A$的极限点,我们不能指望把$i$这样的点叫做极限点。但是如果把“不同”二字从定义中删去,我们就可以对每个$n$取$x_n = i$,有$i = \lim{x_n}$。

\begin{proposition}{}{prop002020304}
(a)一个集合是闭的,当且仅当,它包含它的所有极限点;(b)如果$A \subset X$,那么$A^-=A \cup \{x: x\text{是}A\text{的极限点}\}$。
\end{proposition}

证明留做习题。

从实分析中我们知道,$\mathbb{R}$的基本性质是:任意一个序列,当$n$增大时它的项变得越来越接近,则它一定是收敛的。这种序列称为Cauchy序列。这种序列的属性之一是它的极限一定存在,尽管你不能求出它。

\begin{definition}{Cauchy序列}{def002020305}
序列$\{x_n\}$称为Cauchy序列,如果对于每个$\epsilon > 0$,都存在一个正整数$N$,使得对所有的$n$,$m \ge N$,有$d(x_n, x_m) < \epsilon$。
\end{definition}

如果$(X, d)$有性质:每个Cauchy序列在$X$中有极限,那么$(X, d)$是完备的。

\begin{proposition}{}{prop002020306}
$\mathbb{C}$是完备的。
\end{proposition}

\begin{proof}
如果$\{x_n + iy_n\}$是$\mathbb{C}$中的Cauchy序列,那么$\{x_n\}$和$\{y_n\}$是$\mathbb{R}$中的Cauchy序列,由于$\mathbb{R}$是完备的,所以$x_n \to x$,$y_n \to y$,$x, y$在$\mathbb{R}$中。由此推出,$x+iy=\lim{(x_n + iy_n)}$,所以$\mathbb{C}$是完备的。
\end{proof}

考虑具有度量$d$(见第\ref{section00201}章的\ref{equ002010608}和\ref{equ002010607})的$C_{\infty}$。设$z_n$,$z$是$\mathbb{C}$中的点,可以证明$d(z_n, z) \to 0$,当且仅当$|z_n-z| \to 0$。尽管如此,注意序列$\{z_n\}$,$\lim{|z_n|} = \infty$是$\mathbb{C}_{\infty}$中的Cauchy序列,但是它不是$\mathbb{C}$中的Cauchy序列。

如果$A \subset X$,我们把$\diam{A} = \sup{\{d(x, y):x\text{和}y\text{在}A\text{中}\}}$定义为$A$的直径。

\begin{theorem}{Cantor定理}{thm002020307}
度量空间$(X, d)$是完备的,当且仅当,任意满足条件$F_1 \supset F_2 \supset \cdots$和$\diam{F_n} \to 0$,非空闭集序列$\{F_n\}$,其交集$\bigcap_{n=1}^{\infty}{F_n}$是由一个点所组成。
\end{theorem}

\begin{proof}
设$(X, d)$是完备的,$\{F_n\}$是一个闭集序列,具有性质:(i)$F_1 \supset F_2 \supset \cdots$;(ii)$\lim{\diam{F_n}} = 0$。对于每个$n$,设$x_n$是$F_n$种的任意一点,如果$n, m \ge N$,那么$x_n$,$x_m$在$F_N$中,由定义,$d(x_n, x_m) \le \diam{F_N}$。由假定,$N$可玄德充分大,使得$\diam{F_N} < \epsilon$;这就表明$\{x_n\}$是Cauchy序列。由于$X$是完备的,所以$x_0 = \lim{x_n}$存在。又对于所有的$n \ge N$,因为$F_n \subset F_N$,所以$x_n$在$F_N$中;因此,对于每个$N$,$x_0$在$F_N$中,这就给出$x_0 \in \bigcap_{n=1}^{\infty}{F_n} =F$。所以$F$至少包含一个点。如果$y$也在$F$中,那么对于每个$n$,$x_0$和$y$都在$F_n$中,这就给出$d(x_0, y) \le \diam{F_n} \to 0$,所以$d(x_0, y) = 0$,或者$x_0 = y$。

现在,如果$X$满足所述的条件,我们来证明$X$是完备的。设$\{x_n\}$是$X$中的Cauchy序列,又设$F_n = \{x_n, x_{n+1},\cdots\}^-$;那么$F_1 \supset F_2 \supset \cdots$。如果$\epsilon > 0$,选取$N$,使得对于每个$n, m \ge N$,都有
\[
d(x_n, x_m) < \epsilon;
\]
这就表示对于$n \ge N$,$\diam\{x_n, x_{n+1}, \cdots\} \le \epsilon$,所以对于$n \ge N$,$\diam{F_n} \le \epsilon$(习题\ref{exer002020303})。于是$\diam{F_n} \to 0$,并且,按照假设,在$X$中存在点$x_0$,$\{x_0\} = F_1 \cap F_2 \cap \cdots$。特别地,$x_0$在$F_n$中,$d(x_0, x_n) \le \diam{F_n} \to 0$,所以$x_0 = \lim{x_n}$。
\end{proof}

有一个典型习题与这个定理有联系,就是在$\mathbb{R}$中找一个集序列$\{F_n\}$,它满足下面的条件中的两个条件:
\begin{enumerate}
\item[(a)]每个$F_n$是闭的;
\item[(b)]$F_1 \supset F_2 \supset \cdots$;
\item[(c)]$\diam{F_n} \to 0$。
\end{enumerate}
但是$F = F_1 \cap F_2 \cap \cdot$或者是空的,或者多于一点,对于两个条件的各种可能选择,读者都应举出相应的例子。

\begin{proposition}{}{prop002020308}
设$(X, d)$是一个完备的度量空间,$Y \subset X$,当且仅当$Y$在$X$中是闭时$(Y, d)$是一个完备度量空间。
\end{proposition}

\begin{proof}
当$Y$是闭子集时,$(Y, d)$是完备的。其证明留给读者作为习题。现在设$(Y, d)$是完备的,$x_0$是$Y$的极限点,那么在$Y$中有序列$\{y_n\}$,使得$x_0 = \lim{y_n}$。因此$\{y_n\}$是Cauchy序列(习题\ref{exer002020305}),并且因为$(Y, d)$是完备的,所以$\{y_n\}$一定收敛到$Y$中的$y_0$。由此推得$y_0 = x_0$,所以$Y$包含它的所有极限点。由命题\ref{pro:prop002020304},$Y$是闭的。
\end{proof}

\begin{exercise}
证明命题\ref{pro:prop002020304}。
\end{exercise}

\begin{exercise}
完成命题\ref{pro:prop002020308}的详细证明。
\end{exercise}

\begin{exercise}\label{exer002020303}
证明:$\diam{A} = \diam{A^-}$。
\end{exercise}

\begin{exercise}\label{exer002020304}
设$z_n$,$z$是$\mathbb{C}$中的点,$d$是$\mathbb{C}_{\infty}$中的度量,证明$|z_n - z| \to 0$,当且仅当,$d(z_n, z) \to 0$。证明:如果$|z_n| \to \infty$,那么$\{z_n\}$是$\mathbb{C}_{\infty}$中的Cauchy序列。($\{z_n\}$在$\mathbb{C}_{\infty}$中一定收敛吗?)
\end{exercise}

\begin{exercise}\label{exer002020305}
证明:$(X, d)$中的每个收敛序列一定是Cauchy序列。
\end{exercise}

\begin{exercise}\label{exer002020306}
给出三个不完备度量空间的例子。
\end{exercise}

\begin{exercise}\label{exer002020307}
在$\mathbb{R}$上作一度量$d$,满足条件:$|x_n - x| \to 0$,当且仅当,$d(x_n, x) \to 0$,而当$|x_n| \to \infty$时,$\{x_n\}$是$(\mathbb{R}, d)$中的Cauchy序列。(提示:从$\mathbb{C}_{\infty}$得到启示。)
\end{exercise}

\begin{exercise}\label{exer002020308}
设$\{x_n\}$是Cauchy序列,且$\{x_{n_k}\}$是收敛子序列,证明:$\{x_n\}$一定是收敛的。
\end{exercise}



\section{紧性}\label{section0020204}
紧性的概念是把有限集中一些好的性质推广到无穷集去,紧集的大部分性质类似于有限集的性质,这些性质在有限集是很平凡的。例如,有限集的每个序列有收敛子序列。这是平凡的,因为至少有一点重复无穷多次。当我们把“有限”代之以“紧”时,这个结果仍然成立。

\begin{definition}{紧集}{def002020401}
度量空间$X$的子集$K$是紧的,如果对于$X$中的每个具有性质
\begin{gather}\label{equ002020402}
K \subset \bigcup\{G: G \in \mathscr{G}\},
\end{gather}
的开集族$\mathscr{G}$都可在$\mathscr{G}$中找到有限个集$G_1, G_2, \cdots, G_n$,使得$K \subset G_1 \cup G_2 \cup \cdots \cup G_n$。满足(\ref{equ002020402})的集族$\mathscr{G}$称为$K$的覆盖;如果$\mathscr{G}$的每个集是开的,则称它是$K$的开覆盖。
\end{definition}

显然,空集和所有的有限集是紧的。$D = \{z \in \mathbb{C}:|z|<1\}$是一个非紧集的例子。如果$G_n = \big\{z : |z| < 1 - \frac{1}{n}\big\}$,$n=2,3,\cdots$,那么$\{G_2, G_3, \cdots\}$是$D$的一个开覆盖,但它没有有限子覆盖。

\begin{proposition}{}{prop002020403}
设$K$是$X$的一个紧子集,那么
\begin{enumerate}
\item[(a)]$K$是闭的;
\item[(b)]如果$F$是闭的,且$F \subset K$,则$F$是紧的。
\end{enumerate}
\end{proposition}

\begin{proof}
为了证明(a),我们要证明$F = F^-$。设$x_0 \in K^-$,由命题\ref{pro:prop002020103}(f),对于每个$\epsilon > 0$,$B(x_0;\epsilon) \cap K \neq \emptyset$。设
\[
G_n = X - \bar{B}(x_0; \frac{1}{n}),
\]
并假定$x_0 \not\in K$,那么每个$G_n$是开集,且$K \subset \bigcup_{n=1}^{\infty}{G_n}$(因为$\bigcap_{n=1}^{\infty}{\bar{B}(x_0; \frac{1}{n})} = \{x_0\}$)。因为$K$是紧的,所以存在正整数$m$,使得$K \subset \bigcup_{n=1}^{m}{G_n}$。但是$G_1 \subset G_2 \subset \cdots$,所以$K \subset G_m = X - \bar{B}(x_0;\frac{1}{m})$。但这就给出$B(x_0; \frac{1}{m}) \cap K = \emptyset$,从而得到一个矛盾。于是$K = K^-$。

为了证明(b),设$\mathscr{G}$是$F$的一个开覆盖。那么由于$F$是闭的,$\mathscr{G} \cup \{X-F\}$是$K$的开覆盖。设$G_1, G_2, \cdots, G_n$是$\mathscr{G}$中的集,使得$K \subset G_1 \cup \cdots \cup G_n \cup (X-F)$。显然,$F \subset G_1 \cup \cdots \cup G_n$,所以$F$是紧的。
\end{proof}

设$\mathscr{F}$是$X$的子集族,我们说$\mathscr{F}$有有限交性质$(f, i, p)$如果,只要$\{F_1, F_2, \cdots, F_n\} \subset \mathscr{F}$,总有$F_1 \cap F_2 \cap \cdots \cap F_n \neq \emptyset$。这种子集族的一个例子是$\{D - G_2, D-G_3, \cdots\}$,其中集$G_n$是命题\ref{pro:prop002020403}之前所述例子中的集。

\begin{proposition}{}{prop002020404}
一个集合$K \subset X$是紧的,当且仅当,$K$中每个具有$(f,i,p)$的闭子集\footnote{这里的“闭子集”应是“相对于集合$K$的闭子集”。否则命题的充分性不真,必要性的证明应作相应的修改。}族$\mathscr{F}$,都有$\bigcap\{F: F \in \mathscr{F}\} \neq \emptyset$。
\end{proposition}

\begin{proof}
设$K$是紧的,$\mathscr{F}$是具有$(f,i,p)$的$K$中的闭子集族。假设$\bigcap\{F: F \in \mathscr{F}\} = \emptyset$。令$\mathscr{P} = \{X - F: F \in \mathscr{F}\}$,那么由假设$\bigcup\{X - F: F \in \mathscr{F}\} = X - \bigcap\{F: F \in \mathscr{F}\} = X$。特别地,$\mathscr{P}$是$K$的开覆盖,于是存在$F_1, \cdots, F_n \in \mathscr{F}$,使得$K \subset \bigcup_{k=1}^{n}{(X - F_k)} = X - \bigcap_{k=1}^{n}{F_k}$。但这就给出$\bigcap_{k=1}^{n}{F_k} = X - K$,由于每个$F_k$是$K$的子集,所以必有$\bigcap_{k=1}^{n}{F_k} = \emptyset$。这与$\mathscr{F}$具有$(f,i,p)$相矛盾。

条件的充分性的证明留给读者作为习题。
\end{proof}

\begin{corollary}{}{coro002020405}
每个紧的度量空间是完备的。
\end{corollary}

\begin{proof}
这容易由上面的命题和定理\ref{thm:thm002020307}推出。
\end{proof}

\begin{corollary}{}{coro002020406}
如果$X$是紧的,那么每个无穷集在$X$中至少有一个极限点。
\end{corollary}

\begin{proof}
设$S$是$X$的一无穷子集,假设$S$没有极限点。设$\{a_1, a_2, \cdots\}$是$S$中不同点的序列,那么$F_n = \{a_n, a_{n+1}, \cdots\}$也没有极限点。但是如果一个集合没有极限点,那么也可以说它包含了它的所有极限点,因而它是闭集!于是每个$F_n$是闭的,且$\{F_n : n \ge 1\}$具有$(f,i,p)$。但是由于点$a_1, a_2,\cdots$是不同的,所以$\bigcap_{n=1}^{\infty}{F_n} = \emptyset$。这与上面的命题\ref{pro:prop002020404}相矛盾。
\end{proof}

\begin{definition}{列紧性}{def002020407}
称一个度量空间$(X, d)$是列紧的,如果$X$中的每个序列都有收敛子序列。
\end{definition}

我们将证明度量空间的紧性和列紧性是一回事,为此需要下面的引理。

\begin{lemma}{Lebesque覆盖引理}{lemma002020408}
如果$(X, d)$是列紧的,$\mathscr{G}$是$X$的开覆盖,那么存在$\epsilon > 0$,使得$X$中的每个$x$,都存在$\mathscr{G}$中的一个集$G$,满足$B(x; \epsilon) \subset G$。
\end{lemma}

\begin{proof}
用反证法,设$\mathscr{G}$是$X$的开覆盖,而这样的$\epsilon$不存在。特别地,对于每个正整数$n$,在$X$中有点$x_n$,使得$B(x_n;\frac{1}{n})$不包含在$\mathscr{G}$中任一个集$G$内。因为$X$是列紧的,所以在$X$中存在点$x_0$和序列$\{x_{n_k}\}$,使得$x_0 = \lim{x_{n_k}}$。设$G_0 \in \mathscr{G}$,$x_0 \in G_0$;选取$\epsilon >0$使得$B(x_0;\epsilon) \subset G_0$。现在设$N$是这样的正整数,对于所有的$n_k \ge N$,都有$d(x_0;x_{n_k}) \le \epsilon / 2$。设$n_k$是比$N$和$2/\epsilon$都大的任意正整数,$y \in B(x_{n_k};\frac{1}{n_k})$,那么$d(x_0, y) \le d(x_0, x_{n_k}) + d(x_{n_k}, y) < \epsilon / 2 + 1/n_k < \epsilon$。即$B(x_{n_k}; \frac{1}{n_k}) \subset B(x_0;\epsilon) \subset G_0$,这和$x_{n_k}$的取法相矛盾。
\end{proof}

对于Lebesque覆盖引理通常有两种误解。一是言之未及,一是言过其实。由于$\mathscr{G}$是$X$的开覆盖,所以$X$的每个$x$包含在$\mathscr{G}$的某一个$G$内;因为$G$是开集,于是存在$\epsilon > 0$使得$B(x;\epsilon) \subset G$。但是引理所给出的$\epsilon > 0$是使得对于任意的$x$,$B(x; \epsilon)$都包含在$\mathscr{G}$的某一集内。另一种误解是,以为对于引理中所得到的$\epsilon > 0$,$B(x; \epsilon)$包含在$\mathscr{G}$中含有$x$的每个$G$内。

\begin{theorem}{}{thm002020409}
设$(X, d)$是一个度量空间,那么下列条件是等价的:
\begin{enumerate}
\item[(a)]$X$是紧的;
\item[(b)]$X$中的每个无穷集至少有一个极限点;
\item[(c)]$X$是列紧的;
\item[(d)]$X$是完备的,并且对于每个$\epsilon > 0$,$X$内存在有穷多个点$x_1,x_2,\cdots, x_n$,使得
\[
X = \bigcup_{k=1}^{n}{B(x_k;\epsilon)}.
\]
((d)中所述的性质称为完全有界性。)
\end{enumerate}
\end{theorem}

\begin{proof}
由推论\ref{cor:coro002020406},(a)蕴含(b)。

(b)蕴含(c)。设$\{x_n\}$是$X$中的一个序列,不失一般性,假定点$x_1,x_2,\cdots$是不同的。由(b),集合$\{x_1, x_2,\cdots\}$有一个极限点$x_0$。于是有点$x_{n_1} \in B(x_0; 1)$,类似的,有正整数$n_2 > n_1$,$x_{n_2} \in B(x_0; \frac{1}{2})$,如此继续下去,我们得到正整数$n_1 < n_2 < \cdots$,$x_{n_k} \in B(x_0;\frac{1}{k})$。于是$x_0 = \lim{x_{n_k}}$。所以$X$是列紧的。

(c)蕴含(d)。设$\{x_n\}$是Cauchy序列,应用列紧性的定义和借助于\ref{section0020203}节的习题\ref{exer002020308},便可看出$X$是完备的。

现在设$\epsilon > 0$,固定$x_1 \in X$。如果$X = B(x_1;\epsilon)$,那么结论得证。否则选取$x_2 \in X - B(x_1; \epsilon)$。如果$X = B(x_1;\epsilon) \cup B(x_2;\epsilon)$,结论也得证。否则设$x_3 \in X - [B(x_1;\epsilon) \cup B(x_2;\epsilon)]$。如果这个过程不会终止,我们就得到一序列$\{x_n\}$,使得
\[
x_{n+1} \in X - \bigcup_{k=1}^{n}{B(x_k; \epsilon)}.
\]
但是这蕴含对于$n \neq m$,$d(x_n, x_m) \ge \epsilon > 0$。于是$\{x_n\}$没有收敛子列,这与(c)相矛盾。

(d)蕴含(c)。这部分证明用到“鸽巢原理”,这个原理可表述为:如果物件数多于容器数,那么至少有一个容器里装的物件多于一个。进而,如果无穷多个点包含在有穷多个球里,那么有一个球包含无穷多个点,所以(d)是说,对于每个$\epsilon > 0$和$X$中的无穷集,存在点$y \in X$,使得$B(y; \epsilon)$包含这个集的无穷多个点。设$\{x_n\}$是一个由不同点组成的序列。在$X$中存在点$y_1$和$\{x_n\}$的子序列$\{x_{n}^{(1)}\}$,使得$\{x_n^{(1)}\} \subset B(y_1;1)$。又存在$X$中的$y_2$和$\{x_n^{(1)}\}$的子序列$\{x_n^{(2)}\}$,使得$\{x_n^{(2)}\} \subset B(y_2; 1/2)$。如此继续下去,对于每个正整数$k \ge 2$,存在$X$中的$y_k$和$\{x_n^{(k-1)}\}$的子序列$\{x_n^{(k)}\}$,使得$\{x_n^{(k)}\} \subset B(y_k; 1/k)$。设$F_k = \{x_n^{(k)}\}^-$,那么$\diam{F_k} \le 2/k$,且$F_1 \supset F_2 \supset \cdots$。根据定理\ref{thm:thm002020407},$\bigcap_{k=1}^{\infty}{F_k} = \{x_0\}$。我们断言$x_k^{(k)} \to x_0$($x_k^{(k)}$是$\{x_n\}$的子序列)。事实上,$x_0 \in F_k$,所以$d(x_0;x_k^{(k)}) \le \diam{F_k} \le 2/k$,$x_0 = \lim{x_k^{(k)}}$。

(c)蕴含(a)。设$\mathscr{G}$是$X$的一个开覆盖。上面的引理给出一个$\epsilon > 0$,使得对于每个$x \in X$,$\mathscr{G}$中存在一个集$G$,$B(x; \epsilon) \subset G$。现在已知(c)蕴含(d),因此,在$X$中存在点$x_1, \cdots, x_n$,使得$X = \bigcup_{k=1}^{n}{B(x_k;\epsilon)}$。现在对于$1 \le k \le n$,存在集$G_k \in \mathscr{G}$,$B(x_k;\epsilon) \subset G_k$。因此$X = \bigcup_{k=1}^{n}{G_k}$,即$\{G_1,\cdots, G_n\}$是$\mathscr{G}$的有限子覆盖。
\end{proof}

\begin{theorem}{Heine-Borel定理}{thm002020410}
当且仅当$K$是有界闭集时,$R^{n}$($n \ge 1$)是一个集$K$是紧的。
\end{theorem}

\begin{proof}
如果$K$是紧的,那么由前一定理的(d),$K$是完全有界的。由命题\ref{pro:prop002020403}推出$K$一定是闭的。容易证明完全有界的集也是有界的。

现在假设$K$是有界闭集。因此存在实数$a_1,a_2,\cdots,a_n$和$b_1,b_2,\cdots,b_n$,使得$K \subset F = [a_1, b_1] \times \cdots [a_n, b_n]$。如果能够证明$F$是紧的,那么因为$K$是闭的,就可推知$K$是紧的(命题\ref{pro:prop002020403}(b))。由于$\mathbb{R}^n$是完备的和$F$是闭的,推知$F$是完备的。因此,再次应用前一定理中的(d),我们只需证明$F$是完全有界的。这是容易的,虽然写起来有点繁。设$\epsilon > 0$;现在我们将$F$写成$n$维矩形的和,其中每个矩形的直径小于$\epsilon$。这样,我们有$F \subset \bigcup_{k=1}^{\infty}{B(x_k; \epsilon)}$,其中每个$x_k$属于前面提到的矩形中的某一个。这个做法的细节留给读者作为习题去完成(习题\ref{exer002020403})。
\end{proof}

\begin{exercise}
完成命题\ref{exer002020404}的证明。
\end{exercise}

\begin{exercise}\label{exer002020402}
设$p = (p_1,p_2,\cdots, p_n)$,$q = (q_1, q_2, \cdots, q_n)$是$\mathbb{R}$中的点,并且对于每个$k$,$p_k < q_k$。设$R = [p_1, q_1] \times \cdots \times [p_n,q_n]$。证明
\[
\diam{R} = d(p, q) = [\sum_{k=1}^{n}{(q_k - p_k)^2}]^{\frac{1}{2}}.
\]
\end{exercise}

\begin{exercise}\label{exer002020403}
设$F = [a_1, b_1] \times \cdots \times [a_n, b_n] \subset \mathbb{R}^n$,$\epsilon > 0$,利用习题\ref{exer002020402}证明:存在矩形$R_1, R_2, \cdots, R_m$,使得$F = \bigcup_{k=1}^{m}{R_k}$,并且对于每个$k$,$\diam{R_k} < \epsilon$。如果$x_k \in R_k$,那么由此推出$R_k \subset B(x_k; \epsilon)$。
\end{exercise}

\begin{exercise}\label{exer002020404}
证明:有穷多个紧集的和是紧的。
\end{exercise}

\begin{exercise}
设$X$是所有有界复数序列的集。也就是$\{x_k\} \in X$,当且仅当,$\sup\{|x_n|:n \ge 1\} < \infty$。如果$x = \{x_n\}$和$y = \{y_n\}$,定义$d(x, y) = \sup\{|x_n-y_n|: n \ge 1\}$。证明:对于$X$中的每个$x$和$\epsilon > 0$,$\bar{B}(x; \epsilon)$不是完全有界的,尽管它是完备的。(提示:如果首先证明可以假定$x=(0, 0, \cdots, 0)$,事情就容易了)。
\end{exercise}

\begin{exercise}
证明:完全有界的集的闭包是完全有界的。
\end{exercise}



\section{连续性}\label{section0020205}
函数最基本的性质之一是连续性。有了连续性就保证了一定程度的正则性和光滑性。否则,要得到度量空间上的任何函数理论是困难的。由于本书的主题是具有导数的(所以也是连续的)一个复变数的函数论,所以连续性的研究是基本的。

\begin{definition}{连续}{def002020501}
设$(X, d)$和$(\Omega, \rho)$是度量空间,$f:X \to \Omega$是一个函数。设$a \in X$,$\omega \in \Omega$,如果对于每个$\epsilon > 0$,都存在$\delta > 0$,使得只要$0 < d(x, a) < \delta$就有$\rho(f(x), \omega) < \epsilon$,那么就说$\lim\limits_{x \to a}{f(x)} = \omega$。如果$\lim\limits_{x \to a}{f(x)} = f(a)$就说函数$f$在点$a$是连续的,如果$f$在$X$的每一点都是连续的,那么就称$f$是从$X$到$\Omega$的连续函数。
\end{definition}

\begin{proposition}{}{prop002020502}
设$f: (X, d) \to (\Omega, \rho)$是一个函数,$a \in X$,$\alpha = f(a)$。下列事实是等价的:
\begin{enumerate}
\item[(a)]$f$在$a$点是连续的;
\item[(b)]对于每个$\epsilon > 0$,$f^{-1}(B(\alpha; \epsilon))$包含一个以$a$为中心的球;
\item[(c)]$\alpha = \lim{f(x_n)}$,只要$a = \lim{x_n}$。
\end{enumerate}
\end{proposition}

证明留给读者作为习题。

这是关于函数在一点的连续性的最后一个命题,从现在起,我们将只涉及在$X$的所有点上连续的函数。

\begin{proposition}{}{prop002020503}
设$f: (X, d) \to (\Omega, \rho)$是一个函数,下列事实是等价的:
\begin{enumerate}
\item[(a)]$f$是连续的;
\item[(b)]如果$\Delta$是$\Omega$中的开集,那么$f^{-1}(\Delta)$是$X$中的开集;
\item[(c)]如果$\Gamma$是$\Omega$中的闭集,那么$f^{-1}(\Gamma)$是$X$中的闭集。
\end{enumerate}
\end{proposition}

\begin{proof}
(a)蕴含(b). 设$\Delta$是$\Omega$中的开集,$x \in f^{-1}(\Delta)$。如果$\omega = f(x)$,那么$\omega$在$\Delta$内;由定义,存在$\epsilon > 0$,使得$B(\omega, \epsilon) \subset \Delta$。由于$f$是连续的,所以由上一命题的(b)给出一$\delta > 0$,使得$B(x, \delta) \subset f^{-1}(B(\omega; \epsilon)) \subset f^{-1}(\Delta)$。因此$f^{-1}(\Delta)$是开的。

(b)蕴含(c). 如果$\Gamma \subset \Omega$是闭的,那么$\Delta = \Omega - \Gamma$是开的。由(b),$f^{-1}(\Delta) = X - f^{-1}(\Gamma)$是开的,所以$f^{-1}(\Gamma)$是闭的。

(c)蕴含(a). 假设在$X$中存在一点$x$,$f$在这点不连续,那么存在$\epsilon > 0$,和一个序列$\{x_n\}$,使得$x = \lim{x_n}$,但是对于每个$n$,都有$\rho(f(x_n), f(x)) \ge \epsilon$。令$\Gamma = \Omega - B(f(x); \epsilon)$,那么$\Gamma$是闭的,并且$x_n$在$f^{-1}(\Gamma)$中。由于$f^{-1}(\Gamma)$是闭的(根据(c)),我们有$x \in f^{-1}(\Gamma)$。但这就蕴含$\rho(f(x);f(x)) \ge \epsilon > 0$,故矛盾。
\end{proof}

下述类型的结果大概易为读者所理解,所以证明留给读者作为习题。
\begin{proposition}{}{prop002020504}
设$f$和$g$是$X$到$\mathbb{C}$内的连续函数,$\alpha, \beta \in \mathbb{C}$。那么$\alpha{}f + \beta{}g$和$fg$也是连续的。如果对于$X$中的每个$x$,$g(x) \neq 0$,那么$f/g$也是连续的。
\end{proposition}

\begin{proposition}{}{prop002020505}
设$f: X \to Y$及$g:Y \to Z$是连续函数,那么$g \circ f$(这里$g \circ f(x) = g(f(x))$)是$X$到$Z$内的一个连续函数。
\end{proposition}

\begin{proof}
如果$U$是$Z$中的开集,那么$g^{-1}(U)$是$Y$中的开集。因此$f^{-1}(g^{-1}(U))=(g \circ f)^{-1}(U)$是$X$中的开集。
\end{proof}

\begin{definition}{一致连续}{def002020506}
函数$f:(X, d) \to (\Omega, \rho)$是一致连续的,如果对于每个$\epsilon > 0$,存在$\delta > 0$(只依赖于$\epsilon$),使得当$d(x, y) < \delta$时,就有$\rho(f(x), f(y)) < \epsilon$。我们称$f$是一个Lipschitz函数。如果存在常数$M > 0$,使得对于$X$中的所有$x$和$y$,都有$\rho(f(x), f(y)) \le Md(x, y)$。
\end{definition}

容易看出,每个Lipschitz函数是一致连续的。事实上,如果给定$\epsilon > 0$,取$\delta = \epsilon / M$即可。也容易看出每个一致连续的函数是连续的。上述诸类函数有些什么例子呢?如果$X = \Omega = \mathbb{R}$,那么$f(x)=x^2$是连续的,但不是一致连续的。如果$X = \Omega = [0,1]$,那么$f(x) = x^{\frac{1}{2}}$是一致连续的,但不是Lipschitz函数。下述命题为Lipschitz函数提供了一个丰富的来源。

设$A \subset X$,$x \in X$。我们定义$x$到集$A$的距离$d(x, A)$为
\[
d(x, A) = \inf\{d(x, a) : a \in A\}.
\]

\begin{proposition}{}{prop002020507}
设$A \subset X$,那么:
\begin{enumerate}
\item[(a)]$d(x, A) = d(x, A^-)$。
\item[(b)]$d(x, A) = 0$,当且仅当,$x \in A^-$。
\item[(c)]对于$X$中的所有$x$,有$|d(x, A) - d(y, A)| \le d(x, y)$。
\end{enumerate}
\end{proposition}

\begin{proof}
(a) 如果$A \subset B$,那么由定义显然有$d(x, B) \le d(x, A)$。因此$d(x, A^-) \le d(x, A)$。另一方面,如果$\epsilon > 0$,则存在$A^-$中的一点$y$,使得$d(x, A^-) \ge d(x, y) - \epsilon / 2$。再在$A$中找一点$a$,满足$d(y, a) < \epsilon/2$。但是由三角不等式$|d(x, y) - d(x, a)| \le d(y, a) < \epsilon/2$。特别地,$d(x, y) > d(x, a) - \epsilon/2$。这就给出$d(x, A^-) \ge d(x, a) - \epsilon \ge d(x, A) - \epsilon$。因为$\epsilon$是任意的,所以$d(x, A^-) \ge d(x, A)$。(a)得证。

(b) 如果$x \in A^-$,那么$0 = d(x, A^-) = d(x, A)$。现在对于$X$中的任意$x$,在$A$中存在一最小序列$\{a_n\}$,使得$d(x, A) = \lim{d(x, a_n)}$。所以如果$d(x, A) = 0$,那么$\lim{d(x, a_n)} = 0$,即$x = \lim{a_n}$,$x \in A^-$。

(c) 对于$A$中的$a$,$d(x, a) \le d(x, y) + d(y, a)$,因此$d(x, A) = \inf\{d(x, a):a \in A\} \inf\{d(x, y) + d(y, a):a \in A\} = d(x, y) + d(y, A)$。这就给出了$d(x, A) - d(y, A) \le d(x, y)$。类似的,$d(y, A) - d(x, A) \le d(x, y)$。所以不等式得证。
\end{proof}

注意,命题的(c)是说:由$f(x)=d(x, A)$所定义的函数$f: X \to \mathbb{R}$是一个Lipschitz函数。如果我们变动集$A$,便得到许多这种函数。

两个一致连续的(Lipschitz)函数的乘积仍是一致连续的(Lipschitz)函数,这一命题是不真的。例如,$f(x)=x$是Lipschitz函数,但是$f \cdot f$甚至都不是一致连续的。不过如果$f$和$g$都是有界的,那么命题就成立了(见习题\ref{exer002020503})。

下面的定理包含连续函数的两个最重要的性质。

\begin{theorem}{}{thm002020508}
设$f:(X, d) \to (\Omega, \rho)$是一个连续函数。(a) 如果$X$是紧的,那么$f(X)$是$\Omega$种的紧子集;(b) 如果$X$是连通的,那么$f(X)$是$\Omega$中的连通子集。
\end{theorem}

\begin{proof}
为了证明(a)和(b),不失一般性,可以假设$f(X) = \Omega$。(a) 设$\{\omega_n\}$是$\Omega$中的一个序列,那么对于每个$n \ge 1$,在$X$中存在一点$x_n$,使得$\omega_n = f(x_n)$。由于$X$是紧的,所以在$X$中存在一点$x$和一个子序列$\{x_{n_k}\}$,使得$x = \lim{x_{n_k}}$。设$\omega = f(x)$,那么由$f$的连续性,$\omega = \lim{\omega_{n_k}}$,因此根据定理\ref{thm:thm002020409},$\Omega$是紧的。(b)设$\Sigma \subset \Omega$在$\Omega$中既是开的,又是闭的,且$\Sigma \neq \emptyset$,那么因为$f(X) = \Omega$,所以$\emptyset \neq f^{-1}(\Sigma)$。因为$f$是连续的,所以$f^{-1}(\Sigma)$也既是开的,又是闭的。根据$X$的连通性,$f^{-1}(\Sigma) = X$,这就给出$\Omega = \Sigma$,于是$\Omega$是连通的。
\end{proof}

\begin{corollary}{}{coro002020509}
如果$f:X \to \Omega$是连续的,$K$是$X$中的紧集或连通集,那么相应地$f(K)$是$\Omega$中的紧集或连通集。
\end{corollary}

\begin{corollary}{}{coro002020510}
如果$f:X \to \mathbb{R}$是连续的,$X$是连通的,那么$f(X)$是一个区间。
\end{corollary}

这可由$\mathbb{R}$的连通子集的特征是区间这一事实推出。

\begin{theorem}{中值定理}{thm002020511}
如果$f:[a, b] \to \mathbb{R}$是连续的,且$f(a) \le \xi \le f(b)$,那么存在一点$x$,$a \le x \le b$,使得$f(x) = \xi$。
\end{theorem}

\begin{corollary}{}{coro002020512}
如果$f:X \to \mathbb{R}$是连续的,$K \subset X$是紧的,那么在$K$中存在点$x_0$和$y_0$,使得$f(x_0) = \sup\{f(x):x \in K\}$,$f(y_0) = \inf\{f(x): x \in K\}$。
\end{corollary}

\begin{proof}
如果$\alpha = \sup\{f(x):x\in K\}$,那么因为$f(K)$在$\mathbb{R}$中是有界闭集,所以$\alpha$在$f(K)$内。类似地,$\beta = \inf\{f(x): x \in K\}$在$f(K)$内。
\end{proof}

\begin{corollary}{}{coro002020513}
如果$K \subset X$是紧的,$f: X \to \mathbb{C}$是连续的,那么在$K$内存在点$x_0$和$y_0$,使得
\[
\begin{aligned}
|f(x_0)| &= \sup\{|f(x)|:x \in K\},\\ 
|f(y_0)| &= \inf\{|f(x)|: x \in K\}.
\end{aligned}
\]
\end{corollary}

\begin{proof}
这个系由上一系推出,因为$g(x)=|f(x)|$定义一个从$X$到$\mathbb{R}$的连续函数。
\end{proof}

\begin{corollary}{}{coro002020514}
如果$K$是$X$的紧子集,$x$在$X$内,那么在$K$内存在一点$y$,使得$d(x, y) = d(x, K)$。
\end{corollary}

\begin{proof}
定义$f: X \to \mathbb{R}$为$f(y) = d(x, y)$,那么$f$是连续的,并且由系\ref{cor:coro002020512},在$K$上取到最小值。这就是说,在$K$内存在一点$y$,使得对于每个$z \in K$,都有$f(y) \le f(z)$。这就给出了$d(x, y) = d(x, K)$。
\end{proof}

下面的两个定理极为重要,在全书中将反复用到它,用时不再注明。

\begin{theorem}{}{thm002020515}
设$f:X \to \Omega$是连续的,$X$是紧的,那么$f$是一致连续的。
\end{theorem}

\begin{proof}
设$\epsilon > 0$,我们要找一个$\delta > 0$,使得$d(x, y) < \delta$蕴含$\rho(f(x), f(y)) < \epsilon$。假如不存在这样的$\delta$;特别地,每个$\delta = \frac{1}{n}$都不满足上述要求。那么对于每个$n \ge 1$,在$X$内有点$x_n, y_n$,使得$d(x_n, y_n) < \frac{1}{n}$,但是$\rho(f(x_n), f(y_n)) \ge \epsilon$。因为$X$是紧的,所以存在子序列$\{x_{n_k}\}$和点$x \in X$,使得$x = \lim{x_{n_k}}$。

我们断言$x = \lim{y_{n_k}}$。事实上,$d(x, y_{n_k}) \le d(x, x_{n_k}) + \frac{1}{n_k}$;当$k$趋于$\infty$时,它是趋于零的。

设$\omega = f(x)$,那么$\omega = \lim{f(x_{n_k})} = \lim{f(y_{n_k})}$,所以不等式
\[
\epsilon \le \rho(f(x_{n_k}), f(y_{n_k})) \le \rho(f(x_{n_k}), \omega) + \rho(\omega, f(y_{n_k})),
\]
的右边趋于零。这是一个矛盾。因而定理得证。
\end{proof}

\begin{definition}{}{def002020516}
如果$A$,$B$是$X$的子集,那么定义$A$到$B$的距离$d(A, B)$为
\[
d(A, B) = \inf\{d(a, b): a \in A, b \in B\}.
\]
\end{definition}

注意,如果$B$是一个点所组成的集$\{x\}$,那么$d(A, \{x\}) = d(x, A)$。如果$A = \{y\}$,$B = \{x\}$。那么$d(\{x\}, \{y\}) = d(x, y)$。又如果$A \cap B \neq \emptyset$,那么$d(A, B) = 0$。但是$A, B$不相交,我们也可能有$d(A, B) = 0$。最典型的例子是取$A = \{(x, 0): x \in \mathbb{R}\}$,$B = \{(x, e^x): x \in \mathbb{R}\}$。注意$A, B$都是闭的且不相交,但仍有$d(A, B) = 0$。

\begin{theorem}{}{thm002020517}
如果$A$,$B$是$X$中不相交的集合,$B$是闭的,$A$是紧的,那么$d(A, B) > 0$。
\end{theorem}

\begin{proof}
定义$f: X \to \mathbb{R}$为$f(x) = d(x, B)$。因为$A \cap B = \emptyset$及$B$是闭集,所以对于$A$内的每一个$a$,$f(a) > 0$。但是因为$A$是紧的,所以在$A$内存在一点$a$,使得$0 < f(a) = \inf\{f(x): x \in A\} = d(A, B)$。
\end{proof}

\begin{exercise}
证明命题\ref{pro:prop002020502}。
\end{exercise}

\begin{exercise}
如果$f$和$g$是从$X$到$\mathbb{C}$的一致连续(Lipschitz)函数,那么$f+g$也是一致连续(Lipschitz)函数。
\end{exercise}

\begin{exercise}\label{exer002020503}
我们说$f:X \to \mathbb{C}$是有界的
\end{exercise}



\begin{exercise}\label{exer002020507}
设$G$是$\mathbb{C}$中的一个开子集
\end{exercise}





