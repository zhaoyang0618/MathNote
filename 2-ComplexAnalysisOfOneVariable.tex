\part{单复变函数}
《单复变函数》(Functions of One Complex Variable)的作者是J.B.康威(John B. Conway)。参考:\cite{FunctionsofOneComplexVariable1978}。

\chapter{复数系}\label{section00201}

\section{实数}\label{subsection0020101}
我们用$\mathbb{R}$表示所有实数组成的集。假定读者熟悉实数系及其性质,特别地,假定读者具备下面的知识:$\mathbb{R}$的序,上确界和下确界的定义和性质,以及$\mathbb{R}$的完备性($\mathbb{R}$中的每一个有上界的集必有上确界)。我们也假定读者熟知$\mathbb{R}$中的序列的收敛性与无穷级数。最后,一个人只有在单变量实函数方面有了坚实的基础之后,才可以着手学习复变函数。虽然在学习解析函数理论之前,传统上是先学习多变数实函数。但是对于本书来说,本质上这不是必要的条件,因为本书中任何地方都不需要这个领域里深入的结果。

\section{复数域}\label{subsection0020102}
我们把复数集$\mathbb{C}$定义为所有有序数对$(a, b)$的集,其中$a,b$是实数。加法和乘法由下式定义:
\begin{gather*}
(a, b) + (c, d) = (a+c, b+d), \\
(a, b)(c, d) = (ac-bd, bc + ad).
\end{gather*}
容易验证,这样定义后,$\mathbb{C}$满足域(field)的所有公理。这就是说,$\mathbb{C}$满足加法和乘法的结合律、交换律、分配了;$(0, 0)$和$(1,0)$分别是加法和乘法的单位元素,并且$\mathbb{C}$内的每一个非零元素有加法和乘法的逆元素。

对于复数$(a, 0)$,我们将写为$a$,这个映照$a \mapsto (a, 0)$定义了一个$\mathbb{R}$到$\mathbb{C}$的域同构\footnote{这恐怕不能说是同构,因为明显不是一一映射,应该是$\mathbb{R}$和$\mathbb{C}$的一个子集同构},所以我们可以把$\mathbb{R}$考虑为$\mathbb{C}$的一个子集。如果令$i=(0, 1)$,那么$(a, b) = a + ib$,从现在起,我们对复数就不再使用有序数对的记号了。

注意到$i^2=-1$,所以方程$z^2+1=0$在$\mathbb{C}$内有根。事实上,对于$\mathbb{C}$内的每个$z$,$z^2+1=(z+i)(z-i)$。更一般地,如果$z$和$w$是复数,我们得到
\[
z^2+w^2 = (z+iw)(z-iw),
\]
令$z$和$w$是实数$a$和$b$($a$和$b$都不为0\footnote{这里只需要$a$和$b$不全为0即可。}),我们得到
\[
\frac{1}{a+bi} = \frac{a-bi}{a^2+b^2} = \frac{a}{a^2+b^2} - i(\frac{b}{a^2+b^2}),
\]
这样我们就有了一个复数的倒数的公式。

当我们写$z = a + bi$($a, b \in \mathbb{R}$)时,我们称$a$,$b$为$z$的实部和虚部,并且用$a = \Re{z}$,$b=\Im{z}$表示。

作为本节的结尾,我们在$\mathbb{C}$内引进两个运算。这两个运算不是域的运算。如果$z = x+iy$($x, y \in \mathbb{R}$),那么我们定义$|z| = (x^2+y^2)^{\frac{1}{2}}$为$z$的绝对值,$\bar{z}=x-iy$为$z$的共轭数。注意:
\begin{gather}\label{equ00202001}
|z|^2=z\bar{z},
\end{gather}
特别地,如果$z \neq 0$,那么
\[
\frac{1}{z} = \frac{\bar{z}}{|z|^2}.
\]

下面是绝对值和共轭数的基本性质,其证明留给读者。
\begin{gather}
\Re{z} = \frac{1}{2}(z + \bar{z}), \quad \Im{z} = \frac{1}{2i}(z - \bar{z}). \label{equ00202002} \\
(\bar{z+w}) = \bar{z} + \bar{w}, \quad \bar{zw} = \bar{z}\bar{w}. \label{equ00202003}\\
|zw| = |z||w|. \label{equ00202004} \\
|z/w| = |z|/|w|.\label{equ00202005}\\
|\bar{z}| = |z|.\label{equ00202006}
\end{gather}
读者证明后面三个式子的时候,应当尽量避免将$z$和$w$展开为它们的实部和虚部,而最好利用(\ref{equ00202001}),(\ref{equ00202002})和(\ref{equ00202003})。

\begin{exercise}
求下列各复数的实部和虚部:
\begin{gather*}
\begin{aligned}
&\frac{1}{z}; \frac{z-a}{z+a}(a \in \mathbb{R}); z^2; \frac{3+5i}{7i+1}; (\frac{-1+i\sqrt{3}}{2})^3;\\
&(\frac{-1-i\sqrt{3}}{2})^6; i^n; (\frac{1+i}{\sqrt{2}})^n, 2 \le n \le 8.
\end{aligned}
\end{gather*}
\end{exercise}

\begin{exercise}
求下列各复数的绝对值和共轭数:
\[
\begin{aligned}
&-2+i; -3; (2+i)(4+3i);\frac{3-i}{\sqrt{2}+3i};\frac{i}{i+3}; \\
&(1+i)^6; i^{17}.
\end{aligned}
\]
\end{exercise}

\begin{exercise}
证明:当且仅当$z=\bar{z}$时,$z$才是实数。
\end{exercise}

\begin{exercise}\label{exer002010204}
若$z$和$w$是复数,证明下列等式:
\[
\begin{aligned}
&|z+w|^2 = |z|^2 + 2\Re{z\bar{w}} + |w|^2 \\
&|z-w|^2 = |z|^2 - e\Re{z\bar{w}} + |w|^2 \\
&|z+w|^2 + |z-w|^2 = 2(|z|^2+|w|^2)
\end{aligned}
\]
\end{exercise}

\begin{exercise}
设$z=z_1+\cdots+z_n$,$w= w_1+\cdots+w_n$,利用归纳法证明:
\[
|w| = |w_1|\cdots|w_n|; \bar{z}=\bar{z_1}+\cdots+\bar{z_n}; \bar{w}=\bar{w_1}\cdots\bar{w_n}.
\]
\end{exercise}

\begin{exercise}
设$R(z)$是$z$的有理函数,如果$R(z)$的所有系数是实数,则$\overline{R(z)} = R(\bar{z})$。
\end{exercise}

\section{复平面}\label{subsection0020103}
从复数的定义易见,$\mathbb{C}$中每一点$z$都可以和平面$\mathbb{R}^2$上唯一确定的点$(\Re{z}, \Im{z})$相等同。复数的加法恰好就是向量空间$\mathbb{R}^2$的加法,如果$z$和$w$是$\mathbb{C}$中的点,那么从$z$和$w$到$0(=(0,0))$画两条直线,这两条直线形成了以$0$、$z$、$w$为三个顶点的平行四边形的两条边,平行四边形的第四个顶点就是$z+w$。

注意,$|z-w|$恰好是$z$和$w$之间的距离,理会到这一点,上节习题\ref{exer002010204}中的最后一个等式说的就是平行四边形法则:平行四边形各边长的平方和等于其对角线的平方和。

距离函数的基本性质是它满足三角不等式(见下一章)。在这种情况下,对复数$z_1$,$z_2$,$z_3$,这个不等式变为
\[
|z_1-z_2| \le |z_1-z_3| + |z_3-z_2|.
\]

利用$z_1-z_2 = (z_1-z_3) + (z_3-z_2)$,容易看出,我们只需证明
\begin{gather}\label{equ002010301}
|z+w| \le |z| + |w| \quad (z, w \in \mathbb{C})
\end{gather}
为了证明这个不等式,首先看出,对于$\mathbb{C}$中任意$z$,
\begin{gather}\label{equ002010302}
\begin{aligned}
&-|z| \le \Re{z} \le |z|, \\
&-|z| \le \Im{z} \le |z|.
\end{aligned}
\end{gather}
因此,$\Re{z\bar{w}} \le |z\bar{w}| = |z||w|$。于是
\[
\begin{aligned}
|z+w|^2 &= |z|^2 + 2\Re{z\bar{w}} + |w|^2\\
&\le |z|^2 + 2|z||w| + |w|^2 = (|z|+|w|)^2,
\end{aligned}
\]
由此推出\ref{equ002010301}。(这个式子称为三角不等式,因为如果我们把$z$和$w$表示在平面上,(\ref{equ002010301})式表明,三角形$[0,z,z+w]$的一边的长度小于另外两边长度的和。或者说两点间的最短距离是直线)。在遇到一个不等式时,人们总应当问一问等号成立的必要充分条件是什么,考察一个三角形并考虑到(\ref{equ002010301})的几何意义,我们就引出条件$z=tw$,对某一$t \in \mathbb{R}$,$t \ge 0$。(或者如果$w=0$,则$w=tz$)。显然,当这两点和原点共线时,等号成立。事实上,如果我们看一下(\ref{equ002010301})式的证明,便知道$|z+w|=|z|+|w|$成立的必要充分条件是$|z\bar{w}|=\Re(z\bar{w})$。这等价于$z\bar{w}\ge 0$(即$z\bar{w}$使非负实数)。如果$w \neq 0$,两边乘以$w/w$,我们得到$|w|^2(z/w) \ge 0$,令
\[
t = z/w = (\frac{1}{|w|^2})|w|^2(z/w),
\]
那么$z=tw$,$t \ge 0$。

由归纳法,我们也有
\begin{gather}\label{equ002010303}
|z_1+z_2+\cdots+z_n| \le |z_1|+|z_2+\cdots+|z_n|,|
\end{gather}
不等式
\begin{gather}\label{equ002010304}
||z| - |w|| \le |z-w|
\end{gather}
也是有用的。

既然我们给出了绝对值的几何解释,让我们再来看一看,平面上一点的共轭复数是什么,这是容易的,事实上,$\bar{z}$是$z$关于$x$轴(即实轴)的对称点。

\begin{exercise}
证明(\ref{equ002010304})并给出等号成立的必要充分条件。
\end{exercise}

\begin{exercise}
证明:(\ref{equ002010302})中的等号成立,当且仅当,对任意整数$k$和$l$,$1 \le k,l \le n$,只要$z_l \neq 0$,就有$z_k/z_l \le 0$。
\end{exercise}

\begin{exercise}
设$a \in \mathbb{R}$,$c > 0$是固定的。对于每个可能选取的$a$和$c$,试描画出满足条件
\[
|z-a| - |z+a|=2c
\]
的点集。现在设$a$施任意复数,利用平面的旋转画出满足上述方程的点的轨迹。
\end{exercise}

\section{复数的极坐标表示与复数的方根}\label{subsection0020104}
考虑复平面$\mathbb{C}$的点$z=x+iy$。这个点有极坐标$(r, \theta)$:$x=r\cos{\theta}$,$y=r\sin{\theta}$。显然$r=|z|$,$\theta$是正实轴与从0到$z$的直线段的夹角。注意,在上述等式中的$\theta$可以代之以$\theta$加上$2\pi$的任意整数倍,角$\theta$称为$z$的幅(还是这个“辐)”角,记为$\theta=\arg{z}$。由于$\theta$的不确定性,“$\arg$”不是一个函数。我们引进记号
\begin{gather}\label{equ002010401}
\cis{\theta}=\cos{\theta}+i\sin{\theta}.
\end{gather}

设$z_1=r_1\cis{\theta_1}$,$z_2=r_2\cis{\theta_2}$,那么
\[
\begin{aligned}
z_1z_2=r_1r_2\cis{\theta_1}\cis{\theta_2}&=r_1r_2[(\cos{\theta_1}\cos{\theta_2}-\sin{\theta_1}\sin{}\theta_2)\\
+i(\sin{\theta_1}\cos{\theta_2}+\sin{\theta_2}\cos{\theta_1})]
\end{aligned}
\]
由正弦和余弦的和角公式,我们得到
\begin{gather}\label{equ002010402}
z_1z_2 = r_1r_2\cis(\theta_1+\theta_2).
\end{gather}
换句话说$\arg{z_1z_2} = \arg{z_1} + \arg{z_2}$(什么实函数把成绩变为和?\footnote{对数函数$\ln{(ab)}=\ln{a}+\ln{b}$})。由归纳法,对于$z_k=r_k\cis{\theta_k}$,$1 \le k \le n$,我们有
\begin{gather}\label{equ002010403}
z_1z_2\cdots{}z_n=r_1r_2\cdots{}r_n\cis{(\theta_1+\theta_2+\cdots+\theta_n)}.
\end{gather}
特别地,对于每个整数$n \ge 0$,有
\begin{gather}\label{equ002010404}
z^n = r^n\cis(n\theta).
\end{gather}
进而若$z \neq 0$,则$z[r^{-1}\cis(-\theta)]=1$;所以如果$z \neq 0$,那么对于一切整数$n$,正的,负的。或0,(\ref{equ002010404})也成立。作为(\ref{equ002010404})的一个特别情形,我们得到棣莫佛(de Moivre)公式:
\[
(\cos{\theta}+i\sin{\theta})^n=\cos{n\theta} + i\sin{n\theta}.
\]

现在我们可以来考虑下面的问题了。对于给定的一个复数$a \neq 0$,和一个整数$n \ge 2$,你能否找到满足$z^n=a$的数$z$?这样的$z$你能找到多少个?由于(\ref{equ002010404})式,解答这个问题是容易的。设$a=|a|\cis{\alpha}$;由(\ref{equ002010404}),$z=|a|^{\frac{1}{n}}\cis{(\alpha/n)}$就满足要求。但是这个解不是唯一解,因为$z'=|a|^{\frac{1}{n}}\cis{\frac{1}{n}(\alpha+2\pi)}$也满足$(z')^n=a$。事实上,每一个数
\begin{gather}\label{equ002010405}
|a|^{\frac{1}{n}}\cis{\frac{1}{n}(\alpha+2\pi{}k)}, \quad 0 \le k \le n-1.
\end{gather}
都是$a$的$n$次方根。借助(\ref{equ002010404})我们得到下述结果:对于$\mathbb{C}$中的每一个不等于零的数$a$,都有$a$的$n$个不同的$n$次方根,它们由公式(\ref{equ002010405})给出。

\textbf{例子} \quad 计算$n$次单位根。由于$1=\cis{0}$,(\ref{equ002010405})式给出如下这些根:
\[
1,\cis{\frac{2\pi}{n}},\cis{\frac{4\pi}{n}},\cdots\cis{\frac{2\pi}{n}(n-1)}.
\]
特别地,立方单位根是
\[
1, \frac{1}{2}(-1 + i\sqrt{3}),\frac{1}{2}(-1-i\sqrt{3}).
\]

\begin{exercise}
求出6次单位根。
\end{exercise}

\begin{exercise}
计算:
\begin{enumerate}
\item[(a)]$i$的平方根;
\item[(b)]$i$的立方根;
\item[(c)]$\sqrt{3}+3i$的平方根。
\end{enumerate}
\end{exercise}

\begin{exercise}
$n$次单位原根是一复数$a$,使得$1,a,a^2,\cdots,a^{n-1}$是$n$个不同的$n$次单位根。证明:如果$a,b$分别是$n$次和$m$次单位原根,则$ab$是$k$次单位根,$k$是某一整数。$k$的最小值是什么?如果$a,b$是非单位原根,你能说些什么?
\end{exercise}

\begin{exercise}
试利用二项式
\[
(a+b)^n=\sum_{k=0}^{n}{{n \choose k}a^{n-k}b^{k}},
\]
其中${n \choose k}=\frac{n!}{k!(n-k)!}$,并比较棣莫佛公式两边的实部和虚部,得到公式
\[
\begin{aligned}
&\cos{n\theta} = \cos^{n}{\theta} - {n \choose 2}\cos^{n-2}{\theta}\sin^{2}{\theta} + {n \choose 4}\cos^{n-4}{\theta}\sin^{4}{\theta}-\cdots\\
&\sin{n\theta} = {n \choose 1}\cos^{n-1}{\theta}\sin{theta}-{n \choose 3}\cos^{n-3}{\theta}\sin^{3}{\theta}+\cdots.
\end{aligned}
\]
\end{exercise}

\begin{exercise}
设$z=\cis{\frac{2\pi}{n}}$,整数$n \ge 2$。证明:$1+z+\cdots+z^{n-1}=0$.
\end{exercise}

\begin{exercise}
证明:$\phi(t)=\cis{t}$是加法群$\mathbb{R}$到乘法群$T=\{z:|z|=1\}$上的群同态。
\end{exercise}

\begin{exercise}
如果$z \in \mathbb{C}$,并且对于每个正整数$n$,$\Re{z^n} \ge 0$,证明:$z$是正实数。
\end{exercise}

\section{复平面上的直线和半平面}\label{subsection0020105}
设$L$表示$\mathbb{C}$中的直线。从初等解析几何知道,$L$是由$L$上的一个点和一个方向向量决定的。于是,如果$a$是$L$上任一点,$b$是它的方向向量,那么
\[
L = \{z = a+tb:-\infty < t < \infty\}.
\]
由于$b \neq 0$,这就给出,对于$L$上的$z$,有
\[
\Im{(\frac{z-a}{b})} = 0.
\]
事实上,如果$z$满足等式
\[
0 = \Im{(\frac{z-a}{b})},
\]
那么
\[
t = \frac{z-a}{b},
\]
蕴含$z = a + tb$,$-\infty < t < \infty$。这就是说
\begin{gather}\label{equ002010501}
L = \big{\{} z: \Im{(\frac{z-a}{b})} = 0 \big{\}}.
\end{gather}
集合
\[
\begin{aligned}
&\big{\{} z: \Im{(\frac{z-a}{b})} > 0 \big{\}},\\
&\big{\{} z: \Im{(\frac{z-a}{b})} < 0 \big{\}}.
\end{aligned}
\]
的轨迹是什么呢?作为回答这个问题的第一步,注意到$b$是一个方向,我们可以假定$|b|=1$。我们暂时考虑$a=0$的情形。并且令$H_0=\{z:\Im{(z/b)}>0\}$,$b = \cis{\beta}$。如果$z = r\cis{\theta}$,那么$z/b = r\cis{(\theta-\beta)}$。于是$z$在$H_0$中,当且仅当$\sin(\theta-\beta)>0$,即$\beta < \theta < \pi + \beta$。所以,如果我们“按照$b$的方向沿着$L$前进”,$H_0$是位于$L$左边的半平面。如果我们令
\[
H_a = \big{\{} z: \Im{(\frac{z-a}{b})} > 0 \big{\}},
\]
那么容易看出,$H_a = a + H_0 \equiv \{ a + w: w \in H_0\}$;即$H_a$是由$H_0$平移$a$而得到的,因此,$H_a$是位于$L$的左边的半平面。类似地,
\[
K_a = \big{\{} z: \Im{(\frac{z-a}{b})} < 0 \big{\}}
\]
是位于$L$的右边的半平面。

\begin{exercise}
设$C$是圆周$\{z:|z-c|=r\}$,$r > 0$,$a = c + r\cis{\alpha}$;并且令
\[
L_{\beta} = \big{\{} z: \Im{(\frac{z-a}{b})} = 0 \big{\}},
\]
其中$b=\cis{\beta}$。找出$L_{\beta}$在$a$处切于圆周$C$的关于$\beta$的充分必要条件。
\end{exercise}

\section{扩充平面及其球面表示}\label{subsection0020106}
在复分析中,我们常常涉及到这样一些函数,当自变量趋于给定点时,它们趋于无穷。为了讨论这种情形,我们引进扩充平面$\mathbb{C}_{\infty}\equiv \mathbb{C} \bigcup \{\infty\}$。同时为了讨论到取到无穷作为它的值的函数的连续性。我们也希望在$\mathbb{C}_{\infty}$内引进距离函数。为了这个目的以及为了给出$\mathbb{C}_{\infty}$的具体图像,我们把$\mathbb{C}_{\infty}$表示为$\mathbb{R}^3$中的单位球面
\[
S = \{(x_1,x_2,x_3) \in \mathbb{R}^3:x_1^2 + x_2^2 + x_3^2=1\}.
\]

设$N=(0, 0,1)$;即$N$是$S$上的北极。同时,把$\mathbb{C}$等同于$\{(x_1,x_2,0):x_1,x_2 \in \mathbb{R}\}$,于是$\mathbb{C}$沿赤道切割$\mathbb{C}$。现在对于$\mathbb{C}$中每个点$z$,考虑$\mathbb{R}^3$中通过$z$和$N$的直线。这条直线与球面恰好交于一点$Z \neq N$。如果$|z| > 1$,那么$Z$在北半球面上;如果$|z|<1$,那么$Z$在南半球面上;如果$|z|=1$,那么$Z = z$。当$|z| \to \infty$时,$Z$怎样呢?很清楚,$Z$趋于$N$。因此,我们就把$N$与$\mathbb{C}_{\infty}$中的$\infty$等同起来。这样一来,$\mathbb{C}_{\infty}$就被表示为球面$S$了。

让我们来考察这种表示法。令$z=x+iy$,设$Z = (x_1, x_2, x_3)$是$S$上相应的点,我们要找出用$x$,$y$表示$x_1, x_2, x_3$的方程。在$\mathbb{R}^3$中通过$z$和$N$的直线由$\{tN + (1-t)z:-\infty<t<\infty\}$或
\begin{gather}\label{equ002010601}
\{((1-t)x, (1-t)y, t): -\infty < t < \infty\}
\end{gather}
给出。因此,如果能够找到直线和$S$的交点的$t$值,我们就能够找到$Z$的坐标。如果$t$是这个值,那么
\[
1 = (1-t)^2x^2 + (1-t)^2y^2 + t^2 = (1-t)^2|z|^2 + t^2.
\]
由此注意到
\[
1-t^2 = (1-t)^2|z|^2.
\]
因为$t \neq 1$($z \neq \infty$),所以
\[
t = \frac{|z|^2-1}{|z|^2+1}.
\]
于是
\begin{gather}\label{equ002010602}
x_1 = \frac{2x}{|z|^2+1}, x_2 = \frac{2y}{|z|^2+1}, x_3 = \frac{|z|^2-1}{|z|^2+1}.
\end{gather}
这就给出
\begin{gather}\label{equ002010603}
x_1 = \frac{z + \bar{z}}{|z|^2+1}, x_2 = \frac{-i(z-\bar{z})}{|z|^2+1}, x_3 = \frac{|z|^2-1}{|z|^2+1}.
\end{gather}

如果$Z$是给定的($Z \neq N$),我们希望找$z$。这时,通过令$t = x_3$并利用(\ref{equ002010601}),我们得到
\begin{gather}\label{equ002010604}
z = \frac{x_1 + ix_2}{1 - x_3}
\end{gather}
现在让我们用下面的方式定义扩充平面上点之间的距离函数:对于$\mathbb{C}_{\infty}$中的$z$,$z'$,定义$z$到$z'$的距离$d(z,z')$为它们在$\mathbb{R}^3$中相应两点$Z$和$Z'$的距离。如果$Z=(x_1,x_2,x_3)$,$Z'=(x_1',x_2',x_3')$,那么
\begin{gather}\label{equ002010605}
d(z,z') = [(x_1-x_1')^2 + (x_2-x_2')^2 + (x_3-x_3')^2]^{\frac{1}{2}}.
\end{gather}
利用$Z$和$Z'$在$S$上这一事实,(\ref{equ002010605})给出
\begin{gather}\label{equ002010606}
[d(z,z')]^2 = 2 - 2(x_1x_1' + x_2x_2'+x_3x_3').
\end{gather}
由(\ref{equ002010603}),我们得到
\begin{gather}\label{equ002010607}
d(z,z') = \frac{2|z-z'|}{[(1+|z|^2)(1+|z'|^2)]^{\frac{1}{2}}}, \quad (z,z' \in \mathbb{C}).
\end{gather}
用类似的方法,对于$\mathbb{C}$中的$z$,我们得到
\begin{gather}\label{equ002010608}
d(z, \infty) = \frac{2}{(1 + |z|^2)^{\frac{1}{2}}},
\end{gather}
球面$S$和$\mathbb{C}_{\infty}$的点之间这种对应关系称为球极平面投影。

\begin{exercise}
给出(\ref{equ002010607})和(\ref{equ002010608})的详细推导。
\end{exercise}

\begin{exercise}
对于下列$\mathbb{C}$中的点给出$S$上对应的点:$0, 1+i,3+2i$。
\end{exercise}

\begin{exercise}
$S$上哪些子集对应$\mathbb{C}$中的实轴和虚轴。
\end{exercise}

\begin{exercise}
设$\Lambda$是$S$上的一个圆周,那么在$\mathbb{R}^3$中有唯一的平面$P$,使得$P \bigcap S = \Lambda$。由解析几何知道
\[
P = \{(x_1,x_2,x_3):x_1\beta_1 + x_2\beta_2 + x_3\beta_3 = l\},
\]
其中$(\beta_1,\beta_2,\beta_3)$是与$P$正交的一个向量,$l$是某一实数。可以假设$\beta_1^2+\beta_2^2+\beta_3^2=1$。利用这一事实,证明:如果$\Lambda$包含点$N$,则它在$\mathbb{C}$上的投影是一直线。否则,$\Lambda$投影到$\mathbb{C}$中的一个圆周上。
\end{exercise}

\begin{exercise}
设$Z$和$Z'$是$S$上分别与$z$和$z'$相应的两点。$W$是$S$上与$z+z'$对应的点。试用$Z$和$Z'$的坐标表示出$W$的坐标。
\end{exercise}

\chapter{度量空间与$\mathbb{C}$的拓扑}\label{section00202}

\section{度量空间的定义和例子}\label{subsection0020201}
一个度量空间是一个序偶$(X, d)$,这里$X$是一个集,$d$是一个从$X \times X$到$\mathbb{R}$的函数,称之为距离函数或度量,它满足下列条件:
\[
d(x, y) \ge 0;
\]
当且仅当$x=y$时,$d(x, y)=0$;
\begin{gather*}
d(x, y) = d(y,x) \quad (\text{对称性});\\
d(x, z) \le d(x, y) + d(y, z)\quad (\text{三角不等式}).
\end{gather*}
如果$x$和$r > 0$是固定的,那么定义
\begin{gather*}
B(x; r) = \{y \in X: d(x, y) < r\},\\
\bar{B}(x; r) = \{y \in X: d(x, y) \le r\}.
\end{gather*}
$B(x; r)$和$\bar{B}(x; r)$分别称为以$x$为中心,$r$为半径的开球和闭球。

\textbf{例子}

\begin{example}\label{exam002020101}
设$X = \mathbb{R}$或$\mathbb{C}$,定义$d(z, w)=|z-w|$,这就使$(\mathbb{R}, d)$和$(\mathbb{C}, d)$都成为度量空间。事实上,$(\mathbb{C}, d)$将是我们最感兴趣的例子。如果读者在此以前从来未接触过度量空间的概念,那么在学习这一章的过程中应当时常想到$(\mathbb{C}, d)$。
\end{example}

\begin{example}\label{exam002020102}
设$(X, d)$是一个度量空间,$Y \subset X$;那么$(Y, d)$也是一个度量空间。
\end{example}

\begin{example}\label{exam002020103}
设$X = \mathbb{C}$,定义$d(x+iy, a+ib)=|x-a|+|y-b|$。那么$(\mathbb{C}, d)$是一个度量空间。
\end{example}

\begin{example}\label{exam002020104}
设$X = \mathbb{C}$,定义$d(x+iy, a+ib)=\max{|x-a|, |y-b|}$。
\end{example}

\begin{example}\label{exam002020105}
设$X$是任意一个集,定义$d(x, y) = 0$,如果$x=y$;$d(x, y)=1$,如果$x \neq y$。为了证明函数$d$满足三角不等式,只要考虑在$x,y,z$当中出现相等的各种可能情形。注意,如果$r \le 1$,则$B(x; r)$只由一个点$x$所组成;如果$r > 1$,则$B(x;r) = X$。这个度量空间在解析函数论的研究中并不出现。
\end{example}

\begin{example}\label{exam002020106}
设$X = \mathbb{R}^n$,对于$\mathbb{R}^n$中的$x=(x_1,\cdots, x_n)$和$y=(y_1,\cdots,y_n)$定义
\[
d(x, y) = [\sum_{j=1}^{n}{(x_j-y_j)^2}]^{\frac{1}{2}}.
\]
\end{example}

\begin{example}\label{exam002020107}
设$S$是任意一个集,$B(S)$表示满足条件
\[
\left\|f\right\|_{\infty} = \sup{\{\left|f(s)\right|: s \in S\}} < \infty
\]
的函数$f: S \to \mathbb{C}$的集。这就是说,$B(S)$由所有其值域位于某一有穷半径的圆内的复值函数所构成。对于$B(S)$中的$f$和$g$定义$d(f, g) = \left\|f-g\right\|_{\infty}$。我们来证明$d$满足三角不等式。事实上,如果$f$,$g$和$h$在$B(S)$中,$s$是$S$中的任意一点,那么$|f(s)-g(s)| = |f(s)-h(s)+h(s)-g(s)| \le |f(s)-h(s)| + |h(s)-g(s)| \le \left\|f-h\right\|_{\infty} + \left\|h - g\right\|_{\infty}$。于是若对于$S$中所有的$s$取上确界,则有$\left\|f-g\right\|_{infty} \le \left\|f-h\right\|_{\infty} + \left\|h - g\right\|_{\infty}$,这就是对于$d$的三角不等式。
\end{example}

\begin{definition}{开集}{def002020101}
对于度量空间$(X, d)$,一个集$G \subset X$是开集,如果$G$内的每一个$x$,都存在一个$\epsilon > 0$,使得$B(x;\epsilon) \subset G$。
\end{definition}

于是,一个集在$\mathbb{C}$内是开的,如果它没有“边”。例如,
\[
G = \{z \in G: a < \Re{(z)} < b\}
\]
是开的;但是$\{z: \Re{(z)} < 0\} \bigcap \{0\}$不是开的,因为不管我们把$\epsilon$取得多么小,$B(0;\epsilon)$都不能包含在这个集内。

我们用$\emptyset$表示空集,就是一个元素也没有的集。

\begin{proposition}{}{prop002020101}
设$(X, d)$是一个度量空间,那么:
\begin{enumerate}
\item[(a)]集$X$和$\emptyset$是开集。
\item[(b)]如果$G_1,\cdots, G_n$是$X$中的开集,则$\bigcap_{k=1}^{n}{G_k}$也是$X$中的开集。
\item[(c)]如果$\{G_j:j \in J\}$是$X$中的开集族,$J$是任一指标集,则$G = \bigcup\{G_j:j \in J\}$也是开集。
\end{enumerate}
\end{proposition}

\begin{proof}
(a)的证明是平凡的。为了证明(b),设$x \in G = \bigcap_{k=1}^{n}{G_k}$;那么$x \in G_k$,$k=1,2,\cdots,n$。于是由定义,对于每个$k$有$\epsilon_k > 0$,使得$B(x;\epsilon_k) \subset G_k$。如果取$\epsilon=\min(\epsilon_1,\epsilon_2,\cdots, \epsilon_n)$,那么,对于$1 \le k \le n$,$B(x;\epsilon) \subset B(x;\epsilon_k) \subset G_k$,于是$B(x; \epsilon) \subset G$,$G$是开集。

(c)的证明留给读者作为习题。
\end{proof}

在度量空间里还有另一类著名的子集。这类子集包含它们的全部“边”,换一个说法,它们的余集没有“边”。

\begin{definition}{闭集}{def002020102}
一个集$F \subset X$是闭的,如果它的余集$X-F$是开的。
\end{definition}

下面的命题是命题\ref{pro:prop002020101}的补命题。对于前一命题应用Morgan法则便可完成其证明,我们把它留给读者。

\begin{proposition}{}{prop002020102}
设$(X, d)$是一个度量空间,那么:
\begin{enumerate}
\item[(a)]集$X$和$\emptyset$是闭的。
\item[(b)]如果$F_1,\cdots, F_n$是$X$中的闭集,则$\bigcup_{k=1}^{n}{F_k}$也是$X$中的闭集。
\item[(c)]如果$\{F_j:j \in J\}$是$X$中的闭集族,$J$是任一指标集,则$F = \bigcap\{F_j:j \in J\}$也是闭集。
\end{enumerate}
\end{proposition}

在学习开集和闭集时,最普遍的错误是把闭集的定义解释为一个集不是开集便是闭集。这种理解当然是错误的。只要看集$\{z \in \mathbb{C} : \Re{(z)} > 0\} \cup \{0\}$就清楚了,这个集既不是开的,也不是闭的。

\begin{definition}{}{def002020103}
设$A$是$X$的子集,那么,$A$的内部$\intset{A}$就是集合$\bigcup\{G: G\text{是开集,且}G \subset A\}$。$A$的闭包$A^-$就是集$\bigcap\{F: F\text{是闭集,且}F \supset A\}$。注意,$\intset{A}$可以是空集,$A^-$可以是$X$。如果$A = \{a + ib : a\text{和}b\text{是有理数}\}$,那么同时有$A^-=\mathbb{C}$和$\intset{A} = \emptyset$。根据命题\ref{pro:prop002020101}和\ref{pro:prop002020102},$A^-$是闭集,$\intset{A}$是开集。$A$的边界记为$\partial{A}$,定义为$\partial{A} = A^- \cap (X-A)^-$。
\end{definition}

\begin{proposition}{}{prop002020103}
设$A$和$B$是度量空间$(X, d)$的子集,那么:
\begin{enumerate}
\item[(a)]当且仅当$A = \intset{A}$时$A$是开集。
\item[(b)]当且仅当$A = A^-$时$A$是闭集。
\item[(c)]$\intset{A} = X - (X-A)^-$;$A^- = X- \intset(X-A)$;$\partial{A} = A^- - \intset{A}$。
\item[(d)]$(A \bigcup B)^- = A^- \bigcup B^-$。
\item[(e)]当且仅当存在$\epsilon > 0$,使得$B(x_0; \epsilon) \subset A$时,$x_0 \in \intset{A}$。
\item[(f)]当且仅当,对每一$\epsilon > 0$,$B(x_0; \epsilon) \cap A \neq \emptyset$时,$x_0 \in A^-$。
\end{enumerate}
\end{proposition}
\begin{proof}
(a)至(e)的证明留给读者。为了证明(f),假设$x_0 \in A^- = X - \intset(X-A)$;于是,$x_0 \not\in \intset(X-A)$。由(e),对于每一$\epsilon > 0$,$B(x_0; \epsilon)$不包含在$X-A$内,这就是说,存在一个点$y \in B(x_0; \epsilon)$,$y$不在$X-A$内。所以$y \in B(x_0;\epsilon) \cap A$。现在设$x_0 \not\in A^- = X - \intset{(X-A)}$,那么$x_0 \in \intset(X-A)$,由(e),存在$\epsilon > 0$,使得$B(x_0; \epsilon) \subset X-A$。即$B(x_0; \epsilon) \cap A = \emptyset$,所以$x_0$不满足条件。
\end{proof}

最后,再定义一类著名的集合。

\begin{definition}{稠密}{def002020104}
度量空间$X$的一个子集$A$是稠密的,如果$A^- = X$。
\end{definition}

有理数集$\mathbb{Q}$在$\mathbb{R}$中是稠密的,$\{x + iy : x, y, \in \mathbb{Q}\}$在$\mathbb{C}$中是稠密的。

\begin{exercise}
证明:(\ref{exam002020102})至(\ref{exam002020106})中给出的那些例子都确实是度量空间,只有例子(\ref{exam002020106})的证明可能会有些困难,对于这些例子给出$B(x;r)$。
\end{exercise}

\begin{exercise}
$\mathbb{C}$的下列子集,哪些是开集,哪些是闭集?
\begin{enumerate}
\item[(a)]$\{z:|z|<1\}$;
\item[(b)]实轴;
\item[(c)]$\{z: z^n=1,\text{对某一整数}n \ge 1\}$;
\item[(d)]$\{z \in \mathbb{C}: z\text{是实数,且}0 \le z <1\}$;
\item[(e)]$\{z \in \mathbb{C}: z\text{是实数,且}0 \le z \le 1\}$。
\end{enumerate}
\end{exercise}

\begin{exercise}
如果$(X, d)$是任一度量空间,证明:每一个开球是开集,每一个闭球是闭集。
\end{exercise}

\begin{exercise}
给出(\ref{pro:prop002020101}c)的详细证明。
\end{exercise}

\begin{exercise}
证明命题\ref{pro:prop002020102}。
\end{exercise}

\begin{exercise}
证明:一个集$G \subset X$是开的,当且仅当$X-G$是闭的。
\end{exercise}

\begin{exercise}
证明:$(\mathbb{C}_{\infty}, d)$是一度量空间,其中$d$是由第一章的(\ref{equ002010607}),(\ref{equ002010608})给出的。
\end{exercise}

\begin{exercise}
设$(X, d)$是一度量空间,$Y \subset X$,又设$G \subset X$是开的,证明$G \cap Y$是$(Y, d)$中的开集。反之,如果$G_1 \subset Y$是$(Y, d)$中的开集,则存在开集$G \subset X$,使得$G_1 = G \cap Y$。
\end{exercise}

\begin{exercise}
在上题中用“闭的”代替“开的”。
\end{exercise}

\begin{exercise}
证明命题\ref{pro:prop002020103}
\end{exercise}

\begin{exercise}
证明:$\{\cis{k}:k \ge 0\}$在$T=\{z \in \mathbb{C}:|z|=1\}$中是稠密的。对于哪些$\theta$的值,$\{\cis(k\theta):k>0\}$在$T$中是稠密的?
\end{exercise}

\section{连通性}\label{subsection0020202}
作为这一节的开始,让我们先给出一个例子。设$X = \{ x \in \mathbb{C} : |z| \le 1\} \cup \{z: |z-3| < 1\}$,并且把$\mathbb{C}$的度量赋于$X$(今后,当我们把$\mathbb{R}$或$\mathbb{C}$的子集$X$看作一个度量空间时,如果不作相反的声明,总假定$X$继承度量$d(z, w) = |z-w|$),那么集合$A = \{z:|z| \le 1\}$既是开的,又是闭的。它是闭的,因为它在$X$中的余集$B = X-A = \{z:|z-3| < 1\}$是开的;$A$是开的,因为如果$a \in A$,那么$B(a;1) \subset A$(注意:$\{z \in \mathbb{C}: |z-a| < 1\}$并不总包含在$A$中,当$a=1$时就是一例。但当按定义,$B(a;1)$是$z \in X: |z-a|<1$,它是包含在$A$中的)。类似的,$B$在$X$中也是既开又闭的。

这是一个非连通空间的例子。

\begin{definition}{连通}{def002020201}
一个度量空间$(X, d)$是连通的,如果只有$\emptyset$和$X$既是开的又是闭的。设$A \subset X$,如果度量空间$(A, d)$是连通的,那么$A$是$X$的连通子集。
\end{definition}

连通性的一个等价说法是:$X$是不连通的,如果存在$X$中的互不相交的非空开集$A$和$B$,使得$X = A \cup B$。事实上,如果这个条件成立,那么$A = X-B$也是闭的。

\begin{proposition}{}{prop002020201}
一个集$X \subset \mathbb{R}$是连通的,当且仅当$X$是一个区间。
\end{proposition}

\begin{proof}
设$X = [a, b]$,$a$,$b$是$\mathbb{R}$的元素。设$A \subset X$是$X$的开子集,满足$a \in A$,$A \neq X$。我们将证明$A$不可能也是闭的,因此$X$必是连通的。因为$A$是开的,$a \in A$,所以存在$\epsilon > 0$,使得$[a, a+\epsilon) \subset A$,设
\[
r = \sup\{\epsilon: [a, a+\epsilon) \subset A\}.
\]
则有断言:$[a, a+r) \subset A$。事实上,如果$a \le x < a+r$,令$h = a + r -x > 0$,由上确界的定义,存在$\epsilon$,$r - h < \epsilon < r$且$[a, a+\epsilon) \subset A$。但是$a \le x = a + (r-h) < a + \epsilon$蕴含$x \in A$。断言得证。

但是$a + r \not\in A$\footnote{这里有两种可能:(1)$a+r=b$;(2)$a+r < b$.当$a+r=b$时,$a + r \in A$导致$A=X$,与原来的假定$A \neq X$矛盾;作者忽略了这种情况。},因为在相反的情形,$a + r \in A$,那么由于$A$是开的,存在$\delta > 0$,使得$[a + r, a+r+\delta) \subset A$。但这就给出$[a, a+r+\delta) \subset A$。这与$r$的定义相矛盾。现在假定$A$也是闭的,那么$a + r \in B = X- A$,$B$是开的,因此我们可以找到$\delta > 0$,使得$(a+r-\delta, a+r] \subset B$。这和上述断言矛盾。

其他类型的区间的连通性的证明是类似的,留给读者作为习题。

$\mathbb{R}$中的连通集必是一区间,其证明留做习题。
\end{proof}










