\documentclass[cn,11pt,chinese]{elegantbook}

\usepackage[all]{xy}
\usepackage{amsmath}
\usepackage{asymptote}
\usepackage{subfig}
\usepackage{graphicx}



\newcount\mycount
%\def\ConvColor{rgb:yellow,5;red,2.5;white,5}
%\def\ConvReluColor{rgb:yellow,5;red,5;white,5}
%\def\PoolColor{rgb:red,1;black,0.3}
%\def\FcColor{rgb:blue,5;red,2.5;white,5}
%\def\FcReluColor{rgb:blue,5;red,5;white,4}
%\def\SoftmaxColor{rgb:magenta,5;black,7}
\def\cis{\,\text{cis}\,}
\def\intset{\operatorname{int}}
\def\diam{\operatorname{diam}}
\def\dist{\operatorname{dist}}
\def\ulim{\operatorname{u-lim}}
\def\cinfty{\mathbb{C}_{\infty}}
\def\sh{\operatorname{sh}}
\def\argsh{\operatorname{argsh}}
\def\ch{\operatorname{ch}}
\def\argch{\operatorname{argch}}

%\newtheorem{lemma}{法则}
% 方程编号以section为准
\numberwithin{equation}{section}


% title info
\title{高等数学基础}
%\subtitle{读书笔记}

% bio info
\author{虞朝阳}
\institute{西北工业大学}
%\date{\today}

% extra info
\version{1.00}
\extrainfo{Wir m\"ussen wissen, wir werden wissen. (我们必须知道,我们必将知道) - David.Hilbert}
%\logo{logo.png}
\cover{cover.jpg}

\begin{document}

%$\mathop {\min }\limits_x f(x)$
%
%\begin{displaymath}
%\mathop{\sum \sum}_{i,j=1}^{N} a_i a_j 
%{\sum \sum}_{i,j=1}^{N} a_i a_j
%\end{displaymath}


\maketitle


\chapter*{前言}
\addcontentsline{toc}{chapter}{Preface}
\markboth{Preface}{}
我希望能够形成一本书,这本书能够从高中知识作为基础,然后覆盖高等代数,数学分析,数论,复分析,实分析,泛函分析,微分方程,微分几何,抽象代数,拓扑学,概率论,组合学等等相关的知识。

我会从基本逻辑以及朴素集合论出发,引入抽象代数的群环域的概念,然后构造出以自然数为基础,构造出整数,有理数,实数,复数,涉及整数,有理数的时候,应该引入一些基本的数论知识,以及组合数学的一些知识,构造实数的时候,引入实分析的初步知识,例如不可数,极限等概念,,构造复数的时候,引入代数基本定理,但是不给出证明。

目前还没想好如何引入几何相关的概念。



\tableofcontents
\mainmatter
\hypersetup{pageanchor=true}


\chapter{逻辑和集合论}
这一章主要论述一些基本的逻辑,引入常用的逻辑符号,以及朴素集合论的知识。


% \bibliographystyle{plain}
\bibliography{mathreference}
\appendix

\end{document}
