\documentclass[12pt,a4paper,openany]{book}
\usepackage{amssymb}
\usepackage{fontspec}
\usepackage{amsmath}
\usepackage{amsfonts}
\usepackage[mathscr]{eucal}
%\usepackage{accents}
%\usepackage{statex}

\setromanfont{SimSun}
\setmainfont{SimSun}

\XeTeXlinebreaklocale "zh"
\XeTeXlinebreakskip = 0pt plus 1pt

\newtheorem{example}{例} 
\newtheorem{theorem}{定理}[section]
\newtheorem{definition}{定义}[section]
\newtheorem{axiom}{公理}[section]
\newtheorem{property}{性质}[section]
\newtheorem{proposition}{命题}[section]
\newtheorem{lemma}{引理}[section]
\newtheorem{corollary}{推论}[section]
\newtheorem{remark}{注解}
\newtheorem{condition}{条件}
\newtheorem{conclusion}{结论}
\newtheorem{assumption}{假设}
\newtheorem{exercise}{解答}[chapter]
\newtheorem{answer}{解答}[chapter]

\newcommand\relphantom[1]{\mathrel{\phantom{#1}}}
\newcommand\num[1]{\left\Vert{#1}\right\Vert}
\newcommand\Hom{\text{Hom\,}}
\newcommand\Tr{\text{Tr\,}}
\newcommand\Ker{\text{Ker\,}}
\newcommand\Image{\text{Im\,}}
\newcommand\rank{\text{rank\,}}
\newcommand\tr{\text{tr\,}}

\DeclareMathOperator{\esssup}{ess\,sup}

\makeatletter
\def\wideubar{\underaccent{{\cc@style\underline{\mskip10mu}}}}
\def\Wideubar{\underaccent{{\cc@style\underline{\mskip8mu}}}}
\def\diam{\text{diam}}
\makeatother

\makeatletter
\def\widebar{\accentset{{\cc@style\underline{\mskip10mu}}}}
\def\Widebar{\accentset{{\cc@style\underline{\mskip8mu}}}}
\makeatother


\title{朴素集合论}
\author{P.R.Halmos}

\begin{document}

\frontmatter
\frontmatter
\begin{titlepage}
\maketitle
\end{titlepage}
\setcounter{page}{0}
\chapter{前言}
最近开始看P.R.Halmos的《Naive Set Theory》,这里是一些简单的笔记.

\tableofcontents

\mainmatter
\section{2013年04月21日}
集合是一个不加定义的概念,就像几何中的点,线之类的.首先引入的概念是"属于"和"相等",这是集合论的两个基本关系.他们也是通过外延公理相互联系.

\begin{axiom}[外延公理(Axiom of extension)]
两个集合相等,当且仅当他们有相同的元素.
\end{axiom}

外延公理界定了一个集合的元素,它非常明显,却绝不寻常:

我们考虑人类之间的关系:设$x \in A$,如果$x$是$A$的祖先,此时,"仅当(only if)"部分是成立的,"当(if)"部分不成立.也就是有相同的祖先,不一定能得出相同的$A$.

属于是"元素"和集合之间的关系,不过应该注意,我们这里的元素是没有预先定义,足够宽泛.集合也可以作为元素.有了属于,我们可以考虑集合之间的关系:$x \in A \Rightarrow x \in B$,则称$A \subset B$.包含关系满足:

(1)自反的:$A \subset A$;

(2)传递的:$A \subset B$,$B \subset C$ $\Rightarrow$ $A \subset C$;

(3)反对称的:$A \subset B$不能得出$B \subset A$,如果两者同时成立,必有$A = B$,这也可以认为是外延公理的等价描述.

显然,"属于"和"包含"是完全不同的东西:"属于"一般来说不是自反的,也不满足传递性.对于满足$A \in A$这样的集合,一般情况不在我们的考虑之内.

要想得到更多的集合,我们需要其他的公理,还需要一些基本的逻辑:"属于"和"等于"是两个基本的语句,其余的通过以下七个组合得到:and(且);or(或者,包括either ... or ...);nnot(非);if - then -(如果...就...);if and only if(当且仅当);for some(对某些...);for all(对所有...).

\begin{axiom}[Axiom of specification]
对于每一个集合$A$和条件$S(x)$,存在集合$B$,使得$B$中的元素$x$属于$A$,且$S(x)$成立.
\end{axiom}

我搜索了一下网络,大部分把"Axiom of specification"翻译为"分类公理".有此公理,通常把$B$记作$B=\{x \in A: S(x)\}$,这里有一个极为有趣也非常有意义的集合:$B=\{x \in A: x \notin x\}$.对于这个集合,必有$B \notin A$,证明比较简单,反证法即可,书中给出了详细讨论,这意味着:不存在万物之集,或者说,不存在包含一切的集合,这实际上就是罗素悖论的另一个说法.而哥德尔关于数学基础的一些结论,说明了存在缺陷是世界的本性,上帝也不是万能的,有阴才有阳.

为了有具体的讨论对象,我们需要构造出一些集合.如果我们假设:存在一个集合,那么由axiom of specification可知,存在一个不包含任何元素的集合$\emptyset$,$\emptyset \subset A$对任意的集合$A$成立.这几个结论的证明有些意思,书中也给出了详细的描述,它涉及到了对于空集的处理:反证,如果不成立,那就是存在一个元素$x \in \emptyset$,但是$x \notin A$,但是这样的元素不存在.

\begin{axiom}[Axiom of pairing]
对任意两个集合,存在一个集合,使得这两个集合是该集合的元素.
\end{axiom}

用数学语言描述是:$\forall a,b$,$\exists A$,$a \in A$,$b \in A$.$B=\{a,b\}$.

有了这个公理,我们可以构造出无限多个集合了:
\begin{gather*}
\emptyset,\{\emptyset\},\{\{\emptyset\}\},\cdots\\
\{\{\emptyset\},\{\{\emptyset\}\}\},\cdots
\end{gather*}

这里面的任意两个集合互不相等.这是书中的一道习题,至于严格证明他们互不相等,目前还没有思路.我觉得至少应该证明:上述公理中的集合$A$和$a$,$b$是不相同.$\emptyset$和$\{\emptyset\}$不相等比较好说明:$\emptyset$不包含任何元素,而$\{\emptyset\}$包含了一个元素:$\emptyset$,两者自然不相等,习题中的,我觉得只能是通过外延公理,一层一层脱去花括号,最后归结到这个结论上.

\section{2013年04月29日}
并集和交集:

前面无序对给出了拥有两个元素的集合,我们通过并集可以得到包含更多元素的集合.

\begin{axiom}[并集公理-Axiom of unions]
对任意一簇集合,存在一个集合,包含所有这样的元素,这个元素至少是这一簇集合中的某一个集合的元素.
\end{axiom}

使用数学符号表示为:$\forall \mathscr{C}$,$\exists \mathscr{U}$,若$x \in X$,$X \in \mathscr{C}$,则$x \in \mathscr{U}$.

关于并集有如下公式:
\begin{enumerate}
\item $A \cup \emptyset = A$;
\item $A \cup B = B \cup A$.交换律(commutativity)
\item $A \cup (B \cup C) = (A \cup B) \cup C$.结合律(associativity)
\item $A \cup A = A$.幂等(idempotence)
\item $A \subset B$当且仅当$A \cup B = B$.
\end{enumerate}

注意到无序对只有两个元素,通过它和并集的关系:$\{a\} \cup \{b\} = \{a,b\}$,可以推广到三元集,四元集,...
\[
\{a,b,c\}=\{a\}\cup\{b\}\cup\{c\}.
\]

交集:
\[
A \cap B = \{x : x \in A \text{且} x \in B\}
\]
它可以推广到一簇集合$\mathscr{C}$的交集,此时$\mathscr{C} \neq \emptyset$;此时
\[
\cap\{X; X \in \mathscr{C}\} = \{x: x \in X, \forall X \in \mathscr{C}\}
\]

交集的性质:
\begin{enumerate}
\item $A \cap \emptyset = \emptyset$;
\item $A \cap B = B \cap A$.交换律(commutativity)
\item $A \cap (B \cap C) = (A \cap B) \cap C$.结合律(associativity)
\item $A \cap A = A$.幂等(idempotence)
\item $A \subset B$当且仅当$A \cap B = A$.
\end{enumerate}

特殊的,当$A \cap B = \emptyset$时,称$A$与$B$不相交(disjoint).两两不相交(pairwise disjoint).

交与并的分配律(distributive law)
\begin{gather*}
A \cap (B \cup C) = (A \cap B) \cup (A \cap C)\\
A \cup (B \cap C) = (A \cup B) \cap (A \cup C)
\end{gather*}

习题:$(A \cap B) \cup C = A \cap (B \cup C)$的充要条件是$C \subset A$.

若$(A \cap B) \cup C = A \cap (B \cup C)$,要证明$C \subset A$:$\forall x \in C$,则$x \in (A \cap B) \cup C$,由此得到,$x \in A \cap (B \cup C)$,因此$x \in A$,获证.

反之,如果$C \subset A$,需要证明:$(A \cap B) \cup C = A \cap (B \cup C)$.

$x \in (A \cap B) \cup C$,则或者$x \in (A \cap B)$,或者$x \in C$,若$x \in A \cap B$,则$x \in A$且$x \in B$,$x \in A$,且$x \in B \cup C$,于是$x \in A \cap (B \cup C)$.

若$x \in C$,则$x \in A$,$x \in B \cup C$,于是$x \in A \cap (B \cup C)$.

$x \in A \cap (B \cup C)$,$x \in A$,且$x \in B \cup C$,若$x \in B$,则$x \in A \cap B$,这样$x \in (A \cap B) \cup C$,若$x \in C$,则$x \in (A \cap B) \cup C$.

补集和幂集:

相对补集:$A - B = \{x \in A: x \notin B\}$,很多时候,我们考虑一种特殊情形下的补集,固定一个集合$E$,此时$E-A=A'$.

补集的一些性质:$(A')'=A$;$\emptyset'=E$,$E'=\emptyset$;$A \cap A'=\emptyset$;$A \cup A'=E$;$A \subset B$当且仅当$B' \subset A'$.

和补集相关的一个非常重要的性质(De morgan法则):
\begin{gather*}
(A \cup B)' = A' \cap B' \\
(A \cap B)' = A' \cup B'
\end{gather*}
这一法则揭示了一个重要的现象,常称为对偶原理,在包含了并集,交集,以及补集的包含关系或者等式中,只要把每一个集合替换为它的补集,交换并与交的位置,把包含关系反向,则对应的关系仍旧成立.

关于补集的一些结论:
\begin{enumerate}
\item $A - B = A \cap B'$;
\item $A \subset B$当且仅当$A - B = \emptyset$;
\item $A - (A - B) = A \cap B$;
\item $A \cap (B - C) = (A \cap B) - (A \cap C)$;
\item $A \cap B \subset (A \cap C) \cup (B \cap C')$;
\item $(A \cup C) \cap (B \cup C') \subset A \cup B$;
\end{enumerate}

$A$与$B$的对称差(布尔和)定义为:$A + B = (A-B) \cup (B-A)$.

对称差满足交换律;结合律以及$A + \emptyset = A$,$A + A = \emptyset$.

关于一组集合的交集,有一点需要讨论一下,在定义中要求这一组集合中至少有一个集合,如果出现空的集合簇,即讨论:$x \in X$,对任意的$X \in \emptyset$,会引发出$\cap{\emptyset}$为一个万物之集,为了取消非空的集合簇这一限制,可以把所有的集合限制在一个所谓全集$E$上,此时$\cap{\emptyset}=E$,$\mathscr{C} \cup \{E\}$.

\begin{axiom}[幂集公理:Axiom of powers]
对每一个集合,存在一个集合簇,它包含所有这个给定集合的子集.
\end{axiom}

集合$E$的幂集常记作$P(E)$.

对于有限集$E$,其中有$n$个元素,则$P(E)$中有$2^n$个元素,这也是幂集这个称谓的来源吧.这个结论可以使用归纳法证明,不过这里只能使用以前的关于自然数的信息,从更严格的角度来看,需要首先定义自然数的含义,也就是这里$n$和$2^n$的含义.

习题:证明$P(E) \cap P(F) = P(E \cap F)$,$P(E) \cup P(F) \subset P(E \cup F)$,它们可以推广到
\[
\bigcap_{X \in \mathscr{C}}{P(X)} = P(\bigcap_{X \in \mathscr{C}}{X}),\quad \bigcup_{X \in \mathscr{C}}{P(X)} \subset P(\bigcup_{X \in \mathscr{C}}{X}).
\]

$P(E) \cap P(F)$中的元素是集合$X$,$X \subset E$,$X \subset F$ $\Leftrightarrow$ $A \subset E \cap F$ $\Leftrightarrow$ $X \in P(E \cap F)$.

$X \in P(E) \cup P(F)$ $\Rightarrow$ $X \in P(E)$或$X \in P(F)$ $\Rightarrow$ $X \subset E$或$X \subset F$ $\Rightarrow$ $X \subset E \cup F$, $\Rightarrow$ $X \in P(E \cup F)$.

反过来为什么不一定成立?$X \in P(E \cup F)$ $\Rightarrow$ $X \subset E \cup F$;此时$X$中可能一部分属于$E$,一部分属于$F$,$X \notin P(E)$或$P(F)$.只要$E-F\neq \emptyset$,或$F-E \neq \emptyset$,上面的包含关系就是真包含.

\section{2013年5月1日}
有序对:对于$A=\{a,b,c,d\}$,如何来定义一个次序呢?对于一个次序$c,b,d,a$,和如下集合有一个一一对应关系:
\[
\{\{c\},\{c,b\},\{c,b,d\},\{c,b,d,a\}\}.
\]

$a$和$b$的有序对定义为$(a,b)=\{\{a\}, \{a,b\}\}$.这里需要证明这个定义的合理性,也就是,若$(a,b)=(x,y)$,必有$a=x$,$b=y$.书中给出了详细的证明.

接下里的问题是:对于集合$A$,$B$,是否存在集合包含所有的$(a,b)$,其中$a \in A$,$b \in B$,通过说明$(a,b) \in P(P(a \cup B))$,可以证明这样的集合石存在的,由此引出了笛卡尔积的定义,反过来,对于任一笛卡尔积,都能表示为$A \times B$的形式,书中同样给出了讨论.

习题:
\begin{enumerate}
\item $(A \cup B) \times X = (A \cup X) \cup (B \cup X)$;
\item $(A \cap B) \times (X \cap Y) = (A \times X) \cap (B \times Y)$;
\item $(A-B) \times X = (A \times X) - (B \times Y)$.
\end{enumerate}

\begin{enumerate}
\item $\forall (a,b) \in (A \cup B) \times X$ $\Rightarrow$ $a \in A \cup B$,$b \in X$ $\Rightarrow$ $a \in A$或$a \in B$ $\Rightarrow$ $(a,b) \in A \times X$或$(a,b) \in B \times X$ $\Rightarrow$ $(a,b) \in (A \times X) \cup (B \times X)$;反之也是成立的.
\item $(a,b) \in (A \cap B) \times (X \cap Y)$ $\Leftrightarrow$ $a \in A \cap B$,$b \in X \cap Y$ $\Leftrightarrow$ $(a,b) \in A \times X$,$(a,b) \in B \times Y$ $\Leftrightarrow$ $(a,b) \in (A \times X) \cap (B \times Y)$.
\item $(a,b) \in (A-B) \times X$ $\Leftrightarrow$ $a \in A-B$,$b \in X$ $\Leftrightarrow$ $a \in A$,$a \notin B$ $\Leftrightarrow$ $(a,b) \in A \times X$,$(a,b) \notin B \times X$ $\Leftrightarrow$ $(A \times X)-(B \times X)$.
\end{enumerate}

我们看看:$(A \cap B) \times X = (A \times X) \cap (B \times X)$成立吗?

$(a,b) \in (A \cap B) \times X$ $\Leftrightarrow$ $a \in A \cap B$,$b \in X$ $\Leftrightarrow$ $a \in A$且$a \in B$,$b \in X$ $\Leftrightarrow$ $(a,b) \in A \times X$, $(a,b) \in B \times X$ $\Leftrightarrow$ $(a,b) \in (A \times X) \cap (B \times X)$.

关系:关系是一组有序对的集合,$a$和$b$有关系$R$是指$(a,b) \in R$.

几个例子:相等关系$(x,x) \in X \times X$;属于关系$(x,A) \in X \times P(X)$.

一种特殊的关系:等价关系是指满足自反,对称和传递三个性质的关系.所谓自反是指$aRa$,对称是指$aRb\Rightarrow bRa$;传递是指$aRb$,$bRc$ $\Rightarrow$ $aRc$.这里书中有一道题目,实际上是要求说明这三条性质不能从其中两条件性质推出第三条.这里试着做一下:

(1)集合的包含关系满足自反和传递,不一定满足对称.

(2)考虑$X=\{1,2,3\}$,$R=\{(1,1),(2,2),(3,3),(1,2),(2,3),(2,1),(3,2)\}$,那么自反成立,对称也是成立的,但是不满足传递:$1R2$,$2R3$,无法得到$1R3$.

(3)接下来需要构造一个自反不成立,其他两个性质成立的关系,可是我们应该注意到,一旦有两个不同的$a$,$b$,满足$aRb$,那么根据对称$bRa$,根据传递,必有$aRa$和$bRb$.于是我们需要的是一个小队孤立的$a$.构造如下:$X=\{1,2,3\}$,R=\{(2,2),(2,3),(3,2),(3,3)\},注意到自反不成立,因为存在$1\in A$,$1R1$不成立.对称和传递是成立的.

等价关系在数学中具有特殊重要性,尤其在抽象代数中.等价关系和集合的划分有着直接的联系,首先引进几个概念和记号.

集合$X$的划分是指一簇集合$\mathscr{C}$,这簇集合的并集等于$X$,但是集合簇$\mathscr{C}$是两两不相交的.

所谓等价关系$R$的$x$的等价类是指所有和$x$等价的元素组成的集合,记作$x/R$:$x/R=\{y \in X:xRy\}$.我们把$X$的所有等价类组成的集合记作$X/R$.

我们说的等价类和集合划分的关系实际上可以用等式
\[
\mathscr{C} = X/R
\]
表示:等价关系$R$生成的等价类的集合是$X$的一个划分,同样任何一个划分$\mathscr{C}$都能够决定一个等价关系(记作$X/\mathscr{C}$).

证明参考课本,难度不大.

\section{2013年05月11日}
首先讨论映射,书中使用的是函数(function),我在这里为了和其他书本相一致,采取映射这个术语.首先需要了解前面遗漏的两个符号或者术语:关系$R$的定义域和值域.
\[
\text{dom}{R}=\{x: \exists y (xRy)\};\quad \text{ran}{R}=\{y: \exists x(xRy)\}.
\]

集合$X$到$Y$的映射是一种特殊的关系$f$:$\text{dom}{f}=X$,对每一个$x \in X$,存在唯一的$y \in Y$,使$(x,y) \in f$,此时,满足$(x,y) \in f$的$y$记作$f(x)$.$X$到$Y$的所有映射组成的集合是$P(X \times Y)$的子集,记为$Y^X$.

书中讨论了一些术语相关的知识,这里不叙述了.有时$\{(a,b)|(a,b) \in f\}$,称为$f$的图像(graph).

如果$\text{ran}{f}=Y$,则称$f$是映上(onto)的($X$映到$Y$上的).

设$A \subset X$,记号$f(A)$的含义如下:
\[
f(A)=\{y : y=f(x), \exists x \in A\},
\]
这里存在一个问题,我们考虑的集合中,可能出现$A \in X$这个情形,此时$f(A)$的含义具有歧义.这个问题以前从没想过,因为以前很少遇到,或者说几乎不会遇到.

函数的限制与延拓.

设$f$是$Y$到$Z$的函数,$X \subset Y$,对于如下方式构造的$X$到$Z$的函数$g$:
\[
g(x)=f(x),x \in X,
\]
称$g$为$f$在$X$上的限制,$f$为$g$在$Y$上的延拓.记$g=f|X$,$\text{ran}{(f|X)}=f(X)$.

下面是几个映射的例子:

(1)$X \times Y$到$X$的映射:$f(x,y)=x$;$X \times Y$到$Y$的映射:$f(x,y)=y$.

(2)$R$是$X$上的等价关系,$X$到$X/R$的映射:$f(x)=x/R$,称为canonical map.

对于映射$f$,可以定义$X$上的等价关系如下:
\[
x/R=\{y:f(y)=f(x)\},
\]
也就是说$aRb\Leftrightarrow f(a)=f(b)$.令$y \in Y$,$g(y)=\{x \in X:f(x)=y\}$,此时,$g(y)=x/R$.

(3)$A \subset X$,$A$的特征函数:$\chi(x)=1$,$x \in A$,$\chi(x)=0$,$x \notin A$,对于$A \subset X$,或者$A \in P(X)$,则$A \to \chi_A$也是一个一一映射.

习题:(i)$Y^{\emptyset}$恰好有一个元素,$\emptyset$,无论$Y$是否为空集;(ii)如果$X$不是$\emptyset$,则$\emptyset^X$为空.

这里涉及到了空集,证明方法通常是反证.

$Y^{\emptyset}$表示的是所有$\emptyset$到$Y$的映射,需要证明$\emptyset$是映射,而映射又是特殊的关系$R$,需要证明$\emptyset$是关系,或者说$\emptyset$是有序对的集合,反证,如果$\emptyset$不是关系,那么应该存在元素$\alpha$不是有序对,而这是不可能的.接下来,$\emptyset$是一个映射,也就是对每一个$x \in \emptyset$,存在唯一的$y \in Y$,使$x\emptyset{}y$,这是成立的,也就是如果$\emptyset$不是一个映射,也就是存在$x \in \emptyset$,或者不存在$y \in Y$,使得$x \emptyset y$,或者有两个以上的$y_1$,$y_2$,使得$x \emptyset y_1$,$x \emptyset y_2$,这都不可能.对于任何其他映射或者关系,都要求有元素$x \in \emptyset$,这是不可能的,因此,$\emptyset$是$Y^{\emptyset}$的唯一元素.

假设存在$f \in \emptyset^X$,则对于$x \in X$,$\exists y \in \emptyset$使$f(x)=y$,这是不可能的.

簇:从指标集$I$到$X$的映射.把并集和交集运算推广到集合簇,
\begin{gather*}
\bigcup_{i \in I}{A_i} \text{或} \bigcup_{i}{A_i}\\
\bigcap_{i \in I}{A_i} \text{或} \bigcap_{i}{A_i}
\end{gather*}
$\{I_j\}$是$J$上的集合簇,$K=\bigcup_{j}{I_j}$,$\{A_k\}$为$K$上的集合簇,结合律就是
\[
\bigcup_{k \in K}{A_k} = \bigcup_{j \in J}{(\bigcup_{i \in I_j}{A_i})}.
\]

交换律的推广:
\[
(\bigcup_{i \in I}{A_i}) \cup (\bigcup_{j \in J}{A_j}) = (\bigcup_{j \in J}{A_j}) \cup (\bigcup_{i \in I}{A_i}).
\]

设$\{A_i\}$是$X$的子集簇,$B \subset X$,则
\begin{gather*}
B \cap \bigcup_{i}{A_i} = \bigcup_{i}{(B \cap A_i)} \\
B \cup \bigcap_{i}{A_i} = \bigcap_{i}{(B \cup A_i)}
\end{gather*}
这是分配律.

如果$\{A_i\}$和$\{B_j\}$均为集合簇,则
\begin{gather*}
(\bigcup_{i}{A_i}) \cap (\bigcup_{j}{B_j}) = \bigcup_{ij}{(A_i \cap B_j)} \\
(\bigcap_{i}{A_i}) \cup (\bigcap_{j}{B_j}) = \bigcap_{ij}{(A_i \cup B_j)}
\end{gather*}
这里$\bigcup_{ij}$是指$\bigcup_{(i,j) \in I \times J}$.

设$x \in (\bigcup_{i}{A_i}) \cap (\bigcup_{j}{B_j})$$\Leftrightarrow$$x \in \bigcup_{i}{A_i}$且$x \in \bigcup_{j}{B_j}$,$\exists i_0$,$j_0$,$x \in A_{i_0}$,$x \in B_{j_0}$$\Leftrightarrow$$x \in A_{i_0} \cap B_{j_0}$$\Leftrightarrow$$x \in \bigcup_{ij}{(A_i \cap B_j)}$.

另一等式的证明方法类似.

笛卡尔积的推广,笛卡尔积是集合
\[
X \times Y = \{(x,y)|x \in X, y \in Y\},
\]
考虑集合$\{a,b\}$,$a \neq b$;$Z$为$\{a,b\}$上的集合簇
\[
Z = \{\{z_a,z_b\}|z_a \in X, z_b \in Y\},
\]
于是$Z$到$X \times Y$有一个映射:$f(z)=(z_a,z_b)$.推广到一般情形,
\[
\times_{i \in I}{X_i} = \{\{x_i\}_{i \in I}|x_i \in X_i\},
\]
若$X_i=X$,则记号$\times_i{X_i}$变为$X^I$.

设$\{X_i\}_{i \in I}$为集合簇,$X = \times_i{X_i}$,$J \subset I$,此时存在一个自然的映射,所谓的投影映射.$X \to \times_{i \in J}{X_i}$.

$x \in X$,$f(x)=y \in \times_{i \in J}{X_i}$,其中$y_i=x_i$,$i \in J$.

习题:证明$(\bigcup_{i}{A_i}) \times (\bigcup_{j}{B_j}) = \bigcup_{ij}{(A_i \times B_j)}$,对交也是成立的,另外,$\bigcap_{i}{X_i} \subset X_j \subset \bigcup_{i}{X_i}$,有这个结论,可以把交集和并集定义为包含关系的极大极小值.

(1)设$(a,b) \in (\bigcup_{i}{A_i}) \times (\bigcup_{j}{B_j})$$\Leftrightarrow$$a \in \bigcup_{i}{A_i}$,$b \in \bigcup_{j}{B_j}$$\Leftrightarrow$$a \in A_{i_0}$,$b \in B_{j_0}$$\Leftrightarrow$$(a,b)\in A_{i_0} \times B_{j_0}$$\Leftrightarrow$$(a,b)\in \bigcup_{ij}{(A_i \times B_j)}$.

(2)$\forall x \in \bigcap_{i}{X_i}$$\Rightarrow$$x \in X_j$$\Rightarrow$$\bigcap_{i}{X_i}\subset X_j$;$\forall x \in X_j$$\Rightarrow$$x \in \bigcup_{i}{X_i}$$\Rightarrow$$X_j \subset \bigcup_{i}{X_i}$.

设$X_j \subset Y$,对任意$j$成立,则$\bigcup_{i}{X_i} \subset Y$成立;这意味着$\bigcup_{i}{X_i}$是满足对所有的$j$,$X_j \subset Y$的$Y$中的最小的.

若对每一个$j$,$Y \subset X_j$,则$Y \subset \bigcap_{i}{X_i}$,这意味着$\bigcap_{i}{X_i}$是满足对所有的$j$,$Y \subset X_j$中的最大的集合.


\section{2013年05月12日}
反函数与复合函数

首先是概念.

(1)$f$为$X$到$Y$的映射,定义$f^{-1}$为$P(Y)$到$P(X)$的映射,即$B \subset Y$,
\[
f^{-1}(B) = \{x \in X:f(x) \in B\},
\]
这里$f^{-1}(B)$称为$B$在$f$下的逆像(inverse image).$f^{-1}$还有另一个含义:从$f$的值域到$X$的一个函数,$f^{-1}(y)=\{x\}$$\Leftrightarrow$$f(x)=y$,不过这只是对于一一对应成立.

(2)$f$是$X$到$Y$的映射,$g$是$Y$到$Z$的映射,可以定义一个$X$到$Z$的映射$h$如下:$h(x)=g(f(x))$,$x \in X$,称$h$为$f$与$g$的复合映射.

下面是一些重要的关系式:

(1)$\{A_i\}$为$X$的子集簇,则
\begin{gather*}
f(\bigcup_{i}{A_i}) = \bigcup_{i}{f(A_i)} \\
f(\bigcap_{i}{A_i}) \subset \bigcap_{i}{f(A_i)}
\end{gather*}

第一个关系式:$y \in f(\bigcup_{i}{A_i})$$\Leftrightarrow$$\exists x \in \bigcup_{i}{A_i}$,$f(x)=y$$\Leftrightarrow$$x \in A_{i_0}$,$f(x)=y$$\Leftrightarrow$$y \in f(A_{i_0})$$\Leftrightarrow$$y \in \bigcup_{i}{f(A_i)}$.

第二个关系式:$y \in f(\bigcap_{i}{A_i})$$\exists x \in \bigcap_{i}{A_i}$,$f(x)=y$$\Rightarrow$$y \in f(A_i)$,$\forall i$$\Leftrightarrow$$y \in \bigcap_{i}{f(A_i)}$.

第二个等式之所以不能成立等式,原因在于$y \in f(A_i)$,$\forall i$,是无法得出$\exists x \in \bigcap_{i}{A_i}$,$f(x)=y$.除非映射是一一的.下面是一个例子:$A_1=\{1,2\}$,$f(1)=1$,$f(2)=2$,$A_2=\{1,3\}$,$f(1)=1$,$f(3)=1$,此时$f(A_1)=\{1,2\}$,$f(A_2)=\{1,2\}$,这说明$2 \in f(A_2)$,但是不存在$x \in A_1 \cap A_2$,使$f(x)=2$.

(2)$f$是$X$到$Y$上(onto)的函数的充要条件是$Y$的任一非空子集在$f$下的逆像是$X$的非空子集.

$\Rightarrow$ 设$y$属于$Y$的任一非空子集B,则由于$f$是$X$到$Y$上(onto)的,则存在$x$,使得$f(x)=y$,于是$x \in f^{-1}(y)$,$f^{-1}(y) \subset f^{-1}(B)$,$f^{-1}(B)$非空.

$\Leftarrow$ 考虑单元素集$\{y\}$,由于$f^{-1}(y)$非空,存在$x \in f^{-1}(y)$,则$f(x)=y$.

$f$是一一对应(one-to-one)的充要条件是$f$的值域中的每一个单元素集合在$f$下的逆像是$X$的单元素集合.

(3)$B \subset Y$,$f(f^{-1}(B)) \subset B$.$f$是$X$到$Y$上(onto)的函数,$f(f^{-1}(B))=B$.

证明难度不大,书中给出详细的过程.

(4)$A \subset X$,则$A \subset f^{-1}(f(A))$.$f$是一一对应,则$A = f^{-1}(f(A))$.

证明难度不大,书中给出详细的过程.

(5)$\{B_i\}$为$Y$的子集簇,则
\[
f^{-1}(\bigcup_{i}{B_i}) = \bigcup_{i}{f^{-1}(B_i)};\quad f^{-1}(\bigcap_{i}{B_i}) = \bigcap_{i}{f^{-1}(B_i)}.
\]

(6)$f^{-1}(Y-B) = X - f^{-1}(B)$;

(7)函数的复合不满足交换律,但是满足结合律:$h(gf)=(hg)f$.

(8)把逆映射和复合映射联系起来的等式特别重要.$f$是$X$到$Y$的映射,$g$是$Y$到$Z$的映射,此时,$f^{-1}$是$P(Y)$到$P(X)$的映射,$g^{-1}$映$P(Z)$到$P(Y)$,此时,$gf$与$f^{-1}g^{-1}$都有意义,且有$(gf)^{-1}=f^{-1}g^{-1}$.

这里的一些关系可以推广到更一般的关系上去,书中给出了详细的过程,这里不叙述了.考虑$X$上的关系:$I$为$X$上的相等关系,此时$I$类似于乘法单位元的作用,$IR=RI=R$对于$X$中的每一个关系成立,使用代数形式来表示等价关系为:(i)自反:$I \subset R$;(ii)对称:$R \subset R^{-1}$;(iii)传递:$RR \subset R$.

习题:$f$为$X$到$Y$的映射:(i)设$g$是$Y$到$X$的映射,若$gf$是$X$上的恒等映射,则$f$是一一对应,而$g$是$Y$到$X$上(onto)的.(ii)对$X$的任意子集$A$,$B$,$f(A \cap B)=f(A) \cap f(B)$成立的充要条件是$f$是一一对应;(iii)对$X$的任意子集$A$,$f(X-A) \subset Y-f(A)$成立的充要条件是$f$是一一对应;(iv)对$X$的任意子集$A$,$Y-f(A) \subset f(X-A)$成立的充要条件是$f$是$X$到$Y$上(onto)的映射.

(i)设$f(x_1)=f(x_2)$ $\Rightarrow$ $gf(x_1)=gf(x_2)$ $\Rightarrow$ $x_1=x_2$,从而$f$是一一对应;对于$\forall x \in X$,令$y=f(x)$,则$g(y)=gf(x)=x$,即$g$是$Y$到$X$上的.

(ii)设$f(A \cap B) = f(A) \cap f(B)$,对任意$A$,$B$成立,欲证$f$是一一对应,设$f(x_1)=f(x_2)$,令$A=\{x_1\}$,$B=\{x_2\}$,则$f(A) \cap f(B) \neq \emptyset$,设$y \in f(A) \cap f(B)$,这意味着$A \cap B \neq \emptyset$,这只有在$x_1=x_2$时才可能.

反之,如果$f$是一一对应.$\forall y \in f(A \cap B)$ $\Leftrightarrow$ $\exists x \in A \cap B$,使$y=f(x)$ $\Leftrightarrow$ $\exists x \in A$且$x \in B$ $\Leftrightarrow$ $y \in f(A)$,且$y \in f(B)$(这一步使用了$f$是一一的)$\Leftrightarrow$ $y \in f(A) \cap f(B)$.

(iii)设$f(X-A) \subset Y-f(A)$对任意$A \subset X$成立,欲证$f$是一一对应,设$f(x_1)=f(x_2)=y$,令$A=\{x_1\}$,我们要证明$x_2 \in A$,从而$x_1=x_2$,若$x_2 \neq A$,则$y \notin Y - f(x_1)$,$x_2 \in X-A$,从而$y \in f(X-A)$,矛盾.

反之,若$f$是一一对应,$\forall y \in f(X-A)$ $\Rightarrow$ $\exists x \in X-A$,使$f(x)=y$ $\Rightarrow$ $x \notin A$ $\Rightarrow$ $f(x) \notin f(A)$ (这里面使用了$f$是一一的) $\Rightarrow$ $f(x) \in Y-f(A)$.

(iv)$Y-f(A) \subset f(X-A)$,$\forall y \in Y$,我们要找一个$x$,使$f(x)=y$.对于$B=Y-\{y\}$,令$A=f^{-1}(B)$,则$f(A)=B$ $\Rightarrow$ $Y-f(A)=\{y\} \subset f(X-A)$ $\Rightarrow$ $\exists x \in X-A$,使$f(x)=y$.

反之,$f$是$X$到$Y$上(onto)的,则$\forall y \in Y-f(A)$,设$f(x)=y$ $\Leftrightarrow$ $y \notin f(A)$ $\Rightarrow$ $x \notin A$ $\Leftrightarrow$ $x \in X-A$ $\Rightarrow$ $f(x) \in f(X-A)$.

上面的推导过程中有几个$\Rightarrow$,这也是关系式为$\subset$而不是$=$的原因所在,这里简单讨论:$f(x) \notin f(A) \Rightarrow x \notin A$,若$x \in A \Rightarrow f(x) \in f(A)$.反过来是不一定成立的.也就是$f(x) \in f(A)$,不一定能够得出$x \in A$,因为有可能有另一个$x'$,使得$f(x')=f(x)$.

\section{2013年5月18日}
数是什么?这一节开始从集合论的角度建立起自然数体系,我个人的看法是:整个过程中有些思想方法值得学习,其余的可以仅仅做个了解,或者说类似书中引言的说法(read it,absorb it, and forget it),掌握之后,就可以把它忘记了,我们在使用的时候,继续遵循我们一致接收的中小学教育即可.北师大郇中丹老师的一个网络课程(数学分析,强烈推荐,即使不看其他的,至少应该看看第一讲绪论,这一讲虽然不涉及任何数学知识,但是很精彩!)中表示过这样一个意思(在第二讲集合论初步中):数学就是把我们生活中的一些事实说明白(我稍微扩充了一下).那么这一节以及接下来的几节内容,就是为了把数说明白,为数建立更严密的逻辑基础.我认为它有理论上的重要性,对于应用基本上意义不大.书中作了一个类比,我觉得很能说明一些东西.我们如何来定义"米"这个单位呢?或者如何定义长度呢?我们是通过指定一个物体的长度作为基准,然后所有其他的长度与之作比较.这里面,其实我们还有一个疑问,长度本身又是什么含义?在这里,我们其实并没有给出长度或者"米"本身的明确的定义,我们通过比较来给出我们需要的.

类似的,我们如何来定义自然数呢?我们其实无法说出自然数是什么?但是我们可以通过一些关系建立起自然数.例如,我们不知道2到底是什么?但是我们知道它是排在1后面的,在3前面的,2是1的后继,3是2的后继,这说明我们需要一个后继的概念,用什么方法来定义这个后继呢:$x^+=x \cup \{x\}$.这是目前常用的一个方法,首先这个方法只涉及集合的概念,需要的很少,我们再定义$0=\emptyset$,于是通过$\emptyset$,并集等就得到了自然数,这里我们需要一个公理:

\begin{axiom}[无限公理(Axiom of infinity)]
存在一个集合,包含0和它的每一个元素的后继(successor).
\end{axiom}

集合$A$称为successor set,如果$\emptyset \in A$,且$\forall x \in A$,必有$x^+ \in A$.于是无限公理可以描述为:存在一个successor set $A$.

任一非空的successor簇的交集是一个successor set.证明如下:令$A=\bigcup_{i}{A_i}$,$i \in I$,$I$非空,由于$0 \in A_i$,$\forall i \in I$,于是$0 \in \bigcap_{i}{A_i}$,也就是$0 \in A$.设$x \in A$,则$x \in A_i$,$\forall i \in I$,于是$x^+ \in A_i$,$\forall i \in I$,故$x^+ \in \bigcap_{i}{A_i}$.

由此我们可以令$\omega$为每一个successor set的子集,这样的$\omega$是存在的并且是唯一的.$\omega$中的元素就是自然数.

书中的这一节的后面是定义了一个序列sequence,并且把交,并,笛卡尔积推广到序列上,这里不讨论了.

\section{2013年05月25日}
Peano公理

这一节继续自然数的讨论.从上一节关于自然数的定义,可以得到如下结论:

(I)$0 \in \omega$;

(II)如果$n \in \omega$,则$n^+ \in \omega$;这里$n^+= n \cup \{n\}$.

(III)(数学归纳原理)设$S \subset \omega$,若$0 \in \S$,并且当$n \in S$时必有$n^+ \in S$,则$S=\omega$;

(IV)$n^+ \neq 0$,对所有$n \in \omega$成立;因为$n^+$包含$n$,非空,自然有$n^+ \neq 0$.

(V)若$n,m \in \omega$,且$n^+=m^+$,则$n=m$.

这一个结论的证明比较费劲,书中给出了详细的过程,由于这个过程是归纳法,反证法相关方法的典型应用,这里给出详细过程.它的证明需要两个辅助命题:(i)不存在自然数,它是其元素的子集,由此可以得到$n \notin n$.(ii)一个自然数的每一个元素都是它的子集.对于(ii),引入一个概念:transitive set.集合$E$称为transitive set,如果集合$E$包含所有它的元素,或者说:$x \in y$,$y \in E$,必有$x \in E$,也就是$y \subset E$.于是(ii)实际上是说每一个自然数都是transitive set.

(i)我们实际上需要证明:对于每一个自然数$n$,$x \in n$,那么$n$不可能是$x$的子集.如果令$S=\{n \in \omega| \forall x\in n, n \subset x \text{不成立}\}$,我们需要证明$S = \omega$.使用归纳法:

(a)$0 \in S$成立,因为不存在$x \in 0$,也就无所谓$0 \subset x$问题了.

(b)若$n \in S$,要证明$n^+ \in S$.

首先$n$本身是$n$的子集,这也就意味着不可能有$n \in n$,即$n \notin n$,于是$n^+$不是$n$的子集,因为$n \in n^+=n \cup \{n\}$.若$n^+ \subset x$,则$n \subset x$,而$n \in S$,故$x \notin n$,也就是说$n^+$也不可能是$n$的任一元素的子集,有了这两点,说明$n^+$不是$n^+$的元素的子集,$n^+ \in S$.获证.

(ii)令$S = \{n \in \omega| n\text{为transitive set}\}$.需要证明$S = \omega$,同样使用归纳法.

(a)首先$0 \in S$成立,否则意味着存在$y \in 0$,使得$y$不是$0$的子集,这是不可能的.

(b)若$n \in S$,要证明$n^+ \in S$;要时刻注意$n^+ = n \cup \{n\}$,若$x \in n^+$,则或者$x \in n$,或者$x = n$,若$x \in n$,而$n \in S$,于是$x \subset n \subset n^+$,若$x=n$,则$x \subset n^+$也成立,这意味着$x$是$n^+$的子集.$n^+ \in S$.获证.

有了这两个辅助命题的帮助,可以来证明(V)了.$n^+=m^+$,而$n \in n^+$,于是$n \in m^+$,于是$n \in m$或者$n=m$,由对称性,从另一方面推导将有$m \in n$或$m=n$,若$m \neq n$,则有$n \in m$和$m \in n$同时成立,根据(ii)每一个自然数满足transitive,有$n \in n$,注意到$n \subset n$总是成立,又和(i)发生矛盾.

上述五条也常被称为Peano公理,这可能是用的最多的自然数的公理体系.德国数学家E.Landau写了一本《分析基础》(Foundations of Analysis),其中从Peano公理出发,完整推导了自然数到整数,到有理数,实数,复数的整个过程.

数学归纳法不仅仅可以用于证明,还可以用作定义.

设$f$是$X$到$X$的映射,$a \in X$,一个比较自然的想法是按如下方式定义一个序列$\{u(n)\}$(从$\omega$到$X$的映射):$u(0)=a$,$u(1)=f(u(0))$,$u(2)=f(u(1))$,...使用数学归纳法可以证明只要存在这样的$u(n)$,它就是唯一的,接下来还需要证明存在性,这就是下面的定理:

\begin{theorem}[归纳定理(Recursion Theorem)]
若$a$是$X$中的元素,$f$是$X$到$X$的映射,那么存在一个$\omega$到$X$的映射$u$,使得$u(0)=a$,且对所有$n \in \omega$,有$u(n^+)=f(n(n))$.
\end{theorem}

这个定理的应用就是所谓的递归定义,书中给出了详细的证明.这里复述如下:证明的总的思路是构造一个$u$,然后证明$u$是一个映射,并且满足条件.

首先注意到$\omega$到$X$的映射是$\omega \times X$的一个特殊子集,令$\mathscr{C}$为所有满足如下条件的$A$的集合:$A$是有序对的集合,也就是$A \subset \omega \times X$,且$(0,a) \in A$,对于任意的$(n,x) \in A$,必有$(n^+,f(x)) \in A$.

这个$\mathscr{C}$是非空的,因为$\omega \times X$本身满足条件,令$u$为所有$\mathscr{C}$中元素的交集,
\[
u = \bigcap_{A \in \mathscr{C}}{A}.
\]
下面证明$u \in \mathscr{C}$,首先$(0,a) \in A$,$\forall A \in \mathscr{C}$,故$(0,a) \in u$,其次,若$(n,x) \in u$,则$(n,x) \in A$,$\forall A \in \mathscr{C}$,故$(n^+,f(x)) \in A$,$\forall A \in \mathscr{C}$,于是$(n^+,f(x)) \in u$.

接下来若能证明$u$是一个映射,那么$u$就是满足条件的,也就是说对于每一个自然数$n \in \omega$,都有一个唯一的$x \in X$,使得$(n,x) \in u$.从$u$的构造可以知道,$x$的存在是成立的,接下来只要证明唯一性,也就是如果$(n,x) \in u$,$(n,y) \in u$,那么必有$x=y$.同样使用归纳法来证明,先构造集合$S = \{n \in \omega : (n,x) \in u, (n,y) \in u \Rightarrow x=y\}$.证明中要使用$u$的一个在包含关系下的极小性质,需要一些反证技巧.

(a)$0 \in S$,如果不成立,在存在$b$使得$(0,b) \in u$,于是考虑$u - \{(0,b)\}$,它同样是$\mathscr{C}$中的元素,这与$u$的定义矛盾.

(b)设$n \in S$,欲证$n^+ \in S$.$n \in S$$\Rightarrow$$(n,x) \in u$,$x$是唯一的,于是$(n^+,f(x)) \in u$,(这是$u$的定义),若$n^+ \notin S$,$\exists y \neq f(x)$,$(n^+,y) \in u$,考虑$u-\{(n^+,y)\}$,此时再次得到一个$u$的真子集同样属于$\mathscr{C}$,从而引发矛盾.

习题:(1)若$n$为自然数,则$n \neq n^+$;(2)若$n \neq 0$,则存在$m$使得$n=m^+$;(3)证明$\omega$是reansitive set.(4)若$E$是某个自然数的非空子集,则存在$k \in E$,使得对于任意的异于$k$的$m \in E$,有$k \in m$.

(1)令$S = \{n \in \omega:n \neq n^+\}$,证明$S=\omega$.

首先$0 \in S$,因为$0 \neq o^+$.

其次,若$n \in S$,要证明$n^+ \in S$,否则,意味着$n^+=(n^+)^+$,于是$n=n^+$,与假设矛盾.

(2)结论很明显,问题在于我们如何表示出这个$m$,$n \neq 0$,说明存在$x \in n$.我们令
\[
m = \bigcup_{A \in n}{A}.
\]
下面证明$n=m^+$.

$\forall x \in n$$\Rightarrow$$x \subset m$$\Rightarrow$$x \in m^+$.

$\forall x \in m^+$,意味着$x=m$或$x \in m$,若$x \in m$,$\exists A_0, x \in A_0$,$A_0 \in n$,$A_0 \subset n$(前面的辅助命题(ii)),于是$x \in n$.获证.

这里似乎有问题:第一部分$x \subset m$$\Rightarrow$$x \in m^+$恐怕不严密,因为对于一般集合不成立,只是自然数满足,第二部分缺少$x=m$的情形.$m \in n$的证明并不轻松.

(3)它实际上是要证明每一个自然数是$\omega$的子集.使用归纳法.$0$显然是$\omega$的子集;假设$n$是$\omega$的子集,欲证$n^+$是$\omega$的子集,$\forall \in n^+$,应该证明$x \in \omega$,此时$x=n$或者$x \in n$,$x=n \in \omega$;$x \in n \subset \omega$,同样有$x \in \omega$.获证.

(4)实际上是寻找集合$E$中的最小自然数.令$k=\bigcap_{A \in E}{A}$即可.因为$E$非空,$k$的存在性不是问题.但是需要证明$k \in E$,以及$k \in m$,$m \in E$,且$m \neq k$.

(2)和(4)的构造我觉得是没有问题的,可是严密的证明总是没有得到,想了几天没有太好的思路.

\section{2013年05月26日}
算术

使用归纳定义,可以给出自然数的加法,乘法和幂的定义,减法和除法在自然数并不总是可行的,可以定义为加法和乘法的逆运算.

加法:$s_m(0)=m$,$s_m(n^+)=(s_m(n))^+$;也就是$s_M(n)=m+n$.

乘法:$p_m(0)=0$,$p_m(n^+)=p_m(n)+m$;也就是$p_M(n)=m \cdot n$.

幂次:$e_m(0)=1$,$e_m(n^+)=e_m(n) \cdot m$;也就是$e_M(n)=m^n$.

书中给出了加法结合律的证明,作为数学归纳法的典型应用,这里也记录之:

要证明
\[
(k+m)+n=k+(m+n).
\]
对$n$作归纳.

(i)$(k+m)+0=k+m=k+(m+0)$;

(ii)假设对于$n$成立,对于$n^+$来说,
\[
(k+m)+n^+ = ((k+m)+n)^+=(k+(m+N))^+=k+(m+n)^+=k+(m+n^+).
\]

对于交换律:$m+n=n+m$,在使用归纳法的时候,需要一些技巧.直接对$m$或者$n$做归纳,都不容易成功,因为$m+0=0+m$这一步本身就很麻烦,于是我们分两步走:

(1)$0+n=n+0$.

(i)$n=0$显然成立:$0+0=0$.

(ii)$0+n^+=(0+n)^+=(n+0)^+=n^+=n^++0$.最后一步自然数加法的定义.

(2)$m^++n=(m+n)^+$.

仍然是对$n$归纳:

(i)$m^++0=m^+=(m+0)^+$;

(ii)假设对$n$成立,$m^++n^+=(m^++n)^+=((m+n)^+)^+=(m+n^+)^+$.

接下来证明交换律,对$m$进行归纳.(i)$m=0$已经成立,(ii)假设交换律对于$m$成立,对于$m^+$来说,$m^++n=(m+n)^+=(n+m)^+=n+m^+$.

乘法的交换律和结合律可以用同样的方法证明.

两个自然数$m$和$n$是可比较的(comparable),若$m \in n$,或$m=n$,或$n \in m$.我们有结论:任意两个自然数是可比较的.这个结论比较重要,书中给出了详细证明.令
\[
S(n)=\{m \in \omega: m \text{与} n\text{是可比较的}\},
\]
以及$S = \{n \in \omega:S(n)=\omega\}$.我们需要证明$S=\omega$,这里同样应用数学归纳法.

(i)$0 \in S$,或者$S(0)=\omega$,使用归纳法:(a)$0 \in S(0)$,(b)$m \in S(0)$,则$m^+ \in S(0)$.

(ii)若$n \in S$,则$n^+ \in S$,首先$0 \in S(n^+)$,因为$n^+ \in S(0)$.接下来需要从$m \in S(n^+)$出发推导出$m^+ \in S(n^+)$.$m \in S(n^+)$,意味着$m \in n^+$,或者$m=n^+$,或者$n^+ \in m$,后两者立即可以得到$n^+ \in m$,故只要讨论$m \in n^+=n \cup \{n\}$情形.于是$m=n$或$m \in n$,从$m=n$可以得到$m^+=n^+$.于是接下来讨论$m \in n$这一情形,从$S(n)=\omega$可知$m^+ \in S(n)$,于是$m^+ \in n$,或者$m^+=n$,或者$n \in m^+$.前面两个可以得到$m^+ \in n^+$,这就讨论一个情形:$n \in m^+$,但是它不能和$m \in n$同时成立.因为$n \in m^+$会得到$n=m$,或者$n \in m$,都有$n \subset m$,这与$m \in n$不能同时成立(前一节的结论).

$m \in n$,$m=n$和$n \in m$有且仅有一个成立.

若$m \in n$和$m=n$同时成立,$m=n$$\Rightarrow$$n \subset m$,不可能;若$m \in n$和$n \in m$$\Rightarrow$$m \in m$(transitive),$m \subset m$不可能.若$m=n$,$n \in m$,$\Rightarrow$$m \subset n$,同样不成立.

任一自然数不可能是它的元素的子集,另一个结论是不相同的$m$,$n$满足$m \in n$的充要条件是$m \subset n$.$m \in n \Rightarrow m \subset n$可以同$n$的transitive性质得到;$m \subset n$,$n \in m$不可能,否则$m$是它的某个元素$m$的子集.

$m \in n$定义为$m \subset n$,$m \in n$或者$m=n$定义为$m \le n$.

习题:若$m<n$,则$m+k<n+k$,若$m<n$,$k \neq 0$,则$m \cdot k < n \cdot k$,若$E$是非空的自然数集,则存在$k \in E$,使得$k \le m$对所有的$m \in E$成立.

(i)对$k$施加归纳法,(a)$k=0$时,就是题设本身,结论成立;(b)假设结论对于$k$成立,对于$k^+$来说,$m+k^+=(m+k)^+<(n+k)^+=n+k^+$.我们需要证明若$m<n$,则$m^+<n^+$,这里一点可以这样得到:$m<n$,可以得到$m \in n$$\Rightarrow$$m \subset n$$\Rightarrow$$m \subset n^+$,$m \in n$$\Rightarrow$$m \in n^+$$m \subset n^+$,因此$m^+ \in n^+$,$m^+ < n^+$.

(ii)似乎不是很容易,我们换一个思路,我们证明,对于任意自然数$k$,有$mk^+<nk^+$.使用归纳法以及刚刚证明的关于加法的结论即可.

(a)$k=0$,$mk^+=m<n=nk^+$;(b)$m(k^+)^+=mk^++m<nk^++m<nk^++n=n(k^+)^+$.

(iii)这实际上是前一节已经出现过的题目.

\section{2013年06月01日}
继续上一节的算术.

我们称集合$E$和$F$是对等(equivalent)的,如果存在一个$E$和$F$之间的一一对应,这是一个等价关系.

自然数$n$的每一个真子集对等于某个比$n$小的自然数,或者说$n$的元素.证明使用归纳法.$n=0$是平凡的,如果对于$n$成立,那么对于$n^+$来说,对于$n^+$的真子集$E$,可能出现这样几个情形:$E$是$n$的真子集,此时根据归纳假设,结论成立;若$E=n$,那结论自然成立,$n$中恒等映射;最后一个情形是$n \in E$,此时,存在$k \in n$,但是$k \notin E$,定义$E$上的映射$f$如下:当$i \neq n$时,$f(i)=i$,否则$f(n)=k$,这个$f$是$E$到$n$的一一对应.再使用归纳假设,以及一一对应关系的传递性,可以知道结论成立.

一个多少令人有些意味的事实是:一个集合可以和它的真子集对等.一个最好的例子就是$f(n)=n^+$,自然数集与非零自然数集之间的一个一一对应关系.但是对于自然数$n$,它不能和$n$的任一真子集对等.同样可以使用归纳法证明.从$n$到$n^+$这一步,分$n \in E$和$n \notin E$来讨论,$n \in E$时,取$E-\{n\}$.

集合$E$称为有限(finite)的,如果它与某个自然数对等,否则就称为无限的(infinite),

习题:用这个定义证明自然数集$\omega$是无限的.

我们使用反证法,通过构造出一个自然数$n$和它的真子集之间的一一对应来得到矛盾.首先$\omega$显然不可能和$0$对等,于是假设$\omega \sim n$,那么$n \neq 0$,前面已经证明过,此时存在自然数$m$,$n=m^+$.

假设$f$是$\omega$和$n$之间一一对应,设$f(0)=k$,则$k \le n$,我们构造$\omega-\{0\}$到$m$之间的映射$g$如下:若$x<k$,令$g(x)=f(x)$,当$x \ge m$时,$g(x)=f(x^+)$,那么这个$g$是一一的,于是我们有了$n \sim \omega \sim \omega-\{0\} \sim m$,于是自然数集和它的一个真子集对等了,发生矛盾.

一个集合最多与一个自然数对等.

任何两个不同的自然数$m$,$n$,必有$m \in n$或$n \in m$,无论如何其中一个是另一个的真子集,而自然数不可能与其真子集对等.由此可知有限集不可能和它的真子集对等,即对于有限集来说,整体大于部分成立.

习题:使用有限的定义的这个推论证明$\omega$是无限的.

因为$\omega$可以与$\omega-\{0\}$对等,从而不是有限的.

一个自然数的每一个子集与某个自然数对等,自然说明有限集的每一个子集是有限的.

有限集$E$的元素个数定义为与$E$对等的那个自然数,记为$\#E$,这个自然数是唯一的,$\#E$是$P(X)$到$\omega$的一个映射.

(1)$E \subset F$,则$\#E \le \#F$,$E$,$F$都是有限集.

$E \sim \#E$,$F \sim \#F$,$\#E$对等于$F$的某个子集,$\#E \le \#F$.

(2)$E$,$F$为有限集,则$E \cup F$也是有限的,而且当$E$与$F$不相交时:$\#(E \cup F) = \#(E) + \#(F)$.

若$m$与$n$为自然数,则$m$在$m+n$中的补集与$n$对等.对$n$使用归纳法.

(3)$E$,$F$为有限集,则$E \times F$与$E^F$均为有限集,且$\#(E \times F) - \#(E) \cdot \#(F)$,$\#(E^F)=(\#(E))^{(\#(F))}$.

习题:有限个有限集的并集是有限的,若$E$是有限的,则$P(E)$是有限的,$\#(P(E))=2^{\#(E)}$,若$E$是非空的自然数的有限集,则存在$k \in E$,使得$m \le k$,$\forall m \in E$.

\section{2013年06月09日}
次序(order)

这一节主要是各种定义:

(1)集合$X$中的关系$R$称为是反对称的(antisymmetric),如果$xRy$,$yRx$同时成立,必有$x=y$.

(2)关系$R$称为偏序(partial order),如果关系$R$满足自反的(reflexive),反对称的(antisymmetric),传递的(transitive),此时通常使用符号$\le$.$\forall x,y,z \in X$,(i)$x \le x$;(ii)若$x \le y$,$y \le x$,必有$x=y$;(iii)若$x \le y$,$y \le z$,则$x \le z$.

这里之所以使用偏序,是因为有可能在$X$中存在元素$x$,$y$,无法确定序关系,如果对于$X$中的每一个元素$x$和$y$,或者$x \le y$,或者$y \le x$,则称$\le$为全序(total order, simple order, linear order).一个全序集常称为链(chain).

偏序集是指带有偏序关系的集合,记作$(X, \le)$.类似的,全序集是指带有全序关系的集合.

(3)对于$X$中的偏序$\le$,我们称$y \ge x$,如果$x \le y$;$x < y$或者$y>x$,如果$x \le y$,且$x \neq y$,称$x$是predecessor of $y$,$y$是successor of $x$.

对于关系$<$,有(i)$x<y$和$y<x$不能同时成立;(ii)$x<y$,$y<z$,则有$x<z$,也就是$<$是传递的.

反过来,我们可以从$<$出发定义$\le$,如果关系$<$满足(i)和(ii),然后定义$x \le y$,如果$x<y$或者$x=y$,那么$\le$是一个偏序.

可以把$\le$和$<$的这种关系推广到一半的关系,对于关系$R$和$S$,满足$xSy$,如果$xRy$,且$x \neq y$,此时称$S$是strict relation corresponding $R$.$R$是weak relation corresponding $S$.

(4)对于偏序集$(X, \le)$,$a \in X$,集合$s(a)=\{x \in X: x < a\}$,称为the initial segment determined by $a$,$\bar{s}(a)=\{x \in X: x \le a\}$称为the weak initial segment determined by $a$.

(5)若$x \le y$,$y \le z$,称$y$位于$x$和$z$之间(between $x$ and $z$),若$x<y$,$y<z$,称$y$是严格位于$x$和$z$之间(strictly between $x$ and $z$).若$x<y$,并且不存在元素严格位于$x$和$y$之间,称$x$是immediate predecessor of $y$,或者说$y$是immediate successor of $x$.

(6)$X$为偏序集,若存在$a \in X$,使得$a \le x$,$\forall x \in X$,则称$a$为least (first,smallest) element of $X$,从反对称性可知这个元素是唯一的,类似,若存在元素$a \in X$,使得$x \le a$,$\forall x \in X$,称$a$为greatest (last, largest) element of $X$.对于自然数集,按照通常的次序,$0$是first element,而不存在last element,如果把次序颠倒过来,那么$0$是last element,而不存在first element.

(7)对于偏序集$X$,$a \in X$称为minimal element,如果不存在$X$中的元素严格小于$a$ (strictly smaller than $a$),也就是从$x \le a$,将得到$x=a$;类似的,如果不存在元素严格大于$a$,称$a$为maximal element,于是从$a \le x$,必有$x=a$.

必须注意least element和minimal element是有差别的,非空集合$X$的非空子集构成的集合$\mathscr{C}$,以包含关系作为偏序关系,此时每个单元素集合是minimal的,但是$\mathscr{C}$中一般情况下不存在least element,除非$X$中只有一个元素.

(8)$X$为偏序集,$a \in X$,$E \subset X$,称$a$为$E$的下界(lower bound),若$\forall x \in E$,$a \le x$;称$a$为$E$的上界$upper bound$,若$\forall x \in E$,$x \le a$.注意,$E$中可以不存在上界或下界,也可以有多个上界或下界.后一种情况可以以自然数集作例子,$n$为$E$的上界,那么所有大于$n$的自然数都是$E$的上界.引入两个记号:
\[
\begin{aligned}
E_*&=\{a \in X:\forall x \in E, a \le x\} \\
E^*&=\{a \in X:\forall x \in E, x \le a\}
\end{aligned}
\]
$E_*$以及$E_* \cap E$都可能是$\emptyset$,如果$E_* \cap E \neq \emptyset$,此时它必然只有一个元素.对于$E_*$来说,如果存在一个greatest element $a$,则称$a$为$E$的下确界(greatest lower bound或者infimum),记作g.l.b.或inf,类似,若$E^*$中包含一个least element $a$,则称$a$为$E$的上确界(least upper bound或supremum),记作l.u.b.或者sup.

下面给出几个例子:

(1)作为偏序的第一个例子,自然就是集合的包含关系,它是$P(X)$上的偏序,仅当$X$是单元素集的时候是全序.

(2)全序的例子可以在自然数集中得到,

(3)另一个偏序的例子是映射的扩张,给定集合$X$和$Y$,$F$为所有定义域在$X$中而值域在$Y$中的映射组成的集合.定义$F$上的关系$R$如下:$fRg$如果$\text{dom}{f} \subset \text{dom}{g}$,且$f(x)=g(x)$,$\forall x \in \text{dom}{f}$.也就是这意味着$f$是$g$的限制,而$g$是$f$的扩张.如果注意到映射是$X \times Y$的子集,那么$fRg$实际上就是$f \subset g$.

(4)对于集合$\omega \times \omega$,我们可以定义不同的偏序:(i)$(a,b)R(x,y)$,如果$(2a+1)\cdot 2^y \le (2x+1) \cdot 2^b$,(ii)$(a,b)S(x,y)$,如果$a<x$,或者$a=x$且$b \le y$;这个次序称为字典序(lexicographical order).(iii)$(a,b)T(x,y)$,如果$a \le x$且$b \le y$.

习题:使用$R$以及逆来表述关系$R$的反对称性和totality.

$RR^{-1} \subset I$,$R = E \times E$.

\section{2013年06月10日}
选择公理

对于很多有限情形下的操作,我们基本上不需要任何犹豫,但是在无限的情况下,很多操作需要精心考虑(其实在陶哲轩的《实分析》一书的第一章有不少例子).例如本节中出现的选择公理,它主要是针对无限的.

一个集合,或者是空集,或者非空,对于非空集合,必然存在一个元素.对于两个集合$X$和$Y$,对于它们的笛卡尔积$X \times Y$,如果$X$和$Y$中至少有一个为$\emptyset$时,笛卡尔积为$\emptyset$,如果都不是空集,$X \times Y$必定不是空集.这个结论很容易推广到有限个$\{X_i\}$的情形.可以对于无限个$\{X_i\}$的情形,我们需要本节的选择公理(Axiom of choice).

\begin{axiom}[选择公理]
非空集合的非空簇的笛卡尔乘积是非空的(The Cartesian of a non-empty family of non-empty sets is non-empty).
\end{axiom}

用数学语言表示:考虑集合簇$\{X_i\}_{i \in I}$,这里每一个$X_i$都是非空集合,并且指标集$I$也是非空,此时存在一簇元素$\{x_i\}$,$i \in I$,使得$x_i \in X_i$,对每一个$i \in I$.

设$\mathscr{C}$是非空集合簇,此时我们完全可以把$\mathscr{C}$作为指标集,于是可以应用选择公理,这样就存在一个定义在$\mathscr{C}$上的映射$f$,只要$A \in \mathscr{C}$,就有$f(A) \in A$.特别地,令$\mathscr{C}$为非空集$X$的所有非空子集组成的集合,也就是$P(X)-\{\emptyset\}$,于是存在映射$f$,$\forall A \in \mathscr{C}$,$f(A) \in A$,这个映射称为$X$上的选择映射(choice function).直观上说,我们的映射$f$能够从每一个集合中选择出一个元素,这里也是"选择公理"名称的由来.有限的情形是可以证明的,而对于无限情形,则有这个公理来保证.

设$\mathscr{C}$是两两不相交的非空集合簇,此时存在集合$A$,使得$A \cap C$是单元素集合,$\forall C \in \mathscr{C}$.

作为选择公理的应用,我们证明如下结论:若集合$X$是无限的,那么存在一个子集对等于$\omega$.直观的,既然$X$非空,那么存在$x_0 \in X$,由于$X$不与$1$对等,$X-\{x_0\}$是非空的,于是存在$x_1 \in X - \{x_0\}$,如此继续,存在$x_2 \in X - \{x_0,x_1\}$,等等,这将形成一个无限序列$\{x_n\}$,这里$x_n$互相不等.这里出现了一个需要无限选择的情形,需要依赖于选择公理.书中给出了完整的过程.作为选择公理的应用,这里也给出完整过程.

考虑$X$中的选择映射$f$:$P(X)-\{\emptyset\} \to X$,$f(A) \in A$,$\forall A \in \text{dom}{f}$.令$\mathscr{C}$表示$X$中所有的有限子集组成的集合簇.由于$X$是无限的,$\forall A \in \mathscr{C}$,$X-A$非空,因而$X-A \in \text{dom}{f}$.接下来定义映射$g$:$\mathscr{C} \to \mathscr{C}$,$g(A)=A \cup \{f(X-A)\}$,我们从$\emptyset$出发,对于$g$使用归纳原理,存在$\omega$到$\mathscr{C}$的映射$U$,使得$U(0)=\emptyset$,$U(n^+)=U(n) \cup \{f(X-U(n))\}$,$\forall n \in \omega$.此时我们定义映射$v(n)=f(X-U(n))$,只要证明$v$是一个$\omega$到$X$的一个一一映射.那么$\omega$对等于$X$的一个子集($v$的值域).这只需要注意到:

(i)$v(n) \notin U(n)$,$\forall n \in \omega$.理由:因为$f(X-U(n)) \in X-U(n)$,于是$v(n) \notin U(n)$.

(ii)$v(n) \in U(n^+)$,$\forall n \in \omega$.$U(n^+)$的定义.

(iii)对于自然数$n$,$m$,$n \le m$,则$U(n) \subset U(m)$.这同样可以由$U(n)$的定义得到,$U(m)$可以通过$U(n)$添加一个个元素得到的.

(iv)对于自然数$n$,$m$,$n<m$,则$v(n) \neq v(m)$.因为根据(ii)和(iii),$v(n) \in U(m)$,而根据(i),$v(m) \notin U(m)$,因此两者不可能相等.

这里最后的(iv)就是说明$v$把不同的自然数映射到$X$中不同的元素.因为任意两个不同的自然数,必有一个是严格小于另一个的.

上述结论一个更重要的推论,它涉及到了无限的本质:一个集合是无限的,当且仅当它能够对等于一个真子集.前面已经证明过有限集合不能和真子集对等;下面假设$X$是无限的,设$v$是$\omega$到$X$的一一映射(单射),若$x$属于$v$的值域,即$x=v(n)$,令$h(x)=v(n^+)$,若$x$不属于$v$的值域,则令$h(x)=x$,此时$h$是$X$到$X$的一一映射,并且$h$的值域是$X$的真子集($v(0) \notin \text{ran}{(h)}$).Dedekind使用这个结论来定义无限集.

\section{2013年06月11日}
Zorn引理

这一节只有一个主题:Zorn引理,证明过程很长!

很多存在性定理,经常会归结到一个偏序集以及一个最大元的存在性,这其中Zorn引理是最重要的一个.

\begin{lemma}[Zorn引理(Zorn's lemma)]
若$X$是一个偏序集(partially set),它的每一个链(chain)都存在上界(upper bound),那么$X$中包含一个最大元(maximal element).
\end{lemma}

链(chain)是一个全序集,这里所谓$X$中的chain是指$X$的子集,它自身构成一个全序集.

设$A$是$X$中的一个链,则根据题设要求,$X$中存在$A$的一个上界,这个上界不一定属于$A$,Zorn引理的结论是,存在一个元素$a \in X$,对于任意$x \in X$,如果$a \le x$,那么必有$a=x$.

直观的,既然$X$非空,那么存在$x_0 \in X$,如果它是最大元,那就可以停止了.否则,存在$x_1$严格大于$x_0$,若$x_1$是最大元,停止,否则继续,Zorn引理是说,这个过程最终将能得到有一个最大元.

这里面前面部分没有问题,问题在于最后一步,因为这里是可能是一个无限过程,它会停止吗?这也是这里的困难所在,因为完全可能出现,上面的过程永远得不到最大元,或者说得到是一个non-maximal elements序列,此时怎么办?其实此时这个序列本身是$X$中的链,从而有上界,于是从这个上界开始,继续上述过程,这个过程何时结束,会如何结束,在这里还是不清晰的,我们需要更明确的证明过程,书中的方法来自Zermelo.

首先把抽象的偏序具体化:使用集合的包含关系.把问题进行转化,把抽象的变具体,通常是我们解决问题的思路所在.考虑weak initial segment $\bar{s}(x)$.用$\mathscr{S}$来表示$\bar{s}$的值域.$\mathscr{S}$是$P(X)$的子集.可以用包含关系形成一个偏序.$\bar{s}$是一个一一映射.并且$\bar{s}(x) \subset \bar{s}(y)$的充分必要条件是$x \le y$.于是寻找$X$中的最大元,实际上成了寻找$\mathscr{S}$中的最大元.关于$X$中的链的假设对应于$\mathscr{S}$中的链.

用$\mathscr{X}$表示$X$中所有链的集合.这样的话$\mathscr{X}$也是$P(X)$的子集.$\mathscr{X}$中的每一个元素包含在某个$\bar{s}(x)$中,$\mathscr{X}$非空,我们以包含关系作偏序.若$\mathscr{C}$是$\mathscr{X}$中的一个链,那么$\bigcup_{A \in \mathscr{C}}{A}$属于$\mathscr{X}$,由于$\mathscr{X}$中的每一个集合包含在$\mathscr{S}$的某个集合中,从$\mathscr{S}$到$\mathscr{X}$这一个过程中没有引入新的最大元(maximal element).

$\mathscr{X}$的好处:首先它把条件中关于链的假设更加具体化了,对于$\mathscr{S}$中的每一个链$\mathscr{C}$有上界,$\mathscr{C}$中集合的并集是$\mathscr{C}$的上界,它属于$\mathscr{X}$;另一方面,$\mathscr{X}$包含它的每一个元素的所有子集,这使得我们可以通过每次给non-maximal集合添加一个元素来逐步放大.

至此,我们可以抛开$X$中的偏序,只需要考虑非空集合$X$的子集簇$\mathscr{X}$.根据上面的讨论$\mathscr{X}$满足两个条件:(1)$\mathscr{X}$的每一个元素的任一子集属于$\mathscr{X}$.它说明$\emptyset \in \mathscr{X}$;(2)$\mathscr{X}$中的每一个链中集合的并集属于$\mathscr{X}$.我们需要证明$\mathscr{X}$中存在最大元.

设$f$为$X$上的选择函数,即$f:P(X)-\{\emptyset\}\to X$,并且$f(A) \in A$,$\forall A \in \text{dom}f$,对于$A \in \mathscr{X}$,我们定义$\Hat{A}=\{x \in X: A \cup \{x\} \in \mathscr{X}\}$,定义映射$g:\mathscr{X}\to\mathscr{X}$,
\[
g(A)=\begin{cases}
A \cup \{f(\hat{A}-A)\},&\quad\hat{A}-A\neq\emptyset\\
g(A)=A,&\quad\hat{A}-A=\emptyset
\end{cases}
\]
若$\Hat{A}-A\neq\emptyset$,令$g(A)=A \cup \{f(\Hat{A}-A)\}$,若$\Hat{A}-A=\emptyset$,令$g(A)=A$.根据$\Hat{A}$的定义,$\Hat{A}-A=\emptyset$当且仅当$A$是一个最大元.也就是说我们需要证明存在$A \in \mathscr{X}$,使得$g(A)=A$.注意到$A \subset g(A)$,并且$g(A)$最多比$A$多一个元素.

为方便,引入一个临时定义:称$\mathscr{X}$的一个子集$\mathscr{J}$是tower,如果
\begin{enumerate}
\item[(i)]$\emptyset \in \mathscr{J}$;
\item[(ii)]若$A \in \mathscr{J}$,则$g(A) \in \mathscr{J}$;
\item[(iii)]若$\mathscr{C}$是$\mathscr{J}$中的链,则并集$\bigcup_{A \in \mathscr{C}}{A} \in \mathscr{J}$.
\end{enumerate}
Tower是存在的,$\mathscr{X}$本身就是一个,并且Tower的交集还是一个Tower,于是令$\mathscr{J}_0$表示所有tower的交集.则$\mathscr{J}_0$是最小的tower,我们来证明$\mathscr{J}_0$是一个链.

称$\mathscr{J}_0$中的集合$C$是comparable,如果它和$\mathscr{J}_0$中的任意元素都是comparable.于是$\forall A \in \mathscr{J}_0$,或者$A \subset C$,或者$C \subset A$.我们要证$\mathscr{J}_0$是一个链,意味着要证明$\mathscr{J}_0$中所有元素(集合)是comparable.comparable集合是存在的,$\emptyset$就是其中之一.下面的讨论暂时把注意力集中在一个任意的但是预先固定的comparable集合$C$.

设$A \in \mathscr{J}_0$,$A$是$C$的真子集,我们有$g(A) \subset C$.由于$C$是comparable,于是或者$g(A) \subset C$,或者$C$是$g(A)$的真子集,对于后一情形,$A$是$g(A)$的真子集的真子集,$g(A)-A$将会超过1个元素,不可能.

令$\mathscr{U}=\{A \in \mathscr{J}_0: A \subset C\text{或}g(C) \subset A\}$,$\mathscr{U}$中的所有元素和$g(C)$是comparable.因为若$A \in \mathscr{U}$,则由于$C \subset g(C)$,或者$A \subset g(C)$,或者$g(C) \subset A$.接下来证明$\mathscr{U}$是一个tower.
\begin{enumerate}
\item[(i)]$\emptyset \in \mathscr{U}$,因为$\emptyset \subset C$;

\item[(ii)]欲证$A \in \mathscr{U}$,必有$g(A) \in \mathscr{U}$,分三步:(1)$A$是$C$的真子集;则$g(A) \subset C$,故$g(A) \in \mathscr{U}$;(2)$A=C$,则$g(A)=g(C)$,$g(C) \subset g(A)$,$g(A) \in \mathscr{U}$.(3)$g(C) \subset A$,则$g(C) \subset g(A)$,故$g(A) \in \mathscr{U}$.

\item[(iii)]从$\mathscr{U}$的定义可知,$\mathscr{U}$中一个链的并集属于$\mathscr{U}$.

\end{enumerate}

于是$\mathscr{U}$是一个tower,它包含于$\mathscr{J}_0$中,而$\mathscr{J}_0$是最小的tower,于是$\mathscr{U}=\mathscr{J}_0$.

上面的结论说明对于每一个comparable集合$C$,$g(C)$是comparable集合:给定$C$,按上述方式构造$\mathscr{U}$,从$\mathscr{U}=\mathscr{J}_0$说明若$A \in \mathscr{J}_0$,则或者$A \subset C$(从而$A \subset g(C)$),或者$g(C) \subset A$.

我们已经知道$\emptyset$是comparable,$g$把comparable集合映射到comparable集合,comparable集合构成的链的并集还是comparable集合,这说明$\mathscr{J}_0$中的comparable集合构成一个tower,comparable集合穷尽了$\mathscr{J}_0$,于是$\mathscr{J}_0$是一个链.$\mathscr{J}_0$中任一集合是comparable.既然$\mathscr{J}_0$是一个链,$\mathscr{J}_0$中所有集合的并集$A$属于$\mathscr{J}_0$,于是$g(A) \subset A$,另一方面$A \subset g(A)$,故$A=g(A)$.Zorn引理证毕.

习题:Zorn引理等价于选择公理.考察下面的结论,证明它们也等价于选择公理:(i)任一偏序集有一个最大链(maximal chain),也就是这个链不可能是其他链的真子集;(ii)偏序集的任一链包含在某个最大链中;(iii)每一个偏序集,如果任一链都有下界,那么这个偏序集必有一最大元(这里书中恐怕有问题,似乎应该是最小元).

对于集合$X$,考虑映射$f$:$\text{dom}{f} \subset P(X)$,$\text{ran}{f} \subset X$,$f(A) \in A$,$\forall A \in \text{dom}{f}$.以映射的扩张作为偏序,使用Zorn引理寻找一个最大元,并且证明若$f$是最大元,必有$\text{dom}{f} = P(X)-\{\emptyset\}$.

\section{2013年06月16日}
良序(well ordering)

一个偏序集可能不存在smallest element,即使存在,也可能对于某些子集不存在smallest element.若它的每一个子集有一个smallest element,则称该集合为良序集(well ordered set).它的order称为良序(well ordering).

值得指出:每一个良序集必是全序的(totally order).因为$\{x,y\}$构成一个子集,那么无论是$x$为first element,还是$y$是first element,都有$x \le y$或者$y \le x$.

例子:

(i)每一个自然数$n$,$n$的所有predecessor组成的集合.即集合$n$是良序的,以大小关系为序.

(ii)全体自然数组成的集合$\omega$是良序的,同样以大小为序.

(iii)集合$\omega \times \omega$,序关系$(a,b) \le (x,y)$$\Leftrightarrow$$(2a+1)2^y \le (2x+1)2^b$,不是良序,$(a,b+1) \le (a,b)$$\forall a,b$,这意味着$\omega \times \omega$不存在least element.

考察集合$\omega \times \omega$的子集$E = \{(a,b) | (1,1) \le (a,b)\}$,则$E$有least element $(1,1)$,但是$E$仍然不是良序的,因为$E$的某些子集不存在least element,所有$(a,b) \neq (1,1)$组成的集合不存在least element.

(iv)$\omega \times \omega$对于字典序构成良序集.

对于良序集,我们有类似于数学归纳法的过程:考察良序集$X$的子集$S$,若任意$X$中的元素$x$满足:entire initial segment $s(x)$包含在$S$中,则$x$本身属于$S$.于是Principle of transifinite induction (超限归纳法)断言:$S=X$.

一般的归纳法与超限归纳法有两点明显的差异:(1)一般归纳法是从predecessor到当前元素,而超限归纳法要求当前元素的所有predecessor是一个集合.(2)超限归纳法没有要求归纳基础.对于(1),在良序集中,一个元素可能不存在直接前导(immediate predecessor).在自然数集$\omega$上,超限归纳法等价于数学归纳法,在一般的良序集上,两者不等价.

例:令$X = \omega^+$,即$X=\omega \cup \{\omega\}$,序关系如下:$\omega$中元素的序保持不变,$\forall n \in \omega$,$n < \omega$,此时$X$是一个良序集,问:是否存在一个子集$S \subset X$,使得$0 \in S$,当$n \in S$时,有$n+1 \in S$,答:有,即$S=\omega$.

第二个差异更多的是语言方面的,而不是概念上(本质的)的,或者说其实超限归纳法已经包含这部分内容.设$x_0$为$X$的smallest element,则$s(x_0)=\emptyset$,于是$s(x_0) \subset S$,根据超限归纳法的假设,$x_0 \in S$.

需要证明这个超限归纳原理,证明不是很难:若$X-S$非空,则存在smallest element $x \in X-S$,这意味着它的initial segment $s(x)$中的每一个元素属于$S$,根据假设,$x \in S$,导致矛盾,故$X-S$只能是空集.

\begin{definition}
称良序集$A$为良序集$B$的continuation,如果满足:(1)$B$是$A$的子集,即$B$是$A$的一个initial segment;(2)$B$中元素的序和它们在$A$中元素的序一致.
\end{definition}

$X$为良序集,$a,b \in X$,$b<a$,则$s(a)$是$s(b)$的continuation,自然,$X$是$s(a)$和$s(b)$的continuation.

设$\mathscr{C}$为某个良序集的initialsegments组成的集合,则$\mathscr{C}$相对于continuation是一个链(chain).

$\mathscr{C}$中元素是良序集,任意两个不同的元素来说,其中一个是另一个的continuation.若一簇良序集构成的集合$\mathscr{C}$对于continuation构成一个chain,$U$是$\mathscr{C}$中集合的并集,那么存在一个唯一的$U$中的良序,使得$U$是$\mathscr{C}$中每一个不为$U$集合的continuation.或者说良序集的chain的并仍然是一个良序的,必须注意,这里的序必须是相对于continuation而言的,若order是inclusion,结论不成立.

证明如下:$a,b \in U$,$\exists A, B \in \mathscr{C}$,使得$a \in A$,$b \in B$,于是,或者$A=B$,或者$A$和$B$中的一个是另一个的continuation.无论何种情形,$a,b$属于$\mathscr{C}$中某一个集合,定义$U$中order如下:对于每一对$\{a,b\}$,它的序取$G$中某个集合(它包含$a$和$b$)中的顺序,$\mathscr{C}$是一个chain,这个order是确定的.也就是说刚才定义的序确实是一个order,同时是一个well ordering.

$U$的非空子集必有非空交集(因为这里面的集合非常特殊,一般的交集没有这种性质),必有first element,$\mathscr{C}$为一个continuation,可以得出:集合的first element同时是$U$的first element.

习题:偏序集$X$的子集$A$称为在$X$中是共尾(cofinal)的.若$X$中每一个元素$x$,存在$A$中元素$a$,使得$x \le a$.证明每一个全序集含有一个cofinal well ordered subset.

良序集之所以重要,是因为下面的结论:

\begin{theorem}[Well ordering theorem]
每一个集合可以良序化.
\end{theorem}

更好的说法是:对于每一个集合$X$,存在以$X$为domain的良序,必须注意,这里并没有说此良序和给定集合$X$上的原来的某些结构有关系.因此如果说偏序集或者全序集中的序不是一个良序时,并不意味着无法将它良序化.

证明这个结论使用Zorn引理.给定集合$X$,考虑$X$中所有良序子集构成的集簇$W$,$W$中元素是$X$中的子集$A$和$A$中的良序,我们通过continuation定义$W$中的一个偏序.

$W$非空,因为$\emptyset \in W$.若$X \neq \emptyset$,$\forall x \in X$,$\{(x,x)\}$.

若$\mathscr{C}$是$W$中的一个chain,则$\mathscr{C}$中集合的并集$U$拥有唯一的良序,使得$U$大于或等于$\mathscr{C}$中的每一个集合.这意味着Zorn引理的假设成立,于是存在maximal well ordered set,设它为$M \in W$,集合$M$必然等于整个集合$X$.

若存在$X$中元素$x \notin M$,那么$M$可以继续扩大,把$x$放在$M$中所有元素之后即可.

习题:一个全序集是良序的,当且仅当每一个元素的严格前驱构成的集合是well ordered.这个条件能否用于偏序集?Well ordering theorem包含选择公理.$R$为集合$X$的一个偏序,存在$X$中的一个全序$S$,使得$R \subset S$,也就是任何一个偏序可以在不扩大定义域的情形下扩展为全序.



\section{2013年08月24日}
超限递归(transfinite recursion)

\[
2 < 2\sqrt{2} - 6\sqrt{6}+4
\]


分情况讨论:

如果$x > 0$,那么必然有$-b < 0 < \frac{1}{x}$,只需要考虑后半部分,此时有
\[
\frac{1}{a} < x,
\]
两者结合得到$x > \frac{1}{a}$;

如果$x < 0$,那么必然有$\frac{1}{x} < a$,只需要考虑前半部分,此时有
\[
(-b)x > 1 \\
x < -\frac{1}{b}
\]
这里面一定要注意两边乘以$x$的时候,$x<0$是需要改变不等号的方向的,后面一步同时除以$-b < 0$,不等号再次变号.两者结合得到
\[
x < -\frac{1}{b};
\]
最后的答案结合两个得到
\[
x < -\frac{1}{b} \text{或者} x > \frac{1}{a}
\]

首先是完全的排列一共有$6!=720$个数字.不过这个数字没有什么用途.

下面使用条件(大于345012)逐个计算:

首位数字只能是345中的一个:

345021
345102
345120
345201
345210

35开头的话,后面就是0124的任意排列一共有4!=24个.

其余不存在了如果是4或者5开头,后面是5个数字的任意排列,一共2*5!=240个,所以总数为
5 + 24 + 240 =149






\end{document}
