\part{数学分析}
《数学分析》的作者是克莱鲍尔(G.Klambauer)。参考:\cite{MathematicalAnalysisKlambauer1981}。

\chapter{实数系}\label{ch01001}
在这一章,我们考察实数系的一些性质,它们对于真正地理解微积分的基本概念(如收敛性,连续性,微分和积分)是必不可少的。不过,对于初学者来说,本章前两节中命题的证明细节可以略去不看。

\section{整数,有理数与无理数}\label{sec0100101}
全体正整数(自然数)之集$\mathbb{N}$有两个重要性质。我们将叙述这两个性质并建立其等价性。

\emph{良序原理}\ \ 任何非空正整数集有最小元素。

\emph{数学归纳原理}\ \ 若$P$是具有下列性质的正整数集.
\begin{enumerate}
\item[(i)] $P$含有数$1$;
\item[(ii)] 只要$P$含有正整数$n$, 它也含有正整数$n+1$;
\end{enumerate}
则$P$含有所有正整数,即$P = N$.

\begin{proposition}{}{}
数$1$是最小正整数。
\end{proposition}

证法1 (根据数学归纳原理)设$S$是所有$\ge 1$的正整数集。显然, $1$属于$S$。若整数$n$属于$S$,则$n \ge 1$.因此$n+1 > n \ge 1$, $n+1$属于$S$。由数学归纳原理,$S=\mathbb{N}$, 故所有正整数大于或等于$1$.

证法2 (根据良序原理)从良序原理知道存在最小正整数,比如说是$s$,设$s < 1$. 以$s$乘不等式$0 < s < 1$得$0 < s^2 < s$, 这说明$s$不是最小正整数。由于设$s < 1$导致矛盾,故这个假设不正确。因而$1$是最小正整数。

\begin{corollary}{}{}
若$n$是整数,则$n$与$n+1$之间不存在整数。
\end{corollary}
\begin{proof}
设存在整数$k$满足$n < k < n+1$,则$0 < k-n< 1$,与$1$是最小正整数矛盾。
\end{proof}

\textbf{注}\ \ 断言“存在最大正整数$n$”是错误的。因为由此将有$n^2=n$, 而这就是说$n$应该等于$1$.

\begin{proposition}{}{}
数学归纳原理与良序原理等价,即,只要以整数的通常的算术性质为前提,就能从其中之一推出另一个。
\end{proposition}

\section{Dedekind分割}\label{sec0100102}


\section{不等式}\label{sec0100103}

\section{实数列}\label{sec0100104}

\section{实数级数}\label{sec0100105}

\section{有规则的小数}\label{sec0100106}


\chapter{连续性}\label{ch01002}


\chapter{微分与积分}\label{ch01003}

\chapter{一致收敛性}\label{ch01004}

\chapter{度量空间}\label{ch01005}












