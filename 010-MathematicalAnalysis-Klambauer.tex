\part{数学分析}
《数学分析》的作者是克莱鲍尔(G.Klambauer)。参考:\cite{MathematicalAnalysisKlambauer1981}。

\chapter{实数系}\label{ch01001}
在这一章,我们考察实数系的一些性质,它们对于真正地理解微积分的基本概念(如收敛性,连续性,微分和积分)是必不可少的。不过,对于初学者来说,本章前两节中命题的证明细节可以略去不看。

\section{整数,有理数与无理数}\label{sec0100101}
全体正整数(自然数)之集$\mathbb{N}$有两个重要性质。我们将叙述这两个性质并建立其等价性。

\emph{良序原理}\ \ 任何非空正整数集有最小元素。

\emph{数学归纳原理}\ \ 若$P$是具有下列性质的正整数集.
\begin{enumerate}
\item[(i)] $P$含有数$1$;
\item[(ii)] 只要$P$含有正整数$n$, 它也含有正整数$n+1$;
\end{enumerate}
则$P$含有所有正整数,即$P = N$.

\begin{proposition}{}{}
数$1$是最小正整数。
\end{proposition}

证法1 (根据数学归纳原理)设$S$是所有$\ge 1$的正整数集。显然, $1$属于$S$。若整数$n$属于$S$,则$n \ge 1$.因此$n+1 > n \ge 1$, $n+1$属于$S$。由数学归纳原理,$S=\mathbb{N}$, 故所有正整数大于或等于$1$.

证法2 (根据良序原理)从良序原理知道存在最小正整数,比如说是$s$,设$s < 1$. 以$s$乘不等式$0 < s < 1$得$0 < s^2 < s$, 这说明$s$不是最小正整数。由于设$s < 1$导致矛盾,故这个假设不正确。因而$1$是最小正整数。

\begin{corollary}{}{}
若$n$是整数,则$n$与$n+1$之间不存在整数。
\end{corollary}
\begin{proof}
设存在整数$k$满足$n < k < n+1$,则$0 < k-n< 1$,与$1$是最小正整数矛盾。
\end{proof}

\textbf{注}\ \ 断言“存在最大正整数$n$”是错误的。因为由此将有$n^2=n$, 而这就是说$n$应该等于$1$.

\begin{proposition}{}{}
数学归纳原理与良序原理等价,即,只要以整数的通常的算术性质为前提,就能从其中之一推出另一个。
\end{proposition}
\begin{proof}
(第一部分)设良序原理成立,$S$是具有性质
\begin{enumerate}
\item[(i)] $1$属于$S$;
\item[(ii)] 若$n$属于$S$,则$n+1$也属于$S$
\end{enumerate}
的正整数集. 我们应该证明$S$是所有正整数之集$\mathbb{N}$.

设$T$是所有不属于$S$的正整数之集. 若$T$非空, 则由良序原理知有最小元素,设为$t$. 因为$1$属于$S$,且$1$是最小正整数(见用良序原理建立的命题1),故$t > 1$. 这样$t-1$必定是属于$S$的正整数, 因为$t-1<t$. 由于$t = (t-1) + 1$, $S$的第二个性质保证$t$也属于$S$.但$S$与$T$不相交,故矛盾。由于我们假设$T$非空导致矛盾,故$T$是空集,即$S = \mathbb{N}$.

(第二部分)假定数学归纳原理成立,其次,设存在非空正整数集$S$没有最小元素。因为$1$是最小正整数(见用数学归纳原理建立的命题1),故$1$不属于$S$且小于$S$的所有元素。

设$T$是比$S$的所有元素小的全体正整数之集. 我们已经知道$1$属于$T$. 设$n$属于$T$. 若$n+1$属于$S$, 则由于$n$与$n+1$之间不存在整数(由命题1推论), $n+1$将是$S$的最小元素;但这与我们对$S$的假设矛盾。因而,$n$属于$T$时$n+1$必属于$T$.由数学归纳原理,$T$含有全体正整数,从而$S$是空集。但这又与原来假设$S$非空矛盾。故若$S$是非空正整数集,则$S$有最小元素。证毕。
\end{proof}

数学归纳原理的一种修改过的、但等价的形式说:

满足下列两个性质的关于正整数的陈述对所有正整数成立:
\begin{enumerate}
\item[(i)] 此陈述对整数$1$成立;
\item[(ii)] 若这个陈述对正整数$n$成立,则对正整数$n+1$也必成立.
\end{enumerate}

\begin{definition}{}{}
设$a, b$是整数,$b \neq 0$. 若存在第三个整数$c$使$bc = a$,则称$a$能被$b$整除。我们也说$b$整除$a$或$a$是$b$的倍数,记为$b \mid a$. 若$b \neq 0$且$a$不能被$b$整除,则写成$b \nmid a$.
\end{definition}

容易证明下列性质:
\begin{enumerate}
\item[(i)] $b \mid a$与$a > 0$, $b > 0$蕴含$1 \le b \le a$;
\item[(ii)] $c \mid b$与$b \mid a$蕴含$c \mid a$;
\item[(iii)] 对所有整数$m,n$, $c \mid a$与$c \mid b$蕴含$c \mid ma + nb$.
\end{enumerate}

\begin{definition}{}{}
设$p > 1$是整数,若除$1$以外$p$不能被比它小的正整数整除,则称$p$为素数。不是素数的大于$1$的正整数叫合数。
\end{definition}

\begin{proposition}{}{}
大于1的整数是素数或素数之积。其次,如果不计因数的次序,则分解为素因数之积的方法唯一。
\end{proposition}

\begin{proof}
先证明合数可分解为素数之积。

设这个结论不正确,则存在不能写成素数之积的合数。设$n$是这样的数里最小的一个(由良序原理,这样的$n$是存在的)。因为$n$是合数,故可写作
\[
n = ab, 1 < a < n, 1 < b < n.
\]
当$a$比$n$小时,$a$是素数或素数之积;对$b$也一样。但这样一来,$n=ab$是素数之积,矛盾。因此合数都可以分解为素因数之积。

现在证明分解方法唯一。设存在分解方法不唯一的合数。这样的数里最小的一个(这里用到良序原理)设为
\[
Q = p_1p_2\cdots p_k = q_1q_2\cdots q_j,
\]
\end{proof}

\section{Dedekind分割}\label{sec0100102}


\section{不等式}\label{sec0100103}

\section{实数列}\label{sec0100104}

\section{实数级数}\label{sec0100105}

\section{有规则的小数}\label{sec0100106}


\chapter{连续性}\label{ch01002}


\chapter{微分与积分}\label{ch01003}

\chapter{一致收敛性}\label{ch01004}

\chapter{度量空间}\label{ch01005}












