\part{数学分析}
《数学分析》的作者是克莱鲍尔(G.Klambauer)。参考:\cite{MathematicalAnalysisKlambauer1981}。

\chapter{实数系}\label{ch01001}
矩阵是本书的中心角色,它是理论的重要组成部分,并且许多具体例子都基于矩阵。因而,发展处理矩阵的方法是非常重要的。因为矩阵遍及数学的各个分支,所以这里用到的技巧在其他地方也一定会用到。

\section{证书,有理数与无理数}\label{sec0100101}


\section{Dedekind分割}\label{sec0100102}


\section{不等式}\label{sec0100103}

\section{实数列}\label{sec0100104}

\section{实数级数}\label{sec0100105}

\section{有规则的小数}\label{sec0100106}


\chapter{连续性}\label{ch01002}


\chapter{微分与积分}\label{ch01003}

\chapter{一致收敛性}\label{ch01004}

\chapter{度量空间}\label{ch01005}












