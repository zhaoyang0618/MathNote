\part{流形上的微积分}
《流形上的微积分》的作者是M.Spivak. 参考:\cite{CalculusOnManifoldsSpivak1980}. 

\chapter{欧几里得空间上的函数}\label{chapter00901}
\section{范数与内积}\label{section0090101}
欧几里得(Euclid)$n$维空间(也简称欧氏空间)$\real^n$定义为一切实数$x^i$的$n$数组$(x_1,\cdots, x_n)$(一个“1数组”就是一个数,而$\real^1=\mathbb{R}$则是一切实数的集)的集合. $\real^n$的元通常称为$\real^n$的点,而$\real^1,\real^2,\real^3$通常分别称为直线、平面和空间。如$x$表示$\real^n$的一元素,则$x$是一个$n$数组,其中第$i$个记作$x^i$; 于是我们可以写成
\[
x = (x_1, \cdots, x_n).
\]

$\real^n$中的点也常常称为$\real^n$中的向量,因为,按照$x+y=(x_1+y_1,\cdots, x_n+y_n)$以及$ax=(ax_1,\cdots, ax_n)$作为运算,$\real^n$是一个向量空间(在实数域上,维数为$n$)。在这向量空间中,向量$x$的长度的概念,通常称为$x$的范数$|x|$, 并定义为$|x| = \sqrt{(x_1)^2 + \cdots + (x_n)^2}$。如$n=1$,则$|x|$就是$x$的通常的绝对值。范数和$\real^n$的向量空间结构间的下一关系极为重要。
\begin{theorem}{}{thm009010101}
如$x,y \in \real^n$且$a \in \real$,则
\begin{enumerate}
\item[(1)] $|x| \ge 0$,当且仅当$x=0$时,$|x|=0$.
\item[(2)] $\left|\sum\limits_{i=1}^{n}{x_iy_i}\right| \le |x| \cdot |y|$,当且仅当$x$与$y$线性相关时等式成立。
\item[(3)] $|x+y| \le |x| + |y|$.
\item[(4)] $|ax| = |a| \cdot |x|$.
\end{enumerate}
\end{theorem}

\begin{proof}
\begin{enumerate}
\item[(1)] 根据定义,显然有$|x| \ge 0$,而且$|x|=0$时,必有各个$x_i=0$,也就是$x=0$,反过来,如果$x=0$,显然有$|x|=0$。而$x \neq 0$时,至少存在一个$x_i \neq 0$,于是$|x| \ge |x_i|>0$.

\item[(2)] 如$x$与$y$线性相关,等式明显成立。如不是这样,则对一切$\lambda \in \real$,$\lambda{}y - x \neq 0$,因此
\[
\begin{aligned}
0 &< |\lambda{}y - x|^2 = \sum_{i=1}^{n}{(\lambda{}y_i - x_i)^2}\\
&= \lambda^2\sum_{i=1}^{n}{(y_i)^2} - 2\lambda\sum_{i=1}^{n}{x_iy_i} + \sum_{i=1}^{n}{(x_i)^2}.
\end{aligned}
\]
所以右方是$\lambda$的没有实根的二次式,其判别式必须为负。于是
\[
4\left(\sum_{i=1}^{n}{x_iy_i}\right)^2 - 4\sum_{i=1}^{n}{(x_i)^2}\sum_{i=1}^{n}{(y_i)^2}<0.
\]

\item[(3)]
\begin{equation*}
\begin{aligned}
&|x + y|^2 = \sum_{i=1}^{n}{(x_i+y_i)^2} \\
& = \sum_{i=1}^{n}{(x_i)^2} + \sum_{i=1}^{n}{(y_i)^2} + 2\sum_{i=1}^{n}{x_iy_i} \\
& \le |x|^2+|y|^2+2|x|\cdot|y| \\
&=(|x|+|y|)^2.
\end{aligned}
\end{equation*}

\item[(4)] $|ax| = \sqrt{\sum\limits_{i=1}^{n}{(ax_i)^2}} = \sqrt{a^2\sum\limits_{i=1}^{n}{(x_i)^2}} = |a|\cdot|x|$.
\end{enumerate}
\end{proof}

在(2)中出现的量$\sum\limits_{i=1}^{n}{x_iy_i}$称为$x$与$y$的内积并记作$\left<x,y\right>$。内积的一些最重要的性质如下。
\begin{theorem}{}{thm009010102}
如$x,x_1,x_2$与$y,y_1,y_2$是$\real^n$中的向量,且$a \in \real$, 则
\begin{enumerate}
\item[(1)] $\left<x,y\right>=\left<y,x\right>$ (对称性).
\item[(2)] (双线性)
\begin{gather*}
\left<ax, y\right> = \left<x, ay\right> = a\left<x, y\right>\\
\left<x_1 + x_2, y\right> = \left<x_1, y\right> + \left<x_2, y\right> \\
\left<x, y_1+y_2\right> = \left<x, y_1\right> + \left<x, y_2\right>
\end{gather*}
\item[(3)] $\left<x, x\right> \ge 0$,且$\left<x, x\right>=0$当且仅当$x=0$(正定性)。
\item[(4)] $|x| = \sqrt{\left<x, x\right>}$.
\item[(5)] 极化等式
\[
\left<x, x\right> = \frac{|x+y|^2 - |x-y|^2}{4}.
\]
\end{enumerate}
\end{theorem}

\begin{proof}
\begin{enumerate}
\item[(1)]$\left<x,y\right> = \sum\limits_{i=1}^{n}{x^iy^i} = \sum\limits_{i=1}^{n}{y^ix^i} = \left<y, x\right>$.
\item[(2)]由(1)只须证明
\begin{gather*}
\left<ax,y\right> = a\left<x,y\right>,\\
\left<x_1+x_2, y\right> = \left<x_1, y\right> + \left<x_2, y\right>.
\end{gather*}
这些可由下列等式得出:
\[
\begin{aligned}
\left<ax,y\right> &= \sum_{i=1}^{n}{(ax^i)y^i} = a\sum_{i=1}^{n}{x^iy^i} = a\left<x,y\right>,\\
\left<x_1+x_2, y\right> &= \sum_{i=1}^{n}{(x_1^i+x_2^i)y^i} = \sum_{i=1}^{n}{x_1^iy^i} + \sum_{i=1}^{n}{x_2^iy^i}\\
&= \left<x_1, y\right> + \left<x_2, y\right>.
\end{aligned}
\]
\item[(3)]和(4)留给读者.
\item[(5)]
\[
\begin{aligned}
&\quad\frac{|x+y|^2 - |x-y|^2}{4} \\
& = \frac{1}{4}[\left<x+y, x+y\right> - \left<x-y, x-y\right>]\\
& = \frac{1}{4}[\left<x, x\right> + 2\left<x, y\right> + \left<y, y\right> - \\
& \qquad (\left<x, x\right> - 2\left<x, y\right> + \left<y, y\right>)] = \left<x, y\right>.
\end{aligned}
\]
\end{enumerate}
\end{proof}

我们对记号作一些重要注解以结束本节。向量$(0, \cdots, 0)$通常简记为$0$. $\real^m$的通常基底是$e_1,\cdots, e_n$, 其中$e_i=(0,\cdots,1,\cdots, 0)$, 在第$i$个位置上是$1$.如$T: \real^n \to \real^m$是一个线性变换,$T$关于$\real^n$与$\real^m$的通常基底的矩阵是$m \times n$矩阵$A = (a_{ij})$,其中$T(e_i) = \sum\limits_{j=1}^{n}{a_{ji}e_j}$---$T(e_i)$的系数出现在矩阵的第$i$列。如$S:\real^m \to \real^p$有$p \times m$矩阵$B$, 则$S \circ T$有$p \times n$矩阵$BA$[这里$S \circ T(x) = S(T(x))$;绝大多数线性代数书籍把$S \circ T$简记为$ST$]. 为要找出$T(x)$,我们来计算$m \times 1$矩阵.
\[
\begin{pmatrix}
y^1\\
\vdots\\
y^m
\end{pmatrix}
=\begin{pmatrix}
a_{11}, &\cdots ,&a_{1n}\\
\vdots && \vdots \\
a_{m1}, & \cdots,&a_{mn}
\end{pmatrix}
\cdot\begin{pmatrix}
x^1\\
\vdots\\
x^n
\end{pmatrix};
\]
则$T(x)=(y^1,\cdots,y^m)$.下一习惯记法大大简化许多公式:如$x \in \real^n$与$y \in \real^m$,则$(x,y)$表示
\[
(x^1,\cdots,x^n,y^1,\cdots,y^m) \in \real^{n+m}.
\]

\begin{problemset}
\item 求证$|x| \le \sum\limits_{i=1}^{n}{|x^i|}$.

注意到不等式两边都大于等于0,两边平方,展开,就会发现等式成立。
\[
(\sum_{i=1}^{n}{|x^i|})^2 = \sum_{i=1}^{n}{|x^i|^2} + 2\sum_{i \neq j}{|x^i||x^j|} = |x| + 2\sum_{i \neq j}{|x^i||x^j|}
\]

\item 定理\ref{thm:thm009010101}(3)中的等式何时成立?提示:重新检查证明;答案不是“当$x$与$y$线性相关”。

在证明过程中,等式成立首先要求$x$与$y$线性相关,然后要求$\sum_{i=1}^{n}{x^iy^i} \ge 0$,把线性相关代入,可以得出等号成立当且仅当:$ax+by=0$且$ab \le 0$,也就是$x$和$y$同方向。

\item 求证$|x-y| \le |x| + |y|$,何时等式成立?

和上一道题目类似,这一次要求$x$与$y$共线,但是需要相反方向才能取等号。

\item 求证$\left||x|-|y|\right| \le |x-y|$.
\begin{gather*}
|x| = |x-y+y| \le |x-y|+|y|\\
|x| - |y| \le |x-y|
\end{gather*}
由对称性可以得出另一个不等式:$|y|-|x| \le |x-y|$,结合起来就是所要证的不等式。

\item 量$|y-x|$称为$x$与$y$间的距离,求证并在几何上解释“三角形不等式”:
\[
|z-x| \le |z-y| + |y-x|.
\]

只要注意到,对于三个点$x$, $y$和$z$构成的三角形来说,不等式中的恰好就是三条边长,几何上三角形两边之和大于第三边。至于代数证明很简单.
\[
|z-x| = |z-y + y-x| \le |z-y| + |y-x|.
\]

\item 设$f$与$g$在$[a,b]$上平方可积。
\begin{enumerate}
\item[(a)]求证:$\left|\int_{a}^{b}{fg}\right| \le \left(\int_{a}^{b}{f^2}\right)^{1/2}\left(\int_{a}^{b}{g^2}\right)^{1/2}$, 提示:分别考虑下面二情况:对某一$\lambda \in\real$, $0 = \int_{a}^{b}{(f - \lambda{}g)^2}$; 对一切$\lambda \in \real$, $0 < \int_{a}^{b}{(f - \lambda{}g)^2}$.

\item[(b)]如等式成立,$f = \lambda{}g$必定对某个$\lambda \in \real$成立吗?如$f$与$g$连续又怎样?

\item[(c)]证明定理\ref{thm:thm009010102}是(a)的一个特殊情形。
\end{enumerate}

(a)$f$与$g$平方可积,说明$(f-\lambda{}g)$也是平方可积的,假设存在一$\lambda \in \real$使得$0 = \int_{a}^{b}{(f - \lambda{}g)^2}$, 那么说明除了一个零测集$\Lambda$之外,有$f - \lambda{}g = 0$. 也就是几乎处处有$f = \lambda{}g$,从而几乎处处$fg = \lambda{}g^2$,由此不等式左边$\left|\int_{a}^{b}{fg}\right| = \left|\int_{a}^{b}{\lambda{}g^2}\right| = |\lambda|\int_{a}^{b}{g^2}$, 不等式右边:$\left(\int_{a}^{b}{f^2}\right)^{1/2}\left(\int_{a}^{b}{g^2}\right)^{1/2} = |\lambda|\int_{a}^{b}{g^2}$,不等式成立等号。如果对于所有$\lambda \in \real$, 
\[
\begin{aligned}
0 &< \int_{a}^{b}{(f - \lambda{}g)^2} = \int_{a}^{b}{(f^2 -2\lambda{}fg + \lambda^2g^2)}\\
&=\int_{a}^{b}{f^2} - 2\lambda\int_{a}^{b}{fg} + \lambda^2\int_{a}^{b}{g^2}.
\end{aligned}
\]
如果$\dsint_{a}^{b}{g^2} = 0$,那么讨论和前面类似,只是此时是$g$几乎处处等于0,从而$fg$几乎处处等于0,于是$\dsint_{a}^{b}{fg} = 0$,所以只需考虑$\dsint_{a}^{b}{g^2} > 0$的情形,此时二次型的系数大于零,只有判别式小于0
\[
\Delta = 4\left(\int_{a}^{b}{fg}\right)^2 - 4\int_{a}^{b}{f^2}\int_{a}^{b}{g^2} < 0,
\]
获证。

(b)等式成立,说明存在$\lambda \in \real$,在除了某个零测集之外有$f = \lambda{}g$,或者$f$和$g$至少有一个几乎处处等于0,如果$f$和$g$连续,那么在$g\neq 0$的情形下,必然存在一$\lambda \in \real$使得$f = \lambda{}g$. 因为对于非负连续函数,如果存在某点不等于0,那么由于连续函数的保号性,必然在区间上的积分大于0.

(c)考虑如下定义在$[0, n]$区间上的阶梯函数$f(x)$和$g(x)$:
\[
f(x) = \left\{
\begin{aligned}
&x_1, \quad 0 \le x \le 1,\\
&x_2, \quad 1 < x \le 2,\\
&\cdots, \\
&x_n, \quad (n-1) < x \le n.
\end{aligned}
\right.
\]
$g(x)$类似,只是在相同区间上,$x_i$换成$y_i$,那么显然$f(x)$和$g(x)$都是平方可积的,并且
\[
\begin{aligned}
\int_{0}^{n}{fg} &= \sum_{i=1}^{n}{x_iy_i},\\
\int_{0}^{n}{f^2} &= \sum_{i=1}^{n}{(x_i)^2}\\
\int_{0}^{n}{g^2} &= \sum_{i=1}^{n}{(y_i)^2}.
\end{aligned}
\]

\item 一线性变换$T:\real^n \to \real^n$,如果$|T(x)| = |x|$,则称为保范数的。如果$\left<Tx, Ty\right> = \left<x, y\right>$,则称为保内积的。
\begin{enumerate}
\item 求证$T$是保范数的当且仅当$T$是保内积的。

\item 求证这种线性变换$T$是1-1的,而且$T^{-1}$也是同一种变换。

\end{enumerate}

使用定理\ref{thm:thm009010102}中的(5)极化等式和线性变换的性质即可。从保范数和保内积的定义就可以得到.
\[
\begin{aligned}
|T(x)| = |x| &\Rightarrow \left<Tx,Ty\right> = \frac{1}{4}[|Tx + Ty|^2 - |Tx-Ty|^2] \\
&\quad = \frac{1}{4}[|T(x+y)|^2 - |T(x-y)|^2] \\
&\quad = \frac{1}{4}[|x+y|^2 - |x-y|^2] = \left<x,y\right>
\end{aligned}
\]
反过来,保内积的话,只需要令$x = y$,立即可得$|Tx| = |x|$.

至于要证明$T$是一一的,也就是要证明它既是单射又是满射。
\begin{enumerate}
\item[(1)]$T$是单射,也就是如果$T(x_1) = T(x_2)$,则$x_1 = x_2$.
\[
0 = |T(x_1) - T(x_2)| = |T(x_1 - x_2)| = |x_1 - x_2|.
\]
\item[(2)]$T$是满射。证明$T$的矩阵$A$可逆即可,也就是证明$A$的任意两列正交。这一点只需要注意到
\[
\left<T(e_i),T(e_j)\right> = \left<e_i, e_j\right> = \delta_{ij},
\]
而$T(e_i)$的系数就是$A$的第i列,从而$A$是正交矩阵,可逆。至于说$T^{-1}$也是保范的,可以从下列等式得到
\[
|T(T^{-1}(x))| = |T^{-1}(x)| = |x|.
\]
\end{enumerate}


\item 如$x, y \in \real^n$不为零,$x$与$y$间的夹角记作$\angle{(x, y)}$定义为$\arccos(\left<x, y\right>/|x|\cdot|y|)$.由定理\ref{thm:thm009010101}的(2),这是有意义的。线性变换$T$称为是保角的,如$T$是1-1的,且对$, y \neq 0$,我们有$\angle(Tx, Ty) = \angle(x, y)$.
\begin{enumerate}
\item[(a)] 求证:如$T$是保范数的,则$T$是保角的。

\item[(b)] 如$\real^n$有一基底$x_1, \cdots, x_n$, 又有数$\lambda_1, \cdots, \lambda_n$使得$Tx_i = \lambda_ix_i$,求证$T$是保角的,当且仅当所有$|\lambda_i|$皆相等。

\item[(c)] 所有保角的$T: \real^n \to \real^n$是些什么?
\end{enumerate}

(a)首先从上一道题目可知,保范数意味着$T$是一一的,且保内积,于是$\left<Tx,Ty\right>=\left<x,y\right>$.于是$x \neq y$时
\[
\angle(Tx, Ty) = \arccos{(\frac{\left<Tx, Ty\right>}{|Tx||Ty|})} = \arccos{\frac{\left<x,y\right>}{|x||y|}} = \angle(x, y).
\]

(b)我怎么感觉这道题目的结论有点问题,应该是所有$\lambda_i$都相等才是合理的。下面的证明方法来自微信网友枫树(gusongduping)

充分性:若$\lambda_1 = \lambda_2 = \cdots = \lambda_n \neq 0$, 则$\forall x \in \real^n$有$Tx = \lambda_1x$, 于是$\forall x, y \real^n$有
\[
\angle{Tx, Ty} = \arccos{\frac{(Tx, Ty)}{|Tx||Ty|}} = \arccos{\frac{\lambda_1^2(x, y)}{|\lambda_1|^2|x||y|}} = \angle(x,y).
\]
因而$T$是保角的。

必要性:若$T$是保角的,从而是一一的,于是$\lambda_i \neq 0$, 对于给定的$\lambda_k$与$\lambda_s$, 我们考虑$(x_k, x_s)$,

若$(x_k,x_s) = 0$,也就是两个向量正交,则对于$\alpha = x_k + x_s$, $\beta = -(x_s,x_s)x_k + (x_k,x_k)x_s$, 有$(\alpha, \beta) = 0$,于是
\[
(T\alpha, T\beta) = (\alpha, \beta) = 0 \Leftrightarrow \lambda_k^2 = \lambda_s^2.
\]
又由于$\angle(Tx_k, Tx_s) = \angle(x_k, x_s)$,可得$\lambda_k\lambda_s > 0$,从而$\lambda_k = \lambda_s$.

若$(x_k, x_s) \neq 0$,则对于$\alpha = x_k$, $\beta = -(x_k, x_s)x_k + (x_k, x_k)x_s$,有$(\alpha, \beta) = 0$,于是
\[
(T\alpha, T\beta) = 0 \Leftrightarrow \lambda_k = \lambda_s.
\]

获证。

上面的证明过程,主要是$\alpha$和$\beta$的构造,我原来有一些思路,不过傻了,使用了一般情形进行讨论,没有想到构造正交向量。上面$x_k$和$x_s$正交的时候,要想得到$\lambda_k=\lambda_s$,似乎需要依赖直观,也就是$\lambda_k$和$\lambda_s$同号这一步有些麻烦,好像不成立,只能推到$\lambda_k^2 = \lambda_s^2$. 可以试试相似三角形的概念。正交的时候,最多得到绝对值相等。

(c)由(b),首先必然存在一组基底$x_1,\cdots, x_n$使得$Tx_i = \lambda{x_i}$,从而$\forall x \in \real^n$有$Tx = \lambda{x}$,那么对于标准基底来说,由于$x_1,\cdots, x_n$和标准基底之间存在一一变换,也就是存在克你矩阵$A$使得$x_i = Ae_i$,于是,对于标准基底,就有$Tx = \lambda{Ax}$, 从几何上,应该是平移,旋转,伸缩以及它们的组合的结果。这个做法似乎也不严格,没有证明除了这个变换,就不存在其他的保角变换。

\item 如果$0 \le \theta < \pi$, 设$T: \real^2 \to \real^2$有矩阵
\[
\begin{pmatrix}
\cos{\theta} & \sin{\theta}\\
-\sin{\theta} & \cos{\theta}
\end{pmatrix},
\]
求证$T$是保角的,且若$x \neq 0$,则$\angle(x, Tx) = \theta$.

比较笨,直接计算。点$(x_1,x_2) \in \real^2$,有
\[
\begin{aligned}
|T(x_1,x_2)|^2 &= \left|\begin{pmatrix}\cos{\theta} & \sin{\theta} \\-\sin{\theta} & \cos{\theta}\end{pmatrix}\begin{pmatrix}x_1\\x_2\end{pmatrix}\right|\\
&=\left|\begin{pmatrix}x_1\cos{\theta} + x_2\sin{\theta}\\-x_1\sin{\theta} + x_2\cos{\theta}\end{pmatrix}\right|\\
&=(x_1\cos{\theta} + x_2\sin{\theta})^2+(-x_1\cos{\theta} + x_2\sin{\theta})^2\\
&=x_1^2 + x_2^2 = |(x_1,x_2)|^2.
\end{aligned}
\]
这里省略了展开,合并的过程,展开后刚好抵消一部分,留下部分使用$\cos^2{\theta} + \sin^2{\theta}=1$即可。上述等式说明这是一个保范数的线性变换,从而是保角的。至于后面部分,也是直接计算$\left<x, Tx\right>/|x|\cdot|Tx|$,并注意到$|Tx|=|x|$,于是就有
\[
\begin{aligned}
\frac{\left<x, Tx\right>}{|x|\cdot|Tx|} &= \frac{x_1(x_1\cos{\theta} + x_2\sin{\theta}) + x_2(-x_1\sin{\theta} + x_2\cos{\theta})}{x_1^2+x_2^2} \\
&= \frac{(x_1^2+x_2^2)\cos{\theta}}{x_1^2+x_2^2} = \cos{\theta},
\end{aligned}
\]
因而$\angle(x, Tx) = \theta$.


\item\label{exer009010110} 如$T:\real^m \to \real^n$是一线性变换,证明有这样的数$M$使得对于$h \in \real^m$有$|T(h)| \le M|h|$. 提示:用$|h|$以及$T$的矩阵中的元估计$|T(h)|$.

设$T$的矩阵为$A = (a_{ij})$是一$n \times m$矩阵,令$a=\max{|a_{ij}|}$,则
\[
\begin{aligned}
|T(h)|^2 &= \left<Th,Th\right> = \sum_{i=1}^{n}{(\sum_{j=1}^{m}{a_{ij}h_j})^2}\\
&\le \sum_{i=1}^{n}{(\sum_{j=1}^{m}{a_{ij}^2})^2(\sum_{j=1}^{m}{h_j^2})^2}\\
&\le \sum_{i=1}^{n}{ma^2\sum_{j=1}^{m}{h_j^2}} = mna^2\sum_{j=1}^{m}{h_j^2} = mna^2|h|^2
\end{aligned}
\]
于是,只要取$M = \sqrt{mn}a$,这里$a = \max{|a_{ij}|}$就有$|Th| \le M|h|$.

\item 如果$x, y \in \real^n$, $z, w \in \real^m$,证明:$\left<(x, z), (y,w)\right> = \left<x, y\right> + \left<z, w\right>$以及$|(x, z)| = \sqrt{|x|^2 + |z|^2}$. 注意$(x, z)$与$(y,w)$表示$\real^{n+m}$中的点。

直接展开
\[
\begin{aligned}
\left<(x,z),(y,w)\right> &= (x_1y_1+\cdots+x_ny_n) + (z_1w_1+\cdots+z_mw_m)\\
&=\left<x, y\right> + \left<z,w\right>,\\
|(x,z)|^2 &= \left<(x,z),(x,z)\right> = \left<x,x\right>+\left<z,z\right>\\
&=|x|^2+|z|^2.
\end{aligned}
\]

\item 设$(\real^n)^*$表示向量空间$\real^n$的对偶空间,如$x \in \real^n$,用$\varphi_x(y) = \left<x, y\right>$定义$\varphi_x \in (\real^n)^*$. 用$T(x) = \varphi_x$定义$T: \real^n \to (\real^n)^*$.证明$T$是一个1-1线性变换,并作出结论:每一个$\varphi \in (\real^n)^*$是关于唯一的一个$x \in \real^n$的$\varphi_x$.

(1)$T$是线性变换,这一点因为内积是双线性函数。$\forall z \in \real^n$,
\[
\begin{aligned}
T(\alpha{}x+\beta{}y)(z) &= \varphi_{\alpha{}x+\beta{}y}(z) = \left<\alpha{}x + \beta{}y, z\right> \\
&=\alpha\left<x,z\right> + \beta\left<y, z\right> = \alpha\varphi_x(z) + \beta\varphi_y(z) \\
&= \alpha{}Tx(z)+\beta{}Ty(z).
\end{aligned}
\]

(2)$T$是单射.也就是需要从$Tx_1=Tx_2$推出$x_1=x_2$. 从$Tx_1=Tx_2$可知,对于任意的$y \in \real^n$有$\left<x_1, y\right> = \left<x_2,y\right>$, 分别令$y=x_1$和$y=x_2$,可以得到
\[
\left<x_1,x_1\right> = \left<x_2,x_1\right>=\left<x_1, x_2\right> = \left<x_2,x_2\right>,
\]
由此可得$x_1=x_2$.


(3)$T$是满射. 也就是对于任一元素$\varphi \in (\real^n)^*$, 我们需要找到一个$x$使得$Tx = \varphi$. 那么如何构造这个$\varphi$呢?首先回忆一下什么是线性空间的对偶空间,所谓对偶空间,是所有$\real^n$上的线性泛函构成的线性空间,所谓线性泛函,实际上就是$L:\real^n \to \real$的线性变换。既然是线性空间,我们考虑$\real^n$的标准基底$\{e_1,\cdots,e_n\}$, 在$\varphi$作用下的值分别记为$\alpha_i$,也就是$\varphi(e_i) = \alpha_i$. 于是对于任一$x \in \real^n$,有$x = \sum\limits_{i=1}^{n}{x_ie_i}$,从而
\[
\varphi(x) = \varphi(\sum_{i=1}^{n}{x_ie_i}) = \sum_{i=1}^{n}{x_i\varphi(e_i)} = \sum_{i=1}^{n}{x_i\alpha_i},
\]
这一点提示我们,我们如果记$\alpha=(\alpha_1,\cdots, \alpha_n)$,那么$\varphi(x)=\left<x,\alpha\right>$. 也就是$T\alpha=\varphi$.


\item 如$x, y \in \real^n$,则若$\left<x,y\right>=0$,就称$x$与$y$垂直(或正交)。如$x$与$y$垂直,求证:$|x+y|^2 = |x|^2 + |y|^2$.

直接展开并利用对称性以及正交的定义即可。
\[
|x+y|^2 = \left<x+y,x+y\right> = \left<x,x\right> + 2\left<x,y\right> + \left<y,y\right> = |x|^2+|y|^2.
\]
\end{problemset}



\section{欧几里得空间的子集}\label{section0090102}
闭区间$[a,b]$在$\real^2$中有一自然的类比。这就是闭矩形$[a,b]\times[c,d]$,定义为一切数对$(x,y)$的全体,其中$x \in [a,b]$,$y \in [c,d]$。更一般的,如$A \subset \real^m$, $B \subset \real^n$,则$A \times B \subset \real^{m+n}$定义为一切$(x, y) \in \real^{m+n}$的集,其中$x \in A$, $y \in B$.特别,$\real^{m+n} = \real^m \times \real^n$.如$A \subset \real^m$, $B \subset \real^n$, 和$C \subset \real^p$,则$(A \times B) \times C = A \times (B \times C)$,二者皆简记为$A \times B \times C$;这一记法也推广到任意个数的集的乘积。集$[a_1, b_1] \times \cdots \times [a_n, b_n] \subset \real^n$称作$\real^n$中的闭矩形,而集$(a_1, b_1) \times \cdots \times (a_n, b_n) \subset \real^n$称作开矩形。更一般地,一个集$U \subset \real^n$称作开集,如果对每一个$x \in U$, 有一个开矩形$A$使得$x \in A \subset U$.

$\real^n$的一个子集$C$称为闭集如$\real^n-C$是开集。例如,如$C$只含有限多个点,则$C$是闭。读者应该补充证明:$\real^n$中的闭矩形确为一闭集。

如$A \subset \real^n$且$x \in \real^n$, 则下列三种可能性之一必成立。
\begin{enumerate}
\item[1.]存在一个开矩形$B$使得$x \in B \subset A$.
\item[2.]存在一个开矩形$B$使得$x \in B \subset \real^n - A$.
\item[3.]如$B$是任一个开矩形使得$x \in B$者,则$B$同时含有$A$与$\real^n-A$的点。
\end{enumerate}

满足(1)的那些点构成$A$的内域, 满足(2)的那些点构成$A$的外域,满足(3)的那些点构成$A$的边界。习题1-16到1-18表明这些术语有时可能有意想不到的意义。

不难看出,任何集$A$的内域是开的;对$A$的外域,它实际上是$\real^n-A$的內域,所以也是如此。于是(习题1-14)它们的并集是开的,而所剩下的,即其边界,必定是闭得。

我们把一组开集称为$A$的一个开覆盖(或简称覆盖$A$)\footnote{原文意思是若一个集族$\mathscr{O}$是$A$的覆盖,就说$\mathscr{O}$覆盖$A$. 而不是说开覆盖可以简称为覆盖,而应说开集族$\mathscr{O}$覆盖$A$}$\mathscr{O}$.如果任一点$x \in A$是在$\mathscr{O}$的某开集中。例如,如$\mathscr{O}$是一切开区间$(a, a+1)$的集合,其中$a \in \real$, 则$\mathscr{O}$是$\real$的一(开)覆盖。很明显,$\mathscr{O}$的有限个开集不能覆盖$\real$,也不能覆盖$\real$的任一无界集。类似情况对有界集也可能发生。设对一切正整数$n>1$,$\mathscr{O}$是一切开区间$(1/n,1-1/n)$的集合,则$\mathscr{O}$是$(0, 1)$的一开覆盖,但$\mathscr{O}$中的有限个集仍不能覆盖$(0, 1)$。虽然这一现象可能不会出现特别的坏处,但这种状况不会发生的集至为重要,它们有一个特殊的名称:一集$A$称为紧的,如它的任何开覆盖$\mathscr{O}$包含着一个有限个开集的组仍能覆盖$A$.

只有有限个点的集显然是紧的,包含$0$以及数$1/n$(对一切整数$n$)的无限集$A$也是紧的(理由:如$\mathscr{O}$是一覆盖:则对$\mathscr{O}$中某一开集$U$有$0 \in U$;$A$中只有有限个别的点不在$U$中,每个这样的点至多只要再加上一个开集)。

下列几个结果使对紧集的认识大大简化了,其中只有第一个结果有一定的深度(也就是,用到了有关实数的一些事实)。

\begin{theorem}{海涅-波雷耳(Heine-Borel)}{thm009010103}
闭区间$[a,b]$是紧的。
\end{theorem}

\begin{proof}
若$\mathscr{O}$是$[a,b]$的一个开覆盖,设$A = \{x: a \le x \le b\text{且}[a, x]\text{能被}\mathscr{O}\text{中某有限个开集所覆盖}\}$.注意$a \in A$,且$A$显然有上界(以$b$为上界)。我们希望证明$b \in A$。这只要对$\alpha = A$的上确界($\alpha = \sup{A}$)证明两件事:(1)$\alpha \in A$, (2)$b = \alpha$就行了。

因$\mathscr{O}$是一覆盖,故对某一$U$有$\alpha \in U$.那么在某区间中$\alpha$左边的一切点也在$U$中。因为$\alpha$是$A$的上确界,故在这区间中有一$x \in A$.于是$[a, x]$能被$\mathscr{O}$中某有限个开集所覆盖,而$[x, \alpha]$被一个集$U$所覆盖。所以$[a, \alpha]$能被$\mathscr{O}$中有限个开集所覆盖,即$\alpha \in A$。这就证明了(1).

要证(2)为真,假设不然:$\alpha < b$. 因此在$\alpha$与$b$之间有一点$x'$使$[\alpha, x'] \subset U$.因$\alpha \in A$,区间$[a, \alpha]$能被$\mathscr{O}$中有限个开集所覆盖,而$[\alpha, x']$已被$U$覆盖。所以$x' \in A$,这和$\alpha$是$A$的上确界相矛盾。
\end{proof}

若$B \subset \real^m$是紧的且$x \in \real^n$,易见$\{x\}\times B \subset \real^{n+m}$是紧的。但是,可以作出一个强得多的论断。

\begin{theorem}{}{thm009010104}
若$B$是紧的,$\mathscr{O}$是$\{x\} \times B$的一开覆盖,则有包含$x$的一开集$U \subset \real^n$使得$U \times B$能被$\mathscr{O}$中有限个集所覆盖。
\end{theorem}

\begin{proof}
因为$\{x\} \times B$是紧的,我们可以一开始就认为$\mathscr{O}$是有限的,我们只要找出开集$U$使$U \times B$被$\mathscr{O}$所覆盖。

对每一个$y \in B$, 点$(x, y)$在$\mathscr{O}$的某开集$W$中。因$W$是开的,对某一开矩形$U_y \times V_y$我们有$(x, y) \in U_y \times V_y \subset W$.这些集$V_y$覆盖了紧集$B$,所以有限个$V_{y_1}, \cdots, V_{y_k}$也覆盖$B$.令$U = U_{y_1} \cap \cdots \cap U_{y_k}$.于是,若$(x', y') \in U \times B$,对某一$i$我们有$y' \in V_{y_i}$,当然$x' \in U_{y_i}$.所以$(x',y') \in U_{y_i} \times V_{y_i}$,它包含在$\mathscr{O}$的某个$W$中。
\end{proof}

\begin{corollary}{}{coro009010105}
若$A \subset \real^n$与$B \subset \real^m$是紧的,则$A \times B \subset \real^{n+m}$也是紧的。
\end{corollary}

\begin{proof}
若$\mathscr{O}$是$A \times B$的一开覆盖,则对每一个$x \in A$, $\mathscr{O}$覆盖了$\{x\} \times B$。由定理\ref{thm:thm009010104},有一个包含$x$的开集$U_x$,使得$U_x \times B$能被$\mathscr{O}$中有限个集覆盖。因为$A$是紧的,$U_x$中的有限个$U_{x_1}, \cdots U_{x_m}$覆盖$A$.因为$\mathscr{O}$中有限个集覆盖每一个$U_{x_i} \times B$,所以$\mathscr{O}$中有限个集也就整个覆盖了$A \times B$.
\end{proof}

\begin{corollary}{}{coro009010106}
若每一个$A_i$是紧的,则$A_1 \times \cdots \times A_k$也是紧的。特别,$\real^k$中的闭矩形是紧的。
\end{corollary}

\begin{corollary}{}{coro009010107}
$\real^n$中的有界闭集是紧的。
\end{corollary}

(逆定理也真)

\begin{proof}
若$A \subset \real^n$是有界闭的, 则对某一个闭矩形$B$, $A \subset B$, 若$\mathscr{O}$是$A$的一个开覆盖,则$\mathscr{O}$是$\mathscr{O}$与$\real^n-A$一起是$B$的一个开覆盖。所以$\mathscr{O}$中有限个集$U_1, \cdots, U_n$,可能再加上$\real^n-A$,覆盖了$B$,因此, $U_1, \cdots, U_n$覆盖了$A$.
\end{proof}

\begin{problemset}
\item\label{exer009010114} 求证任何一个(即使是无穷多个)开集的并集是开的。求证两个(从而是有限个)开集的交集是开的,给出对于无穷多个开集的一个反例。

设$U = \bigcup_{i \in I}{U_i}$,这里$U_i$为开集,$I$是指标集,需要证明$U$是开集。对于任意的$a \in U$, 存在$i \in I$,使得$a \in U_i$,由于$U_i$是开集,从而存在开矩形$B$使得$a \in B \subset U_i \subset U$,于是$U$为开集。

对于交集,设$U = U_1 \cap U_2$, 那么对于$x \in U$,于是$x \in U_i$, $i=1,2$,存在开矩形$B_i$,使得$x \in B_i$,我们记开矩形$B_i = (a_i^{(1)}, b_i^{(1)}) \times \cdots \times (a_i^{(n)}, b_i^{(n)})$, 那么只需要取
\[
a^{(j)} = \max{(a_1^{(j)}, a_2^{(j)})},b^{(j)} = \min(b_1^{(j)}, b_2^{(j)}), j = 1,\cdots, n,
\]
即可,这样的$B = (a^{(1)}, b^{(1)}) \times \cdots \times (a^{(n)}, b^{(n)})$包含$x$,并且是$B_1 \cap B_2$的子集。从而是$U_1 \cap U_2$的子集,也就是$U$的子集。

至于反例,考虑集合$U_i = (-\frac{1}{i}, 1 + \frac{1}{i})$,这里$i=2,3,\cdots$,显然每一个都是开集,但是它们的交集等于$[0, 1]$是闭集。


\item\label{exer009010115} 求证$\{x \in \real^n:|x-a|<r\}$是开的(参见习题\ref{exer009010127})。

书中是使用开矩形来定义开集的,所以需要找到开矩形属于这个集合$U$。对于任意一点$y \in U$,记$s = \min{(|y-a|, r - |y-a|)}$,于是考虑以$y$为中心,边长是$s/\sqrt{2}$的开矩形B,那么对于任意一点$z \in B$,有$|z-y| < s$, 于是
\[
|z - a| < |z - y| + |y - a| < s + |y-a| < r - |y-a| + |y-a| = r,
\]
说明$z \in U$,也就是$B \subset U$,因而$U$是开集。

有了这个结论,后面的证明中,有时会这样使用开集:如果$U$是开集,$\forall a \in U$,那么存在$r>0$,使得$U(a,r) = \{x\in \real^n : |x-a| < r\} \subset U$.


\item\label{exer009010116} 求下列集的内域,外域和边界:
\begin{gather*}
\{x \in \real^n: |x| \le 1 \}\\
\{x \in \real^n: |x| = 1\}\\
\{x \in \real^n:\text{每一}x_i\text{是有理数}\}.
\end{gather*}

上述三个集合分别记作$A, B, C$,

(1)$A$是一个特别常见的集合,它的內域是集合
\[
\{x \in \real^n: |x| < 1 \},
\]
外域是
\[
\{x \in \real^n: |x| > 1 \},
\]
边界是
\[
\{x \in \real^n: |x| = 1 \},
\]

(2)$B$是一个闭集,其內域是空集$\emptyset$,外域为
\[
\{x \in \real^n: |x| \neq 1 \},
\]
边界就是它自身.
\[
\{x \in \real^n: |x| = 1 \}.
\]

(3)这个集合比较特殊,需要使用有理数的稠密性。它的內域和外域都是空集,边界是$\real^n$,全集。


\item\label{exer009010117} 求作一个集$A \subset [0, 1]\times[0,1]$,使得$A$在每一条水平线和铅直线上至多只含一点,但$A$的边界$=[0, 1] \times [0, 1]$。提示:只要能保证$A$在正方形$[0, 1] \times [0, 1]$的每四分之一中含有点,又在每十六分之一中含有点,如此等等,这就够了。

按照提示操作,需要使用不可数集和可数集的概念,这样来选择:首先把$[0,1]\times [0,1]$四等分,也就是把$[0,1]$二等分,分别在四个区域正方形区域中各自选择一个点,记为$a_1^1, a_1^2, a_1^3, a_1^4$,要求,任意两个点不在同一水平线或者铅直线上,然后把$[0,1]\times [0,1]$16(=$4^2$)等分,也就是把$[0, 1]$四等分($2^2$),然后分别在16个正方形区域各自选择一个点,记作$a_2^1,a_2^2, \cdots, a_2^{16}$,要求是:所有点不能在同一水平线或者铅直线上, 如此继续,$4^3$等分,等等,第$n$次操作,需要$4^n$等分,得到$a_n^{1}, a_n^2,\cdots, a_n^{4^n}$个点,要求还是:所有的点不能在同一水平线或者铅直线上,这个是可以办到的,因为无论是线段还是正方形区域,里面都是不可数个点,而前面选择的只有最多可数个点。这样得到无穷多个点,可数多个点,用$A$表示所有这些点的集合。这个集合的边界是$[0, 1] \times [0,1]$。

首先,根据$A$中点的构造过程可知,每一条水平线或者铅直线上最多只含一个$A$中的点。其次,对于任意一点$(x, y) \in [0, 1] \times [0, 1]$, 对于任意包含$(x, y)$的开矩形$B=(x_1, y_1) \times (x_2, y_2)$来说, 令$d = \min\{x_2-x_1, y_2-y_1\}$, 那么只要$n$足够大,使得$\frac{1}{2^n} < d$, 那么第$n$操作中,分成了$4^n$个边长为$\frac{1}{2^n}$个正方形的时候,至少有一个正方形落入上述开矩形$B$之中,从而必然有$B$中一点$(x',y') \in A$,另外由于$B$中的点是不可数的,而$A$中只有可数个点,因而也必有$B$中的点不属于$A$.所以$(x, y)$是$A$的边界上的点。

\item 如$A \subset [0, 1]$是这样一些开区间$(a_i, b_i)$的并集,使得$(0, 1)$中的每一有理数包含在某个$(a_i, b_i)$内,求证$A$的边界$=[0, 1]-A$.

首先$A$是开区间的并集,从而使开集。其次,从有理数的稠密性可知,对于任一点$x_0 \in [0, 1] - A$, 包含$x_0$的任意一个开区间,必然包含某个有理数,而有理数属于$A$, 于是$x_0$是$A$的边界点。至于$[0, 1]$之外的点,首先$\real - [0, 1]$是开集,从而属于$A$的外域。

\item 如$A$是包含任何有理数$r \in [0, 1]$的一个闭集,求证$[0, 1] \subset A$.

这道题目还是因为有理数的稠密性。对于任意的$x \in [0, 1]$,由于$[0, 1]$中的所有有理数属于$A$, 从而$x$是$A$的聚点($x$的任意邻域内,存在$A$的点)又因为$A$是闭集,因而$x \in A$.

\item\label{exer009010120} 求证推论\ref{cor:coro009010107}的逆:$\real^n$的紧集是闭有界集(参见习题\ref{exer009010128})。

设$T \subset \real^n$是紧集。

(1)考虑$T$的开覆盖:$U_m = \{x \in \real^n : |x| < m\}$,这里$m \in \mathbb{Z}^+$, 由于$T$是紧集,存在有限个开集$U_{m_1}, \cdots, U_{m_k}$覆盖$T$, 令$M = \max\{m_1\cdots, m_k,\}$,那么$T \subset U_M$,从而是有界的。

(2) 要证明$T$是闭集,也就是证明$\real^n-T$是开集。对于任意一点$x \in \real^n - T$, 对于任意一点$a \in T$,如果记$r_a = |x - a| / 2$,那么开球$U_a = \{x \in \real^n : |x - a| < r_a\}$为一开集,考虑所有这样的开球$U_a$, $a \in T$, 显然它们构成$T$的一个开覆盖,从而存在有限个开球$U_{a_1}, \cdots, U_{a_k}$覆盖$T$. 我们把这些半径中最小的那个记为$r$,也就是$r = \min\{r_{a_1}, \cdots, r_{a_k}\}$, 我们来看看集合$U(x, r) = \{y : \real^n : |y-x|<r\}$。我们希望证明$U(x, r)\cap T = \emptyset$. 那么对于任一点$y \in T$, 存在$a_i$,使得$y \in U_{a_i}$,也就是有
\[
|y - a_i| < r_{a_i},
\]
于是
\[
|y - x| \ge ||y - a_i| - |a_i - x|| = |2r_{a_i} - |y-a_i|| = 2r_{a_i} - |y-a_i| > r_{a_i} > r,
\]
因而$y \not \in U(x, r)$. 于是$U(x, r) \subset \real^n - T$, $\real^n-T$为开集,$T$是闭集。

\item (a)如$A$是闭的且$x \not\in A$,求证存在一数$d > 0$使对一切$y \in A$有$|y-x| \ge d$.

(b)如$A$是闭的,$B$是紧的,且$A \cap B =\emptyset$,求证存在$d>0$使对一切$y \in A$与$x \in B$有$|y-x| \ge d$. 提示:对每一个$b \in B$找出包含$b$的一开集$U$使得这一关系式对$x \in U \cap B$成立。

(c) 若$A$与$B$都是闭的但都不是紧的,试在$\real^2$中给出一个反例。

(a) $A$为闭集,那么$\real^n-A$为开集,而$x \in \real^n-A$, 从而存在开球$U(x, r) \subset \real^n-A$,选择$d=r$即可,此时,对于所有的$y \in A$,有$|y-a| \ge d$.

(b)从(a)可知,对于任一$b \in B$, 存在$d_b$使得$\forall a \in A$, $|a-b| \ge d_b$, 那么考虑开球$U_b = \{x \in \real^n : |x - b| < d_b/2\}$, $U_b$具有如下性质:任意点$a \in A$, $x \in U_b$有$|x-a| \ge d_b/2$. 理由如下:$|a-b| \ge d_b$, $|x - b| < d_b/2$, 于是
\[
|x-a| \ge ||x-b| - |b-a|| = |b-a| - |x-b| \ge d_b - d_b/2 = d_b/2.
\]
于是所有的$U_b$, $b \in B$构成$B$的一个开覆盖,由于$B$的紧性,存在有限个$U_{b_1}, \cdots, U_{b_k}$覆盖$B$.此时选取$d = \min\{d_{b_1}/2,\cdots, d_{b_k}/2\}$, 此时,$\forall y \in A$, $\forall x \in B$有$|y-x| \ge d$.证明方法和前面$U_b$类似。

(c)可以考虑由孤立点组成的两个集合
\[
\begin{aligned}
A &= \{(n, \frac{1}{n}) : n=2,3,\cdots\},\\ 
B &= \{(n, -\frac{1}{n}) : n=2,3,\cdots\}.
\end{aligned}
\]
这两个集合都是闭集,但不是紧集。它们之间的距离可以任意接近。

\item\label{exer009010122} 如$U$是开的且$C \subset U$是紧的,证明存在一紧集$D$使得$C \subset D$的内域且$D \subset U$.

这里需要使用上一题的结论。$U$为开集,那么$\real^n-U$为闭集。$C$为紧集,那么存在$d>0$,使得$\forall x \in C$, $\forall y \in \real^n - U$,有$|x-y| \ge d$. 由此,我们这样来构造$D$:
\[
D = \{x \in \real^n : \exists y \in C, |y-x| \le d / 2\}.
\]

(1)由于$C$是紧集,从而是有界,根据$D$的构造,$D$也是有界的,事实上,$\forall x \in D$,存在$y \in C$, $|x - y| \le d / 2$, 于是
\[
|x| \le |x - y| + |y| < M + d/2.
\]

(2)$D$是闭集,我们证明$\real^n-D$是开集。对于任一$x \in \real^n - D$,根据$D$的定义,$\forall y \in C$, $|y-x| > d / 2$。我们希望找到开球$U(x, r) \cap D = \emptyset$. 我们这样来选取这个$r$: 根据上一题的结论,存在$d_x$,使得对于任意$y \in C$,有$|y-x|\ge d_x$, 显然可以认为$d_x \ge d/2$, 是否可以更强一点,$d_x \> d/2$呢,对于紧集$C$来说,这是可以的(需要后面一节中,紧集上连续函数可以取到最大值和最小值),我们取$r = d_x - d/2$, 于是$\forall z \in U(x, r)$, 对于任一$y \in C$, 有
\[
|y - z| \ge ||y-x| - |z-x|| = |y-x| - |z-x| > d_x - r = d/2,
\]
因此$z \in \real^n - D$,从而$\real^n-D$为开集.

(3)还需要证明最后一步$D \subset U$. $\forall x \in D$, $y \in \real^n - U$, 存在$a \in C$, $|x-a| \le d/2$, $|y - a| \ge d$,于是
\[
|y-x| \ge ||y-a| - |x-a|| = |y-a| - |x-a| \ge d - d/2 = d/2.
\]
它说明$D$中的点不可能属于$\real^n-U$, 因此$D \subset U$.

\end{problemset}



\section{函数与连续性}\label{section0090103}
从$\real^n$到$\real^m$的一个函数(有时称为$n$个变元的(向量值)函数)是一个规则,它把$\real^n$中的每一点对应到$\real^m$中的某一点;一个函数$f$使$x$所对应的点记作$f(x)$. 我们写$f:\real^n \to \real^m$(按上下文读作“$f$把$\real^n$映入$\real^m$”或“$f$映$\real^n$入$\real^m$”)表明$f(x) \in \real^m$是对$x \in \real^n$定义的。记号$f: A \to \real^m$表示$f(x)$仅对集$A$中的$x$有意义,$A$称为$f$的定义域。如$B \subset A$,我们把$f(B)$定义为对$x \in B$的一切$f(x)$的集。又若$C \subset \real^m$,我们定义$f^{-1}(C) = \{x \in A:f(x) \in C\}$.记号$f:A \to B$表示$f(A) \subset B$.

通过作出一函数$f: \real^2 \to \real$的图,我们可以得到它的一方便的表示,这个图就是一切形如$(x, y, f(x, y))$的3数组的集,它实际上是3维空间中的一个图形(例如,见第\ref{chapter00902}章图)。

若$f,g:\real^n \to \real$,则函数$f+g$, $f-g$, $f \cdot g$与$f/g$可以确切地像单变量情况一样来定义。如$f:A \to \real^m$, $g: B \to \real^p$, 其中$B \subset \real^m$,则复合函数$g \circ f$定义为$g\circ f(x)= g(f(x))$; $g \circ f$的定义域是$A \cap f^{-1}(B)$. 如$f: A \to \real^m$是1-1的,也就是,当$x \neq y$时$f(x) \neq f(y)$,我们定义$f^{-1}:f(A) \to \real^n$,这里要求$f^{-1}(z)$是唯一的$x \in A$并且$f(x)=z$.

一个函数$f:A \to \real^m$用$f(x)=(f_1(x), \cdots, f_m(x))$确定$m$个分量函数$f_1, \cdots, f_m:A \to \real$.反过来,如果已给$m$个函数$g_1,\cdots, g_m:A \to \real$,则有唯一的函数$f:A \to \real^m$使得$f_i = g_i$,即$f(x)=(g_1(x),\cdots, g_m(x))$。这个函数$f$将记作$(g_1, \cdots, g_m)$,所以我们总有$f=(f_1,\cdots, f_m)$.如$\pi:\real^n \to \real^n$是恒等函数, $\pi(x)=x$,则$\pi_i(x)=x_i$;函数$\pi_i$称作第$i$个投影函数。

和单变量情况一样,记号$\lim\limits_{x \to a}{f(x)} = b$表示,当选取$x$足够接近于$a$但不等于$a$时,我们可以使$f(x)$任意地接近于$b$.用数学术语讲,这表明:对任一数$\epsilon>0$,存在一数$\delta>0$使对$f$的定义域中的一切满足$0 < |x-a| < \delta$的$x$有$|f(x) - b| < \epsilon$.函数$f:A \to \real^m$称为在$a \in A$连续,如果$\lim\limits_{x \to a}{f(x)} = f(a)$. $f:A \to \real^m$在每一$a \in A$处连续就简称$f$是连续的。关于连续性概念的有趣的意想不到点之一是,它可以不用极限来定义。由下一定理得知,$f:\real^n \to \real^m$连续,当且仅当只要$U \subset \real^m$是开的$f^{-1}(U)$就是开的;如$f$的域不是$\real^n$的全部,则需要一稍微复杂的条件。

\begin{theorem}{}{thm009010108}
如$A \subset \real^n$,函数$f:A \to \real^m$连续当且仅当对任一开集$U \subset \real^m$存在某开集$V \subset \real^n$使得$f^{-1}(U) = V \cap A$.
\end{theorem}

\begin{proof}
设$f$连续。如$a \in f^{-1}(U)$,则$f(a) \in U$.因$U$是开的,故有开矩形$B$使$f(a) \in B \subset U$.因$f$在$a$点连续,我们只要把$x$选取在包含$a$的某充分小的矩形$C$内,就能保证$f(x) \in B$.对每一$a \in f^{-1}(U)$这样做,并令$V$为所有这些$C$的并集。显然$f^{-1}(U) = V \cap A$.其逆也类似,留给读者去证明。
\end{proof}

定理\ref{thm:thm009010108}的下一推断极为重要。
\begin{theorem}{}{thm009010109}
如$f:A \to \real^m$是连续的,其中$A \subset \real^n$,而$A$是紧的,则$f(A)$也是紧的。
\end{theorem}

\begin{proof}
设$\mathscr{O}$是$f(A)$的一个开覆盖。对于$\mathscr{O}$中每一个开集$U$存在一个开集$V_U$使得$f^{-1}(U) = V_U \cap A$.一切$V_U$的集合是$A$的一开覆盖。因$A$是紧的,故有有限个$V_{U_1}, \cdots, V_{U_n}$覆盖$A$. 于是$U_1, \cdots, U_n$覆盖$f(A)$.
\end{proof}

若$f:A \to \real$有界,则$f$在$a \in A$处不连续的程度可以用一个确切的方法加以度量。对$\delta > 0$令
\begin{gather*}
\begin{aligned}
M(a,f,\delta) &= \sup\{f(x):x \in A\text{且}|x-a|<\delta\},\\
m(a,f,\delta) &= \inf\{f(x):x \in A\text{且}|x-a|<\delta\}.
\end{aligned}
\end{gather*}
$f$在$a$处的振幅$o(f,a)$定义为$o(f,a) = \lim\limits_{\delta \to 0}{[M(a,f,\delta) - m(a,f,\delta)]}$.因为$M(a,f,\delta)-m(a,f,\delta)$当$\delta$下降时也下降,所以这一极限恒存在。关于$o(f,a)$有两个重要事实。

\begin{theorem}{}{thm009010110}
有界函数$f$当且仅当$o(f, a)=0$时在$a$点连续。
\end{theorem}

\begin{proof}
设$f$在$a$点连续,对任一数$\epsilon>0$我们可以选取一数$\delta>0$使对一切$x \in A$且$|x-a|<\delta$者恒有$|f(x)-f(a)|<\epsilon$; 于是$M(a,f,\delta) - m(a,f,\delta) \le 2\epsilon$.因这对任意$\epsilon$为真,故有$o(f, a) = 0$.其逆类似,并留给读者。
\end{proof}

\begin{theorem}{}{thm009010111}
设$A \subset \real^n$是闭的。如$f:A \to \real$是任一有界函数,又$\epsilon>0$,则$\{x \in A: o(f, x) \ge \epsilon\}$是闭的。
\end{theorem}

\begin{proof}
设$B = \{x \in A: o(f, x) \ge \epsilon\}$.我们要证明$\real^n - B$是开的。如果$x \in \real^n-B$,那么或者有$x \not\in A$,不然的话,就有$x \in A$以及$o(f, x) < \epsilon$. 在第一种情况下,因$A$是闭的,故存在包含$x$的开矩形$C$使得$C \subset \real^n-A \subset \real^n-B$. 在第二种情况下,存在一$\delta>0$使得$M(x, f, \delta) - m(x, f, \delta) < \epsilon$. 令$C$是一包含$x$的开矩形,使得对一切$y \in C$, 有$|x-y|<\delta$. 则若$y \in C$, 就有一$\delta_1$, 使对所有满足$|z-y|<\delta_1$的$z$有$|x-z|<\delta$。于是$M(y,f,\delta_1) - m(y, f,\delta_1) < \epsilon$,从而$o(y, f) < \epsilon$.所以$C \subset \real^n-B$.
\end{proof}

\begin{problemset}
\item 若$f:A \to \real^m$且$a \in A$,证明$\lim\limits_{x \to a}{f(x)} = b$当且仅当对于$i=1,\cdots,m$有$\lim\limits_{x \to a}{f_i(x)} = b_i$.

\item 求证$f:A \to \real^m$在$a$点连续当且仅当每一个$f_i$都如此。

\item 求证线性变换$T:\real^n \to \real^m$是连续的。提示:利用习题\ref{exer009010110}.

\item 设$A = \{(x, y) \in \real^2: x > 0 \text{且}0 < y < x^2\}$.
\begin{enumerate}
\item[(a)]证明通过$(0, 0)$的任一直线包含一个围绕$(0, 0)$的在$\real^2-A$中的区间。
\item[(b)]这样定义$f:\real^2 \to \real$,当$x \not\in A$时,$f(x)=0$,当$x \in A$时$f(x)=1$.对$h \in \real^2$定义$g_h:\real \to \real$, $g_h(t)=f(th)$.求证每个$g_h$在0点连续,但$f$在$(0, 0)$点不连续。 
\end{enumerate}

\item\label{exer009010127} 由考察$f(x)=|x-a|$确定的$f:\real^n \to \real^1$来证明$\{x \in \real^n:|x-a|<r\}$是开的。

\item\label{exer009010128} 如$A \subset \real^n$不是闭的,证明存在一无界的连续函数$f:A \to \real$.提示:如$x \in \real^n - A$但$x \not\in$[$(\real^n-A)$的内域],令$f(y) - 1/|y-x|$.

\item 如$A$是紧的,求证任何连续函数$f:A \to \real$有最大值和最小值。

\item 设$f:[a, b] \to \real$是一增函数。如$x_1,\cdots, x_n \in [a, b]$,证明
\[
\sum_{i=1}^{n}{o(f, x_i)} < f(b) - f(a).
\]

\end{problemset}


\chapter{微分}\label{chapter00902}
\section{基本定义}\label{section0090201}
回想一函数$f: \real \to \real$在$a \in \real$处可微是指:有一数$f'(a)$使得
\begin{equation}\label{equ009020101}
\lim_{h \to 0}{\frac{f(a+h) - f(a)}{h}} = f'(a).
\end{equation}
对一般情形的函数$f:\real^n \to \real^m$这个式子当然没有意义,但可以用一种方式将其重写使之有意义。如$\lambda:\real \to \real$是由$\lambda(h) = f'(a)\cdot h$定义的线性变换,则(\ref{equ009020101})式等价于
\begin{equation}\label{equ009020102}
\lim_{h \to 0}{\frac{f(a+h) - f(a) - \lambda(h)}{h}} = 0.
\end{equation}
(\ref{equ009020102})式常常可解释为$\lambda+f(a)$是$f$在$a$处的一个好的近似(见习题). 因而我们集中注意力于线性变换$\lambda$,而把可微性重述如下:

函数$f:\real \to \real$在$a \in \real$点可微,如果有一线性变换$\lambda:\real \to \real$使得
\[
\lim_{h \to 0}{\frac{f(a+h) - f(a) - \lambda(h)}{h}} = 0.
\]

在这一形式下,这个定义对于高维有简单的推广:
\begin{definition}{}{def009020101}
函数$f:\real^n \to \real^m$在$a \in \real^n$点可微,如果存在一线性变换$\lambda:\real^n \to \real^m$使得
\[
\lim_{h \to 0}{\frac{|f(a+h) - f(a) - \lambda(h)|}{|h|}} = 0.
\]
\end{definition}
注意$h$是$\real^n$中的点,$f(a+h)-f(a)-\lambda(h)$是$\real^m$中的点,所以范数记号是不可少的,这个线性变换$\lambda$记作$Df(a)$,称作$f$在$a$点的\textbf{导数}\footnote{这里不同的书略有差异,从而会给人造成混淆,有些书把这个称为微分,而作为线性函数,给定一组基底之后,是存在一个对应的矩阵的,把这个矩阵称为导数。}。短语“这个线性变换$\lambda$”的正确性证明如下。
\begin{theorem}{}{thm009020101}
如$f: \real^n \to \real^m$在$a \in \real^n$点可微,则存在一个唯一的线性变换$\lambda:\real^n \to \real^m$使得
\[
\lim_{h \to 0}{\frac{|f(a+h) - f(a) - \lambda(h)|}{|h|}} = 0.
\]
\end{theorem}

\begin{proof}
假定$\mu:\real^n \to \real^m$也满足
\[
\lim_{h \to 0}{\frac{|f(a+h) - f(a) - \mu(h)|}{|h|}} = 0.
\]
如$d(h) = f(a+h) - f(a)$,则
\[
\begin{aligned}
&\lim_{h \to 0}{\frac{|\lambda(h) - \mu(h)|}{|h|}} \\
& = \lim_{h \to 0}{\frac{|\lambda(h) - d(h) + d(h) - \mu(h)|}{|h|}}\\
& \le \lim_{h \to 0}{\frac{|\lambda(h) - d(h)|}{|h|}} + \lim_{h \to 0}{\frac{|d(h) - \mu(h)|}{|h|}}\\
& = 0.
\end{aligned}
\]
另一方面,因$\frac{|\lambda(h) - \mu(h)|}{|h|} \ge 0$, 所以$\lim\limits_{h \to 0}{\frac{|\lambda(h) - \mu(h)|}{|h|}} = 0$. 如$x \in \real^n$,则当$t \to 0$时$tx \to 0$.因此对$x \neq 0$我们有
\[
0 = \lim_{t \to 0}{\frac{|\lambda(tx) - \mu(tx)|}{|tx|}} = \frac{|\lambda(x) - \mu(x)|}{|x|}
\]
所以$\lambda(x) = \mu(x)$.
\end{proof}

我们以后将会发现求$Df(a)$的一个简单方法。目前我们来考察由$f(x, y) = \sin{x}$定义的函数$f:\real^2 \to \real$. 那么$Df(a,b) = \lambda$满足$\lambda(x,y) = (\cos{a}) \cdot x$. 为要证明它,注意
\[
\begin{aligned}
&\lim_{(h,k) \to 0}{\frac{|f(a+h, b+k) - f(a, b) - \lambda(h, k)|}{|(h, k)|}}
&\quad = \lim_{(h,k) \to 0}{\frac{|\sin{(a+h)} - \sin{a} - (\cos{a}) \cdot h|}{|(h, k)|}}
\end{aligned}
\]
因为$\sin'(a) = \cos{a}$, 我们有
\[
\lim_{h \to 0}{\frac{|\sin(a + h) - \sin{a} - (\cos{a}) \cdot h|}{|h|}} = 0.
\]
因为$|(h, k)| \ge |h|$,所以还有
\[
\lim_{(h,k) \to 0}{\frac{|\sin{(a+h)} - \sin{a} - (\cos{a}) \cdot h|}{|(h, k)|}} = 0.
\]

考察$Df(a):\real^n \to \real^m$关于$\real^n$与$\real^m$的通常基底的矩阵,常常是方便的。这个$m \times n$矩阵称为$f$在$a$处的雅可比(Jacobi)矩阵,记作$f'(a)$. 如$f(x, y) = \sin{x}$, 则$f'(a,b) = (\cos{a}, 0)$. 如$f: \real \to \real$,则$f'(a)$是$1 \times 1$矩阵,其唯一元就是在初等微积分中记作$f'(a)$的那个数。

如果$f$仅在包含$a$的某个开集上定义,那么还可以定义$Df(a)$.为使定理的叙述流畅而又不失其普遍性,我们只考虑定义在$\real^n$上的函数。设有$f:\real^n \to \real^m$,如果$f$在每一个$a \in A$处可微就称$f$在$A$上可微。如$f:A \to \real^m$,又若$f$可以扩张为在包含$A$的某开集上的可微函数,则称$f$是可微的。

\begin{problemset}
\item 求证:如$f:\real^n \to \real^m$在$a \in \real^n$处可微,则它在$a$点连续。提示,利用习题\ref{exer009010110}.

\item 一函数$f: \real^2 \to \real$,如对每一$x \in \real$,对所有$y_1, y_2 \in \real$我们都有$f(x, y_1) = f(x, y_2)$,则称$f$与第二变元无关。试证$f$与第二变元无关当且仅当存在一函数$g:\real \to \real$使得$f(x, y) = g(x)$. $f'(a, b)$用$g'$表示时是什么?

\item 试决定何时一函数$f:\real^2 \to \real$与第一变元无关,并对这种$f$求出$f'(a,b)$.什么样的函数即与第一变元无关也与第二变元无关?

\item\label{exer009020104} 设$g$是单位圆周$\{x : \in \real^2 : |x| = 1\}$上的连续函数且有$g(0, 1) = g(1,0) = 0$, $g(-x) = -g(x)$.定义$f:\real^2 \to \real$为:
\[
f(x) = \left\{
\begin{aligned}
|x|\cdot g\left(\frac{x}{|x|}\right)&\quad x \neq 0,\\
0 &\quad x = 0.
\end{aligned}
\right.
\]
\begin{enumerate}
\item[(a)] 如$x \in \real^2$且$h:\real \to \real$定义为$h(t) = f(tx)$,证明$h$是可微的。
\item[(b)] 证明$f$在$(0, 0)$处不可微,除非$g = 0$.提示:当$k=0$时(再当$h=0$时)考察$(h, k)$,先证明$Df(0, 0)$必须为零。
\end{enumerate}


\item\label{exer009020105} 设$f:\real^2 \to \real$用下式定义:
\[
f(x, y) = \left\{
\begin{aligned}
\frac{x|y|}{\sqrt{x^2 + y^2}} &\quad (x, y) \neq 0,\\
0 &\quad (x, y) = 0.
\end{aligned}
\right.
\]
证明$f$是习题\ref{exer009020104}中考虑过的那种函数,所以$f$在$(0, 0)$处不可微.

\item 设$f:\real^2 \to \real$定义为$f(x, y) = \sqrt{|xy|}$.求证$f$在$(0, 0)$处不可微.

\item 设$f:\real^n \to \real$是一函数使得$|f(x)| \le |x|^2$. 证明$f$在0处可微.

\item 设$f:\real \to \real^2$. 求证:当且仅当$f_1$与$f_2$在$a \in \real$处可微时,$f$在$a$处可微,且这时
\[
f(a) = \begin{pmatrix}
(f_1)'(a)\\
(f_2)'(a)
\end{pmatrix}.
\]


\item 两函数$f,g:\real \to \real$称为在$a$点直到$n$阶相等, 如果
\[
\lim_{h \to 0}{\frac{f(a+h) - g(a+h)}{h^n}} = 0.
\]
\begin{enumerate}
\item[(a)] 试证:$f$在$a$点可微,当且仅当$f$在$a$点连续,且存在形如$g(x)=a_0 + a_1(x-a)$的函数$g$使得$f$与$g$在$a$点直到一阶相等。
\item[(b)] 如$f'(x),\cdots,f^{(n)}(x)$在$x=a$附近存在, $f^{(n)}(x)$在$a$点连续,试证$f$与下式
\[
g(x) = \sum_{i=0}^{n}{\frac{f^{(i)}(a)}{i!}(x-a)^i}
\]
定义的函数$g$在$a$点直到$n$阶相等。提示:极限
\[
\lim_{x \to a}{\frac{f(x) - \sum_{i=0}^{n}{\frac{f^{(i)}(a)}{i!}(x-a)^i}}{(x-a)^n}}
\]
可用洛必达(L'Hospital)法则计算。
\end{enumerate}

\end{problemset}


\section{基本定理}\label{section0090202}
\begin{theorem}{锁链法则}{thm009020202}
如$f:\real^n \to \real^m$在$a$点可微,$g:\real^m \to \real^p$在$f(a)$点可微,则其复合$g \circ f: \real^n \to \real^p$在$a$点可微,且
\[
D(g \circ f)(a) = Dg(f(a)) \circ Df(a).
\]
\end{theorem}

注。此式可写成
\[
(g \circ f)'(a) = g'(f(a)) \circ f'(a).
\]
如$m=n=p=1$,我们便得到老的锁链规则。

\begin{proof}
令$b = f(a)$, $\lambda = Df(a)$, $\mu = Dg(f(a))$.如果我们定义
\begin{align}
\varphi(x) &= f(x)- f(a) - \lambda(x - a), \label{equ009020201}\\
\psi(x) &= g(y) - g(b) - \mu(y - b), \label{equ009020202}\\
\rho(x) &= g\circ f(x) - g \circ f(a) - \mu \circ \lambda(x - a),\label{equ009020203}
\end{align}
则
\begin{align}
\lim_{x \to a}{\frac{|\varphi(x)|}{|x - a|}} &=0,\label{equ009020204}\\
\lim_{y \to b}{\frac{|\psi(x)|}{|y - b|}} &= 0, \label{equ009020205}
\end{align}
而我们必须证明
\[
\lim_{x \to a}{\frac{|\rho(x)|}{|x - a|}} = 0.
\]
现在
\[
\begin{aligned}
\rho(x) &= g(f(x)) - g(b) - \mu(\lambda(x - a)) \\
&= g(f(x)) - g(b) - \mu(f(x) - f(a) - \varphi(x)) \quad\text{由(\ref{equ009020201})}\\
&= [g(f(x)) - g(b) - \mu(f(x) - f(a))] + \mu(\varphi(x)) \\
&= \psi(f(x)) + \mu(\varphi(x)) \text{由(\ref{equ009020202})}
\end{aligned}
\]
于是我们必须证明
\begin{align}
&\lim_{x \to a}{\frac{|\psi(f(x))|}{|x - a|}} = 0, \label{equ009020206}\\
&\lim_{x \to a}{\frac{|\mu(\varphi(x))|}{|x - a|}} = 0. \label{equ009020207}
\end{align}
(\ref{equ009020207})式容易从(\ref{equ009020204})式和习题\ref{exer009010110}推得。如果$\epsilon > 0$, 从(\ref{equ009020205})式推知,对某一个$\delta > 0$,我们有
\[
|\psi(f(x))| < \epsilon|f(x) - b|, \text{只要}|f(x) - b| < \delta,
\]
而这一点只要对某个$\delta_1$, 由$|x - a|<\delta_1$就总成立,于是,由习题\ref{exer009010110},对某个$M$,
\[
\begin{aligned}
|\psi(f(x))| & < \epsilon|f(x) - b| \\
&= \epsilon|\varphi(x) + \lambda(x - a)|\\
&= \epsilon|\varphi(x)| + \epsilon{}M|x - a|.
\end{aligned}
\]
(\ref{equ009020206})式现就容易得出。

\end{proof}

\begin{theorem}{}{thm009020203}
\begin{enumerate}
\item[(a)] 如$f:\real^n \to \real^m$是一常值函数(也就是,若对某$y \in \real^m$,我们有:对一切$x \in \real^n$, $f(x) = y$),那么
\[
Df(a) = 0.
\]

\item[(2)] 如$f:\real^n \to \real^m$是一个线性变换,则
\[
Df(a) = f.
\]

\item[(3)] 如$f:\real^n \to \real^m$,则$f$在$a \in \real^n$处可微当且仅当每个$f_i$是如此,且
\[
Df(a) = (Df_1(a), \cdots, Df_m(a)).
\]
于是$f'(a)$是$m \times n$矩阵,其第$i$行是$f_i'(a)$.

\item[(4)]如$s:\real^2 \to \real$定义为$s(x, y) = x + y$,则
\[
Ds(a, b) = s.
\]

\item[(5)]如$p:\real^2 \to \real$定义为$p(x, y) = x \cdot y$, 则
\[
Dp(a, b)(x, y) = bx + ay.
\]
于是$p'(a,b) = (b, a)$.
\end{enumerate}
\end{theorem}

\begin{proof}
\begin{enumerate}
\item[(1)]
\[
\lim_{h \to 0}{\frac{|f(a + h) - f(a) - 0|}{|h|}} = \lim_{h \to 0}{\frac{|y - y - 0|}{|h|}} = 0.
\] 

\item[(2)]
\[
\lim_{h \to 0}{\frac{|f(a + h) - f(a) - f(h)|}{|h|}} = \lim_{h \to 0}{\frac{|f(a) = f(h) - f(a) - f(h)|}{|h|}} = 0.
\]

\item[(3)]如每一个$f_i$在$a$处可微,且
\[
\lambda = (Df_1(a), \cdots, Df_m(a)),
\]
则
\[
\begin{aligned}
& f(a + h) - f(a) - f(h) \\
&\qquad = (f_1(a + h) - f_1(a) - Df_1(a)(h), \cdots, \\
&\qquad f_m(a+h) - f_m(a) - Df_m(a)(h)).
\end{aligned}
\]
所以
\[
\begin{aligned}
&\lim_{h \to 0}{\frac{|f(a + h) - f(a) - \lambda(h)|}{|h|}}\\
&\le \lim_{h \to 0}{\sum_{i=1}^{m}{\frac{|f_i(a+h) - f_i(a) - Df_i(a)(h)|}{|h|}}}\\
&=0.
\end{aligned}
\]

另一方面,如$f$在$a$处可微,则由(2)与定理\ref{thm:thm009020202}, $f_i = \pi_i\circ f$在$a$处可微。

\item[(4)]由(2)推得。

\item[(5)]令$\lambda(x, y) = bx + ay$. 那么
\[
\begin{aligned}
&\lim_{(h, k) \to 0}{\frac{|p(a +h, b+k) - p(a, b) - \lambda(h, k)|}{|(h, k)|}}\\
&\qquad=\lim_{(h, k) \to 0}{\frac{|hk|}{|(h, k)|}}.
\end{aligned}
\]
现在
\[
|hk| \le \left\{
\begin{aligned}
|h|^2 &\quad\text{如}|k| \le |h|,\\
|k|^2 &\quad\text{如}|h| \le |k|,
\end{aligned}
\right.
\]
因此$|hk| \le |h|^2 + |k|^2$.所以
\[
\frac{|hk|}{|(h,k)|} \le \frac{h^2 + k^2}{\sqrt{h^2 + k^2}} = \sqrt{h^2 + k^2},
\]
因而
\[
\lim_{(h, k) \to 0}{\frac{|hk|}{|(h, k)|}} = 0.
\]

\end{enumerate}
\end{proof}

\begin{corollary}{}{coro009020204}
如$f,g:\real^n \to \real$在$a$处可微,则
\begin{gather*}
D(f+g)(a) = Df(a) + Dg(a),\\
D(f \cdot g)(a) = g(a)Df(a) + f(a)Dg(a).
\end{gather*}
此外,如果$g(a) \neq 0$,则
\[
D(f/g)(a) = \frac{g(a)Df(a) - f(a)Dg(a)}{[g(a)]^2}.
\]
\end{corollary}

\begin{proof}
我们将证明第一式而把其余的留给读者。因为$f + g = s \circ (f, g)$, 我们有
\[
\begin{aligned}
D(f+g)(a) &= Ds(f(a), g(a)) \circ D(f, g)(a)\\
&= s \circ (Df(a), Dg(a))\\
&= Df(a) + Dg(a).
\end{aligned}
\]
\end{proof}

下面这样的函数$f: \real^n \to \real^m$的可微性现在得到了保证。其各分量函数可以从函数$\pi_i$(它们是线性变换)以及我们在初等微积分中早已会求导的函数经过加法、乘法、除法和复合而获得。但是,求$Df(x)$或$f'(x)$可能是一项相当艰巨的工作。例如,设$f:\real^2 \to \real$定义为$f(x, y) = \sin(xy^2)$.因为$f = \sin\circ(\pi_1 \cdot [\pi_2]^2)$,我们有
\[
\begin{aligned}
f'(a, b) &= \sin'(ab^2) \cdot [b^2(\pi_1)'(a, b) + a([\pi_2]^2)'(a, b)]\\
&= \sin'(ab^2) \cdot [b^2(\pi_1)'(a, b) + 2ab(\pi_2)'(a, b)]\\
&=(\cos(ab^2)) \cdot [b^2(1, 0) + 2ab(0, 1)]\\
&= (b^2\cos(ab^2), 2ab\cos(ab^2)).
\end{aligned}
\]
幸而我们很快将会发现计算$f'$的一种简单得多的方法。

\begin{problemset}
\item 利用本节定理求以下的$f'$:
\begin{enumerate}
\item[(a)] $f(x, y, z) = x^y$.
\item[(b)] $f(x, y, z) = (x^y, z)$.
\item[(c)] $f(x, y) = \sin(x\sin{y})$.
\item[(d)] $f(x, y, z) = \sin(x\sin(y\sin{z}))$.
\item[(e)] $f(x, y, z) = x^{y^z}$.
\item[(f)] $f(x, y, z) = x^{y+z}$.
\item[(g)] $f(x, y, z) = (x + y)^z$.
\item[(h)] $f(x, y) = \sin(xy)$.
\item[(i)] $f(x, y) = [\sin(xy)]^{\cos{3}}$.
\item[(j)] $f(x, y) = (\sin(xy), \sin(x\sin{y}), x^y)$.
\end{enumerate}

\item 求以下的$f'$(其中$g:\real \to \real$是连续的):
\begin{enumerate}
\item[(a)] $f(x, y) = \int_{a}^{x+y}{g}$.
\item[(b)] $f(x, y) = \int_{a}^{x \cdot y}{g}$.
\item[(c)] $f(x, y, z) = \int_{x^y}^{\sin(x\sin(y\sin{z}))}{g}$.
\end{enumerate}

\item 一函数$f: \real^n \times \real^m \to \real^p$, 如果对$x,x_1,x_2 \in \real^n$, $y, y_1, y_2 \in \real^m$以及$a \in \real$,我们有
\begin{gather*}
f(ax, y) = af(x,y) = f(x, ay),\\
f(x_1 + x_2, y) = f(x_1, y) + f(x_2, y),\\
f(x, y_1+y_2) = f(x, y_1) + f(x, y_2),
\end{gather*}
则称$f$是双线性的。
\begin{enumerate}
\item[(a)] 求证若$f$是双线性的,则
\[
\lim_{(h, k) \to 0}{\frac{|f(h, k)|}{|(h, k)|}} = 0.
\]

\item[(b)]求证$Df(a, b)(x, y) = f(a, y) + f(x, b)$.

\item[(c)]证明定理\ref{thm:thm009020203}中$Dp(a, b)$的公式是(b)的一特殊情况。
\end{enumerate}

\item 定义$IP: \real^n \times \real^n \to \real$为$IP(x, y) = \left<x,y\right>$.
\begin{enumerate}
\item[(a)] 求$D(IP)(a, b)$与$(IP)'(a, b)$.
\item[(b)] 如$f, g: \real \to \real^n$可微且$h:\real \to \real$定义为$h(t) = \left<f(t), g(t)\right>$, 证明
\[
h'(a) = \left<f'(a)^{T}, g(a)\right> + \left<f(a), g'(a)^{T}\right>.
\]
(注意$f'(a)$是一$n \times 1$矩阵;其转置矩阵$f'(a)^{T}$是一$1 \times n$矩阵,我们把它看作$\real^n$的元.)
\item[(c)] 如$f:\real \to \real^n$可微且对一切$t$,$|f(t)| = 1$,证明$\left<f'(t)^{T}, f(t)\right> = 0$.
\item[(d)]举出一可微函数$f: \real \to \real$使得$|f|(t) = |f(t)|$定义的函数$|f|$不可微。
\end{enumerate}

\item 设$E_i(i=1,\cdots, k)$是各维数不必相同的欧氏空间。一函数$f: E_1 \times \cdots \times E_k \to \real^p$称为重线性的,如果对于每个选定的$x_j \in E_j$($j \neq i$),由$g(x) = f(x_1,\cdots,x_{i-1}, x, x_{i+1},\cdots, x_k)$定义的函数$g:E_i \to \real^p$是一个线性变换。
\begin{enumerate}
\item[(a)]如果$f$是重线性的而且$i \neq j$, 证明对于$h = (h_1,\cdots, h_k)$,其中$h_l \in E_l$, 我们有
\[
\lim_{h \to 0}{\frac{|f(a_1,\cdots,h_i,\cdots,h_j,\cdots,a_k)|}{|h|}} = 0.
\]
提示:如果$g(x, y) = f(a_1,\cdots,x,\cdots,y,\cdots, a_k)$,则$g$是双线性的。
\item[(b)] 求证
\[
Df(a_1,\cdots, a_k)(x_1,\cdots, x_k) = \sum_{i=1}^{k}{f(a_1,\cdots, a_{i-1}, x_i, a_{i+1},\cdots, a_k)}.
\]
\end{enumerate}

\item 把一个$n \times n$矩阵的每一列视作$\real^n$的一元从而把矩阵本身当作$n$重乘积$\real^n \times \cdots \times \real^n$中的一个点。
\begin{enumerate}
\item[(a)] 求证$\det: \real^n \times \cdots \times \real^n \to \real$可微且
\[
D(\det)(a_1,\cdots, a_n)(x_1,\cdots, x_n) = \sum_{i=1}^{n}{\det\begin{bmatrix}a_1\\ \vdots \\x_i\\ \vdots\\ a_n\end{bmatrix}}.
\]

\item[(b)] 如果$a_{ij}: \real \to \real$可微而$f(t) = \det(a_{ij}(t))$,试证
\[
f'(t) = \sum_{j=1}^{n}{\det\begin{bmatrix}
a_{11}(t) &\cdots & a_{1n}(t) \\
\vdots & & \vdots\\
a_{j1}'(t) & \cdots & a_{jn}'(t)\\
\vdots & & \vdots\\
a_{n1}(t) &\cdots & a_{nn}(t) \\
\end{bmatrix}}.
\]

\item[(c)] 如对一切$t$, $\det{(a_{ij}(t))} \neq 0$, 且$b_1, \cdots, b_n: \real \to \real$都是可微的,又设$s_1,\cdots, s_n:\real \to \real$是这样的一些函数使得$s_1(t),\cdots, s_n(t)$是方程组
\[
\sum_{j=1}^{n}{a_{ij}s_j(t) = b_j(t)}, \quad i=1,\cdots, n
\]
的解。求证$s_i$可微并求出$s_i'(t)$,
\end{enumerate}

\item 设$f:\real^n \to \real^n$可微且有可微的逆$f^{-1}:\real^n \to \real^n$。证明:$(f^{-1})'(a) = [f'(f^{-1}(a))]^{-1}$. 提示:$f \circ f^{-1}(x) = x$.

\end{problemset}

\section{偏导数}\label{section0090203}
我们从讨论“每次对一个变元”求导数的问题开始,如$f: \real^n \real$且$a \in \real^n$,如极限
\[
\lim_{h \to 0}{\frac{f(a_1, \cdots, a_i + h, \cdots, a_n) - f(a_1,\cdots, a_n)}{h}}
\]
存在,就记作$D_if(a)$,称为$f$在$a$点的偏导数。注意$D_if(a)$是某函数的通常导数,这很重要;实际上,如$g(x) = f(a_1,\cdots, x,\cdots, a_n)$,则$D_if(a) = g'(a)$. 这表明, $D_if(a)$是$f$的图形和平面$x_j = a_j$($j \neq i$)的交线在$(a, f(a))$点切线的斜率(图)。这也表明,计算$D_if(a)$是我们早已会做的问题。如$f(x_1,\cdots,x_n)$已由含有$x_1,\cdots,x_n$的某公式给出,则我们可这样来求$D_if(x_1,\cdots, x_n)$,即把所有$x_j$($j \neq i$)都看作常数,而对所得的$x_i$的函数在$x_i$求导。例如,$f(x,y) = \sin(xy^2)$,则$D_1f(x,y) = y^2\cos(xy^2)$, $D_2f(x, y) = 2xy\cos(xy^2)$. 又如,$f(x,y) = x^y$, 则$D_1f(x, y) = yx^{y-1}$, $D_2f(x, y) = x^y\ln{x}$.

稍经练习(例如,作本节末的习题),就和已经会计算的通常导数一样,也会很容易地计算$D_if$.

如果对一切$x \in \real^n$, $D_if(x)$存在,我们便得到一个函数$D_if: \real^n \to \real$.这个函数在$x$点的第$j$个偏导数,也就是$D_j(D_if)(x)$,常常记作$D_{i,j}f(x)$。注意这个记号把$i$和$j$的次序颠倒了。实际上,这个次序通常是没有关系的,因为绝大多数函数(在习题中给出例题)满足$D_{i,j}f = D_{j,i}f$. 有好些细致的定理保证这个等式:下面这个定理已经完全够用了。我们把它的陈述放在这里面而把证明放在后面(习题\ref{exer009030328})。
\begin{theorem}{}{thm009020305}
如$D_{i,j}f$与$D_{j,i}f$在包含$a$的一开集中连续,则
\[
D_{i,j}f(a) = D_{j,i}f(a).
\]
\end{theorem}

函数$D_{i,j}f$叫做$f$的二阶(混合)偏导数。高阶(混合)偏导数可用明显的方式来定义。显然定理\ref{thm:thm009020305}能用来证明在适当条件下高阶混合偏导数的相应等式。如$f$有一切阶的偏导数,则$D_{i_1,\cdots, i_k}f$中$i_1,\cdots, i_k$的次序是完全无所谓的。具有这种性质的函数称为$C^{\infty}$函数。在以下各章中,为方便起见,经常仅限于讨论$C^{\infty}$函数。

在下节,将用偏导数来求导数。它们还有另外一个重要的用处---求函数的极大值和极小值。

\begin{theorem}{}{thm009020306}
设$A \subset \real^n$, 如$f:A \to \real$在$A$的內域中点$a$处达到极大(或极小),且$D_if(a)$存在,则$D_if(a) = 0$.
\end{theorem}

\begin{proof}
设$g_i(x) = f(a_1,\cdots,x,\cdots,x_n)$。显然$g_i$在$a_i$处有极大值(或极小值),且$g_i$在包含$a_i$的一开区间中有定义。因此$0 = g_i'(a_i) = D_if(a)$.
\end{proof}

提醒读者,定理\ref{thm:thm009020306}的逆即使当$n=1$时已不成立(如$f:\real \to \real$由$f(x) = x^3$定义,则$f'(0)=0$,但$0$点甚至不是一个局部极大值或极小值). 如$n > 1$,定理\ref{thm:thm009020306}的逆可以在一种更为奇特的方式下不再为真。例如,设$f:\real^2 \to \real$由$f(x, y) = x^2 - y^2$定义(图)。则因$g_1$在$0$处有一极小值,故$D_1f(0,0)=0$;而因$g_2$在$0$处有一个极大值,故$D_2f(0, 0) = 0$.显然$(0,0)$既不是相对极大点也不是相对极小点。

如用定理\ref{thm:thm009020306}来寻求$f$在$A$上的最大值或最小值,那么还必须单另检查$f$在边界点上的值----这是一件可怕的事情,因为$A$的边界可能是整个$A$! 习题\ref{exer009020227}指明一种作法,习题\ref{exer009050516}陈述了一个经常可用的好方法。

\begin{problemset}
\item\label{exer009020217} 求下列函数的偏导数:
\begin{enumerate}
\item[(a)] $f(x, y, z) = x^y$.
\item[(b)] $f(x, y, z) = z$.
\item[(c)] $f(x, y) = \sin(x\sin{y})$.
\item[(d)] $f(x, y, z) = \sin(x\sin(y\sin{z}))$.
\item[(e)] $f(x, y, z) = x^{y^z}$.
\item[(f)] $f(x, y, z) = x^{y+z}$.
\item[(g)] $f(x, y, z) = (x + y)^z$.
\item[(h)] $f(x, y) = \sin(xy)$.
\item[(i)] $f(x, y) = [\sin(xy)]^{\cos{3}}$.
\end{enumerate}

\item 求下列函数的偏导数(其中$g:\real\to\real$连续):
\begin{enumerate}
\item[(a)] $f(x, y) = \int_{a}^{x+y}{g}$.
\item[(b)] $f(x, y) = \int_{y}^{x}{g}$.
\item[(c)] $f(x, y) = \int_{a}^{xy}{g}$.
\item[(d)] $f(x, y, z) = \int_{a}^{\int_{b}^{y}{g}}{g}$.
\end{enumerate}

\item 如$f(x, y) = x^{x^{x^{x^y}}} + (\ln{x})(\arctan(\arctan(\arctan(\sin(\cos{xy}) - \ln(x+y)))))$, 求$D_2f(1, y)$.提示:有一很容易的做法。

\item 通过$g$与$h$的导数求$f$的偏导数,如果
\begin{enumerate}
\item[(a)] $f(x, y) = g(x)h(y)$.
\item[(b)] $f(x, y) = g(x)^{h(y)}$.
\item[(c)] $f(x, y) = g(x)$.
\item[(d)] $f(x, y) = g(y)$.
\item[(e)] $f(x, y) = g(x + y)$.
\end{enumerate}

\item 设$g_1, g_2: \real^2 \to \real$连续,定义$f:\real^2 \to \real$为
\[
f(x, y) = \int_{0}^{x}{g_1(t, 0)dt} + \int_{0}^{y}{g_2(x, t)dt}.
\]
\begin{enumerate}
\item[(a)] 证明$D_2f(x,y) = g_2(x, y)$.
\item[(b)] $f$应怎样定义使得$D_1f(x, y) = g_1(x, y)$?
\item[(c)] 求一函数$f:\real^2 \to \real$使得$D_1f(x, y) = x$, $D_2f(x, y) = y$.再求一个使得$D_1f(x, y) = y$, $D_2f(x, y) = x$.
\end{enumerate}

\end{problemset}


\chapter{积分}\label{chapter00903}
\section{基本定义}\label{section0090301}



%\begin{eqnarray}
%\lefteqn{\int_0^\infty some expression =} \\
%& & some terms \\
%& & \mbox{} + more terms
%\end{eqnarray}


\section{测度零和容度零}\label{section0090302}


\section{可积函数}\label{section0090303}


\section{富比尼定理}\label{section0090304}

\begin{problemset}
\item\label{exer009030328} 设$D_{1,2}f$与$D_{2,1}f$都连续,应用富比尼定理对$D_{1,2}f = D_{2,1}f$给一简证。提示:

\end{problemset}



\chapter{链上的积分}\label{chapter00904}




\chapter{流形上的积分}\label{chapter00905}

