% 一些参考网页
% https://blog.csdn.net/appmathw/article/details/82831807
% https://tex.stackexchange.com/questions/66655/numbered-definitions-theorems-in-beamer
% https://tex.stackexchange.com/questions/174257/blocks-in-beamer
% https://tex.stackexchange.com/questions/94481/how-to-adjust-the-beamer-block-width-to-the-size-of-its-content
% https://tex.stackexchange.com/questions/43652/standalone-latex-editor-renderer-for-windows
% https://blog.csdn.net/xueshengke/article/details/53045880
% https://zhuanlan.zhihu.com/p/36868831

\documentclass[UTF8]{ctexbeamer}

%\mode<presentation>
%{
%    \setbeamercovered{dynamic}  % translucent when using pause
%    \setbeamertemplate{navigation symbols}{}    % hide navigation bars
%    \setbeamertemplate{caption}[numbered]   % numerate captions
%    \setbeamertemplate{background}{\includegraphics[height=\paperheight]{ISEE.pdf}} % set background image
%    \setbeamertemplate{footline}{\footnotesize \insertframenumber/\inserttotalframenumber \hfill}   % display page number at bottom left corner 
%}

\usetheme{Boadilla}
\usecolortheme{beaver}
\usepackage{graphicx}
\usepackage{color}
\usepackage{amsmath}
\usepackage{mathrsfs}
\setbeamertemplate{navigation symbols}{}    % hide navigation bars

\begin{document}
\title{数学分析}
\subtitle{集合}
\author{\songti 虞朝阳}
\institute{zhaoyang0618@msn.com}
\date{\today}
\frame{\titlepage}

%\begin{frame}
%\frametitle{目录}
%\tableofcontents
%\end{frame}

%\section{进化树之后要回答的问题}
%\begin{frame}
%\frametitle{目录}
%\tableofcontents[currentsection]
%\end{frame}

\begin{frame}
\frametitle{集合运算}
\begin{itemize}
\item 交集
\item 并集
\item 差集
\end{itemize}
\end{frame}

\begin{frame}[t]
\frametitle{问题Q0001}
\begin{block}{\textbf{基本思路}}
	\begin{itemize}
		\item<0-> 使用编辑距离矩阵将类似的消息归于一张连通图中。
		\item<0-> 使用固定值替换感兴趣的消息,如代码、email地址。
		\item<0-> 查找归一化距离小于阈值的消息,并确定聚类边界。
	\end{itemize}
\end{block}
\end{frame}

\begin{frame}[t]
\frametitle{问题Q0002}
假设$f:\mathbb{R} \to \mathbb{R}$为一连续函数
\end{frame}


\end{document}
