\documentclass[UTF8]{ctexbeamer}
\usetheme{Boadilla}
\usecolortheme{beaver}
\usepackage{graphicx}
\usepackage{color}
\usepackage{amsmath}
\usepackage{mathrsfs}
\setbeamertemplate{navigation symbols}{}    % hide navigation bars

\begin{document}
\title{数学分析}
\subtitle{集合}
\author{\songti 虞朝阳}
\institute{zhaoyang0618@msn.com}
\date{\today}
\frame{\titlepage}

%\begin{frame}
%\frametitle{目录}
%\tableofcontents
%\end{frame}

%\section{进化树之后要回答的问题}
%\begin{frame}
%\frametitle{目录}
%\tableofcontents[currentsection]
%\end{frame}

\begin{frame}
\frametitle{集合运算}
\begin{itemize}
\item 交集
\item 并集
\item 差集
\end{itemize}
\end{frame}

\begin{frame}[t]
\frametitle{问题Q0001}
\begin{block}{\textbf{基本思路}}
	\begin{itemize}
		\item<0-> 使用编辑距离矩阵将类似的消息归于一张连通图中。
		\item<0-> 使用固定值替换感兴趣的消息,如代码、email地址。
		\item<0-> 查找归一化距离小于阈值的消息,并确定聚类边界。
	\end{itemize}
\end{block}
\end{frame}

\begin{frame}[t]
\frametitle{问题Q0002}
假设$f:\mathbb{R} \to \mathbb{R}$为一连续函数
\end{frame}


\end{document}
