\documentclass[12pt,a4paper,openany]{book}
\usepackage{amssymb}
\usepackage{fontspec}
\usepackage{amsmath}
\usepackage{amsfonts}
\usepackage{harpoon}
\usepackage{extarrows}
\usepackage{hhtensor}
\usepackage{esvect}
\usepackage{bm}
\usepackage{mathabx}
%\usepackage{accents}
\usepackage[bf,small,center,indentafter,pagestyles]{titlesec}
\usepackage{subfigure}

\setromanfont{SimSun}
\setmainfont{SimSun}
%\setCJKmainfont[BoldFont=LiHei Pro]{KaiTi_GB2312}

\XeTeXlinebreaklocale "zh"
\XeTeXlinebreakskip = 0pt plus 1pt
\newtheorem{example}{例} 
\newtheorem{theorem}{定理}[section]
\newtheorem{definition}{定义}[section]
\newtheorem{axiom}{公理}[section]
\newtheorem{property}{性质}[section]
\newtheorem{proposition}{命题}[section]
\newtheorem{lemma}{引理}[section]
\newtheorem{corollary}{推论}[section]
\newtheorem{remark}{注解}
\newtheorem{condition}{条件}
\newtheorem{conclusion}{结论}
\newtheorem{assumption}{假设}

\newcommand\relphantom[1]{\mathrel{\phantom{#1}}}
\newcommand\num[1]{\left\Vert{#1}\right\Vert}
\newcommand\Hom{\text{Hom\,}}
\newcommand\Tr{\text{Tr\,}}
\newcommand\Ker{\text{Ker\,}}
\newcommand\Image{\text{Im\,}}
\newcommand\rank{\text{rank\,}}
\newcommand\tr{\text{tr\,}}

\title{数学专业研究生考试试题}
\author{虞朝阳}

\begin{document}
\renewcommand{\contentsname}{目\quad{}录}
\renewcommand{\chaptername}{}
%\renewcommand\thechapter{{第~\arabic{chapter}~章}}

\renewcommand{\chaptername}{第~\thechapter~章}

\titleformat{\chapter}[hang]{\centering\LARGE\bfseries}{\chaptername}{1em}{}

%\newcommand\prechaptername{第}
%\newcommand\postchaptername{章}
%\renewcommand\thechapter{{\prechaptername \value{chapter} \postchaptername}}
%\renewcommand\thechapter{{第~\arabic{chapter}~章\ #1}}
%\renewcommand\chapter{\thechapter}

\frontmatter
\begin{titlepage}
\maketitle
\end{titlepage}
\setcounter{page}{0}
\chapter{前言}
本文档是我从网络(主要是博士家园)上收集的各个学校的研究生考试数学方面的试题. 这里无法保证每一道题目的准确性, 甚至有一些题目还是不完整的.
\tableofcontents

\mainmatter
\chapter{北京大学硕士研究生入学试题}

\section{1995年}
\subsection{解析几何与高等代数}
\begin{enumerate}
\item 在空间仿射坐标系中, 直线$L_1, L_2$有方程
\[
\begin{array}{cc}
L_1: \left\{
\begin{aligned}
&x + 2y - z + 1 = 0, \\
&x - 4y - z - 2 = 0,
\end{aligned}
\right. 
&
L_2: \left\{
\begin{aligned}
&x - y + z - 2 = 0, \\
&4x - 2y + 1 = 0,
\end{aligned}
\right.
\end{array}
\]
\begin{enumerate}
\item 设直线$L$过原点$O$, 并且与$L_1, L_2$都相交, 求$L$的方程(普通方程或标准方程);
\item 设平面$\pi$过$L_1$, 并且与$L_2$平行, 求$\pi$的方程.
\end{enumerate}

\item 设直角坐标系中, 一圆柱面的轴线$L$有方程
\[
\frac{x-1}{2} = \frac{y}{-1} = \frac{z+1}{2},
\]
并且点$P(1, 0, 1)$在这个圆柱面上, 求这个圆柱面的方程.

\item \begin{enumerate}
\item 设$\alpha_1, \alpha_2, \alpha_3 \in R^n$, 证明: 如果向量组$\alpha_1, \alpha_2, \alpha_3$线性无关, 则向量组$\alpha_1+2\alpha_2, -\alpha_2+3\alpha_3, 4\alpha_1-\alpha_3$也线性无关.

\item 设
\[
A = \begin{pmatrix}
-1 & 5 & 3 & 1 \\
4 & 1 & -2 & 7 \\
0 & 3 & 4 & -1 \\
2 & 0 & -1 & 3
\end{pmatrix}
\]
求$A$的秩以及$A$的列向量组的一个极大线性无关组.
\end{enumerate}

\item 用正交线性替换把下述二次型化成标准型, 并且写出所作的正交线性替换:
\[
f(x_1, x_2, x_3) = 2x_1^2 + 5x_2^2 + 5x_3^2 +4x_1x_2 - 4x_1x_3 - 8x_2x_3.
\]

\item 设$V$是数域$K$上所有2级矩阵组成的线性空间, 定义$V$的一个变换$\underline{A}$如下: 
\[
\underline{A}(X) = 
\begin{pmatrix}
1 & 1 \\
1 & 2
\end{pmatrix}
X
\begin{pmatrix}
1 & 1 \\
1 & 2
\end{pmatrix}^{-1}, \quad \forall X \in V
\]
\begin{enumerate}
\item 证明$\underline{A}$是$V$的一个线性变换;
\item 求$\underline{A}$在$V$的一组基$E_{11}, E_{12}, E_{21}, E_{22}$下的矩阵, 其中$E_{ij}$表示$(i, j)$元为1, 其余元全为零的2级矩阵.
\end{enumerate}

\item 设$A$是一个$n$级正定矩阵, 证明: 存在一个$n$级正定矩阵$B$, 使得$A = B^2$.

\item 设$A$是数域$K$上$n$维线性空间$V$的一个线性变换, 设$f(x)$是数域$K$上的一个一元多项式, 并且$f(x) = f_1(x)f_2(x)$, 其中$f_1(x)$与$f_2(x)$互素, 用$ker{B}$表示线性变换$B$的核. 证明:
\[
\ker{f(A)} = \ker{f_1(A)} \oplus \ker{f_2(A)}.
\]
\end{enumerate}

\subsection{数学分析}
\begin{enumerate}
\item 求下列极限:
\begin{enumerate}
\item $\displaystyle\lim\limits_{x \rightarrow 0}{\frac{\tan{x} - \sin{x}}{x^3}}$;
\item $\displaystyle\lim\limits_{n \rightarrow +\infty}{\frac{1}{n}\sqrt[n]{n(n+1)\cdots(2n-1)}}$.
\end{enumerate}

\item \begin{enumerate}
\item 设$f(x)$在$[0, +\infty)$上可微, 且满足
\[
\int_{0}^{x}{tf(t)dt} = \frac{x}{3}\int_{0}^{x}{f(t)dt}, \quad x > 0,
\]
求$f(x)$.

\item 设$z = f(x, y)$是二次可微函数, 又有关系式:
\[
u = x + ay, \quad v = x - ay. \qquad (a\text{是常数})
\]
证明
\[
a^2\frac{\partial^2{z}}{\partial{x^2}} - \frac{\partial^2{z}}{\partial{y^2}} = 4a^2\frac{\partial^2{z}}{\partial{u}\partial{v}}.
\]
\end{enumerate}

\item \begin{enumerate}
\item 设$f(x)$是$[1, +\infty)$上的可微函数, 且当$x \rightarrow +\infty$时, $f(x)$是单调下降趋于零. 若积分
\[
\int_{1}^{+\infty}{f(x)dx}
\]
收敛, 证明积分$\displaystyle\int_{1}^{+\infty}{xf'(x)dx}$收敛.

\item 判别级数
\[
\sum_{n=2}^{+\infty}{\frac{\sin{nx}}{\log{n}}}
\]
的敛散性.
\end{enumerate}

\item \begin{enumerate}
\item 设$f(x, y)$是$R^2$上的连续函数, 试交换累次积分
\[
\int_{-1}^{1}{dx}\int_{x^2+x}^{x+1}{f(x, y)dy}
\]
的求积次序.

\item 求线积分
\[
\]
在下列两种曲线$C$的情形下的值.
\begin{enumerate}
\item $C: (x - 1)^2 + (y - 1)^2 = 1$, 逆时针方向;
\item $C: |x| + |y| = 1$, 逆时针方向.
\end{enumerate}
\end{enumerate}

\end{enumerate}

\section{1996年}
\subsection{解析几何与高等代数}
\begin{enumerate}
\item 在仿射坐标系中, 求过点$M_0(0, 0, -2)$, 与平面$\pi_1: 3x - y + 2z - 1 = 0$平行, 且与直线$l_1$:
\[
\frac{x-1}{4} = \frac{y-3}{-2} = \frac{z}{1}
\]
相交的直线$l$的方程.

\item 作直角坐标变换, 把下述二次曲面方程化成标准方程, 并且指出它是什么曲面:
\[
x^2 + 4y^2 + z^2 - 4xy - 8xz - 4yz + 2x + y + 2z - \frac{25}{16} = 0.
\]

\item 设线性空间$V$中的向量组$\alpha_1, \alpha_2, \alpha_3, \alpha_4$线性无关.
\begin{enumerate}
\item 试问: 向量组$\alpha_1 + \alpha_2, \alpha_2 + \alpha_3, \alpha_3 + \alpha_4, \alpha_4 + \alpha_1$是否线性无关? 要求说明理由.
\item 求向量组$\alpha_1 + \alpha_2, \alpha_2 + \alpha_3, \alpha_3 + \alpha_4, \alpha_4 + \alpha_1$生成的线性子空间$W$的一个基以及$W$的维数.
\end{enumerate}

\item 设$V$是数域$K$上的$n$维线性空间, 并且$V \ U \oplus W$. 任给$\alpha \in V$, 设$\alpha = \alpha_1 + \alpha_2$, $\alpha_1 \in U, \alpha_2 \in W$, 令
\[
P(\alpha) = \alpha_1.
\]
证明:
\begin{enumerate}
\item $P$是$V$上的线性变换, 并且$P^2 = P$;

\item $P$的核$\ker{P} = W$, $P$的象(值域)$\text{Im\,}P = U$;

\item $V$中存在一个基, 使得$P$在这个基下的矩阵是
\[
\begin{pmatrix}
I_r & 0 \\
0 & 0
\end{pmatrix}
\]
其中$I_r$表示$r$级单位矩阵; 请指出$r$等于什么?
\end{enumerate}

\item $n$级矩阵$A$称为周期矩阵, 如果存在正整数$m$, 使$A^m = I$, 其中$I$是单位矩阵. 证明: 复数域上的周期矩阵一定可以对角化.

\item 用$R[x]_4$表示实数域上次数小于4的一元多项式组成的集合, 它是一个欧几里德空间, 其上的内积:
\[
(f, g) = \int_{0}^{1}{f(x)g(x)dx}.
\]
设$W$是由零次多项式组成的子空间, 求$W^{\perp}$以及它的一个基.

\end{enumerate}

\subsection{数学分析}
\begin{enumerate}
\item 判断下列命题的真伪, 不必说明理由.
\begin{enumerate}
\item 对数列$\{a_n\}$作和$s_n = \sum\limits_{k=1}^{n}{a_k}$, 若$\{s_n\}$是有界数列, 则$\{a_n\}$是有界列.

\item 数列$\{a_n\}$存在极限$\lim\limits_{n \rightarrow +\infty}{a_n} = a$的充要条件是: 对任一自然数$p$, 都有
\[
\lim_{n \rightarrow }{|a_{n+p} - a_n|} = 0.
\]

\item 设$f(x)$是$[a, +\infty)$上的递增连续函数, 若$f(x)$在$[a, +\infty)$上有界, 则$f(x)$在$[a, +\infty)$上一致连续.

\item 设$f(x)$在$[a, b]$上连续, 且在$(a, b)$上可微, 若存在极限
\[
\lim_{x \rightarrow a+0}{f'(x)} = l,
\]
则右导数$f_+'(a)$存在且等于$l.$

\item 若$f(x)$是$[a, +\infty)$上的非负连续函数, 且积分
\[
\int_{a}^{+\infty}{f(x)dx}
\]
收敛, 则$\lim\limits_{x \rightarrow +\infty}{f(x)} = 0$.
\end{enumerate}

\item 设$f(x)$在$x = a$处可微, $f(a) \neq 0$, 求极限
\[
\lim_{n \rightarrow +\infty}{\Bigl{(} \frac{f(a + \frac{1}{n})}{f(a)} \Bigr{)}^n}.
\]

\item 
\begin{enumerate}
\item 求幂级数
\[
\sum_{n=1}^{\infty}{nx^{n-1}} \qquad (|x| < 1)
\]
的和.

\item 求级数$\sum\limits_{n=1}^{\infty}{\displaystyle\frac{2n}{3^n}}$的和.
\end{enumerate}

\item 求积分
\[
I = \iiint\limits_{D}{(x + y + z)dxdydz},
\]
的值, 其中$D$是由平面$x + y + z = 1$以及三个坐标平面所围成的区域.

\item 设$a_n \neq 0\,(n=1,2,\cdots)$, 且$\lim\limits_{n \rightarrow +\infty}{a_n} = 0$. 若存在极限
\[
\lim_{n \rightarrow +\infty}{\frac{a_{n+1}}{a_n}} = l,
\]
证明: $|l| \le 1$.

\item 设在$[a, b]$上, $f_n(x)$一致收敛于$f(x)$, $g_n(x)$一致收敛于$g(x)$. 若存在正数列$\{M_n\}$, 使得
\[
|f_n(x)| \le M_n, |g_n(x)| \le M_n \quad (x \in [a, b], n = 1, 2, \cdots)
\]
证明$f_n(x)g_n(x)$在$[a, b]$上一致收敛于$f(x)g(x)$.
\end{enumerate}

\section{1997年}
\subsection{解析几何与高等代数}
\begin{enumerate}
\item 判断下列二次曲线类型
\begin{enumerate}
\item $x^2 - 3xy + y^2 + 10x - 10y + 21 = 0$;
\item $x^2 + 4xy + 4y^2 - 20x + 10y - 50 = 0$.
\end{enumerate}

\item 过$x$轴和$y$轴分别作动平面, 交角$\alpha$是常数, 求交线轨迹的方程, 并证明它是一个锥面.

\item 设$A, B$是数域$K$上的$n$阶方阵, $X$是未知量$x_1, \cdots, x_n$所成的$n \times 1$矩阵. 已知齐次线性方程组$AX = 0$和$BX = 0$分别有$l$, $m$个线性无关解向量, 这里$l \ge 0$, $m \ge 0$.
\begin{enumerate}
\item 证明$(AB)X = 0$至少有$\max(l, m)$个线性无关解向量.
\item 如果$l + m > n$, 证明$(A + B)X = 0$必有非零解.
\item 如果$AX = 0$和$BX = 0$无公共非零解向量, 且$l + m = n$, 证明$R^n$中任一向量$\alpha$可唯一表成$\alpha = \beta + \gamma$, 这里$\beta, \gamma$分别是$AX = 0$和$BX = 0$的解向量.
\end{enumerate}

\item 设$A$是实数域$R$上的3维线性空间$V$内的一个线性变换, 对$V$的一组基$\epsilon_1, \epsilon_2, \epsilon_3$, 有
\[
\begin{aligned}
A\epsilon_1 &= 3\epsilon_1 + 6\epsilon_2 + 6\epsilon_3, \\
A\epsilon_2 &= 4\epsilon_1 + 3\epsilon_2 + 4\epsilon_3, \\
A\epsilon_3 &= -5\epsilon_1 - 4\epsilon_2 - 6\epsilon_3.
\end{aligned}
\]
\begin{enumerate}
\item 求$A$的全部特征值和特征向量.
\item 设$B = A^3 - 5A$, 求$B$的一个非平凡的不变子空间.
\end{enumerate}

\item 设$f(x)$是有理数域$Q$上的一个$m$次多项式$(m \ge 0)$, $n$是大于$m$的正整数, 证明: $\sqrt[n]{2}$不是$f(x)$的实根.

\item 设$A$是$n$维欧氏空间$V$内的一个线性空间, 满足
\[
(A\alpha, \beta) = - (\alpha, A\beta), \quad (\forall \alpha, \beta \in V)
\]
\begin{enumerate}
\item 若$\lambda$是$A$的一个特征值, 证明$\lambda = 0$.
\item 证明$V$内存在一组标准正交基, 使$A^2$在此组基下的矩阵为对角矩阵.
\item 设$A$在$V$的某组标准正交基下的矩阵为$A$, 证明: 把$A$看作复数域$C$上的$n$阶方阵, 其特征值必为0或纯虚数.
\end{enumerate}

\end{enumerate}

\subsection{数学分析}
\begin{enumerate}
\item 将函数$f(x) = \arctan{\displaystyle\frac{2x}{1 - x^2}}$, 在$x = 0$点展开为幂级数, 并指出收敛区间.

\item 判别广义积分的收敛性:
\[
\int_{0}^{+\infty}{\frac{\ln(1 + x)}{x^p}dx}.
\]

\item 设$f(x)$在$(-\infty, +\infty)$上有任意阶导数$f^{(n)}(x)$, 且对任意有限闭区间$[a, b]$, $f^{(n)}(x)$在$[a, b]$上一致收敛于$\phi(x)$($n \rightarrow +\infty$), 求证: $\phi(x) = ce^x$, $c$为常数.

\item 设$x_n > 0\,(n = 1, 2, \cdots)$及$\lim\limits_{n \rightarrow +\infty}{x_n} = a$. 用$\epsilon-N$语言证明$\lim\limits_{n \rightarrow +\infty}{\sqrt{x_n}} = \sqrt{a}$.

\item 求第二型曲面积分
\[
\oiint\limits_{S}{(xdydz + \cos{y}dzdx + dxdy)}
\]
其中$S$为$x^2 + y^2 + z^2 = 1$的外侧.

\item 设$x = f(u, v), y = g(u, v)$, $w = w(x, y)$有二阶连续偏导数, 满足
\[
\frac{\partial{f}}{\partial{u}} = \frac{\partial{g}}{\partial{v}}, \frac{\partial{f}}{\partial{v}} = -\frac{\partial{g}}{\partial{u}}, \frac{\partial^2{w}}{\partial{x^2}} + \frac{\partial^2{w}}{\partial{y^2}} = 0,
\]
证明:
\begin{enumerate}
\item $\displaystyle\frac{\partial^2(fg)}{\partial{u^2}} + \frac{\partial^2(fg)}{\partial{v^2}} = 0$;
\item $w(u, v) = w(f(u, v), g(u, v))$满足$\displaystyle\frac{\partial^2{w}}{\partial{u^2}} + \frac{\partial^2{w}}{\partial{v^2}} = 0$.
\end{enumerate}

\item 计算三重积分
\[
\iiint\limits_{\varOmega: x^2 + y^2 + z^2 \le 2x}{(x^2 + y^2 + z^2)^{5/2}dxdydz}.
\]

\end{enumerate}

\section{1998年}
\subsection{解析几何与高等代数}
\begin{enumerate}
\item 设在直角坐标系中给出了两条互相异面的直线$l_1, l_2$的普通方程:
\[
\begin{array}{cc}
l1: \left\{
\begin{aligned}
&x + y + z - 1 = 0 \\
&x + y + 2z + 1 = 0
\end{aligned}
\begin{aligned}
&3x + y + 1 = 0 \\
&y + 3z + 2 = 0
\end{aligned}
\right.
&
l2: \left\{
\right.
\end{array}
\]
\begin{enumerate}
\item 过$l_1$作平面$\pi$, 使$\pi$与$l_2$平行;
\item 求$l_1$与$l_2$间的距离;
\item 求$l_1$与$l_2$的公垂线的方程.
\end{enumerate}

\item 在直角坐标系中, 球面的方程为:
\[
(x - 1)^ + y^2 + (z + 1)^2 = 4,
\]
求所有与向量$u(1, 1, 1)$平行的球面的切线所构成的曲面的方程.

\item 讨论$a, b$满足什么条件时, 数域$K$上的下述线性方程组有唯一解, 有无穷多个解, 无解? 当有解时, 求出该方程组全部解.
\[
\left\{
\begin{aligned}
ax_1 + 3x_2 + 3x_3 &= 3 \\
x_1 + 4x_2 + x_3 &= 1 \\
2x_1 + 2x_2 + bx_3 &= 2
\end{aligned}
\right.
\]

\item 设$V$是定义域为实数集$R$的所有实值函数组成的集合, 对于$f, g \in V$, $a \in R$, 分别用下列式子定义$f + g$与$af$:
\[
(f + g)(x) = f(x) + g(x), \quad (af)(x) = a(f(x)), \quad \forall x \in R
\]
则$V$成为实数域上的一个线性空间.

设$f_0(x) = 1$, $f_1(x) = \cos{x}$, $f_2(x) = \cos{2x}$, $f_3(x) = \cos{3x}$,

\begin{enumerate}
\item 判断$f_0, f_1, f_2, f_3$是否线性无关, 写出理由;
\item 用$<f, g>$表示$f, g$生成的线性子空间, 判断$<f_0, f_1> + <f_2, f_3>$是否为直和, 写出理由.
\end{enumerate}

\item 用$J$表示元素全为1的$n$级矩阵, $n \ge 2$. 设$f(x) = a + bx$是有理数域上的一元多项式, 令$A = f(J)$.
\begin{enumerate}
\item 求$J$的全部特征值和全部特征向量;
\item 求$A$的所有特征子空间;
\item $A$是否可以对角化? 如果可对角化, 求出有理数域上的一个可逆矩阵$P$, 使得$P^{-1}AP$为对角矩阵, 并且写出这个对角矩阵.
\end{enumerate}

\item 用$M_2(C)$表示复数域$C$上所有2级矩阵组成的集合. 令
\[
V = \{ A \in M_2(C) | \Tr(A) = 0, \text{且}A^* = A \},
\]
其中$\Tr(A)$表示$A$的迹, $A^*$表示$A$的转置共轭矩阵.
\begin{enumerate}
\item 证明$V$对于矩阵的加法, 以及实数与矩阵的数量乘法成为实数域上的线性空间, 并且说明$V$中元素形如
\[
\begin{pmatrix}
a_1 & a_2 + ia_3 \\
a_2 - ia_3 & - a_1
\end{pmatrix},
\]
其中$a_1, a_2, a_3$都是实数, $i = \sqrt{-1}$.

\item 设
\[
A = 
\begin{pmatrix}
a_1 & a_2 + ia_3 \\
a_2 - ia_3 & - a_1
\end{pmatrix}, \quad 
B = 
\begin{pmatrix}
b_1 & b_2 + ib_3 \\
b_2 - ib_3 & -b_1
\end{pmatrix}
\]
考虑$V$上的一个二元函数:
\[
(A, B) = a_1b_1 + a_2b_2 + a_3b_3,
\]
证明这个二元函数是$V$上的一个内积, 从而$V$成为欧几里德空间; 并且求出$V$的一个标准正交基, 要求写出理由.

\item 设$T$是一个酉矩阵(即, $T$满足$T^*T = I$, 其中$I$是单位矩阵), 对任意$A \in V$, 规定$\varPsi_T(A) = TAT^{-1}$, 证明$\varPsi_T$是$V$上的正交变换.

\item $\varPsi_T$的意义同第(3)小题, 求下述集合
\[
S = \{ T | \det{T} = 1, \text{且} \varPsi_T = 1_V \},
\]
其中$\det{T}$表示$T$的行列式, $1_V$表示$V$上的恒等变换.

\end{enumerate}
\end{enumerate}

\subsection{数学分析}
\begin{enumerate}
\item 选一个最确切的答案, 填入括号中.
\begin{enumerate}
\item 设$f(x)$定义在$[a, b]$上. 若对任意的$g \in R([a, b])$, 有$f \cdot g \in R([a, b])$, 则 \hfill(\ )
\begin{enumerate}
\item $f \in R([a, b])$,
\item $f \in C([a, b])$,
\item $f$可微,
\item $f$可导.
\end{enumerate}

\item 设$f \in C((a, b))$. 若存在
\[
\lim_{x \rightarrow a+}{f(x)} = 1, \quad \lim_{x \rightarrow b-}{f(x)} = 2,
\]
则 \hfill (\ )
\begin{enumerate}
\item $f(x)$在$[a, b]$一致连续,
\item $f(x)$在$[a, b]$连续,
\item $f(x)$在$(a, b)$一致连续,
\item $f(x)$在$(a, b)$可微.
\end{enumerate}

\item 若反常(广义)积分$\displaystyle\int_{0}^{1}{f(x)dx}$, $\displaystyle\int_{0}^{1}{g(x)dx}$都存在, 则反常积分
\[
\int_{0}^{1}{f(x)g(x)dx}
\]
\hfill (\ )
\begin{enumerate}
\item 收敛,
\item 发散,
\item 不一定收敛,
\item 一定不收敛.
\end{enumerate}

\item 若$\lim\limits_{n \rightarrow +\infty}{na_n} = 1$, 则$\sum\limits_{n=1}^{+\infty}{a_n}$ \hfill (\ )
\begin{enumerate}
\item 发散,
\item 收敛,
\item 不一定收敛,
\item 绝对收敛.
\end{enumerate}

\item 设$f(x, y)$在区域$\{ (x, y) : x^2 + y^2 < 1 \}$上有定义, 若存在偏导数
\[
f_x'(0, 0) = 0 = f_y'(0, 0).
\]
则$f(x, y)$ \hfill (\ )
\begin{enumerate}
\item 在点$(0, 0)$处连续,
\item 在点$(0, 0)$处可微,
\item 在点$(0, 0)$处不一定连续,
\item 在点$(0, 0)$处不可微.
\end{enumerate}
\end{enumerate}

\item 计算下列极限(写出演算过程)
\begin{enumerate}
\item $\lim\limits_{n \rightarrow \infty}{\sqrt[n]{1 + a^n}} \quad (a > 0)$;
\item $\lim\limits_{x \rightarrow 0}{(\displaystyle\frac{1}{x^2} - \frac{\cot{x}}{x}})$;
\item $\lim\limits_{x \rightarrow 0+}{\displaystyle\sum\limits_{n=1}^{\infty}{\frac{1}{2^nn^x}}}$.
\end{enumerate}

\item 求下列积分值.
\begin{enumerate}
\item $\displaystyle\iint\limits_{S}{x^3dydz + x^2ydzdx + x^2zdxdy} \qquad S: z = 0, z = b, x^2 + y^2 = a^2$;
\item $\displaystyle\int_{C}{\frac{1}{y}dx + \frac{1}{x}dy} \qquad C: y = 1, x= 4, y = \sqrt{x}$逆时针一周.
\end{enumerate}

\item 解答下列问题
\begin{enumerate}
\item 求幂级数$\displaystyle\sum\limits_{n=1}^{\infty}{\frac{(-1)^n}{n!}(\frac{n}{e})^nx^n}$的收敛半径.
\item 求级数$\displaystyle\sum\limits_{n=0}^{\infty}{\frac{2^n(n+1)}{n!}}$的和.
\end{enumerate}

\item 试证明下列命题:
\begin{enumerate}
\item 反常积分$\displaystyle\int_{0}^{+\infty}{\frac{\sin{x^2}}{1 + x^p}dx}$($p \ge 0$)是收敛的.
\item 设$f(x, y)$在$G = \{ (x, y): x^2 + y^2 < 1\}$上有定义. 若$f(x, 0)$在$x = 0$处连续, 且$f_y'(x, y)$在$G$上有界, 则$f(x, y)$在$(0, 0)$处连续.
\end{enumerate}
\end{enumerate}

\section{1999年}
\subsection{解析几何与高等代数}
\begin{enumerate}
\item 在仿射坐标系中, 已知直线$l_1, l_2$的方程分别是
\[
\frac{x+13}{2} = \frac{y-5}{3} = \frac{z}{1}, \quad  \frac{x-10}{5} = \frac{y+7}{4} = \frac{z}{1}.
\]
\begin{enumerate}
\item 判断直线$l_1$与$l_2$的位置关系s, 要求写出理由;
\item 设直线$l$的一个方向向量为$\vec{v}(8, 7, 1)$, 并且$l$与$l_1$和$l_2$都相交, 求直线$l$的方程.
\end{enumerate}

\item 在直角坐标系$Oxyz$中, 设顶点在原点的二次锥面$S$的方程为
\[
a_{11}x^2 + a_{22}y^2 + a_{33}z^2 + 2a_{12}xy + 2a_{13}xz + 2a_{23}yz = 0
\]
\begin{enumerate}
\item 如果三条坐标轴都是$S$的母线, 求$a_{11}, a_{22}, a_{33}$;
\item 证明: 如果$S$有三条互相垂直的直母线, 则
\[
a_{11} + a_{22} + a_{33} = 0.
\]
\end{enumerate}

\item 设实数域上的矩阵$A$为
\[
A = \begin{pmatrix}
1 & 1 & 0 \\
-1 & 0 & 1 \\
-3 & 0 & 0
\end{pmatrix}
\]
\begin{enumerate}
\item 求$A$的特征多项式$f(\lambda)$;
\item $f(\lambda)$是否为实数域上的不可约多项式;
\item 求$A$的最小多项式, 要求写出理由;
\item 实数域上的矩阵$A$是否可对角化, 要求写出理由.
\end{enumerate}

\item 设实数域上的矩阵$A$为
\[
A = \begin{pmatrix}
1 & 0 & 1 \\
0 & 6 & -2 \\
1 & -2 & 2
\end{pmatrix}
\]
\begin{enumerate}
\item 判断$A$是否为正定矩阵, 要求写出理由;
\item 设$V$是实数域上的3维线性空间, $V$上的一个双线性函数$f(\alpha, \beta)$在$V$的一个基$\alpha_1, \alpha_2, \alpha_3$下的度量矩阵为$A$. 证明$f(\alpha, \beta)$是$V$的一个内积; 并且求出$V$对于这个内积所成的欧氏空间的一个标准正交基.
\end{enumerate}

\item 设$V$是数域$K$上的一个$n$维线性空间, $\alpha_1, \alpha_2, \cdots, \alpha_n$是$V$的一个基, 用$V_1$表示由$\alpha_1 + \alpha_2 + \cdots + \alpha_n$生成的线性子空间; 令
\[
V_2 = \{ \sum_{i=1}^{n}{k_i\alpha_i} | \sum_{i=1}^{n}{k_i} = 0, k_i \in K \}.
\]
\begin{enumerate}
\item 证明$V_2$是$V$的子空间;
\item 证明$V = V_1 \oplus V_2$;
\item 设$V$上的一个线性变换$A$在基$\alpha_1, \alpha_2, \cdots, \alpha_n$下的矩阵$A$是置换矩阵(即, $A$的每一行与每一列都只有一个元素是1, 其余元素全为0), 证明$V_1$与$V_2$都是$A$的不变子空间.
\end{enumerate}

\item 设$V$和$U$分别是数域$K$上的$n$维, $m$维线性空间, $A$是$V$到$U$的一个线性映射, 即$A$是$V$到$U$的映射, 且满足
\[
\begin{aligned}
A(\alpha + \beta) &= A\alpha + A\beta, \quad \forall \alpha, \beta \in V, \\
A(k\alpha) &= kA\alpha, \quad \forall \alpha \in V, k \in K,
\end{aligned}
\]
令
\[
\Ker{A} := \{ \alpha \in V | A\alpha = 0 \},
\]
称$\Ker{A}$是$A$的核, 它是$V$的一个子空间, 用$\Image{A}$表示$A$的象(即值域).
\begin{enumerate}
\item 证明: $\dim{(\Ker{A})} + \dim{(\Image{A})} = \dim{V}$;
\item 证明: 如果$\dim{V} = \dim{U}$, 则$A$是单射当且仅当$A$是满射.
\end{enumerate}

\item 设$V$是实数域$R$上的$n$维线性空间. $V$上的所有复值函数组成集合, 对于函数的加法以及复数与函数的数量乘法, 形成复数域$C$上的一个线性空间, 记作$C^V$.

证明: 如果$f_1, f_2, \cdots, f_{n+1}$是$C^V$中$n+1$个不同的函数, 并且它们满足
\[
\begin{aligned}
f_i(\alpha + \beta) &= f_i(\alpha) + f_i(\beta), \quad \alpha, \beta \in V, \\
f_i(k\alpha) &= kf_i(\alpha), \quad \forall k \in R, \alpha \in V.
\end{aligned}
\]
则$f_1, f_2, \cdots, f_{n+1}$是$C^V$中的线性相关的向量组.
\end{enumerate}

\subsection{数学分析}
\begin{enumerate}
\item 判断下列命题的真伪:
\begin{enumerate}
\item 设$\{a_n\}$是一个数列. 若在任一子列$\{a_{n_k}\}$中均存在收敛子列$\{a_{n_{k_i}}\}$, 则$\{a_n\}$必为收敛列.

\item 设$f \in C((a, b))$. 若存在
\[
\lim_{x \rightarrow a+}{f(x)} = A < 0, \quad \lim_{x \rightarrow b-}{f(x)} = B > 0,
\]
则必存在$\xi \in (a, b)$, 使得$f(\xi) = 0$.

\item 设$f(x)$在$[a, b]$上有界. 若对任意的$\delta > 0$, $f(x)$在$[a + \delta, b]$上可积. 则$f(x)$在$[a, b]$上可积.

\item 设$f(x)$, $g(x)$在$[0, 1]$上的瑕积分均存在, 则乘积$f(x) \cdot g(x)$在$[0, 1]$上的瑕积分必存在.

\item 设级数$\sum\limits_{n=1}^{\infty}{b_n}$收敛, 若有$a_n \le b_n$, ($n = 1, 2, \cdots$), 则级数$\sum\limits_{n=1}^{\infty}{a_n}$必收敛.
\end{enumerate}

\item 求下列极限(写出计算过程).
\begin{enumerate}
\item $\lim\limits_{x \rightarrow 0}{\displaystyle\frac{a\tan{x} + b(1 - \cos{x})}{\alpha\log(1 - x) + \beta(1 - e^{-x^2})}}, \quad (a^2 + \alpha^2 \neq 0)$;

\item $\lim\limits_{n \rightarrow \infty}{(\displaystyle\frac{\sin\frac{\pi}{n}}{n + 1} + \frac{\sin\frac{2\pi}{n}}{n + \frac{1}{2}} + \cdots + \frac{\sin\pi}{n + \frac{1}{n}})}$;

\item $\lim\limits_{n \rightarrow \infty}{\displaystyle\int_{0}^{1}{(1 - x^2)^ndx}}$;

\item $\lim\limits_{n \rightarrow \infty}{\sqrt[n]{1 + a^n}}, \quad (a > 0)$.
\end{enumerate}

\item 求解下列命题.
\begin{enumerate}
\item 求级数$\displaystyle\sum\limits_{n=0}^{\infty}{\frac{n}{3^n}2^n}$之和;

\item 设$f \in C([0, 1])$, 且在$(0, 1)$上可微, 若有$8\displaystyle\int_{7/8}^{1}{f(x)dx} = f(0)$, 证明: 存在$\xi \in (0, 1)$, 使得$f'(\xi) = 0$.

\item 证明: 级数$\sum\limits_{n=1}^{\infty}{(-1)^n\frac{\arctan{n}}{\sqrt{n}}}$收敛.

\item 证明: 积分$\displaystyle\int_{0}^{+\infty}{xe^{-xy}dy}$在$(0, +\infty)$上不一致收敛.

\item 设$u = f(x, y, z)$, $g(x^2, e^y, z) = 0$, $y = \sin{x}$, 且已知$f$与$g$都有一阶连续偏导数, $\displaystyle\frac{\partial{g}}{\partial{z}} \neq 0$. 求$\displaystyle\frac{\partial{u}}{\partial{x}}$.

\item 设$f(x)$在$[-1, 1]$上二次连续可微, 且有
\[
\lim_{x \rightarrow 0}{\frac{f(x)}{x}} = 0,
\]
证明: 级数$\displaystyle\sum\limits_{n=1}^{\infty}{f(\frac{1}{n})}$绝对收敛.
\end{enumerate}

\end{enumerate}

\section{2000年}
\subsection{解析几何与高等代数}
\begin{enumerate}
\item 
\begin{enumerate}
\item 在直角坐标系中, 一个柱面的准线方程为
\[
\left\{
\begin{aligned}
xy &= 4, \\
z &= 0,
\end{aligned}
\right.
\]
母线方向为$(1, -1, 1)$, 求这个柱面的方程.

\item 在平面直角坐标系$Oxy$中, 二次曲线的方程为
\[
x^2 - 3xy + y^2 + 10x - 10y + 21 = 0,
\]
求$I_1, I_2, I_3$; 指出这是什么二次曲线, 并且确定其形状.
\end{enumerate}

\item 
\begin{enumerate}
\item 设实数域上的矩阵
\[
A = \begin{pmatrix}
2 & 0 & 4 \\
0 & 6 & 0 \\
4 & 0 & 2
\end{pmatrix},
\]
求正交矩阵$T$, 使得$T^{-1}AT$为对角矩阵, 并且写出这个对角矩阵.

\item 在直角坐标系$Oxyz$中, 二次曲面$S$的方程为
\[
2x^2 + 6y^2 + 2z^2 + 8xz = 1,
\]
作直角坐标变换, 把$S$的方程化成标准方程. 并且指出它是什么二次曲面.

\end{enumerate}

\item 设实数域上的$s \times n$矩阵$A$的元素只有0和1, 并且$A$的每一行元素的和是常数$r$, $A$的每两个行向量的内积为常数$m$, 其中$m < r$.
\begin{enumerate}
\item 求$|AA'|$;
\item 证明$s \le n$;
\item 证明$AA'$的特征值全为正实数.
\end{enumerate}

\item 设$V$是数域$K$上的$n$维线性空间, $A$是$V$上的线性变换, 且满足$A^3 - 7A = -6I$, 其中$I$表示$V$上的恒等变换. 判断$A$是否可对角化, 写出理由.

\item 设$V$和$V'$都是数域$K$上的有限维线性空间, $A$是$V$到$V'$的一个线性映射, 证明: 存在直和分解
\[
V = U \oplus W, \quad V' = M \oplus N,
\]
使得$\Ker{A} = U$, 并且$W \cong M$.

\item 设$f(x)$和$p(x)$都是首项系数为1的整系数多项式, 且$p(x)$在有理数域$Q$上不可约. 如果$p(x)$与$f(x)$有公共复根$\alpha$, 证明:
\begin{enumerate}
\item 在$Q[x]$中, $p(x)$整除$f(x)$;
\item 存在首项系数为1的整系数多项式$g(x)$, 使得
\[
f(x) = p(x)g(x).
\]
\end{enumerate}

\item 
\begin{enumerate}
\item 设$V$是实数域上的线性空间, $f$是$V$上的正定的对称双线性函数, $U$是$V$的有限维子空间, 证明
\[
V = U \oplus U^{\perp},
\]
其中$U^{\perp} = \{ \alpha \in V | f(\alpha, \beta) = 0, \forall \beta  \in U \}$.

\item 设$V$是数域$K$上的$n$维线性空间, $g$是$V$上的非退化的对称双线性函数, $W$是$V$的子空间, 令
\[
W^{\perp} = \{ \alpha \in V | g(\alpha, \beta) = 0, \forall \beta \in W \}.
\]
证明:
\begin{enumerate}
\item $\dim{V} = \dim{W} + \dim{W^{\perp}}$;
\item $(W^{\perp})^{\perp} = W$.
\end{enumerate}
\end{enumerate}

\end{enumerate}

\subsection{数学分析}
\begin{enumerate}
\item 计算:
\begin{enumerate}
\item 求极限
\[
\lim_{x \rightarrow 0}{\frac{(a + x)^x - a^x}{x^2}}, \quad a > 0.
\]

\item 求$e^{2x - x^2}$到含$x^3$项的Taylor展开式.

\item 求积分
\[
\int_{0}^{1}{\frac{x^b - x^a}{\ln{x}}dx},
\]
其中$a > b > 0$.

\item 求积分
\[
\iiint\limits_{V}{(x^2 + y^2 + z^2)^{\alpha}dxdydz}.
\]
$V$是实心球$x^2 + y^2 + z^2 \le R^2$, $\alpha > 0$.

\item 求积分
\[
\iint\limits_{S}{x^3dydz + y^3dxdz + z^3dxdy},
\]
$S$是$x^2 + y^2 + z^2 = a^2$的外表面.
\end{enumerate}

\item 叙述定义
\begin{enumerate}
\item $\lim\limits_{x \rightarrow -\infty}{f(x)} = +\infty$;
\item 当$x \rightarrow a-0$时, $f(x)$不以$A$为极限.
\end{enumerate}

\item 函数$f(x)$在$[a, b]$上一致连续, 又在$[b,c]$上一致连续, $a < b< c$, 用定义证明$f(x)$在$[a, c]$上一致连续.

\item 构造一个二元函数$f(x, y)$, 使得它在原点$(0, 0)$两个偏导数都存在, 但在原点不可微.

\item 函数$f(x)$在$[a, b]$连续, 证明不等式
\[
[\int_{a}^{b}{f(x)dx}]^2 \le (b - a)\int_{a}^{b}{f^2(x)dx}.
\]

\item 
\begin{enumerate}
\item 在区间$(0, 2\pi)$内展开$f(x)$的Fourier级数:
\[
f(x) = \frac{\pi - x}{2};
\]

\item 证明它的Fourier级数在$(0, 2\pi)$内每一点上收敛于$f(x)$.
\end{enumerate}
\end{enumerate}

\section{2001年}
\subsection{解析几何与高等代数}
\begin{enumerate}
\item 在空间直角坐标系中, 点$A,B,C$的坐标依次为:
\[
(-2, 1, 4), \quad (-2, -3, -4), \quad (-1, 3, 3)
\]
\begin{enumerate}
\item 求四面体$OABC$的体积;
\item 求三角形$ABC$的面积.
\end{enumerate}

\item 在空间直角坐标系中,
\[
l_1: \frac{x-a}{1} = \frac{y}{-2} = \frac{z}{3}
\]
与
\[
l_2: \frac{x}{2} = \frac{y-1}{1} = \frac{z}{-2}
\]
是一对相交直线.
\begin{enumerate}
\item 求$a$.
\item 求$l_2$绕$l_1$旋转出的曲面的方程.
\end{enumerate}

\item 设$\omega$是复数域$C$上的本原$n$次单位根(即$\omega^n = 1$, 而当$0 < l < n$时, $\omega^l \neq 1$), $s, b$都是正整数, 而且$s < n$, 令
\[
A = \begin{pmatrix}
1 & \omega^b & \omega^{2b} & \cdots & \omega^{(n-1)b} \\
1 & \omega^{b+1} & \omega^{2(b+1)} & \cdots & \omega^{(n-1)(b+1)} \\
\cdots & \cdots & \cdots & \cdots & \cdots \\
1 & \omega^{b+s-1} & \omega^{2(b+s-1)} & \cdots & \omega^{(n-1)(b+s-1)}
\end{pmatrix}
\]
任取$\beta \in C^s$, 判断线性方程组$AX = \beta$有无解? 有多少解? 写出理由.

\item 
\begin{enumerate}
\item 设
\[
A = \begin{pmatrix}
0 & 1 & 0 \\
0 & 0 & 1 \\
-2 & 3 & -1
\end{pmatrix}
\]
\begin{enumerate}
\item 若把$A$看成有理数域上的矩阵, 判断$A$是否可对角化, 写出理由;
\item 若把$A$看成复数域上的矩阵, 判断$A$是否可对角化, 写出理由.
\end{enumerate}

\item 设$A$是有理数域上的$n$级对称矩阵, 并且在有理数域上$A$合同于单位矩阵$I$. 用$\delta$表示元素全为1的列向量, $b$是有理数. 证明: 在有理数域上
\[
\begin{pmatrix}
A & b\delta \\
b\delta' & b
\end{pmatrix} \simeq
\begin{pmatrix}
I & 0 \\
0 & b - b^2\delta'A^{-1}\delta
\end{pmatrix}
\]
\end{enumerate}

\item 在实数域上的$n$维列向量空间$R^n$中, 定义内积为$(\alpha, \beta) = \alpha'\beta$, 从而$R^n$成为欧几里得空间.
\begin{enumerate}
\item 设实数域上的矩阵
\[
A = \begin{pmatrix}
1 & -3 & 5 & -2 \\
-2 & 1 & -3 & 1 \\
-1 & -7 & 9 & -4
\end{pmatrix}
\]
求齐次线性方程组$AX = 0$的解空间的一个正交基.

\item 设$A$是实数域$R$上的$s \times n$矩阵, 用$W$表示齐次线性方程组$AX = 0$的解空间, 用$U$表示$A'$的列空间(即$A'$的列向量组生成的子空间). 证明: $U = W^{\perp}$.
\end{enumerate}

\item 设$A$是数域$K$上$n$维线性空间$V$上的一个线性变换. 在$K[x]$中, $f(x) = f_1(x)f_2(x)$, 且$f_1(x)$与$f_2(x)$互素, 用$\Ker{A}$表示线性变换$A$的核. 证明:
\[
\Ker{f(A)} = \Ker{f_1(A)} \oplus \Ker{f_2(A)}.
\]

\item 设$A$是数域$K$上$n$维线性空间$V$上的一个线性变换, $I$是恒等变换. 证明: $A^2 = A$的充分必要条件是
\[
\rank(A) + \rank(A - I) = n.
\]

\end{enumerate}

\subsection{数学分析}
\begin{enumerate}
\item 求极限
\[
\lim_{n \rightarrow \infty}{\frac{a^{2n}}{1 + a^{2n}}}.
\]

\item 设$f(x)$在点$a$可导, $f(a) \neq 0$. 求极限
\[
\lim_{n \rightarrow \infty}{\Bigl{(} \frac{f(a + \frac{1}{n})}{f(a)} \Bigr{)}^n}.
\]

\item 证明函数$f(x) = \sqrt{x}\ln{x}$在$[1, +\infty]$上一致连续.

\item 设$D$是包含原点的平面凸区域, $f(x, y)$在$D$上可微,
\[
x\frac{\partial{f}}{\partial{x}} + y\frac{\partial{f}}{\partial{y}} = 0.
\]
证明: $f(x, y)$在$D$上恒为常数.

\item 计算第一型曲面积分
\[
\iint_{\varSigma}{xdS},
\]
其中$\varSigma$是锥面$z = \sqrt{x^2 + y^2}$被柱面$x^2 + y^2 = ax (a > 0)$割下的部分.

\item 求极限
\[
\lim_{t \rightarrow 0+}{\frac{1}{t^4}\iiint_{x^2 + y^2 + z^2 \le t^2}{f(\sqrt{x^2 + y^2 + z^2})dxdydz}},
\]
其中$f$在$[0, 1]$上连续, $f(0) = 0$, $f'(0) = 1$.

\item 求常数$\lambda$, 使得曲线积分
\[
\int_{L}{\frac{x}{y}r^{\lambda}dx - \frac{x^2}{y^2}r^{\lambda}dy} = 0 \qquad (r = \sqrt{x^2 + y^2})
\]
对上半平面的任何光滑闭曲线$L$成立.

\item 证明函数$f(x) = \sum\limits_{n=1}^{\infty}{\frac{1}{n^x}}$在$(1, \infty)$上无穷次可微.

\item 求广义积分
\[
\int_{0}^{\infty}{\frac{\arctan{bx^2} - \arctan{ax^2}}{x}dx}, \quad b > a > 0.
\]

\item 设$f(x)$是以$2\pi$为周期的周期函数, 且$f(x) = x$, $-\pi \le x < \pi$. 求$f(x)$与$|f(x)|$的Fourier级数. 它们的Fourier级数是否一致收敛(给出证明)?

\end{enumerate}

\section{2002年}
\subsection{解析几何与高等代数}
\begin{enumerate}
\item 在空间直角坐标系中, 直线$l_1$和$l_2$分别有方程
\[
\begin{array}{cc}
\left\{
\begin{aligned}
x + y + z - 1 &= 0 \\
x + y + 2z + 1 &= 0
\end{aligned}
\right. & 
\left\{
\begin{aligned}
3x + y + z &= 0 \\
x + 3z + 2 &= 0
\end{aligned}
\right.
\end{array}
\]
\begin{enumerate}
\item 求过$l_1$平行于$l_2$的平面的方程;
\item 求$l_1$和$l_2$的距离;
\item 求$l_1$和$l_2$的公垂线的方程.
\end{enumerate}

\item 在空间直角坐标系中, 求直线
\[
\left\{
\begin{aligned}
x &= 3x + 2 \\
y &= 2y - 1
\end{aligned}
\right.
\]
绕$z$轴旋转所得旋转曲面的方程.

\item 用正交变换化下面二次型为标准形
\[
f(x_1, x_2, x_3) = x_1^2 + x_2^2 + x_3^2 - 4x_1x_2 - 4x_1x_3 - 4x_2x_3.
\]
(要求写出正交变换的矩阵和相应的标准形)

\item 对于任意非负整数$n$, 令$f_n(x) = x^{n+2} - (x+1)^{2n+1}$, 证明:
\[
(x^2 + x + 1, f_n(x)) = 1.
\]

\item 设正整数$n \ge 2$, 用$M_n(K)$表示数域$K$上全体$n \times n$矩阵关于矩阵加法和数乘所构成的$K$上的线性空间. 在$M_n(K)$中定义变换$\sigma$如下:
\[
\sigma((a_{ij})_{n \times n}) = (a_{ij}')_{n \times n}, \quad \forall (a_{ij})_{n \times n} \in M_n(K),
\]
其中
\[
a_{ij}' = \left\{
\begin{array}{ll}
a_{ij}, & i \neq j; \\
i \cdot \tr(A), & i = j.
\end{array}
\right.
\]
\begin{enumerate}
\item 证明$\sigma$是$M_n(K)$上的线性变换.
\item 求出$\ker{\sigma}$的维数与一组基.
\item 求出$\sigma$的全部特征子空间.
\end{enumerate}

\item 用$R$表示实数域, 定义$R^n$到$R$的映射$f$如下
\[
f(X) = |x_1| + \cdots + |x_r| - |x_{r+1}| - \cdots - |x_{r+s}|, \quad \forall X = (x_1, x_2, \cdots, x_n)^T \in R^n,
\]
其中$r \ge s \ge 0$. 证明:
\begin{enumerate}
\item 存在$R^n$的一个$n - r$维子空间$W$, 使得$f(X) = 0$, $\forall X \in W$;
\item 若$W_1, W_2$是$R^n$的两个$n - r$维子空间, 且满足
\[
f(X) = 0, \quad \forall X \in W_1 \cup W_2,
\]
则一定有$\dim(W_1 \cap W_2) \ge n - (r + s)$.
\end{enumerate}

\item 设$V$是数域$K$上的$n$维线性空间, $V_1, \cdots V_n$是$V$的$s$个真子空间, 证明:
\begin{enumerate}
\item 存在$\alpha \in V$, 使得$\alpha \notin V_1 \cup V_2 \cup \cdots \cup V_n$;
\item 存在$V$中的一组基$\epsilon_1, \cdots, \epsilon_n$, 使得
\[
\{ \epsilon_1, \cdots, \epsilon_n \} \cap (V_1 \cup V_2 \cup \cdots \cup V_s) = \emptyset.
\]
\end{enumerate}

\end{enumerate}

\subsection{数学分析}
\begin{enumerate}
\item 求极限: $\lim\limits_{x \rightarrow 0}{\displaystyle(\frac{\sin{x}}{x})^{\frac{1}{1 - \cos{x}}}}$.

\item 设$\alpha \ge 0$, $x_1 = \sqrt{2 + \alpha}$, $x_{n+1} = \sqrt{2 + x_n}$, $n = 1, 2, \cdots$, 证明极限$\lim\limits_{n \rightarrow \infty}{x_n}$存在, 并求极限值.

\item 设$f(x)$在$[a, a + 2\alpha]$上连续, 证明存在$x \in [a, a + \alpha]$, 使得$f(x + \alpha) - f(x) = \displaystyle\frac{1}{2}(f(a + 2\alpha) - f(a))$.

\item 设$f(x) = x\sqrt{1 - x^2} + \arcsin{x}$, 求$f'(x)$.

\item 设$u(x, y)$有二阶连续偏导数, 证明$u$满足偏微分方程$\displaystyle\frac{\partial^2u}{\partial{x^2}} - 2\frac{\partial^2u}{\partial{x}\partial{y}} + \frac{\partial^2u}{\partial{y^2}} = 0$当且仅当: 存在二阶连续可微函数$\varphi(t), \psi(t)$, 使得$u(x, y) = x\varphi(x+y) + y\psi(x + y)$.

\item 计算三重积分$\displaystyle\iiint\limits_{\varOmega}{x^2\sqrt{x^2 + y^2}dxdydz}$, 其中$\varOmega$是曲面$z = \sqrt{x^2 + y^2}$与$z = x^2 + y^2$围成的有界区域.

\item 计算第二型曲面积分$I = \displaystyle\iint\limits_{\varSigma}{x^2dydz + y^2dzdx + z^2dxdy}$, 其中$\varSigma$是球面$x^2 + y^2 + z^2 = az (a > 0)$的外侧.

\item 判断级数$\sum\limits_{n=1}^{\infty}{\displaystyle\ln\cos\frac{1}{n}}$的收敛性并给出证明.

\item 证明: (1) 函数项级数$\sum\limits_{n=1}^{\infty}{nxe^{-nx}}$在区间$(0, \infty)$上不一致收敛; (2) 函数项级数$\sum\limits_{n=1}^{\infty}{nxe^{-nx}}$在区间$(0, \infty)$上可逐项求导.

\item 设$f(x)$连续, $g(x) = \displaystyle\int_{0}^{x}{yf(x - y)dy}$, 求$g''(x)$.

\end{enumerate}

\section{2005年}
\subsection{解析几何与高等代数}
\begin{enumerate}
\item 在直角坐标系中, 求直线$l: \left\{ \begin{aligned} 2x + y - z = 0 \\ x + y + 2z = 1  \end{aligned}\right.$到平面$\pi: 3x + By + z = 0$的正交投影轨迹的方程.
\item 在直角坐标系中对于参数$\lambda$的不同取值, 判断下面平面二次曲线的形状: $x^2 + y^2 + 2\lambda{}xy + \lambda = 0$.

对于中心型曲线, 写出对称中心的坐标;

对于线心型曲线, 写出对称直线的方程.
\item 设数域$K$上的$n$阶矩阵$A$的$(i, j)$元为$a_i - b_j$.
\begin{enumerate}
\item 求$|A|$;
\item 当$n \ge 2$时, $a_1 \neq a_2, b_1 \neq b_2$, 求齐次线性方程组$AX = 0$的解空间的维数和一个基.
\end{enumerate}

\item \begin{enumerate}
\item 设数域$K$上$n$阶矩阵, 对任意正整数$m$, 求$C^m$.
\item 用$M_n(K)$表示数域$K$上所有$n$级矩阵组成的集合, 它对于矩阵的加法和数量乘法成为$K$上的线性空间. 数域$K$上$n$级矩阵$A = \begin{bmatrix} a_1 & a_2 & a_3 & \cdots & a_n \\ a_n & a_1 & a_2 & \cdots & a_{n-1} \\ \cdots & \cdots & \cdots & \ddots & \cdots \\ a_2 & a_3 & a_4 & \cdots & a_1 \end{bmatrix}$称为循环矩阵. 用$U$表示$K$上所有$n$级循环矩阵组成的集合.

证明$U$是$M_n(K)$的一个子空间, 并求$U$的一个基和维数.
\end{enumerate}

\item \begin{enumerate}
\item 设实数域$R$上$n$级矩阵$H$的$(i, j)$元为$\displaystyle\frac{1}{i + j - 1} (n > 1)$. 在实数域上$n$维线性空间$R^n$中, 对于$\alpha, \beta \in R^n$, 令$f(\alpha, \beta) = \alpha^{\bm{T}}H\beta$. 试问: $f$是不是$R^n$上的一个内积, 写出理由.
\item 设$A$是$n$级正定矩阵$(n > 1), \alpha \in R^n$. 且$\alpha$是非零向量. 令$B = A\alpha\alpha'$, 求$B$的最大特征值以及$B$的属于这个特征值的特征子空间的维数和一个基.
\end{enumerate}

\item 设$A$是数域$R$上$n$维线性空间$V$上的一个线性变换, 用$I$表示$V$上的恒等变换, 证明: $A^3 = I \Leftrightarrow \text{rank}(I - A) + \text{rank}(I + A + A^2) = n$.
\end{enumerate}

\subsection{数学分析}
\begin{enumerate}
\item 设$f(x) = \displaystyle\frac{x^2\sin{x} - 1}{x^2 - \sin{x}}\sin{x}$, 试求$\lim\limits_{x \rightarrow +\infty}{\sup{f(x)}}$和$\lim\limits_{x \rightarrow +\infty}{\inf{f(x)}}$.

\item 证明下列各题:
\begin{enumerate}
\item 设$f(x)$在开区间可微, 且$f'(x)$在$(a, b)$有界, 证明$f(x)$在$(a, b)$一致连续.

\item 设$f(x)$在开区间$(a, b)$($-\infty < a < b < +\infty$)可微且一致连续, 试问$f'(x)$在$(a, b)$是否一定有界. (若肯定回答, 请证明; 若否定回答, 举例说明)
\end{enumerate}

\item 设$f(x) = \sin^2{(x^2 + 1)}$.
\begin{enumerate}
\item 求$f(x)$的麦克劳林展开式.
\item 求$f^{(n)}(0)$. ($n = 1, 2, 3, \cdots$)
\end{enumerate}

\item 试作出定义在$R^2$中的一个函数$f(x, y)$, 使得它在原点处同时满足以下三个条件:
\begin{enumerate}
\item $f(x, y)$的两个偏导数都存在;
\item 任何方向导数都存在;
\item 原点不连续.
\end{enumerate}

\item 计算$\displaystyle\int\limits_{L}{x^2ds}$, 其中$L$是球面$x^2 + y^2 + z^2 = 1$与平面$x + y + z = 0$的交线.

\item 设函数列$\{f_n(x)\}$满足下列条件:
\begin{enumerate}
\item $\forall n$, $f_n(x)$在$[a, b]$连续且有$f_n(x) \le f_{n+1}(x)(x \in [a, b])$,
\item $\{f_n(x)\}$点点收敛于$[a, b]$上的连续函数$s(x)$,
\end{enumerate}
证明: $\{f_n(x)\}$在$[a, b]$上一致收敛于$s(x)$.

\end{enumerate}

\section{2006年}
\subsection{解析几何与高等代数}
\begin{enumerate}
\item \begin{enumerate}
\item 设$A$, $B$分别是数域$K$上$s \times n, s \times m$矩阵, 叙述矩阵方程$AX = B$有解的充要条件. 并且给予证明.
\item 设$A$是数域$K$上$s \times n$列满秩矩阵, 试问: 方程$XA = E_n$是否有解? 有解, 写出它的解集; 无解, 说明理由.
\item 设$A$是数域$K$上$s \times n$列满秩矩阵, 试问: 对于数域$K$上任意$s \times m$矩阵$B$, 矩阵方程$AX = B$是否一定有解? 当有解时, 它有多少个解? 求出它的解集; 要求说明理由.
\end{enumerate}

\item \begin{enumerate}
\item 设$A, B$分别是数域$K$上的$s \times n, n \times s$矩阵, 证明:
\[
\text{rank}(A - ABA) = \text{rank}(A) + \text{rank}(E_n - BA) - n.
\]

\item 设$A, B$分别是实数域上$n$阶矩阵. 证明: 矩阵$A$与矩阵$B$的相似关系不随数域扩大而改变.
\end{enumerate}

\item \begin{enumerate}
\item 设$A$是数域$K$上的$n$阶矩阵, 证明: 如果矩阵$A$的各阶顺序主子式都不为0, 那么$A$可以分唯一的分解成$A = BC$, 其中$B$是主对角元都为1的下三角矩阵, $C$是上三角阵.

\item 设$A$是数域$K$上$n$阶可逆矩阵, 试问: $A$是否可以分解成$A = BC$, 其中$B$是主对角元都为1的下三角矩阵, $C$是上三角阵? 说明理由.
\end{enumerate}

\item \begin{enumerate}
\item 设$A$是实数域$R$上的$n$阶对称矩阵, 它的特征多项式$f(\lambda)$的所有不同的复根为实数$\lambda_1, \lambda_2, \cdots, \lambda_s$, 把$A$的最小多项式$m(\lambda)$分解成$R$上不可约多项式的乘积. 说明理由.

\item 设$A$是$n$阶实对称矩阵, 令$A(\alpha) = A\alpha, \forall \alpha \in R^n$, 根据第(1)问中$m(\lambda)$的因式分解, 把$R^n$分解成线性变换$A$的不变子空间的直和. 说明理由.
\end{enumerate}

\item 设$X = \{1, 2, \cdots, n\}$, 用$C^X$表示定义域为$X$的所有复值函数组成的集合, 它对于函数的加法和数量乘法成为复数域$C$上的一个线性空间.

对于$f(x), g(x) \in C^X$, 规定$\rangle f(x), g(x) \langle = \sum\limits_{j=1}^{n}{f(j)\bar{g(j)}}$,

这个二元函数是复线性空间$C^X$上的一个内积, 从而$C^X$成为一个酉空间.

设$p_1(x), p_2(x), \cdots, p_n(x) \in C^X$, 且满足$p_k(j) = \displaystyle\frac{1}{\sqrt{n}}\omega^{kj}, \forall j \in X$, 其中$\omega = e^{\frac{2\pi}{n}i}$.
\begin{enumerate}
\item 求复线性空间$C^X$的维数;
\item 证明: $p_1(x), p_2(x), \cdots, p_n(x)$是酉空间的一个标准正交基;
\item 令$\sigma(f(x)) = \Hat{f}(x), \forall f(x) \in C^X$, 其中$\Hat{f}(x)$在$x = k$处的函数值$\Hat{f}(k)$是$f(x)$在标准正交基$p_1(x), p_2(x), \cdots, p_n(x)$下的坐标的第$k$个分量. 证明: $\sigma$是酉空间$C^X$上的一个线性变换, 并且求$\sigma$在标准正交基$p_1(x), p_2(x), \cdots, p_n(x)$下的矩阵;
\item 证明第(3)题中的$\sigma$是酉空间$C^X$上的一个酉变换.
\end{enumerate}

\item 设$V$是域$K$上的$n$维线性空间, $A_1, A_2, \cdots, A_s$为$V$上的线性变换, 令$A = A_1 + A_2 + \cdots + A_s$, 求证: $A$为幂等变换且$\text{rank}(A) = \text{rank}(A_1) + \cdots + \text{rank}(A_s)$的充要条件是: 各$A_i$均为幂等变换, 且$A_iA_j = 0, i \neq j$.

\item 求一个过$x$轴的平面$\pi$, 使得其与单叶双曲面$\displaystyle\frac{x^2}{4} + y^2 - z^2 = 1$的交线为一个圆.

\item 证明四面体的每一个顶点到对面重心的线段共点, 且这点到顶点的距离是它到对面重心距离的3倍.

\item 一条直线与坐标平面$yoz$面, $xoz$面, $xoy$面的交点分别是$A, B, C$, 当直线变动时, 直线上的三个定点$A, B, C$也分别在坐标平面上变动. 此外, 直线上有第四点$P$, 点$P$到三点的距离分别是$a, b, c$, 求该直线按照保持点$A, B, C$分别在坐标平面上的规则移动时, 点$P$的轨迹.

\item 在一个仿射坐标系中, 已知直线$l_1$的方程为$\left\{ \begin{aligned} &x - y + z + 7 = 0 \\ &2x - y - 6 = 0 \end{aligned} \right.$, $l_2$经过点$M(-1, 1, 2)$, 平行于向量$\bm{u}(1, 2, -3)$. 判别这两条直线的位置关系, 并说明理由.
\end{enumerate}

\subsection{数学分析}
\begin{enumerate}
\item 确界存在原理是关于实数域完备性的一种描述, 试给出一个描述实数域完备性的其他定理, 并证明其与确界存在原理的等价性.

\item 设函数$f(x, y) = x^3 + 3xy - y^2 - 6x + 2y + 1$, 求$f(x, y)$在$(-2, 2)$处二阶二阶带Peano余项的Taloy展开; 问$f(x, y)$在$R^2$上有哪些关于极值的判别点, 这些点是否为极值点, 说明理由.

\item 设$F(x, y) = x^2y^3 + |x|y + y -5$,
\begin{enumerate}
\item 证明方程$F(x, y) = 0$在$(-\infty, +\infty)$上确定唯一的隐函数$y = f(x)$;
\item 求$f(x)$的极值点.
\end{enumerate}

\item 计算第二型曲面积分$\displaystyle\iint\limits_{\Sigma}{x^2dydz + y^2dzdx + z^2dxdy}$, 其中曲面$\Sigma$是椭球面$\displaystyle\frac{x^2}{a^2} + \frac{y^2}{b^2} + \frac{z^2}{c^2} = 1$外侧.

\item 证明广义积分$\displaystyle\int_{0}^{+\infty}{\frac{\sin{x}}{x}dx}$收敛, 并计算此积分.

\item 设$f(x, y)$是定义在$D = (a, b) \times [c, d]$上, $x$固定时, 对$y$连续; 设$x_0 \in (a, b)$取定, 对于任意$y \in [a, b]$, 极限$\lim\limits_{x \rightarrow x_0}{f(x, y)} = g(y)$收敛. 证明: 重极限$\lim\limits_{\substack{x \rightarrow x_0 \\ y \rightarrow y_0}}{f(x, y)} = g(y_0)$对任意$y_0 \in [c, d]$成立的充分必要条件是, 极限$\lim\limits_{x \rightarrow x_0}{f(x, y)} = g(y)$在$[c, d]$上一致连续.

\item 若函数$f(x)$在区间$[a, b]$上有界, 给出并证明$f(x)$在$[a, b]$上Riemann和的极限$\lim\limits_{\lambda{(\Delta)} \rightarrow 0}{\sum\limits_{i=1}^{n}{f(\xi_i)(x_i - x_{i-1})}}$收敛的Cauchy准则.

\item 设$\{f_n(x)\}$是$(-\infty, +\infty)$上一连续函数列, 满足存在常数$M$, 使得对于任意$f_n(x)$和$x \in (-\infty, +\infty)$恒有$|f_n(x)| \le M$. 假定对$(-\infty, +\infty)$中任意区间$[a, b]$都有$\lim\limits_{n \rightarrow \infty}{\displaystyle\int_{a}^{b}{f_n(x)dx}} = 0$. 证明: 对任意区间$[c, d] \subset (-\infty, +\infty)$以及$[c, d]$上绝对可积函数$h(x)$, 恒有$\lim\limits_{n \rightarrow \infty}{\displaystyle\int_{a}^{b}{f_n(x)h(x)dx}} = 0$.

\item 设存在一区间$[a, b]$使得两个Fourier级数$\displaystyle\frac{a_0}{2} + \sum\limits_{n=1}^{\infty}{a_n\cos{nx} + b_n\sin{nx}}$, $\displaystyle\frac{\alpha_0}{2} + \sum\limits_{n=1}^{\infty}{\alpha_n\cos{nx} + \beta_n\sin{nx}}$都在$[a, b]$上收敛, 并且其和函数在$[a, b]$上连续且相等, 问对于任意自然数$n$, $a_n = \alpha_n, b_n = \beta_n$是否成立? 如成立, 请证明; 如不成立, 加上什么条件后能保证成立, 说明理由.

\item 设$f(x)$在$[0, +\infty)$上内闭Riemann可积, 证明: 广义积分$\displaystyle\int_{0}^{+\infty}{f(x)dx}$绝对可积的充分必要条件是: 对于任意满足$x_0 = 0$, $x_n \rightarrow +\infty$的单调递增序列$\{x_n\}$, 级数$\displaystyle\sum\limits_{n=0}^{\infty}{\int_{x_n}^{x_{n+1}}{f(x)dx}}$绝对收敛.
\end{enumerate}

\section{2007年}
\subsection{解析几何与高等代数}
\begin{enumerate}
\item 回答下列问题:
\begin{enumerate}
\item 问何时存在$n$阶方阵$A, B$, 满足$AB - BA = E$(单位矩阵)? 又, 是否存在$n$维线性空间上的线性变换$A, B$, 满足$AB - BA = E$(恒等变换)? 若是, 举出例子; 若否, 给出证明.

\item 设$n$阶矩阵的各行元素之和为常数$c$, 则$A^3$的各行元素之和是否为常数? 若是, 是多少? 说明理由.

\item 设$m \times n$矩阵$A$的秩为$r$, 任取$A$的$r$个线性无关的行向量, 再取$A$的$r$个线性无关的列向量, 组成的$r$阶子式是否一定不为0? 若是, 给出证明; 若否, 举出反例.

\item 设$A, B$都是$m \times n$矩阵, 线性方程组$AX = 0$与$BX = 0$同解, 则$A$与$B$的列向量组是否等价? 行向量组是否等价? 若是, 给出证明; 若否, 举出反例.

\item 把实数域$R$看成有理数域$Q$上的线性空间, $b = p^2q^2r$, 这里的$p, q, r \in Q$是互不相同的素数, 判断向量组$1, \sqrt[n]{b}, \sqrt[n]{b^2}, \cdots, \sqrt[n]{b^{n-1}}$是否线性相关? 说明理由.
\end{enumerate}

\item 设$n$阶矩阵$A, B$可交换, 证明: $\text{rank}(A + B) \le \text{rank}(A) + \text{rank}(B) - \text{rank}(AB)$.

\item 设$f$为双线性函数, 且对任意的$\alpha, \beta, \gamma$都有$f(\alpha, \beta)f(\gamma, \alpha) = f(\beta, \alpha)f(\alpha, \gamma)$. 求证: $f$为对称的或反对称的.

\item 设$V$是欧几里德空间, $U$是$V$的子空间, $\beta \in U$, 求证: $\beta$是$\alpha \in V$在$U$上的正交投影的充分必要条件为: $\forall \gamma \in U$, 都有$|\alpha - \beta| \le |\alpha - \gamma|$.

\item 设$n$阶复矩阵$A$满足: 对于任意正整数$k$, 都有$\text{tr}(A^k) = 0$. 求$A$的特征值.

\item 设$n$维线性空间$V$上的线性变换$A$的最小多项式与特征多项式相同. 求证: $\exists \alpha \in V$, 使得$\alpha, A\alpha, A^2\alpha, \cdots, A^{n-1}\alpha$为$V$的一个基.

\item 设$P$是球内一定点, $A, B, C$是球面上三动点, $\displaystyle\angle{APB} = \angle{BPC} = \angle{CPA} = \frac{\pi}{2}$, 以$PA, PB, PC$为棱作平行六面体, 记与$P$相对的顶点为$Q$, 求$Q$点的轨迹.

\item 设直线$L$的方程为
\[
\left\{
\begin{aligned}
A_1x + B_1y + C_1z + D_1 = 0, \\
A_2x + B_2y + C_2z + D_2 = 0, \\
\end{aligned}
\right.
\]
问系数满足什么条件时, 直线$L$
\begin{enumerate}
\item 过原点;
\item 平行于$x$轴, 但不与$x$轴重合;
\item 与$y$轴相交;
\item 与$z$轴重合.
\end{enumerate}

\item 证明双曲抛物面$\displaystyle\frac{x^2}{a^2} - \frac{y^2}{b^2} = 2z$的相互垂直的直母线的交点在双曲线上.

\item 求椭球面$\displaystyle\frac{x^2}{25} + \frac{y^2}{16} + \frac{z^2}{9} = 1$被点$(2, 1, -1)$平分的弦.
\end{enumerate}

\subsection{数学分析}
\begin{enumerate}
\item 用有限覆盖定理证明连续函数的介值性定理.

\item $f(x), g(x)$在有界区间上一致连续, 证明: $f(x)g(x)$在此区间上也一致连续.

\item 已知$f(x)$在$[a, b]$上有4阶导数, 且有$f^{(4)}(\beta) \neq 0, f'''(\beta) = 0, \beta \in (a, b)$, 证明: 存在$x_1, x_2 \in (a, b)$, 使得$f(x_1) - f(x_2) = f'(\beta)(x_1 - x_2)$成立.

\item 构造一函数在$\mathbb{R}$上无穷次可微, 且$f^{(2n+1)}(0) = n$, $f^{(2n)}(0) = 0$, $n = 0, 1, \cdots$, 并说明满足条件的函数有任意多个.

\item 设$D = [0, 1] \times [0, 1]$, $f(x, y)$是$D$上的连续函数, 证明: 满足$\displaystyle\iint\limits_{D}{f(x, y)dxdy = f(\xi, \eta)}$的点$(\xi, \eta)$有无穷多个.

\item 求$\displaystyle\iint\limits_{\Sigma}{\sin^4{x}dydz + e^{-|y|}dzdx + z^2dxdy}$, 其中$\Sigma$是$x^2 + y^2 + z^2 = 1$, $z > 0$方向向上.

\item $f(x, y)$是$R^2$上的连续函数, 试作一无界区域$D$, 使$f(x, y)$在$D$上的广义积分收敛.

\item $f(x) = \displaystyle\ln{(1 + \frac{\sin{x}}{x^p})}$, 讨论不同$p$对$f(x)$在$(1, +\infty)$上积分的敛散性.

\item 记$F(x, y) = \sum\limits_{n=1}^{+\infty}{nye^{-n(x+y)}}$, 是否存在$a > 0$以及函数$h(x)$在$(1-a, 1+a)$上可导, 且$h(1) = 0$, 使得$F(x, h(x)) = 0$.

\item 设$f(x), g(x)$在$[a, b]$上黎曼可积, 证明: $f(x), g(x)$的傅立叶展开式有相同系数的充要条件是$\displaystyle\int_{a}^{b}{|f(x) - g(x)|dx} = 0$.
\end{enumerate}

\section{2008年}
\subsection{解析几何与高等代数}
\begin{enumerate}
\item \begin{enumerate}
\item 若$A$是$m \times n$矩阵, 非齐次线性方程组$Ax = \beta$有解, 且$r(A) = r$, 则方程组$Ax = \beta$的解向量中线性无关的最多有多少个? 并找出一组最多的线性无关的解向量.

\item 若$Ax = \beta$对所有$m$维非零向量$\beta$都有解, 求$r(A)$.
\end{enumerate}

\item \begin{enumerate}
\item 若$A$是$s \times n$矩阵, $B$是$n \times m$矩阵, $r(AB) = r(B)$. 则对于所有$m \times l$矩阵$C$是否有$r(ABC) = r(BC)$? 并给出理由.

\item $A$是$n$阶实矩阵, $A$的每一元素的代数余子式都等于此元素, 求$r(A)$.
\end{enumerate}

\item \begin{enumerate}
\item 设$A,C$分别为$n, m$阶实对称矩阵, $B$是$n \times m$实矩阵, $\begin{pmatrix} A & B \\ B^{\bm{T}} & C \end{pmatrix}$是正定矩阵(实), 证明: $\begin{vmatrix} A & B \\ B^{\bm{T}} & C \end{vmatrix} \le |A| \cdot |C|$, 等号当且仅当$B = 0$时成立.

\item 设$A = (a_{ij})_{n \times n}$是$n$阶实矩阵, $|a_{ij}| \le 1$, 求证: $|A|^2 \le n^n$.
\end{enumerate}

\item 设$f(x)$为一整系数多项式, $n$不能整除$f(0), f(1), \cdots, f(n-1)$, 证明: $f(x)$无整数根.

\item $A$是数域$K$上的$n$阶矩阵, $A$的特征多项式的根都属于$K$, 则$A$相似于上三角矩阵.

\item $V$是数域$K$上的线性空间, $A, B$是数域$V$上的线性变换, $A, B$的最小多项式互素, 求满足: $AC = CB$的所有线性变换$C$.

\item $A$是$n$维欧氏空间$V$上的正交变换. 证明: $A$是第一类的当且仅当存在$V$上的正交变换$B$满足$A = B^2$.

\item 求过直线$l: \left\{ \begin{aligned} &x - y + z + 4 = 0 \\ &x + y - 3z = 0 \end{aligned} \right.$且与$\pi_1: x + y + 2z =  0$垂直的平面$\pi_2$.

\item 平面$Ax + By + Cz + D = 0$与单叶双曲面$x^2 + y^2 - z^2 = 1$的交线是两条直线, 证明: $A^2 + B^2 = C^2 + D^2$.

\item 直线$l_1$过点$(1, 1, 1)$, 与$l_2: \left\{ \begin{aligned} &x+y+z = 0 \\ &x-y-3z=0 \end{aligned} \right.$相交, 交角为$\displaystyle\frac{\pi}{3}$, 求直线$l_1$方程.

\item 证明球面$S_1: x^2 + y^2 + z^2 -2x -2y -4z + 2 = 0$与球面$S_2: x^2 + y^2 + z^2 +2x -6y+1=0$有交点, 并求出交圆的圆心坐标.
\end{enumerate}

\subsection{数学分析}
\begin{enumerate}
\item 证明有界闭区间上的连续函数一致连续.

\item 是否存在$(-\infty, +\infty)$上的连续函数$f(x)$, 满足$f(f(x)) = e^{-x}$? 证明你的结论.

\item 数列$\{x_n\}(n \ge 1)$, 满足$\forall n < m, |x_n - x_m| > \displaystyle\frac{1}{n}$, 求证$\{x_n\}$无界.

\item $f(x)$是$(-1, 1)$上的无穷次可导函数, $f(0) = 1, |f'(0)| \le 2$, 令$g(x) = \displaystyle\frac{f'(x)}{f(x)}, |g^{(n)}(0)| \le 2n!$, 证明对所有正整数$n$, $|f^{(n)}(0)| \le (n+1)!$.

\item $\displaystyle\iint_{\Sigma}{(y - z)dydz + (z - x)dzdx + (x - y)dxdy}$,

$\Sigma$: 球面$x^2 + y^2 + z^2 = 2Rx$被圆柱面$x^2 + y^2 = 2rx(- < r< R)$所截得的部分, 定向取外侧.

\item 证明$F(x, y) = 2 - \sin{x} + y^3e^{-y}$在全平面有唯一解$y = y(x)$, 且$y(x)$连续, 可微.

\item $f(x)$在$[0, +\infty)$上内闭Riemann可积, 且$\displaystyle\int_{0}^{+\infty}{f(x)dx}$收敛, 求证$\displaystyle\lim\limits_{a \rightarrow 0^+}{\int_{0}^{+\infty}{e^{-ax}f(x)dx}} = \int_{0}^{+\infty}{f(x)dx}$.

\item $f(x)$是$(-\infty, +\infty)$上的二阶连续可导函数, 满足: 1)$\lim\limits_{|x| \rightarrow +\infty}{(f(x) - |x|)} = 0$; 2)$\exists x_0 \in (-\infty, +\infty)$, 满足$f(x_0) \le 0$. 求证: $f''(x)$在$(-\infty, +\infty)$上变号.

\item $g(x)$是周期为1的连续函数, $\displaystyle\int_{0}^{1}{g(x)dx} = 0$, $f(x)$在$[0, 1]$上有连续一阶导函数, $f(0) = f(1)$, $a_n = \displaystyle\int_{0}^{1}{f(x)g(nx)dx}$, 求证$\lim\limits_{n \rightarrow +\infty}{na_n} = 0$.

\item $f(x)$在$[0, 1]$上Riemann可积, 且对$[0, 1]$上任何有限个两两不交的闭区间$[a_i, b_i]$, $1 \le i \le n$, 都有$\displaystyle|\sum\limits_{i=1}^{n}{\int_{a_i}^{b_i}{f(x)dx}}| \le 1$, 求证$\displaystyle\int_{0}^{1}{|f(x)|dx} \le 2$.
\end{enumerate}

\section{2009年}
\subsection{解析几何与高等代数}
\begin{enumerate}
\item 一般说来一个向量组的极大线性无关部分组是不唯一的, 那么什么向量组的极大线性无关部分组是唯一的? 证明你的结论.

\item 设多项式$f(x)$的所有复根都是实数, 证明: 如果$a$是$f(x)$的导数$f'(x)$的重根, 则$a$也是$f(x)$的根.

\item 设$S$为$n$阶实对称矩阵, $S_1, S_2$都是$m$阶实对称矩阵, 证明: 若准对角矩阵$\begin{pmatrix} S & \\ & S_1 \end{pmatrix}$与$\begin{pmatrix} S & \\ & S_2 \end{pmatrix}$合同, 则$S_1$与$S_2$合同.

\item 解方程组$\left\{\begin{aligned}&x + y + z = 2 \\ & (x - y)^2 + (y - z)^2 + (z - x)^2 = 14 \\ &x^2y^2z + x^2yz^2 + xy^2z^2 = 2 \end{aligned} \right.$.

\item 设$A$为$n$阶实方阵且有$AA' = A^2$, 证明: $A$是对称矩阵.

\item 设$n \le 2$, $M_n(K)$为$K$上所有$n$阶方阵所成集合, $M_n(K)$上的一个函数$f$即为映射$f: M_n(K) \rightarrow K$, $M_n(K)$上的所有函数组成的集合记为$F(K)$, 在$F(K)$中定义加法和数乘运算如下: 对任意$f, g \in F(K)$, 对任意$k \in K$和任意$A \in M_n(K)$, $(f + g)(A) = f(A) + g(A), (kf)(A) = kf(A)$, 则$F(K)$关于此运算成为数域$K$上的一个线性空间. 对于$f \in F(K)$, $f$称为是列线性函数如果$f$对于矩阵的每一列都是线性的, 即对$K^n$中任意列向量$\beta_1, \beta_2, \cdots, \beta_n, \beta$, 任意$1 \le j \le n$, 以及任意$k \in K$, 都有$f(\beta_1, \cdots, \beta_{j-1}, \beta_j+\beta, \beta_{j+1}, \cdots, \beta_n) = f(\beta_1, \cdots, \beta_{j-1}, \beta_j, \beta_{j+1}, \cdots, \beta_n) + f(\beta_1, \cdots, \beta_{j-1}, \beta, \beta_{j+1}, \cdots, \beta_n)$和$f(\beta_1, \cdots, \beta_{j-1}, k\beta_j, \beta_{j+1}, \cdots, \beta_n) = kf(\beta_1, \cdots, \beta_{j-1}, \beta_j, \beta_{j+1}, \cdots, \beta_n)$, (其中的矩阵用它们的列向量组表示出), 而$f$称为是反对称的若$A \in M_n(K)$有两列向量相同时必有$f(A) = 0$. 用$SP(K)$表示$F(K)$中所有反对称列线性函数所成的集合, 证明: $SP(K)$是$F(K)$的一个子空间, 并求$SP(K)$的维数和一组基.

\item 设$U$为齐次线性方程组$ABX = 0$的解空间, 其中$A$为$n \times m$矩阵, $B$为$m \times p$矩阵, $X$为$p \times 1$矩阵, 证明: $m$维向量空间$K^m$中子集合$W = \{ Y = BX | X \in U \}$是子空间, 它的维数等于$\text{rank}(B) - \text{rank}(AB)$, 并利用此结论证明对任意三个矩阵$A, B, C$有$\text{rank}(AB) + \text{rank}(BC) \le \text{rank}(B) + \text{rank}(ABC)$.

\item 设$R$为实数域, $\alpha_1, \alpha_2, \cdots, \alpha_s$是$n$维欧氏空间$R^n$中的一线性无关向量组, 其中$R^n$中的内积为标准内积$(\alpha, \beta) = \alpha\cdot\beta'$, 这里的向量$\alpha$和$\beta$都看成是$1 \times n$矩阵, 用$B$表示$(i, j)$元为$(\alpha_i, \alpha_j)$, $1 \le i, j \le s$, 的$s \times s$矩阵, 对向量组$\alpha_1, \alpha_2, \cdots, \alpha_s$施行施密特(Schmidt)正交化过程后得到向量组$\beta_1, \beta_2, \cdots, \beta_s$, 证明: $|B| = \prod\limits_{i=1}^{s}{\num{\beta_i}^2}$, 其中$\num{\beta_i}$表示向量$\beta_i$的长度.

\item 请问直线$l: \left\{ \begin{aligned} &A_1x + B_1y + C_1z + D_1 = 0 \\ &A_2x + B_2y + C_2z + D_2 = 0 \end{aligned} \right.$的系数满足什么条件时才具有以下性质?
\begin{enumerate}
\item 经过原点;
\item 与$x$轴平行但不重合;
\item 和$y$轴相交;
\item 与$z$轴垂直(不必相交).
\end{enumerate}

\item 设平面$Ax + By + Cz + D =0$与双曲抛物面$2z = x^2 - y^2$的交线为两条直线, 证明: $A^2 - B^2 - 2CD = 0$.

\item 设空间直角坐标系中的曲面$Q$方程为$x^2 + y^2 -z^2 = 1$, 取一个过$z$轴的平面$\Sigma$并考虑全体与之平行的平面族. 问: 这些平行平面与$Q$的截线是什么类型的曲线? 当它们与$\Sigma$的距离变动时, 截线的形状如何变化? 请给出清楚的描述并说明判断理由.

\item 给出空间中半径为1的球面$S$和到球心距离为2的一点$P$, 考虑过$P$点且与$S$相交的任一条直线, 取两个交点的中点, 用解析几何的方法证明这些中点的轨迹在一个球面上, 并求出球心和半径.
\end{enumerate}

\subsection{数学分析}
\begin{enumerate}
\item 证明闭区间上的连续函数能取到最大值和最小值.

\item 设$f(x)$和$g(x)$是$R$上的有界一致连续函数, 求证: $f(x)g(x)$在$R$上一致连续.

\item 设$f(x)$是周期为$2\pi$的连续函数, 且其Fourier级数$\displaystyle\frac{a_0}{2} + \sum\limits_{n=1}^{+\infty}{a_n\cos{nx} + b_n\sin{nx}}$处处收敛, 求证: 这个Fourier级数处处收敛到$d(x)$.

\item 设$\{a_n\}_{n=1}^{\infty}$, $\{b_n\}_{n=1}^{\infty}$都是有界数列, 且$a_{n+1} + 2a_n = b_n$, 若$\lim\limits_{n \rightarrow \infty}{b_n}$存在, 求证$\lim\limits_{n \rightarrow \infty}{a_n}$也存在.

\item 是否存在$R \rightarrow R$的连续可导函数$f(x)$满足: $f(x) > 0$, 且$f'(x) = f(f(x))$?

\item 已知$f(x)$是$[0, +\infty)$上的单调连续函数, 且$\lim]limits_{x \rightarrow \infty}{f(x)} = 0$, 求证: $\displaystyle\lim\limits_{n \rightarrow \infty}\int_{0}^{+\infty}{f(x)\sin{nx}dx} = 0$.

\item 求曲线积分$\displaystyle\int_{L}{(y - z)dx + (z - x) dy + (x - y)dz}$, 其中$L$是球面$x^2 + y^2 + z^2 = 1$与球面$(x-1)^2 + (y-1)^2 + (z-1)^2 = 4$交成的曲线.

\item 设$x, y, z \ge 0, x+y+z = \pi$, 求$2\cos{x} + 2\cos{y} + 4\cos{z}$的最大最小值.

\item 设$f(x) \in C(a, b)$, 对任何$x \in (a, b)$都有$\displaystyle\varliminf\limits_{h \rightarrow 0^+}{\frac{f(x + h) - f(x - h)}{h}} \ge 0$, 求证: $f(x)$在$(a, b)$上单调不减.

\item 已知$f(x)$是$[0, +\infty)$上的正的连续函数, 且$\displaystyle\int_{0}^{+\infty}{\frac{1}{f(x)}dx} < +\infty$, 求证$\displaystyle\lim\limits_{A \rightarrow +\infty}{\frac{1}{A^2}\int_{0}^{A}{f(x)dx}} = +\infty$.
\end{enumerate}

\section{2010年}
\subsection{解析几何与高等代数}
\begin{enumerate}
\item 整系数多项式$f(x) = \sum\limits_{k=0}^{n}{a_kx^k} (n \ge 2010)$. 若存在素数$p$满足: 

i) $p \nmid a_n$, ii) $p \mid a_i, i = 0, 1, 2, \cdots, 2008$, iii) $p^2 \nmid a_0$

证明: $f(x)$必有次数不低于2009的不可约整系数因式.

\item 向量组$\alpha_1, \alpha_2, \cdots, \alpha_s$线性无关, 且可以由向量组$\beta_1, \beta_2, \cdots, \beta_t$线性表出, 证明必存在某个向量$\beta_j$($j = 1,2, \cdots, t$)使得向量组$\beta_j, \alpha_2, \cdots, \alpha_s$线性无关.

\item 设$A$是非零矩阵, 证明$A$可以写成某个列满秩矩阵与某个行满秩矩阵的乘积.

\item $AB$是$n$阶矩阵, 且满足$A = \displaystyle(B - \frac{1}{110}E)'(B + \frac{1}{110}E)$, 证明: 对任意的$n$维列向量$\xi$, 方程组$A'(A^2 + A)X = A'\xi$必有非零解.

\item 设$A$是$n$阶正定矩阵, 向量组$\beta_1, beta_2, \cdots, \beta_n$满足$\beta_i'A\beta_j = 0 (1 \le i < j \le n)$. 问向量组$\beta_1, \beta_2, \cdots, \beta_n$的秩可能是多少, 证明你的结论.

\item 线性变换$A$是对称变换, 且$A$是正交变换, 证明$A$是某个对合(即满足$A^2 = E$, $E$是单位变换)

\item $V$是内积空间, $\xi, \eta$是$V$中两个长度相等的向量, 证明必存在某个正交变换, 将$\xi$变到$\eta$.

\item $A$是复矩阵, $B$是幂零矩阵, 且$AB = BA$, 证明$|A + 2010B| = |A|$.

\item 求过$z$轴且与平面$x + 2y + 3z = 1$夹角为$60^{\circ}$的平面的方程.

\item 求直线$\left\{\begin{aligned} &x - y + z = 1 \\ &x + y - z = 1 \end{aligned}\right.$绕$z$轴旋转所成旋转曲面的方程, 并指出这是什么曲面.

\item 定义仿射坐标系$XOY$中的一个变换$f = \left\{ \begin{aligned} &x' = 7x - y + 1 \\ &y' = 4x + 2y + 4 \end{aligned} \right.$,
\begin{enumerate}
\item 求在$f$下的不变直线.
\item 以两条不变直线为坐标轴建立仿射坐标系$X'O'Y'$, 求此坐标系中$f$的变换公式.
\end{enumerate}

\item 用不过圆锥顶点的平面切割圆锥, 证明所截的曲线只可能为椭圆, 双曲线和抛物线. 并说明曲线类型随切割角度的变换规律.
\end{enumerate}

\subsection{数学分析}
\begin{enumerate}
\item 用有限覆盖定理证明聚点定理.

\item 是否存在数列$\{x_n\}$, 其极限点构成的集合为$M = \displaystyle\{1, \frac{1}{2}, \frac{1}{3}, \cdots\}$, 说明理由.

\item 设$I$是无穷区间, $f(x)$为$I$上的非多项式连续函数. 证明: 不存在$I$上一致收敛的多项式序列$\{P_n(x)\}$, 其极限函数为$f(x)$.

\item $f(x)$在$[0,1]$上连续, 在$(0, 1)$可导, 且满足$f(1) = \displaystyle\frac{1}{2}\int_{0}^{1/2}{e^{1-x^2}f(x)dx}$, 求证: 存在$\xi \in (0, 1)$使得$f'(\xi) = 2\xi{}f(\xi)$.

\item $f(x) \in C^1(R)$, $I$是有界闭区间, $\displaystyle{F_n(x) = n[f(x + \frac{1}{n}) - f(x)]}$, 证明函数序列$\{F_n(x)\}$在$I$上一致收敛. 如果$I$是有界开区间, 问$\{F_n(x)\}$在$I$上是否仍然一致收敛? 说明理由.

\item 构造$R$上的函数$f(x)$, 使其在$Q$上间断, 其他点连续. ($Q$表示有理数集)

\item 广义积分$\displaystyle\int_{0}^{+\infty}{xf(x)dx}$与$\displaystyle\int_{0}^{+\infty}{\frac{f(x)}{x}dx}$均收敛, 证明$I(t) = \displaystyle\int_{0}^{+\infty}{x^tf(x)dx}$在$(-1, 1)$上有定义, 并且有连续导函数.

\item 计算曲线积分$I = \displaystyle\oint_{\varGamma}{ydx + zdy + xdz}$, 其中$\varGamma$为$x^2 + y^2 + z^2 = 1$与$x + y + z = 0$的交线, 从$x$轴正向看是逆时针.

\item 证明下面的方程在点$(0, 0, 0)$附近唯一确定了隐函数$z = f(x, y)$,
\[
x + \frac{1}{2}y^2 + \frac{1}{2}z + \sin{z} = 0
\]
并将$f(x, y)$在点$(0, 0)$展开为带佩亚诺余项的泰勒公式, 展开到二阶.

\item $f(x), g(x)$是$[0, +\infty)$上的非负单调递减连续函数, 且$\displaystyle\int_{0}^{+\infty}{f(x)dx}$和$\displaystyle\int_{0}^{+\infty}{g(x)dx}$均发散, 设$h(x) = \min{(f(x), g(x))}$, 试问$\displaystyle\int_{0}^{+\infty}{h(x)dx}$是否一定发散? 说明理由.
\end{enumerate}

\section{2011年}
\subsection{解析几何与高等代数}
\begin{enumerate}
\item 判断题, 并说明理由:
\begin{enumerate}
\item 矩阵$A$的秩是5, 其中$A$的第3, 4行线性无关, 第1, 3列线性无关, $A$的这些行列组成的子式$A\begin{pmatrix} 3 & 4 \\ 1 & 3 \end{pmatrix} \neq 0$

\item 对于数域$F$, $W$是$F^5$的子空间, 那么存在线性变换$\varphi: F^6 \rightarrow F^5$, 使$\varphi(F^6) = W$;

\item 若$AX = 0$有唯一解, 则$AX = \beta$有唯一解.

\item 有限维空间的非零线性变换必有非零特征根.

\item 对任何正整数$n$, 存在有理数域上的$n$次不可约多项式$p(x) \in Q[x]$.

\item $V^*$是$V$的对称空间, $W$是$V$的真子空间, 则存在$f \in V^*$使$f(W) = 0$;.

\item $\varphi$是复数域上$C^{13}$的线形变换, 一定存在一个$\varphi$的10维不变子空间.

\item $\varphi$是欧氏空间的线性变换, $\varphi^*$是$\varphi$的共轭变换, 那么$\ker{\varphi^*\varphi} = \ker{\varphi}$.

\item 对角元素互不相同的上三角矩阵可以转化为对角矩阵.

\item 对于$A \in M_n(F)$是可逆矩阵, 那么存在$a_0, a_1, \cdots, a_{n-1} \in F$使得
\[
A^{-1} = a_{n-1}A^{n-1} + a_{n-2}A^{n-2} + \cdots + a_1A + a_0I_n.
\]

\end{enumerate}

\item 设$A = \begin{pmatrix} 1 & 1 & 0 & 0 \\ 0 & 1 & 0 & 2 \\ 0 & 1 & 1 & 0 \\ 0 & 0 & 0 & -1 \end{pmatrix}$;
\begin{enumerate}
\item 求$A$的最小多项式;
\item 求$A^{15}$;
\item 求$A$的若尔当标准型.
\item 设$Q[A] = \{ \sum\limits_{i=0}^{n}{a_iA^i} | a_i \in Q, n \in Z^{+} \}$, 求$Q[A]$的维数, 要求说明理由.
\end{enumerate}

\item 设$f(x_1, x_2, x_3) = x_1^2 + x_2^2 + x_3^2 + 4x_1x_2 + 4x_1x_3 + 4x_2x_3$
\begin{enumerate}
\item 把$f$写成$X'AX$的形式, 求$A$的特征值和特征向量.
\item 求正交矩阵$C$和对角矩阵$D$, 使得$A = CDC'$.
\item 求$f(x_1, x_2, x_3)$在$x_1^2 + x_2^2 + x_3^3 = 1$中的最大值和最小值, 并说明何时取到.
\end{enumerate}

\item $U,V,W$分别是$F$上的$r, s, t$维线性空间, $\Hom_F(U, V)$表示$U$到$V$上的线性变换的集合.
\begin{enumerate}
\item 证明$\dim{\Hom_F(U, V)} = rs$;
\item 设$\sigma^*$为$\Hom(W, U)$到$\Hom(W, V)$上线性映射, 则存在单射$\sigma$, 使$\sigma^*(f)w = \sigma \circ (fw)$, 其中$w \in W$;
\item 证明$\dim{Im\sigma^*} = \dim{Ker{(I - \sigma^*)}} + \dim{Im\sigma}$.
\end{enumerate}

\item $a, b, c, d$是起点相同的向量, 证明$a, b, c, d$终点共面当且仅当$[a,b,c]\cdot[b,c,d] + [c,d,a]\cdot[d,a,b] = 0$. ($[a, b, c]$表示向量$a, b, c$的混合积.)

\item 距两条异面直线距离相等的点的轨迹是什么? 用解析几何的方法加以证明.

\item $E$是一个椭球面, 中心是$O$.
\begin{enumerate}
\item 取与$E$相交的一族平行平面, 则截线都是椭圆, 而且中心共线.

\item 对$E$外侧任意的点$p$, 由$p$向$E$作切线, 可能的切点在一个平面$\varPi_p$上.

\item 同上过$E$外侧另一点$q$向$E$作切线, 切点落在平面$\varPi_q$上, 如果$q$在$Oq$连线上, 则$\varPi_p \parallel \varPi_q$, 而且两个平面截$E$所得的椭圆中心$O$与$p, q$共线.
\end{enumerate}

\end{enumerate}

\subsection{数学分析}
\begin{enumerate}
\item 使用确界存在原理证明: 连续函数$f(x)$定义在区间$I$上, 证明$f(I)$是一个区间.

\item 函数$f(x)$在$x_0$连续并且$|f(x)|$在$x_0$可导, 求证$f(x)$在$x_0$可导.

\item 函数$f(x)$在$(0, 1)$可导, $f'$有界, $\lim\limits_{x \rightarrow 0+0}{f(x)}$不存在, 求证存在数列$\{x_n\}$满足条件: (1) $\lim\limits_{n \rightarrow +\infty}{x_n}=0$; (2) 对所有$n$, $f'(x_n) = 0$.

\item 构造两个以$2\pi$为周期的函数, 使之在$[0, \pi]$上其Fourier级数一致收敛于0.

\item 证明$f(x)$在$[0, 1]$上可积, 其充要条件是: 对$F(x, y) = f(x)$在$[0, 1] \times [0, 1]$上可积.

\item $f(x, y)$在其定义域中的某个点上存在方向导数, 且在三个方向上的方向向量均存在且相等. 证明$f(x, y)$不可微.

\item 设$D$为$R^2$上的无界闭集, 试构造一个函数$f(x, y)$, 使它在一个由光滑曲线所围成的无界闭区域$D$上的二重积分$\displaystyle\iint\limits_{D}{f(x, y)dxdy}$发散.

\item 设$T(x)$, $x$属于$R^n$的一个子集$D$, $D$是一个凸区域, $T(x)$在$D$上有连续二阶偏导数, 其Jaccobi行列式正定. 证明$T(x)$是单调的.

\item $a_n > 0$, 并且$\sum\limits_{n=1}^{+\infty}{a_n}$收敛, 求证
\[
\lim_{n \rightarrow +\infty}{n^2(\frac{1}{a_1} + \frac{1}{a_2} + \cdots + \frac{1}{a_n})^{-1}}
\]
收敛.

\item $f_n(x)$在$[a, b]$可导, $f_n'(x)$在$[a, b]$上一致有界, 并且$\{f_n(x)\}$点态收敛于有界函数$f(x)$, 求证$f(x)$在$[a, b]$上连续.
\end{enumerate}

\section{2018年}
\subsection{数学分析}
\begin{enumerate}
\item 设$f \in C(0, 1)$, 且存在$x_1, x_2, x_3, x_4 \in (0,1)$,使得
\[
\alpha = \frac{f(x_2) - f(x_1)}{x_2 - x_1} < \frac{f(x_4) - f(x_3)}{x_4 - x_3} = \beta,
\]
证明:对任意$\lambda \in (\alpha, \beta)$,存在$x_5, x_6 \in (0,1)$,使得$\lambda = \frac{f(x_6) - f(x_5)}{x_6 - x_5}$.

定义
\[
F(x, y) = \frac{f(y) - f(x)}{y - x}, 0 < x < y < 1.
\]
则$F$在连通开集$D=\{(x,y):0 < x < y < 1\}$上连续. 不妨设$x_1 < x_2$, $x_3 < x_4$,则$(x_1, x_2) \in D$, $(x_3, x_4) \in D$,且
\[
\alpha = F(x_1, x_2) < \lambda < F(x_3, x_4) = \beta.
\]
根据连通集上连续函数的节值定理,存在$(x_5, x_6) \in D$,使得
\[
\lambda = F(x_5, x_6),
\]
即
\[
\lambda = \frac{f(x_6) - f(x_5)}{x_6 - x_5}.
\]

另一种方法是考虑函数
\[
G(t) = \frac{f((1-t)x_2 + tx_4) - f((1-t)x_1 + tx_3)}{(1-t)(x_2 - x_1) + t(x_4 - x_3)}
\]
此时$\alpha = G(0)$, $\beta=G(1)$.剩下的就是使用中值定理即可。

\item 设$A$, $B \in R^3$; $\gamma$是以$A$, $B$为端点的光滑曲线, 弧长为$L$; $U$是一个包含$\gamma$的开集; $f$是$U$上连续可微的函数, 它的梯度向量长度的上界是$M$. 求证:
\[
|f(A) - f(B) \le ML.|
\]

不妨设$\gamma = \gamma(t)$, $t \in [0, 1]$, $\gamma(0) = B$, $\gamma(1)=A$,则
\[
\begin{aligned}
|f(A) - f(B)| &= |\int_0^1{\frac{df(\gamma(t))}{dt}dt}| \le \int_0^1{|\frac{df(\gamma(t))}{dt}|dt} \\
&=\int_0^1{|(\text{grad}{f(\gamma(t))}, \gamma'(t))|dt} \le M\int_0^1{|\gamma'(t)|dt} \\
&= ML,
\end{aligned}
\]
其中$(\cdot,\cdot)$是$R^3$中的内积.
\end{enumerate}

\chapter{中国科学院研究生院入学考试}
\section{2007年}
\subsection{高等代数}
\begin{enumerate}
\item 设多项式$f(x)$, $g(x)$, $h(x)$只有非零常数公因子, 证明: 存在多项式$u(x)$, $v(x)$, $w(x)$, 使得$u(x)f(x) + v(x)g(x) + w(x)h(x) = 1$.

\item 设$m, n, p$都是非负整数, 证明: $(x^2 +x + 1)$整除$(x^{3m} + x^{3n+1} + x^{3p+2})$.

\item 设$A$是$n$阶实数矩阵, $A \neq 0$, 而且$A$的每个元素都和它的代数余子式相等. 证明$A$是可逆矩阵.

\item 计算$n$阶行列式
\[
D_n = \begin{vmatrix}
2\cos\alpha & 1 & &  & & \\
1 & 2\cos\alpha & 1 &&& \\
&1 & 2\cos\alpha & 1 && \\
&&1 &\ddots & \ddots & \\
&&& \ddots & 2\cos\alpha & 1 \\
&&&& 1 & 2\cos\alpha
\end{vmatrix}
\]

\item 设$\alpha_1, \alpha_2, \cdots, \alpha_k \in R^n$是齐次线性方程组$AX = 0$的基础解系, $s, t \in R$, $\beta_1 = s\alpha_1 + t\alpha_2$, $\cdots$, $\beta_{k-1} = s\alpha_{k-1} + t\alpha_k$, $\beta_k = s\alpha_k + t\alpha_1$. 试问: $s, t$应该满足什么关系, 使得$\beta_1, \cdots, \beta_{k-1}, \beta_k$是方程组$AX = 0$的基础解系, 反之, 当$\beta_1, \cdots, \beta_{k-1}, \beta_k$是方程组$AX = 0$的基础解系时, 这个关系必须成立.

\item 设$A$是实对称矩阵, 如果$A$是半正定的, 则存在实的半正定矩阵$B$, 使得$A = B^2$.

\item 已知$A = \begin{pmatrix} 1 & 0 & 0 \\ 1 & 0 & 1 \\ 0 & 1 & 0 \end{pmatrix}$, 试证明对于$n \ge 3$有$A^n = A^{n-2} + A^2 - I$, 并计算$A^{100}$, 其中$I$表示单位矩阵.

\item 设二次型$f = x_1^2 + x_2^2 + x_3^2 + 2ax_1x_2 + 2x_1x_3 + 4bx_2x_3$通过正交变换化为标准形$f = y_2^2 + 2y_3^2$, 求参数$a, b$及所用的正交变换.

\item 设$A$是复数域上6维线性空间$V$的线性变换, $A$的特征多项式为$(\lambda - 1)^3(\lambda + 1)^2(\lambda + 2)$, 证明$V$能够分解成三个不变子空间的直和, 而且它们的维数分别是1, 2, 3.

\end{enumerate}

\subsection{数学分析}
\begin{enumerate}
\item 求幂级数$\sum\limits_{n=0}^{\infty}{\displaystyle\frac{n^2 + 1}{2^nn!}x^n}$的收敛域, 并求其和.

\item 讨论积分$\displaystyle\int_{0}^{+\infty}{\frac{e^{\sin{x}}\sin{2x}}{x^p}dx}$的绝对收敛和条件收敛.

\item 计算曲面积分$\displaystyle\iint\limits_{\varSigma}{yzdydz + (x^2 + z^2)ydzdx + xydxdy}$, 其中$\varSigma$为曲面$4 - y = x^2 + z^2$在$xoz$平面的右侧部分的外侧.

\item 证明下列不等式:
\begin{enumerate}
\item $x^n(1 - x) < \displaystyle\frac{1}{ne}$ \quad ($0 < x < 1$, $n$为正整数);
\item $x^y + y^x > 1$ \quad (x, y > 0).
\end{enumerate}

\item 设级数$\sum\limits_{n=1}^{\infty}{b_n}$收敛, 且$\sum\limits_{n=1}^{\infty}{(a_n - a_{n-1})}$绝对收敛. 证明: 级数$\sum\limits_{n=1}^{\infty}{a_nb_n}$收敛.

\item 假设$f(x)$为二次连续可微实值函数, 对于所有的实数$x$, 满足$|f(x)| \le 1$且满足$(f(0))^2 + (f'(0))^2 = 4$.. 证明存在实数$x_0$, 满足$f(x_0) + f''(x_0) = 0$.

\item 假设$|f(x)| \le 1$和$|f''(x)| \le 1$对一切成立, 证明: 在$[0, 2]$上有$|f'(x)| \le 2$.

\item 设$D = [0, 1] \times [0, 1]$, $f(x, y)$是定义在$D$上的二元函数, $f(0, 0) = 0$, 且$f(x, y)$在$(0, 0)$处可微. 求极限:
\[
\lim_{x \rightarrow 0+}{\frac{\displaystyle\int_{0}^{x^2}{dt}\int_{x}^{\sqrt{t}}{f(t, u)du}}{1 - e^{\frac{x^4}{4}}}}
\]

\item 设$-\infty < x_0 < +\infty$, $\varphi(x)$和$f(x)$在$[x_0, x_0 + h]$上连续, 且存在$M > 0$, $K > 0$, 使得
\[
|\varphi(x)| \le M \Bigl{(} 1 + K\int_{x_0}^{x}{|\varphi(t)f(t)|dt}  \Bigr{)}, \quad x \in (x_0, x_0 + h).
\]
证明: $\varphi(x)$必满足
\[
|\varphi(x)| \le M\exp{\{ KM\int_{x_0}^{x}{|f(t)|dt} \}}, x \in (x_0, x_0 + h).
\]

\item 设$\alpha \in (0, 1)$, 记$e = (1, 1, \cdots, 1)^T \in R^n$, $S(\displaystyle\frac{e}{n}, \frac{\alpha}{n}) = \{ x \in R^n : \num{x - \frac{e}{n}} \le \frac{\alpha}{n} \}$, 对于$x \in S(\displaystyle\frac{e}{n}, \frac{\alpha}{n})$且$e^Tx = 1$, 证明:
\[
-\sum_{i=1}^{n}{\ln{x_i}} \le n\ln{n} + \frac{\alpha^2}{2(1 - \alpha)^2}.
\]

\end{enumerate}

\section{2012年}
\subsection{高等代数}
\begin{enumerate}
\item 证明多项式$f(x) = 1 + \frac{x}{1!} + \frac{x^2}{2!} + \cdots + \frac{x^n}{n!}$没有重根.
\item 设多项式$g(x) = p^k(x)g_1(x)$($k \ge 1$), 多项式$p(x)$与$g_1(x)$互素. 证明: 对任意多项式$f(x)$有
$$
\frac{f(x)}{g(x)} = \frac{r(x)}{p^(x)} + \frac{f_1(x)}{p^{k-1}(x)g_1(x)}
$$
其中, $r(x)$, $f_1(x)$都是多项式, $r(x) = 0$或$\deg(r(x)) < \deg(p(x))$.

\item 已知$n$阶方阵
$$
A = \begin{pmatrix}
a_1^2 & a_1a_2 + 1 & \cdots & a_1a_n + 1 \\
a_2a_1 + 1 & a_2^2 & \cdots & a_2a_n + 1 \\
\cdots & \cdots & \cdots & \cdots \\
a_na_1 + 1 & a_na_2+1 & \cdots & a_n^2
\end{pmatrix}
$$
其中, $\sum_{i=1}^{n}{a_i} = 1$, $\sum_{i=1}^{n}{a_i^2} = n$.
\begin{enumerate}
\item[1)]求$A$的全部特征值;
\item[2)]求$A$的行列式$\det(A)$和迹$\tr(A)$.
\end{enumerate}
\item 设数域$k$上的$n$阶方阵$A$满足$A^2 = A$, $V_1$, $V_2$分别是齐次线性方程组$Ax = 0$和$(A - I_n)x = 0$在$k^n$中的解空间, 试证明: $k^n = V_1 \oplus V_2$, 其中$I_n$代表$n$阶单位矩阵, $\oplus$表示直和.

\item 设$n$阶矩阵$A$可逆, $\alpha$, $\beta$均为$n$维列向量, 且$1 + \beta^TA^{-1}\alpha \neq 0$,其中$\beta^T$表示$\beta$的转置.
\begin{enumerate}
\item[1)] 证明矩阵$P = \begin{pmatrix}A & \alpha \\ -\beta^T & 1\end{pmatrix}$可逆, 并求其逆矩阵.

\item[2)] 证明矩阵$Q = A + \alpha\beta^T$可逆, 并求其逆矩阵.
\end{enumerate}

\item 证明: 任何复数方阵$A$都与它的转置矩阵$A^T$相似.

\item 在二阶实数矩阵构成的线性空间$R^{2 \times 2}$中定义:
$$
(A, B) = \tr(A^TB), \quad \forall A, B \in R^{2 \times 2}
$$
其中, $A^T$表示矩阵$A$的转置, $\tr(X)$表示矩阵$X$的迹.
\begin{enumerate}
\item[1)] 证明$(A, B)$是线性空间$R^{2 \times 2}$的内积.
\item[2)] 设$W$是由$A_1 = \begin{pmatrix}1 & 1 \\0 & 0\end{pmatrix}$, $A_2 = \begin{pmatrix}0 & 1 \\1 & 1\end{pmatrix}$生成的子空间. 试求$W^{\bot}$的一组标准正交基.
\end{enumerate}

\item 设$T_1, T_2, \cdots, T_n$是数域上线性空间$V$的非零线性变换, 试证明存在向量$\alpha \in V$, 使得$T_i(\alpha) \neq 0$, $i=1,2,\cdots, n$.
\end{enumerate}

\section{2018}
\subsection{数学分析}
\begin{enumerate}
\item 设$x > 0$, 证明$\sqrt{1 + x} - \sqrt{x} = \frac{1}{2\sqrt{x + \theta}}$,其中$\theta=\theta(x) > 0$,并且$\lim_{x \to 0}{\theta(x)} = \frac{1}{4}$.

根据Lagrange中值定理有
\[
\sqrt{1 + x} - \sqrt{x} = \frac{1}{2\sqrt{x + \theta}}
\]
\[
\begin{aligned}
\theta(x) &= (\frac{\sqrt{1+x} + \sqrt{x}}{2})^2 - x = \frac{1 + 2\sqrt{x(x+1)} - 2x}{4} \\
&= \frac{1}{4} + \frac{1}{2}[\sqrt{x(x+1)} - x]\\
&= \frac{1}{4} + \frac{1}{2}[\frac{x}{\sqrt{x(x+1)} + x}] \\
&= \frac{1}{4} + \frac{1}{2}[\frac{1}{\sqrt{1 + \frac{1}{x}} + 1}]
\end{aligned}
\]
易知
\[
0 < \frac{1}{2}[\frac{1}{\sqrt{1 + \frac{1}{x}} + 1}] < \frac{1}{4}
\]
则有
\[
0 < \frac{1}{4} < \theta(x) < \frac{1}{2}
\]
同时根据
\[
\theta(x) = \frac{1}{4} + \frac{1}{2}[\frac{1}{\sqrt{1 + \frac{1}{x}} + 1}],
\]
可得
\[
\lim_{x \to 0}{\theta(x)} = \frac{1}{4}.
\]

\end{enumerate}

\chapter{南开大学研究生入学考试}
\section{2018}
\subsection{高等代数}
\subsection{数学分析}
\begin{enumerate}
\item 求$f(x)=4\ln{x} + x^2-6x$的极值.

\item 已知区域$D=\{(x,y)|x \ge 0, y \ge 0, x+2y \le 1\}$,求二重积分$\iint_{D}{e^{x+2y}dxdy}$.
\end{enumerate}

\chapter{中山大学研究生入学考试}
\section{数学分析}
\begin{enumerate}
\item 计算
\begin{enumerate}
\item $\displaystyle\int_{0}^{\frac{\pi}{2}}{\frac{\sin{x}}{\sin{x} + \cos{x}}dx}$;

\item $\displaystyle\int{\frac{\arcsin{e^x}}{e^x}dx}$;

\item $\lim\limits_{x \rightarrow 0+}{\frac{\sqrt{x}}{1 - e^{\sqrt{x}}}}$;

\item $\lim\limits_{x \rightarrow \infty}{\displaystyle(\sqrt{\cos\frac{1}{x}})^{x^2}}$;

\item 设$z = z(x, y)$由方程$e^{-xy} - 2z + e^z = 0$确定, 求$\displaystyle\frac{\partial^2z}{\partial{x^2}}$;

\item 求曲面$x^2 + 2y^2 + 3z^2 = 6$在$(1, 1, 1)$点处的切平面方程.
\end{enumerate}

\item 判别下列级数或广义积分的收敛性, 条件收敛还是绝对收敛.
\begin{enumerate}
\item $\sum\limits_{n=1}^{\infty}{\displaystyle(-1)^n\frac{(\ln{n})^2}{(\ln{3})^n}}$;
\item $\sum\limits_{n=1}^{\infty}{\displaystyle(\frac{\pi}{2n^2} + \sin\frac{\pi}{n})}$;
\item $\displaystyle\int_{0}^{+\infty}{x^2e^{-x^2}dx}$;
\item $\displaystyle\int_{0}^{1}{\frac{\ln{x}}{(1 - x)^2}dx}$.
\end{enumerate}

\item 求平面曲线$\left\{ \begin{aligned} x &= a(\cos{t} + t\sin{t}) \\ y &= a(\sin{t} - t\cos{t}) \end{aligned} \right.$上对应于$t = t_0$点的法线方程, 并讨论曲线在$t \in (0, \pi)$一段的凹凸性.

\item 讨论函数$f(x, y) = \left\{ \begin{array}{ll} \displaystyle\frac{xy^2}{x^2 + y^2}, & (x, y) \neq (0, 0) \\ 0, & (x, y) = (0, 0) \end{array} \right.$在$P_0(0, 0)$点处
\begin{enumerate}
\item 连续性;
\item 可微性;
\item 沿$\vec{l} = (\cos\alpha, \sin\alpha)$的方向导数的存在性.
\end{enumerate}

\item 计算曲线积分$\displaystyle\oint_{C}{xyzdy}$, 其中曲线$C: \left\{ \begin{aligned}&x^2 + y^2 + z^2 = 1 \\ y = z & \end{aligned}\right.$, 其方向与$z$轴构成右手系.

\item 对幂级数$\sum\limits_{n=1}^{\infty}{(-1)^{n-1}\frac{2n+1}{n}x^{2n}}$
\begin{enumerate}
\item 求收敛域;
\item 求和函数;
\item 讨论幂级数在收敛域上的一致收敛性.
\end{enumerate}

\item 在$Oxy$平面上, 光滑曲线$L$过$(1, 0)$点, 并且曲线$L$上任意一点$P(x, y)(x \neq 0)$处的切线斜率与直线$OP$的斜率之差等于$ax$($a > 0$为常数).
\begin{enumerate}
\item 求曲线$L$的方程;
\item 如果$L$与直线$y = ax$所围成的平面图形的面积为8, 确定$a$的值.
\end{enumerate}

\item 设$f(x)$在$[0, 1]$连续, 令
\[
f_n(t) = \int_{0}^{t}{f(x^n)dx}, \quad t \in [0, 1], n = 1, 2, \cdots
\]
证明函数列$\{f_n(t)\}$在$[0, 1]$一致收敛于函数$g(t) = tf(0)$.

\end{enumerate}

\chapter{其他}
\begin{enumerate}
\item $\displaystyle\int_{0}^{1}{\sin(\pi{x}) \cdot x^x(1 - x)^{1 - x}dx}$;

\item $\displaystyle\int_{0}^{\frac{1}{2}}{\frac{\ln(1 - x)\ln{x}}{x(1 - x)}dx}$;

\item $\displaystyle\int_{0}^{1}{\frac{\ln{x}\ln^2(1 - x)}{x}dx}$;

\item $\displaystyle\int_{0}^{+\infty}{\tan{\frac{x}{\sqrt{x^2 + x^3}}}\frac{\ln(1 + \sqrt{x})}{x}dx}$;

\item $\displaystyle\int_{0}^{\frac{\pi}{2}}{\cos(k\ln(\tan(x)))dx}, k \in R$;

\item $\displaystyle\int_{0}^{+\infty}{x^n\sin(\sqrt[4]{x})e^{-\sqrt[4]{x}}dx, n \in N}$;

\item $\displaystyle\int_{0}^{\frac{\pi}{2}}{\ln^2(\cos{x})dx}$;

\item $\displaystyle\int_{0}^{+\infty}{\frac{\sin{x}}{xe^x}dx}$;

\item $\displaystyle\int_{0}^{+\infty}{\frac{x\ln{x}}{(1 + x^2)(1 + x^3)^2}dx}$;

\item $\displaystyle\int_{0}^{+\infty}{(\frac{x}{e^x - e^{-x}} - \frac{1}{2})\frac{1}{x^2}dx}$;

\item $\displaystyle\int_{0}^{\frac{\pi}{2}}{\arccos(\frac{\cos{x}}{1 + 2\cos{x}})dx}$;

\item $\displaystyle\int_{0}^{\frac{\pi}{3}}{\arccos(\frac{\cos{x}}{1 + 2\cos{x}})dx}$;

\item 若$f:[0, 1] \times [0, 1] \rightarrow R$可积, 求
\[
\displaystyle\lim\limits_{n \rightarrow +\infty}{(\frac{(2n+1)!}{(n!)^2})^2}\int_{0}^{1}\int_{0}^{1}(xy(1-x)(1-y))^nf(x, y)dxdy;
\]

\item $\displaystyle\lim\limits_{n \rightarrow +\infty}{\int_{0}^{1}{|\sin{nx}|^3dx}}$;

\item $\displaystyle\lim\limits_{x \rightarrow 0}{\int_{0}^{x}{\ln{\frac{|\sin(t - \frac{x}{2})|}{\sin{\frac{x}{2}}}}\frac{dt}{\sin{t}}}}$;

\item $\displaystyle\prod\limits_{n=1}^{+\infty}{(1 + \frac{1}{n^3})}$;

\item $\displaystyle\int_{0}^{\arccos{\frac{1}{3}}}{\arccos\Bigl{(} \frac{1 - \cos{x}}{2\cos{x}} \Bigr{)}dx}$;

\item $\displaystyle\int_{0}^{\frac{\pi}{2}}{\arccos\sqrt{ \frac{\cos{x}}{1 + 2\cos{x}} }dx}$;

\item 求$1 + \displaystyle\Bigl{(} \frac{1 + \frac{1}{2}}{2} \Bigr{)}^2 + \Bigl{(} \frac{1 + \frac{1}{2} + \frac{1}{3}}{3} \Bigr{)}^2 + \Bigl{(} \frac{1 + \frac{1}{2} + \frac{1}{3} + \frac{1}{4}}{4} \Bigr{)}^2 + \cdots$;

\item 求$\sum\limits_{n=1}^{+\infty}{\displaystyle\frac{((n-1)k)!}{(nk)!}}$;

\item 求$\sum\limits_{n=1}^{+\infty}{\frac{\Bigl{(} \frac{3 - \sqrt{5}}{2} \Bigr{)}^n}{n^3}}$;

\item 求$\sum\limits_{n=1}^{+\infty}{\displaystyle\frac{1}{n^2(e^{n\pi} - e^{-n\pi})^2}}$;

\item 求和: $\sum\limits_{n=1}^{+\infty}{\displaystyle\frac{1}{C_{2n}^{n}}}$, $\sum\limits_{n=1}^{+\infty}{\displaystyle\frac{1}{nC_{2n}^{n}}}$, $\sum\limits_{n=1}^{+\infty}{\displaystyle\frac{1}{n^2C_{2n}^{n}}}$, $\sum\limits_{n=1}^{+\infty}{\displaystyle\frac{1}{n^4C_{2n}^{n}}}$;

\end{enumerate}

\end{document}
