\part{多元微积分}
《多元微积分》的作者是C.Goffman。参考:\cite{CalculusOfMultiVariables1978}。

\chapter{欧氏空间}\label{section00301}
欧氏空间是一个向量空间,它具有满足某些条件的距离函数。在这一章中,我们给出这些空间的定义和主要性质,还要讨论向量空间之间的线性映射及其性质,然后给出欧氏空间拓扑的一个简洁的处理。


\section{向量空间}\label{subsection0030101}
实数域$\mathbb{R}$上的\textbf{向量空间}是一个集合$S$,它带有映射
\[
g: S \times S \to S
\]
和
\[
v: \mathbb{R} \times S \to S,
\]
通常用记号$g(x,y) = x+y$和$v(a,x)=ax$记这两个映射,并假定下列条件成立:
\begin{enumerate}
\item[($a$)] $S$对于映射$g$成一Abel群。
\item[($b_1$)]对于每个$x \in S$和$a, b \in \mathbb{R}$,$(ab)x = a(bx)$。
\item[($b_2$)]对于每个$x \in S$和$a, b \in \mathbb{R}$,$(a+b)x=ax+bx$。
\item[($b_3$)]对于每个$x, y \in S$和$a \in \mathbb{R}$,$a(x+y)=ax+ay$。
\item[($b_4$)]对于每个$x \in S$,$1 \cdot x = x$。
\end{enumerate}

我们用记号$\theta$记$S$中群的恒等元。容易证明$0 \cdot x = \theta$对每个$x \in S$成立。不难看出,群的逆元是$(-1)x$,记为$-x$。我们把$y+(-x)$写为$y-x$。

我们只给出向量空间的两个例子。
\begin{example}
设$S$是$n$-实数组的集合。对于
\[
\begin{aligned}
x &= (x_1, \cdots, x_n) \in S,\\
y &= (y_1, \cdots, y_n) \in S,
\end{aligned}
\]
命
\[
x+y = (x_1+y_1, \cdots, x_n+y_n).
\]
对于
\[
x = (x_1, \cdots, x_n) \in S
\]
和$a \in R$,命
\[
ax = (ax_1, \cdots, ax_n).
\]
\end{example}

\begin{example}
设$A$是任一集合,$X$是一向量空间。命$X^A$是$A$到$X$的全体映射所成的集合。对于任意$f, g \in X^A$,定义$f+g$是这样的映射:对于每个$\alpha \in A$,
\[
(f+g)(\alpha)  = f(\alpha)+g(\alpha).
\]
对每个$f \in X^A$和$a \in \mathbb{R}$,定义$af$为这样的映射:对于每个$\alpha \in A$,
\[
(af)(\alpha) = af(\alpha).
\]
\end{example}

设$A, B$是两个集合,映射$f: A \to B$叫做单射的(injective),如果它是一对一的;叫做满射的(surjective),如果它是$A$到整个$B$上的映射;叫做双射(bijective),如果它是$A$到整个$B$上的映射,而且是一对一的。这就是说,如果$x, y \in A$,只要$x \neq y$就有$f(x) \neq f(y)$,$f$就是单射的;如果对于每个$u \in B$,存在$x \in A$(可以多于一个),使得$u = f(x)$,$f$就是满射的。

设$S, T$是两个向量空间,映射
\[
f: S \to T
\]
叫做一个同态映射,如果对任意$x, y \in S$有$f(x+y)=f(x)+f(y)$,对每个$x \in S$,$a \in \mathbb{R}$有$f(ax) = af(x)$。

同态映射称为同构映射,如果它是双射的。向量空间$S$和$T$称为同构的,如果存在一个同构映射$f: S \to T$。

\textbf{注意} \quad 向量空间之间的同态映射也叫做线性映射。今后我们就采用这后一术语。

设$S$是一向量空间,子集$T \subset S$称为一个子空间,如果






