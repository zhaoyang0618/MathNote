\chapter{多元微积分}
《多元微积分》的作者是C.Goffman。参考:\cite{CalculusOfMultiVariables1978}。

\section{欧氏空间}\label{section00301}
欧氏空间是一个向量空间,它具有满足某些条件的距离函数。在这一章中,我们给出这些空间的定义和主要性质,还要讨论向量空间之间的线性映射及其性质,然后给出欧氏空间拓扑的一个简洁的处理。


\subsection{向量空间}\label{subsection0030101}
实数域$\mathbb{R}$上的\textbf{向量空间}是一个集合$S$,它带有映射
\[
g: S \times S \to S
\]
和
\[
v: \mathbb{R} \times S \to S,
\]
通常用记号$g(x,y) = x+y$和$v(a,x)=ax$记这两个映射,并假定下列条件成立:
\begin{enumerate}
\item[($a$)] $S$对于映射$g$成一Abel群。
\item[($b_1$)]对于每个$x \in S$和$a, b \in \mathbb{R}$,$(ab)x = a(bx)$。
\item[($b_2$)]对于每个$x \in S$和$a, b \in \mathbb{R}$,$(a+b)x=ax+bx$。
\item[($b_3$)]对于每个$x, y \in S$和$a \in \mathbb{R}$,$a(x+y)=ax+ay$。
\item[($b_4$)]对于每个$x \in S$,$1 \cdot x = x$。
\end{enumerate}

我们用记号$\theta$记$S$中群的恒等元。容易证明$0 \cdot x = \theta$对每个$x \in S$成立。不难看出,群的逆元是$(-1)x$,记为$-x$。我们把$y+(-x)$写为$y-x$。

我们只给出向量空间的两个例子。
\begin{example}
设$S$是$n$-实数组的集合。对于
\[
\begin{aligned}
x &= (x_1, \cdots, x_n) \in S,\\
y &= (y_1, \cdots, y_n) \in S,
\end{aligned}
\]
命
\[
x+y = (x_1+y_1, \cdots, x_n+y_n).
\]
对于
\[
x = (x_1, \cdots, x_n) \in S
\]
和$a \in R$,命
\[
ax = (ax_1, \cdots, ax_n).
\]
\end{example}








