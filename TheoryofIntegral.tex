\documentclass[12pt,a4paper]{book}
\usepackage{fontspec}
\usepackage{amsmath}
\usepackage{amsfonts}
\usepackage{amssymb}
\usepackage{harpoon}
\usepackage{extarrows}
\usepackage{hhtensor}
\usepackage{esvect}
\usepackage{bm}
\usepackage{index}
\usepackage{esint}
\usepackage[bf,small,center,indentafter,pagestyles]{titlesec}
\usepackage{subfigure}

\setromanfont{SimSun}
\setmainfont{SimSun}
%\setCJKmainfont[BoldFont=LiHei Pro]{KaiTi_GB2312}

\XeTeXlinebreaklocale "zh"
\XeTeXlinebreakskip = 0pt plus 1pt
\newtheorem{example}{例} 
\newtheorem{theorem}{定理}[section]
\newtheorem{definition}{定义}[section]
\newtheorem{axiom}{公理}[section]
\newtheorem{property}{性质}[section]
\newtheorem{proposition}{命题}[section]
\newtheorem{lemma}{引理}[section]
\newtheorem{corollary}{推论}[section]
\newtheorem{remark}{注解}[section]
\newtheorem{condition}{条件}
\newtheorem{conclusion}{结论}
\newtheorem{assumption}{假设}

\newcommand\relphantom[1]{\mathrel{\phantom{#1}}}
\newcommand\num[1]{\left\Vert{#1}\right\Vert}
\newcommand\grad{\text{grad\,}}
\newcommand\ddiv{\text{div\,}}
\newcommand\rot{\text{rot\,}}
\newcommand\sign{\text{sign\,}}
\newcommand\xrightrightarrows[1]{\stackrel{{#1}}{\rightrightarrows}}
\newcommand\nidotsint{\int\overbrace{\cdots}^{n}\int}

\makeatletter
\def\ExtendSymbol#1#2#3#4#5{\ext@arrow 0099{\arrowfill@#1#2#3}{#4}{#5}}
\def\RightExtendSymbol#1#2#3#4#5{\ext@arrow 0359{\arrowfill@#1#2#3}{#4}{#5}}
\def\LeftExtendSymbol#1#2#3#4#5{\ext@arrow 6095{\arrowfill@#1#2#3}{#4}{#5}}
\makeatother

\newcommand\myRightarrow[2][]{\RightExtendSymbol{=}{=}{\rightrightarrows}{#1}{#2}}

\makeindex
\newindex{name}{ndx}{nnd}{索引}
\usepackage{shorttoc}

\title{积分论}
\author{Stanislaw Saks}

\begin{document}
\renewcommand{\contentsname}{目\quad{}录}
\renewcommand{\chaptername}{}
%\renewcommand\thechapter{{第~\arabic{chapter}~章}}

\renewcommand{\chaptername}{第~\thechapter~章}

\titleformat{\chapter}[hang]{\centering\LARGE\bfseries}{\chaptername}{1em}{}

%\newcommand\prechaptername{第}
%\newcommand\postchaptername{章}
%\renewcommand\thechapter{{\prechaptername \value{chapter} \postchaptername}}
%\renewcommand\thechapter{{第~\arabic{chapter}~章\ #1}}
%\renewcommand\chapter{\thechapter}

\frontmatter
\begin{titlepage}
\maketitle
\end{titlepage}
\setcounter{page}{0}
\chapter{前言}
本版本与第一版不同之处:

\chapter{第一版前言}
在19世纪下半叶,实函数的现代理论与古典分析有了显著区别,

观点绝不孤立,Ch. Hermite在给T. J. Stieltjes的一封信中,一一种更强的语气表达了:“”。
 
\tableofcontents

\mainmatter
\chapter{抽象空间上的积分}
\section{导论}
本书中要讨论的函数,其自变量为一点集。这类函数已经在古典分析中几个重要的特殊场合中出现. 但是它们只是在集合论和基于集合论的相关分析知识发展起来后,才得到一般性的研究。

例如,如果给定一个在每个区间上可积的函数$f(x)$, 那么,把每个区间$I$上$f(x)$的积分的值联合起来,我们得到一个区间的函数$F(I)$. 类似的,对于$n$个变元的函数$f(x_1,\cdots, x_n)$的多重积分,我们需要考虑位于高维空间中的更一般的点集上的函数,函数$F(I)$中的参数$I$就会被任意点集替代,在此点集上定义了给定函数$f(x_1,\cdots, x_n)$的积分。

我们详述(dwell on)这些例子,是为了强调(在任何意义下in any sense)积分概念和集合上的函数概念之间的天然联系。无需多言,还有很多其他集合的函数的例子。在初等几何中,例如我们有线段的长度,或者多边形的面积。这两个函数(长度和面积)的参数的分类中,前者是线段,后者是多边形。这些类别的进一步推广引起了测度的一般理论的诞生。在初等几何中,关于图形的长度,面积,体积的概念,推广到了更加多样性的点集上。然而,

\chapter{Caratheodory测度}

\chapter{有界变差函数与Lebesgue-Stieltjes积分}

\chapter{集合与区间上的可加函数的导数}

\chapter{曲面$z = F(x, y)$的面积}

\chapter{极大和极小函数}

\chapter{广义有界变差函数}

\chapter{Denjoy积分}

\chapter{一个或两个实变数的函数的导数}

\chapter{关于Haar侧度的注记}

\chapter{抽象空间的Lebesgue积分的注记}

\printindex[name]
\addcontentsline{toc}{chapter}{索引}


\end{document}
