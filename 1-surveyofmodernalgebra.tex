\chapter{近世代数概论}
《近世代数概论》的作者是G.伯克霍夫和S.麦克莱恩。参考:\cite{surveyofmodernalgebra1979}。

\section{整数}
\subsection{交换环, 整环}
近世代数第一次揭示了数学系统的多变性和丰富性。本书从最基本也是最古老的正整数系统(整数系统,记为$\mathbb{Z}$)开始。

首先假定加法和乘法的八个公设,这些公设不仅对整数成立,而且对于很多数学系统都成立,例如所有有理数,所有实数,所有复数,所有多项式,任意已知区间上的连续函数。

\begin{definition}{交换环}{ring} 
设$R$是由元素$a,b,c,\cdots$组成的集合,在$R$上定义了任意两个元素$a$与$b$的和$a+b$及积$ab$.如果下列公设(i)-(viii)成立,那么$R$称为交换环:
\begin{enumerate}
\item[(i)] 封闭性. 若$a$, $b \in R$, 则$a+b \in R$,$ab \in R$.
\item[(ii)] 唯一性. 若$R$中$a=a'$且$b=b'$,则$a+b=a'+b'$以及$ab=a'b'$。
\item[(iii)]交换律. 对$R$中一切$a$与$b$,
\[
a+b=b+a,\quad ab=ba.
\]
\item[(iv)]对一切$a,b,c \in R$,
\[
\begin{aligned}
a + (b + c) &= (a+b)+c, \\
a(bc) &= (ab)c.
\end{aligned}
\]
\item[(v)]分配律. 对一切$a,b,c \in R$,
\[
a(b + c)=ab + ac. 
\]
\item[(vi)]零. $R$中包含元素$0$,使得对于一切$a \in R$, 
\[
a + 0 = a.
\]
\item[(vii)]单位元素. $R$中包含元素$1 \neq 0$,使得对于一切$a \in R$,
\[
a1=a.
\]
\item[(viii)]加法逆元素.对于每个$a \in R$,方程
\[
a + x = 0
\]
在$R$中有解$x$. $x$称为$a$的逆元素,并记为$-a$.
\end{enumerate}
\end{definition}

首先定义中的$1 \neq 0$,排除只包含一个元素$0$的情形。其次,$0$和$1$其实起着相似的作用,所以可以分别称为加法和乘法单位元。第三,交换中只保证了加法存在逆元素,对于乘法没有这个保证,这样一来,在整数集合$\mathbb{Z}$中,$c \neq 0$,且$ca=cb$,则必有$a=b$,这个结论对于一般的交换环不成立(例如区间上全体实函数组成的集合)。为此引入整环的概念。
\begin{definition}{整环}{integerRing} 
满足下面附加公设的交换环是整环:
\begin{enumerate}
\item[(ix)] 消去律. 若$c \neq 0$且$ca=cb$, 则$a=b$.
\end{enumerate}
\end{definition}

整环并不保证每个非零元素存在乘法逆元素。不过后面会证明$1$是有乘法逆元素的($1$自身),$-1$也有乘法逆元素$-1$。

这里应该多举一些交换环和整环的例子。

集合$\mathbb{Z}[\sqrt{2}] = \{a+b\sqrt{2} | a,b \in \mathbb{Z}\}$是一个整环,$a+b\sqrt{2}=c+d\sqrt{2}$当且仅当$a=c$且$b=d$,加法和乘法分别定义为:
\[
\begin{aligned}
(a + b\sqrt{2}) + (c + d\sqrt{2}) &= (a+c) + (b+d)\sqrt{2}, \\
(a + b\sqrt{2})(c + d\sqrt{2})&=(ac+2bd)+(ad+bc)\sqrt{2}.
\end{aligned}
\]

\subsection{交换环的基本性质}
当我们想要得到对于整个代数系统都正确的结论时,必须多加小心,我们必须确信,所有的证明只用到明显列出的公设和一般逻辑法则,其中最基本的逻辑法则是相等关系的三个基本定律:对一切$a$,$b$,$c$有
\begin{itemize}
\item 自反律 $a=a$.
\item 对称律 若$a=b$,则$b=a$.
\item 传递律 若$a=b$且$b=c$,则$a=c$.
\end{itemize}

下面任意交换环都成立的一些基本法则。 证明的时候只是需要注意只能使用公设或者前面证明的结论。这里省略,参考书本。

\begin{corollary}{法则1}{rule1}
对一切$a,b,c \in R$,有
\[
(a+b)c = ac + bc.
\]
\end{corollary}

这条法则可称为右分配律,可与公设(v)对比,公设(v)是左分配律。

\begin{corollary}{法则2}{rule2}
对一切$a \in R$,$0+a=a$且$1a=a$。
\end{corollary}

\begin{corollary}{法则3}{rule3}
如果$z \in R$满足:对一切$a \in R$,$a+z=a$,那么$z=0$。
\end{corollary}
这个法则说明加法单位元素0的唯一性。

\begin{corollary}{法则4}{rule4}
对一切$a, b, c \in R$成立:由$a+b=a+c$,可推出$b=c$。
\end{corollary}
这个法则称为加法消去律。

\begin{corollary}{法则5}{rule5}
对一切$a \in R$,存在唯一的$x \in R$满足$a+x=0$。
\end{corollary}
公设(viii)只保证了存在性,这个法则说明唯一性。

\begin{corollary}{法则6}{rule6}
对一切$a, b \in R$,存在唯一的$x \in R$使得$a+x=b$。
\end{corollary}
这个法则说明减法是可能的而且差是唯一的。

\begin{corollary}{法则7}{rule7}
对一切$a \in R$,$a \cdot 0 = 0 = 0 \cdot a$。
\end{corollary}

\begin{corollary}{法则8}{rule8}
如果$u \in R$满足:对一切$a \in R$,$au=a$,那么$u=1$。
\end{corollary}
这个法则说明乘法单位元素1的唯一性。

\begin{corollary}{法则9}{rule9}
对一切$a, b \in R$,$(-a)(-b)=ab$。
\end{corollary}

特别的有$(-1)(-1)=1$。这个证明起来稍微麻烦一点,不过只需要注意到$-a$,$-b$的定义,一步一步来还是可以得到的。只需要考虑
\[
[ab + a(-b)] + (-a)(-b) = ab + [a(-b) + (-a)(-b)]
\]
即可。中间的$a(-b)$可以换成$(-a)b$。另外需要使用法则7。

还有一条基本的代数定律是用于解二次方程的:若$ab=0$,则或者$a=0$或者$b=0$。遗憾的是,这个断语不是对一切交换环成立的。但是在任意的整环$D$中成立(可以根据乘法消去律证明)。反之,在任意交换环中,从这个断语可以得到消去律。若$a \neq 0$,从$ab=ac$有$ab-ac=a(b-c)=0$,可得$b-c=0$从而$b=c$.于是我们有:
\begin{theorem}{}{theorem1}
在交换环中,乘法消去律等价于“非零元素之积不为零”这个命题。
\end{theorem}
这里所谓“非零元素之积不为零”这个命题,可以用符号表示为:$a \neq 0$,$b \neq 0$,则必有$ab \neq 0$。我们把满足$ab=0$的非零元素$a$,$b$称为零因子。因此交换环中的消去律等价于“$R$中不包含零因子”。

前面提到$\mathbb{Z}[\sqrt{2}]$是整环,需要证明在$\mathbb{Z}[\sqrt{2}]$中成立消去律,这个可以使用这个定理来完成。证明过程参考书本,需要注意,这里需要用到结论$\sqrt{2}$不是有理数,也就是不能表示为$a/b$的形式,这里$a, b$是整数。

如果承认$\sqrt{2}$是实数,并且承认所有实数的集合构成整环,那么借助于子整环的概念可以非常容易证明$\mathbb{Z}[\sqrt{2}]$是整环。
\begin{definition}{子整环}{defSubIntegerRing}
整环$D$的子整环是$D$的子集,它对于同一种加法和乘法运算也是整环。
\end{definition}

子集$S$是子整环的充分必要条件是:$S$包含0和1;$S$包含其中任意元素$a$的加法逆元素;$S$包含其中任意两个元素$a$与$b$的和$a+b$以及积$ab$。换成集合语言,可以描述如下:
\begin{itemize}
\item $0 \in S$, $1 \in S$;
\item 对任意$a \in S$,必有$-a \in S$;
\item 对任意$a, b \in S$,有$a + b \in S$,$ab \in S$。
\end{itemize}

\subsection{有序整环的性质}
所有整数组成的环$\mathbb{Z}$在数学中起着独特的作用,因此我们将研究它的特殊性质。乘法交换律和消去律仅仅是其中两个,许多其他性质都来源于整数有可能被排成通常的次序:
\[
\cdots,-4,-3,-2,-1,0,1,2,3,4,\cdots
\]
这个次序常用关系$a<b$表示。关系$a<b$成立当且仅当差$b-a$为正整数。假设正整数$1,2,3,\cdots$集合的下列三个性质作为公设。
\begin{itemize}
\item 加法律 两个正整数的和是正整数。
\item 乘法律 两个正整数的积是正整数。
\item 三分律 对于已知整数$a$,下面三种情况中有且仅有一个成立:或者$a$为正整数,或者$a=0$,或者$-a$为正整数。
\end{itemize}

请注意,这里相当于根据这三个公设定义了正整数集合,也就是只要$\mathbb{Z}$的子集$\mathbb{Z}^+$满足这三个公设的就可以作为$\mathbb{Z}$的正整数集合。按照通常的加法,乘法,应该和我们以前学到的是一致的。有必要给这样的整环一个单独的名称。

\begin{definition}{有序整环}{defOrderedIntegerRing}
如果整环$D$中存在某些被称为正元素的元素,它们满足类似于上面对整数指出的加法,乘法和三分律这三个公设,那么称$D$为有序整环。
\end{definition}

明显,整数环$\mathbb{Z}$,有理数环$\mathbb{Q}$,实数环$\mathbb{R}$都是有序整环。所有复数构成的集合是整环,但是无法定义类似整数的序关系,不是有序整环。

\begin{theorem}{}{theorem2}
在任意有序整环中,一切非零元素的平方都是正的。
\end{theorem}

证明使用三分律以及前面的法则9即可。注意所谓平方,意指$a^2 = a \cdot a$。

由此定理,立即可以得到$1=1^2$是正的。从而可以证明在有序整环中$x^2+1=0$无解,也说明所有复数无法构成有序整环。

\begin{definition}{大于,小于关系}{defLess}
在有序整环中,$a<b$和$b>a$这两个等价的说法都意味着$b-a$是正的,还有$a \le b$的意思是$a<b$或者$a=b$。
\end{definition}
根据这个定义,正元素$a$可以描述为大于零的元素,元素$b<0$称为负元素。从定义还可以得出“小于关系”的传递律:
\begin{itemize}
\item 传递律 若$a < b$且$b<c$,则$a<c$。
\end{itemize}

证明直接使用定义以及加法律即可。事实上,根据定义以及正元素的三个公设,正好对应到不等式的三个性质:
\begin{itemize}
\item 不等式两边同时加上一个元素 若$a < b$,则$a+c < b+c$。
\item 不等式两边同时乘以一个正元素 若$a < b$且$c > 0$,则$ac < bc$。
\item 三分律 对任意$a$和$b$,三个关系式$a<b$,$a=b$和$a>b$中有且仅有一个成立。
\end{itemize}

证明不难,需要注意加上一个元素的时候,对这个元素没有限制,但是乘以一个元素的时候,要求这个元素必须是正元素,事实上,乘以负元素的话,不等号反向。

\begin{definition}{绝对值}{defAbsolute}
在有序整环中,当元素$a$为0时,它的绝对值$|a|$是0;否则$|a|$是元素对$a$,$-a$中的正元素。
\end{definition}
也就是$a$的绝对值可以表示如下: 
\[
|a| = \left\{
\begin{aligned}
a, &\quad a \ge 0 \\
-a, &\quad a < 0
\end{aligned}
\right.
\]
适当的分情况讨论,可以得到和的绝对值与积德绝对值的定律: 
\begin{equation}
|a+b| \le |a| + |b|; \quad |ab|=|a||b|.
\end{equation}

和的绝对值的定律也可以这样证明:
\[
-|a| \le a \le |a| \text{且} -|b| \le b \le |b|
\]
于是有
\[
-(|a|+|b|) \le  a + b \le |a|+|b|,
\]
由此得证。

\subsection{良序原则}
如果有序整环的子集$S$的每个非空子集都包含最小元素,那么$S$成为良序的。利用这个概念我们可以阐述整数的重要性质,这性质在特征上不是代数的,并且是其他数系不具备的。
\begin{itemize}
\item 良序原则 全体正整数的集合是良序的。
\end{itemize}

换句话说,正整数的任意非空集合$C$包含某最小元素$m \in C$,使$C$中的$c$总有$m \le c$。不过这里有一点疑惑,本书中这个良序原理是作为公理来接受的吗?还是需要证明?看来需要看其他书了解一下。

\begin{theorem}{}{theorem3}
0和1之间没有整数。
\end{theorem}

这个证明有点意思:假设存在适合$0<c<1$的任意整数$c$,那么所有这种整数的集合$C$是非空的。根据良序原则,这个集合存在最小整数$m$,并且$0 < m < 1$。用正数$m$乘不等式两边,得到$0 < m^2 < m$,于是$m^2$是集合$C$中的另一个整数,它小于已假定的$C$中的最小元素$m$,这个矛盾导出定理成立。

\begin{theorem}{}{theorem4}
如果正整数的一个集合$S$包含1,并且当它包含$n$时必包含$n+1$,那么集合$S$包含任意正整数。
\end{theorem}

证明使用良序原则。由那些不包含于$S$中的正整数组成的集合$S'$,证明$S'$是空集即可。

\subsection{数学归纳法,指数定律}
现在我们可以按加法,乘法及序完整地列出全体整数集合的基本性质,今后我们假定全体整数构成有序整环$\mathbb{Z}$,其中所有正元素的集合是良序的。全体整数的集合的其他每个数学性质,可以由此通过严格的逻辑推导来证明。特别的,可以导出非常重要的
\begin{itemize}
\item \textcolor{main}{数学归纳法原理} 设命题$P(n)$与每个正整数$n$有关,它或者正确,或者错误。如果(i)$P(1)$是正确的,(ii)对一切$k$,由$P(k)$推出$P(k+1)$,那么$P(n)$对一切正整数$n$都是正确的。
\end{itemize}

只需要考虑集合$C = \{k|P(k)\text{成立}\}$,这个集合满足前面的定理\ref{thm:theorem4}的条件。

现在用归纳的方法来证明在任意交换环中成立的各种定律。首先用它来形式地建立任意$n$个被加数的一般分配律。
\begin{equation}\label{equation0012}
a(b_1+b_2+\cdots+b_n) = ab_1+ab_2+\cdots+ab_n.
\end{equation}
为明确起见,定义累加和$b_1+b_2+\cdots+b_n$如下:
\[
\begin{aligned}
&b_1+b_2+b_3 = (b_1+b_2)+b_3,\\
&b_1+b_2+b_3+b_4 = [(b_1+b_2)+b_3]+b_4.
\end{aligned}
\]
一般的通过递推公式:
\begin{equation}\label{equation0013}
b_1+\cdots+b_k+b_{k+1} = (b_1+\cdots+b_k) + b_{k+1}.
\end{equation}

证明使用数学归纳法即可。类似的但更为复杂的归纳论证将得到一般结合律,它断言:和$b_1+\cdots+b_k$或者积$b_1\cdots{}b_k$不管把括号括在哪里都有相同的值。应用这个结果和\ref{equation0012},可以建立双边一般的分配律:
\[
\begin{aligned}
&(a_1+\cdots+a_m)(b_1+\cdots+b_n)\\
=&a_1b_1+\cdots+a_1b_n+\cdots+a_mb_1+\cdots+a_mb_n.
\end{aligned}
\]
注意,根据一般结合律和一般交换律,$k$个项的和不管项的次序与分组如何总有相同的值。

任意交换环$R$中的正整指数也可以归纳定义。如果$n$为正整数,则幂$a^n$表示$n$个因子的积$aa\cdots{}a$,这也可以递归定义:
\begin{equation}\label{equation0016}
a^1 = a, a^{n+1} = a^na. \quad(\forall a \in R)
\end{equation}由这些定义,我们可以对任意正整指数$m$和$n$证明下面常用的定律:
\begin{gather}
a^ma^n=a^{m+n},\label{equation0017}\\
(a^m)^n = a^{mn},\quad (ab)^m=a^mb^m.\label{equation0018}
\end{gather}

证明同样使用数学归纳法和递归定义即可。

最后,我们证明二项公式在任意交换环$R$上成立。首先用递推公式
\[
0!=1,\quad (n+1)!=n!(n+1),
\]
定义非负整数上的阶乘函数$n!$,然后对$\mathbb{Z}$中的$n \ge 0$,类似的用
\[
\binom{n}{0} = \binom{n}{n}=1, \quad \binom{n+1}{k} = \binom{n}{k-1} + \binom{n}{k} 
\]
定义二项系数。由这些定义,再对$n$用归纳法,得到 %\[n \choose m\]
\begin{align}\label{equation0019}
(x+y)^n &= x^n + nx^{n-1}y + \cdots + \binom{n}{k}x^{n-k}y^k + \cdots + y^n \notag \\
&=\sum_{k=0}^{n}{\binom{n}{k}x^{n-k}y^k}.
\end{align}
和
\begin{gather}\label{equation0020}
k!(n-k)!\binom{n}{k}=n!
\end{gather}
也就是
\[
{n \choose k} = \frac{n!}{k!(n-k)!}.
\]

数学归纳原理允许我们在证明$P(n+1)$时,随意假定$P(n)$的正确性,我们指出,人们甚至可以对一切$k \le n$假定$P(k)$的正确性,这称为
\begin{itemize}
\item \textcolor{main}{数学归纳法第二原理} 设命题$P(n)$与每个正整数$n$有关,如果对每个$m$,由假设"$P(k)$对一切$k <m$是正确的",可以推出"$P(m)$本身是正确的",那么$P(n)$对一切$n$都是正确的。
\end{itemize}

令$S$表示使$P(n)$错误的正整数集合,使用良序原理即可。注意,在$m=1$的情形中,所有$k<1$的集合是空的,因此必须暗含$P(1)$的证明。也就是在使用数学归纳法的时候,都需要证明$P(1)$成立。

\subsection{可除性}
整系数方程$ax=b$不总是有整数解$x$,如果有整数解,则称$b$可被$a$整除。在任意整环中也有类似的可除性概念。
\begin{definition}{整除}{defDividable}
在整环$D$中,如果有$D$中某一$q$,使$b=aq$,则称元素$b$可被元素$a$整除。当$b$可被$a$整除时,记作$a|b$,我们说$a$是$b$的因子,$b$是$a$的倍数。$1$的因子称为$D$的单位或可逆元素。
\end{definition}
关系$a|b$满足自反律和传递律:
\begin{itemize}
\item 自反律 $a|a$;
\item 传递律 由$a|b$和$b|c$可推出$a|c$.
\end{itemize}
自反律可以通过$a=a\cdot{}1$得到,至于第二个,使用定义:$a|b$和$b|c$意味着存在元素$d_1$和$d_2$,满足$b=ad_1$和$c=bd_2$,由此得到$c=a(d_1d_2)$,$d_2d_2 \in D$,按照定义$a|c$。


对于全体整数集$\mathbb{Z}$组成的整环来说,$1$和$-1$都是$1$的因子,因而都是$\mathbb{Z}$的单位或者可逆元素,而且也只有这两个单位。
\begin{theorem}{}{theorem0015}
$\mathbb{Z}$中仅有的单位是$\pm1$
\end{theorem}
对于整数$a$和$b$,$ab=1$意味着$a=\pm1$和$b=\pm1$。这个证明需要使用到有序整环中的概念,以及良序原则得到的定理\ref{thm:theorem3}:从$ab=1$得到$|a||b|=1$,而整环中不存在零因子,可以知道$|a|>0$和$|b|>0$,最后通过三分律以及不等式的性质可以知道$|a|$和$|b|$只能是1.

\begin{corollary}{}{corollary0015}
如果整数$a$和$b$彼此可整除,即$a|b$且$b|a$,那么$a=\pm{}b$。
\end{corollary}

证明需要使用到消去律和上述定理。

因为$a=a \cdot 1 = (-a) \cdot (-1)$,任意整数$a$可被$a$,$-a$,$1$和$-1$整除,我们有定义:
\begin{definition}{素数}{defPrime}
如果整数$p$不为0或$\pm{}1$,并且$p$只能被$\pm{}1$和$\pm{}p$整除,那么称$p$为素数。
\end{definition}
这个概念应该是可以被推广到一般整环的。到后面学到理想概念之后再来对比整数里面的素数。

\subsection{欧几里得算法}
整数$a$除以$b$用普通的除法就得到商$q$和余数$r$。也就是
\begin{itemize}
\item \textcolor{main}{除法算式} 对于给定的整数$a$和$b$,$b>0$,存在整数$q$和$r$,使得
\begin{equation}\label{equation0012}
a = bq + r, \quad 0 \le r < b.
\end{equation}
\end{itemize}

从几何上看,说明$a$会在区间$[bq, b(q+1)]$上,去掉右端点。证明使用良序原理,考虑集合$S = \{a-bx|a-bx \ge 0, x \in \mathbb{Z}\}$。要使用良序原理,我们需要证明$S$非空,注意到$b>0$,对于整数,就有$b \ge 1$,$-|a|b \le -|a| \le a$,于是$a - (-|a|b) \ge 0$,$S$非空。

\begin{corollary}{}{corollary0013}
对给定的整数$a$和$b$,满足等式\ref{equation0012}的商$q$和余数$r$是唯一确定的。
\end{corollary}

反证法即可,不过需要结论:$a|b$,并且$|b| < |a|$,那么只能是$b = 0$。或者说$a|b$时,必有$|a| \le |b|$。

我们经常有必要不涉及单个整数,而是去处理某整数集合。如果集合$S$包含$S$中任意两个元素$a$与$b$的和$a+b$及差$a-b$,则称集合$S$在加法与减法之下封闭。所有偶数构成这样的集合。更一般的,任意固定的整数$m$的所有倍数$xm$的集合在加法与减法之下是封闭的,反过来也成立,也就是说:这种倍数的集合是具有这些性质的唯一的整数集合。
\begin{theorem}{}{theorem0016}
在加法与减法之下封闭的任意非空整数集合,不是仅由零组成,就是包含最小正整数并由这个整数的所有倍数组成。
\end{theorem}

证明参考书本,只是提示一点:对于这样的集合$S$,必有$0 \in S$,然后就有$a \in S$,必有$-a \in S$,从而必然有正整数。由此得到最小的正整数$m$,然后归纳证明$S$包含所有$m$的倍数,再证明除了$m$的倍数之外不能有其他。

\begin{definition}{}{defGreateCommonFactor}
如果整数$d$是整数$a$与$b$的公因子,并且是任何其他公因子的倍数,那么称$d$为$a$与$b$的最大公因子(g.c.d.)。也就是$d$满足
\[
d|a; \quad{}d|b;\quad{} c|a\text{和}c|b \text{可推出}c|d.
\]
\end{definition}

例如$3$和$-3$都是6和9的最大公因子。按照定义,两个不同的最大公因子必彼此整除,因此它们仅相差一个符号。$a$和$b$中两个可能的最大公因子$\pm{}d$中,正的最大公因子常用符号$(a, b)$表示。值得注意的是,最大公因子定义中的“最大”,主要不是指$d$的数值比其他公因子$c$大,而是指$d$为任何这种数$c$的倍数。

\begin{theorem}{}{theorem0017}
任意两个整数$a \neq 0$和$b \neq 0$有正的最大公因子$(a, b)$,它可表为$a$和$b$的具有整系数的$s$和$t$的线性组合,形为
\begin{equation}\label{equation0013}
(a,b)=sa+tb.
\end{equation}
\end{theorem}

考虑形为$sa+tb$的所有数,这些数组成的集合对加法和减法封闭,从而存在最小正整数$d$,然后证明它就是正的最大公约数。

类似,$a$和$b$的公倍数的集合$M$在加法和减法之下也是封闭的,它的最小正元素$m$将是$a$和$b$的公倍数,它整除每个公倍数,于是$m$是最小公倍数(l.c.m.)。
\begin{theorem}{}{theorem0018}
任意两个整数$a \neq 0$和$b \neq 0$有最小公倍数$m=[a, b]$,它是$a$和$b$的每个公倍数的因子,并且它自己也是$a$和$b$的公倍数。
\end{theorem}

为找到两个整数$a$和$b$的最大公因子,可应用所谓欧几里得算法。由于$(a, b) = (a, -b)$,我们可以假设$a$和$b$都是正整数。除法公式给出:
\begin{gather}\label{equation0014}
a = bq_1+r_1, \quad 0 \le r_1 < b,
\end{gather}
整除$a$和$b$的每个整数必整除余数$r_1$,反之,$b$和$r_1$的每个公因子是$a$的因子,所以$a$与$b$的公因子和$b$与$r_1$的公因子相同,从而$(a,b)= (b, r_1)$。于是我们可以在$b$和$r_1$继续执行类似操作:
\begin{equation}
\begin{aligned}
b &= r_1q_2 + r_2, \quad  &0 < r_2 < r_1 \\
r_1 &= r_2q_3 + r_3, \quad &0 < r_3 < r_2 \\
&\cdots& \cdots\\
r_{n-2} &= r_{n-1}q_n + r_n, \quad &0 < r_n < r_{n-1}\\
r_{n-1} &= r_nq_{n+1}&
\end{aligned}
\end{equation}
因为余数不断减小,最后必有余数$r_{n+1}$为零。所要求的最大公因子是:
\[
(a, b) = (b, r_1) = (r_1, r_2) = \cdots = (r_{n-1}, r_n) = r_n.
\]

利用欧几里得算法,可以把最大公因子显式地表示为线性组合$sa+tb$,这只需要用$a$和$b$表示逐次的余数$r_i$即可。
\[
\begin{aligned}
r_1 &= a - bq_1 = a + (-q_1)b \\
r_2 &= b - q_2r_1 = (-q_2)a + (1 + q_1q_2)b \\
&\cdots
\end{aligned}
\]

利用$(a, b) = sa+tb$可以证明下面的定理:
\begin{theorem}{}{theorem0019}
如果$p$为素数,那么由$p|ab$可推出$p|a$或$p|b$。
\end{theorem}

当$p$为素数的时候,如果$p|a$不成立,那么必有$(p, a)=1$。于是$1 = sa+tp$,两边乘以$b$即可得到结论。

如果$(a, b)=1$,就称$a$和$b$互素。用前面的方法可以证明:
\begin{theorem}{}{theorem0020}
如果$(c, a)=1$且$c|ab$,那么$c|b$。
\end{theorem}

运用定理\ref{thm:theorem0020},再加上整除的定义,可以证明下面的:
\begin{theorem}{}{theorem0021}
如果$(a, c)=1$,$a|m$且$c|m$,那么$ac|m$。
\end{theorem}

\subsection{算术基本定理}
现在可以证明整数唯一因子分解定理,也成为算术基本定理。
\begin{theorem}{算术基本定理}{theorem0022}
任意非零整数可表为单位($\pm{}1$)乘以正素数的积,如果不计素因子出现的顺序,这种表示是唯一的。
\end{theorem}

存在性证明使用数学第二归纳法,至于唯一性,使用上一节的定理\ref{thm:theorem0019}。

数的因子分解中,同一个素数$p$可以出现多次。把所出现的相同的素数集中起来,分解式可写为:
\begin{gather}\label{equation0015}
a = \pm{}p_1^{e_1}p_2^{e_2} \cdots p_k^{e_k} \quad (1 < p_1 < p_2 < \cdots < p_k).
\end{gather}
由唯一性可知,每个素数$p_i$的指数$e_i$是由给定的$a$唯一确定的。

\subsection{同余式}
两个整数$a$和$b$对模$m$同余定义如下:
\begin{definition}{同余}{defModule}
$a \equiv b (\mod{m})$成立当且仅当$m|(a-b)$。
\end{definition}
我们也可以说$a \equiv b(\mod{m})$的意思是差$a-b$在$m$的所有倍数的集合中。另外还可以根据下述事实来定义:每个整数$a$除以$m$剩下唯一的余数。
\begin{theorem}{}{theorem0023}
两个整数$a$和$b$对模$m$同余当且仅当它们除以$|m|$时剩下相同的余数。
\end{theorem}
注意到$a \equiv b(\mod{m})$当且仅当$a \equiv b(\mod{m})$,只需要对$m>0$进行证明即可。证明使用定义即可。

固定模$m$的同于关系具有和相等类似的性质(很多时候在知道模$m$的时候,经常会省略$(\mod{m})$,就如下面所示):
\begin{itemize}
\item 自反律 $a \equiv a$.
\item 对称律 若$a \equiv b$,则$b \equiv a$.
\item 传递律 若$a \equiv b$且$b \equiv c$,则$a \equiv c$.
\end{itemize}
证明使用定义即可完成。

固定模$m$的同余关系还具有“代换性质”,这也是相等关系的性质之一,即:同余整数之和同余,而且同余整数之积同余。用同余式表示为:$a_1 \equiv b_1$,$a_2 \equiv b_2$,那么$a_1+a_2 \equiv b_1+b_2$,$a_1a_2 \equiv b_1b_2$。
\begin{theorem}{}{theorem0024}
如果$a \equiv b(\mod{m})$,那么对一切整数$x$,有
\[
a+x \equiv b+1, \quad ax \equiv bx, \quad  -a \equiv -b \quad (\mod{m})
\]
\end{theorem}

同样使用定义即可证明。

对于方程成立的消去律对于同余式不一定成立。




