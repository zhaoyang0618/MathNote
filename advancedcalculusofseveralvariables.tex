\documentclass[cn,11pt,chinese]{elegantbook}

\usepackage[all]{xy}
\usepackage{amsmath}
\usepackage{asymptote}
\usepackage{subfig}
\usepackage{graphicx}



\newcount\mycount
%\def\ConvColor{rgb:yellow,5;red,2.5;white,5}
%\def\ConvReluColor{rgb:yellow,5;red,5;white,5}
%\def\PoolColor{rgb:red,1;black,0.3}
%\def\FcColor{rgb:blue,5;red,2.5;white,5}
%\def\FcReluColor{rgb:blue,5;red,5;white,4}
%\def\SoftmaxColor{rgb:magenta,5;black,7}
\def\cis{\,\text{cis}\,}
\def\intset{\operatorname{int}}
\def\diam{\operatorname{diam}}
\def\dist{\operatorname{dist}}
\def\ulim{\operatorname{u-lim}}
\def\cinfty{\mathbb{C}_{\infty}}
\def\sh{\operatorname{sh}}
\def\argsh{\operatorname{argsh}}
\def\ch{\operatorname{ch}}
\def\argch{\operatorname{argch}}

%\newtheorem{lemma}{法则}
% 方程编号以section为准
\numberwithin{equation}{section}


% title info
\title{多元微积分}

% bio info
\author{C.H. Edwards, Jr.}
%\date{\today}

% extra info
\version{1.00}
\extrainfo{Wir m\"ussen wissen, wir werden wissen. (我们必须知道,我们必将知道) - David.Hilbert}
%\logo{logo.png}
\cover{cover.jpg}

\begin{document}

%$\mathop {\min }\limits_x f(x)$
%
%\begin{displaymath}
%\mathop{\sum \sum}_{i,j=1}^{N} a_i a_j 
%{\sum \sum}_{i,j=1}^{N} a_i a_j
%\end{displaymath}


\maketitle

\tableofcontents
\mainmatter
\hypersetup{pageanchor=true}

\chapter{欧氏空间与线性映射}\label{chapter001}
\section{向量空间}\label{section00101}
$n$元组$(x_1, \cdots, x_n)$按照如下运算组成向量空间$R^n$: 
\begin{enumerate}
\item[(I)]\textbf{向量加法(vector addition)} 对于$x = (x_1,\cdots, x_n)$, $y = (y_1,\cdots, y_n)$, 向量加法定义为
\[
x+y = (x_1+y_1,\cdots, x_n+y_n),
\]
\item[(II)]\textbf{数乘(或者叫标量乘法, scalar multiplication)} 对于$x = (x_1,\cdots, x_n)$,以及$a \in R$,定义数乘为
\[
ax = (ax_1, \cdots, ax_n).
\]
\end{enumerate}

可以验证上面定义的两个运算满足下面的条件:
\begin{enumerate}
\item[V1] $x + (y + z) = (x + y) + z$, \quad (加法结合律)
\item[V2] $x + y = y + x$, \quad (加法交换律)
\item[V3] $x+0=x$, \quad (加法单位元,零元的存在性)
\item[V4] $x + (-x) = 0$, \quad (加法逆元的存在性)
\item[V5] $(ab)x = a(bx)$, \quad (数乘结合律)
\item[V6] $(a+b)x = ax+bx$, \quad (分配律)
\item[V7] $a(x+y) = ax+ay$, \quad (分配律)
\item[V8] $1x = x$.
\end{enumerate}

一般的,向量空间定义如下:
\begin{definition}{向量空间}{def0010101}
向量空间是指集合$V$,以及两个映射$V \times V \to V$和$R \times V \to V$,这两个映射满足公设V1-V8。$V$的子集$W$称为$V$的子空间,如果它在$V$的两个映射下,构成向量空间。
\end{definition}

容易证明:$W$是$V$的子空间,当且仅当向量加法和数乘在$W$中封闭,当且仅当$W$中向量的线性组合仍然属于$W$。


% \bibliographystyle{plain}
\bibliography{mathreference}
\appendix

\end{document}
